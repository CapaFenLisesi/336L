\chapter{Inviscid Fluid Dynamics}
\section{Introduction}
This  chapter introduces some of the fundamental concepts that arise in the theory
of incompressible, {\em inviscid}\/ (or, to be more exact, high Reynolds number) fluid motion.

\section{Streamlines,   Stream Tubes, and Stream Filaments}
A line drawn in a fluid such that its tangent at each point is parallel to the local fluid velocity is called
a {\em streamline}. The aggregate of all the streamlines at a given instance in time constitutes
the instantaneous {\em flow pattern}.
The streamlines drawn through each point of a closed curve constitute  a {\em stream tube}. Finally, a
{\em stream filament}\/ is defined as a stream tube whose cross-section is a curve of infinitesimal
dimensions. 

When the flow is unsteady then the configuration of the stream tubes and filaments changes from
time to time. However, when the flow is {\em steady}\/ then the stream tubes and filaments are
{\em stationary}. In the latter case, a stream tube acts like an actual tube through which the fluid
is flowing. This follows because there can be no flow across the walls, and into the tube, since the
flow is, by definition, always tangential to these walls. Moreover, the walls are fixed in space
and time, since the motion is steady. Thus, the motion of the fluid within the
tube would be unchanged were the walls replaced by a rigid  frictionless boundary. 

Consider a stream filament of an incompressible fluid whose motion is steady. Suppose that the cross-sectional area of the
filament is  sufficiently small that the fluid velocity is the same at  each point on the
cross-section. Moreover, let the cross-section be  everywhere normal to the direction of this common velocity. 
Suppose that $v_1$ and $v_2$ are the flow speeds at two points on the filament at which the cross-sectional
areas are $S_1$ and $S_2$, respectively. Consider the section of the filament lying between these points.
Since the fluid is {\em incompressible},  the same
volume of fluid must flow into one end of the section, in a given time interval, as flows out of the other, which implies that
\begin{equation}
v_1\,S_1=v_2\,S_2.
\end{equation}
This is the simplest manifestation of the equation of fluid continuity discussed in Section~\ref{scont}. 
The above result is equivalent to the statement  that the product of the
speed and cross-sectional area is constant along any stream filament of an incompressible fluid in steady motion.
Thus, a stream filament within such a fluid cannot terminate unless the velocity at that
point becomes infinite. Leaving this case out of consideration, it follows that stream filaments
in steadily flowing incompressible fluids are either closed loops, or terminate at the boundaries of the fluid. 
The same is, of course, true of streamlines. 

\section{Bernoulli's Theorem}\label{sqbern}
In its most general form, {\em Bernoulli's theorem}---which was discovered by Daniel Bernoulli (1700--1783)---states that, in the {\em steady}\/ flow of an {\em inviscid}\/ fluid, the quantity
\begin{equation}
\frac{p}{\rho} + {\cal T}
\end{equation}
is constant along a streamline, where $p$ is the pressure, $\rho$ the density, and ${\cal T}$
the total energy per unit mass.

\begin{figure}
\epsfysize=2.75in
\centerline{\epsffile{Chapter05/bernoulli.eps}}
\caption{\em Bernoulli's theorem.}\label{fbern}
\end{figure}

The proof is straightforward. Consider the body of fluid bounded by the cross-sectional areas $AB$ and $CD$ of the stream filament
pictured in Figure~\ref{fbern}.  Let us denote the values of quantities at $AB$ and $CD$ by the suffixes $1$ and
$2$, respectively. Thus, $p_1$, $v_1$, $\rho_1$, $S_1$, ${\cal T}_1$ are the pressure, fluid speed, density, cross-sectional
area, and total energy per unit mass, respectively, at $AB$, {\em etc.} Suppose that, after a short time interval $\delta t$, the body of fluid has moved such
that it occupies the section of the filament bounded by the cross-sections $A'B'$ and $C'D'$, where
$AA'=v_1\,\delta t$ and $CC'=v_2\,\delta t$. Since the motion is {\em steady}, the mass $m$ of the fluid between $AB$
and $A'B'$ is the same as that between $CD$ and $C'D'$, so that
\begin{equation}
m= S_1\,v_1\,\delta t\,\rho_1=S_2\,v_2\,\delta t\,\rho_2.
\end{equation}
Let $T$ denote the total energy of the section of the fluid lying between $A'B'$ and $CD$. Thus, the increase in energy 
of the fluid body in the time interval $\delta t$ is
\begin{equation}\label{q3.4}
(m\,{\cal T}_2+T)-(m\,{\cal T}_1+T)=m\,({\cal T}_2-{\cal T}_1).
\end{equation}
In the absence of viscous energy dissipation, this energy increase must equal the net work done by the fluid pressures at
$AB$ and $CD$, which is
\begin{equation}\label{q3.5}
p_1\,S_1\,v_1\,\delta t - p_2\,S_2\,v_2\,\delta t = m\left(\frac{p_1}{\rho_1}-\frac{p_2}{\rho_2}\right).
\end{equation}
Equating expressions (\ref{q3.4}) and (\ref{q3.5}), we find that
\begin{equation}
\frac{p_1}{\rho_1}+ {\cal T}_1= \frac{p_2}{\rho_2}+{\cal T}_2,
\end{equation}
which demonstrates that $p/\rho+{\cal T}$ has the same value at any two points on a given stream filament, and is therefore constant along the filament. Note that Bernoulli's theorem has only been proved for the case of the {\em steady}\/ motion of an {\em inviscid}\/ fluid. However,
the fluid in question may either be compressible or incompressible.  

For the particular case of an incompressible fluid, moving in a conservative force-field, the total energy per unit mass is the
sum of the kinetic energy per unit mass, $(1/2)\,v^2$, and the potential energy per unit mass, $\Psi$, and
Bernoulli's theorem thus becomes
\begin{equation}
\frac{p}{\rho} + \frac{1}{2}\,v^2 + \Psi = \mbox{constant along a streamline}.
\end{equation}
If we focus on a particular streamline, 1 (say), then Bernoulli's theorem states that
\begin{equation}
\frac{p}{\rho} + \frac{1}{2}\,v^2 + \Psi = C_1,
\end{equation}
where $C_1$ is a constant characterizing that streamline. If we consider a second streamline, 2 (say), then
\begin{equation}
\frac{p}{\rho} + \frac{1}{2}\,v^2 + \Psi = C_2,
\end{equation}
where $C_2$ is another constant. It is not generally the case that $C_1=C_2$. If, however, the fluid motion is
{\em irrotational}\/  then the constant in Bernoulli's theorem is the same for all streamlines (see Section~\ref{qirr}), so
that
\begin{equation}
\frac{p}{\rho} + \frac{1}{2}\,v^2 + \Psi = C
\end{equation}
throughout the fluid.

\section{Vortex Lines,   Vortex Tubes, and Vortex Filaments}
The curl of the velocity field of a fluid, which is generally termed  {\em vorticity},  is usually
represented by the symbol $\bomega$: {\em i.e.},
\begin{equation}
\bomega \equiv \nabla\times {\bf v}.
\end{equation}

A {\em vortex line}\/ is a line whose tangent is everywhere parallel to the local vorticity vector.
The vortex lines drawn through each point of a closed curve 
constitute the surface of a {\em vortex tube}. Finally, a
{\em vortex filament}\/ is a vortex tube whose cross-section is of infinitesimal dimensions. 

\begin{figure}
\epsfysize=2.5in
\centerline{\epsffile{Chapter05/vortex.eps}}
\caption{\em A vortex filament.}\label{fvortex}
\end{figure}

Consider a section $AB$ of a vortex filament. The filament is bounded by the curved surface that forms the filament
wall, as well as two plane surfaces, whose vector areas are ${\bf S}_1$ and ${\bf S}_2$ (say), which form the ends of the section at
points $A$ and $B$, respectively. See Figure~\ref{fvortex}. Let the plane surfaces have outward pointing normals
that are parallel (or anti-parallel) to the vorticity vectors, $\bomega_1$ and $\bomega_2$, at points $A$ and $B$,
respectively.
Gauss's theorem (see Section~\ref{sdiv}), applied to the section, yields
\begin{equation}
\oint \bomega \cdot d{\bf S} = \int \nabla\cdot\bomega\,dV,
\end{equation}
where $d{\bf S}$ is an outward directed surface element, and $dV$ a volume element. However, 
\begin{equation}
\nabla\cdot\bomega=\nabla\cdot\nabla\times {\bf v} \equiv 0
\end{equation}
[see Equation~(\ref{divcurl})],
implying that
\begin{equation}
\oint \bomega \cdot d{\bf S} =0.
\end{equation}
Now, $\bomega\cdot d{\bf S}=0$ on the curved surface of the filament,
since $\bomega$ is, by definition, tangential to this surface. Thus, the only contributions to the surface integral
come from the plane areas ${\bf S}_1$ and ${\bf S}_2$. It follows that
\begin{equation}
\oint \bomega \cdot d{\bf S} = S_2\,\omega_2-S_1\,\omega_1 = 0.
\end{equation}
This result is essentially an equation of continuity for vortex filaments. It implies that the product of the
magnitude of the vorticity and the cross-sectional area, which is termed the vortex  {\em intensity}, is constant
along the filament. It follows that a vortex filament cannot terminate in the interior of the fluid. For, if
it did,  the cross-sectional area, $S$, would have to vanish, and, therefore, the vorticity, $\omega$, would have to become infinite. Thus,
a vortex filament must either form  a closed vortex ring, or must terminate at the fluid boundary. 

Since a vortex tube can be regarded as a bundle of vortex filaments whose net intensity is the sum of the
intensities of the constituent  filaments, we conclude that the intensity of a vortex tube
remains constant along the tube. 

\section{Circulation and Vorticity}\label{scirc}
Consider a closed curve $C$ situated entirely within a moving fluid.  The
vector line integral (see Section~\ref{sveclinei})
\begin{equation}
\Gamma_C = \oint_C {\bf v}\cdot d{\bf r},
\end{equation}
where $d{\bf r}$ is an element of $C$, and the integral is taken around the whole curve, is termed the {\em circulation}\/ of the flow around the curve. 
The sense of circulation ({\em e.g.}, either clockwise or counter-clockwise) is arbitrary. 

Let $S$ be a surface having the closed curve $C$ for a boundary, and let $d{\bf S}$ be an element of this surface (see Section~\ref{sect22})  with that direction of the normal which is related to the chosen sense of circulation around $C$ by the right-hand
circulation rule (see Section~\ref{svecp}).  According to Stokes' theorem (see Section~\ref{scurl}),
\begin{equation}\label{q3.12}
\Gamma_C= \oint_C {\bf v}\cdot d{\bf r} = \int_S \bomega\cdot d{\bf S}.
\end{equation}
Thus, we conclude that circulation and vorticity are intimately related to one another. In fact, according to the
 above expression, the circulation of the fluid around loop $C$ is equal to the net sum of the  intensities of the
 vortex filaments passing through the loop and piercing the surface $S$ (with a filament making a
 positive, or negative, contribution to the sum depending on whether it pierces the surface
 in the direction determined by the chosen sense of circulation around $C$ and the right-hand circulation rule, or in the opposite direction). 
 One important proviso to (\ref{q3.12}) is that the surface $S$ must lie entirely within the fluid. 

\section{Kelvin Circulation Theorem}\label{skelvin}
According to the {\em Kelvin circulation theorem}, which is named after Lord Kelvin  (1824--1907),  the
circulation around any {\em co-moving}\/ loop in an
{\em inviscid}\/ fluid  is {\em independent}\/ of time. The
proof is as follows. The circulation around a given loop $C$ is defined
\begin{equation}
\Gamma_C = \oint_C {\bf v}\cdot d{\bf r}.
\end{equation}
However, for a loop that is co-moving with the fluid, we have $d{\bf v} = d(d{\bf r}/dt)= d(d{\bf r})/dt$. Thus,
\begin{equation}
\frac{d\Gamma_C}{dt} = \oint_C \frac{d{\bf v}}{dt}\cdot d{\bf r} + \oint_C {\bf v}\cdot d{\bf v}.
\end{equation}
However, by definition, $d{\bf v}/dt\equiv D{\bf v}/Dt$ for a co-moving loop (see Section~\ref{sconv}). Moreover, the equation of motion of
an incompressible inviscid fluid can be written [see Equation~(\ref{e5.78})]
\begin{equation}
\frac{D{\bf v}}{Dt}=   - \nabla\left(\frac{p}{\rho}+\Psi\right),
\end{equation}
since $\rho$ is a constant. Hence, 
\begin{equation}
\frac{d\Gamma_C}{dt} =- \oint_C\nabla\!\left(\frac{p}{\rho}- \frac{1}{2}\,v^2+\Psi\right)\cdot d{\bf r}= 0,
\end{equation}
since ${\bf v}\cdot d{\bf v} = d(v^2/2)=\nabla(v^2/2)\cdot d{\bf r}$ (see Section~\ref{sgrad}), and $p/\rho -\frac{1}{2}\,v^2+\Psi$ is obviously a single-valued function. 

\begin{figure}
\epsfysize=2.5in
\centerline{\epsffile{Chapter05/kelvin.eps}}
\caption{\em A vortex tube.}\label{fkelvin}
\end{figure}

One corollary of the Kelvin circulation theorem is that the fluid particles that form the walls of a vortex tube at a given instance in time continue to
form the walls of a vortex tube at all subsequent times. To prove this, imagine a closed loop $C$ that is embedded in the wall of a vortex tube but does
not circulate around the interior of the tube. See Figure~\ref{fkelvin}. The normal component of the
vorticity over the surface enclosed by $C$ is zero, since all vorticity vectors are tangential to this surface.
Thus, from (\ref{q3.12}), the circulation around the loop is zero. By Kelvin's circulation theorem, the circulation around the loop remains zero as the
tube is convected by the fluid. In other words, although the
surface enclosed by $C$  deforms, as it is convected by the fluid, it always remains on the  tube wall, since
no vortex filaments can pass through it. 

Another corollary of the circulation theorem is that the intensity of a vortex tube remains constant as it
is convected by the fluid. This can be proved by considering the circulation around the loop $C'$ pictured
in Figure~\ref{fkelvin}. 

\section{Irrotational Flow}\label{qirr}
Flow is said to be {\em irrotational}\/ when the vorticity $\bomega$ has the magnitude zero everywhere. 
It immediately follows, from Equation~(\ref{q3.12}), that the circulation around any arbitrary loop in an irrotational
flow pattern is zero (provided that the loop can be spanned by a surface that lies entirely within the fluid). Hence, from Kelvin's circulation theorem, if an inviscid fluid is initially irrotational
then it remains irrotational at all subsequent times. This can be seen more directly from the
equation of motion of an inviscid incompressible fluid which, according to  Equations~(\ref{e4.35}) and (\ref{e5.78}),  takes the
form
\begin{equation}
\frac{\partial{\bf v}}{\partial t} + ({\bf v}\cdot \nabla)\,{\bf v} = - \nabla\left(\frac{p}{\rho}+\Psi\right),
\end{equation}
since $\rho$ is a constant. However, from Equation~(\ref{divsp}), 
\begin{equation}
({\bf v}\cdot\nabla)\,{\bf v} \equiv \nabla(v^2/2) -{\bf v}\times \bomega.
\end{equation}
Thus, we obtain
\begin{equation}\label{eq3.24}
\frac{\partial {\bf v}}{\partial t} = -\nabla\left(\frac{p}{\rho}+\frac{1}{2}\,v^2+\Psi\right)+ {\bf v}\times \bomega.
\end{equation}
Taking the curl of this equation, and making use of the vector identities $\nabla\times \nabla\phi\equiv 0$ [see Equation~(\ref{curlgrad})], 
$\nabla\cdot\nabla\times {\bf A}\equiv {\bf 0}$ [see Equation~(\ref{divcurl})], as well as the identity (\ref{curlvp}), and the fact that $\nabla\cdot{\bf v}=0$
in an incompressible fluid, we obtain the vorticity evolution equation
\begin{equation}\label{eq3.25}
\frac{D{\bomega}}{Dt} = (\bomega\cdot\nabla)\,{\bf v}.
\end{equation}
Thus, if $\bomega={\bf 0}$, initially, then $D\bomega/Dt={\bf 0}$, and, consequently, $\bomega={\bf 0}$  at all subsequent times. 

Suppose that $O$ is a fixed point, and $P$ an arbitrary movable point, in an irrotational fluid. Let $O$ and $P$ be joined
by two different paths, $OAP$ and $OBP$ (say). It follows that $OAPBO$ is a closed curve. Now, since the circulation
around such a curve in an irrotational fluid is zero, we can write
\begin{equation}
\int_{OAP} {\bf v}\cdot d{\bf r} + \int_{PBO} {\bf v}\cdot d{\bf r} = 0,
\end{equation}
which implies that
\begin{equation}
\int_{OAP} {\bf v}\cdot d{\bf r} =\int_{OBP} {\bf v}\cdot d{\bf r} = -\phi_P
\end{equation}
(say). It is clear that $\phi_P$ is a scalar function whose value depends  on the
position of $P$ (and the fixed point $O$), but not on the path taken between $O$ and $P$.
Thus, if $O$ is the origin of our coordinate system, and $P$ an arbitrary point whose position
vector is ${\bf r}$, then we have effectively defined a scalar field $\phi({\bf r})\equiv-\int_O^P {\bf v}\cdot d{\bf r}$.

Consider a point $Q$ that is sufficiently close to $P$ that the velocity ${\bf v}$ is constant along $PQ$. 
Let $\bdeta$ be the position vector of $Q$ relative to $P$. It  then follows that (see Section~\ref{sgrad})
\begin{equation}
-\bdeta\cdot\nabla\phi = -\phi_Q+\phi_P = \int_{P}^Q {\bf v}\cdot d{\bf r} \simeq {\bf v}\cdot \bdeta.
\end{equation}
The above equation becomes exact in the limit that $|\bdeta|\rightarrow 0$. Since $Q$ is arbitrary (provided that it is
sufficiently close to $P$), the
direction of the vector $\bdeta$ is also arbitrary, which implies that
\begin{equation}\label{e3.51t}
{\bf v} = -\nabla\phi.
\end{equation}
We, thus, conclude that if the motion of a fluid is irrotational then the associated velocity field can always be expressed as minus the
gradient of a scalar function of position, $\phi({\bf r})$. This scalar function is called the {\em velocity potential}, and
flow which is derived from such a potential is known as {\em potential flow}. Note that the velocity potential
is undefined to an arbitrary additive constant.

We have demonstrated that a velocity potential necessarily exists in a fluid whose velocity field is irrotational. 
Conversely, when a velocity potential exists the flow is necessarily irrotational. This follows because [see Equation~(\ref{curlgrad})]
\begin{equation}
\bomega=\nabla\times {\bf v} = -\nabla\times\nabla\phi\equiv {\bf 0}.
\end{equation}
Incidentally, the fluid velocity at any given point in an irrotational fluid is normal to the constant-$\phi$ surface that
passes through that point. 

If a flow pattern is both irrotational and incompressible then we have 
\begin{equation}
{\bf v} = -\nabla\phi
\end{equation}
and
\begin{equation}
\nabla\cdot{\bf v} = 0.
\end{equation}
These two expressions can be combined to give (see Section~\ref{qlap})
\begin{equation}
\nabla^2\phi = 0.
\end{equation}
In other words, the velocity potential in an incompressible irrotational fluid satisfies Laplace's equation. 

According to Equation~(\ref{eq3.24}),  if the flow pattern in an incompressible inviscid fluid is also irrotational, so that $\bomega={\bf 0}$ and ${\bf v}=-\nabla\phi$, then we can write
\begin{equation}
\nabla\left(\frac{p}{\rho} + \frac{1}{2}\,v^2 + \Psi - \frac{\partial\phi}{\partial t}\right)=0,
\end{equation}
which implies that
\begin{equation}
\frac{p}{\rho} + \frac{1}{2}\,v^2 + \Psi - \frac{\partial\phi}{\partial t} = C(t),
\end{equation}
where $C(t)$ is uniform in space, but can vary in time. In fact, the time variation of $C(t)$ can be eliminated
by adding the appropriate function of time (but not of space) to the velocity potential, $\phi$. Note that such
a procedure does not modify the instantaneous velocity field ${\bf v}$ derived from $\phi$. Thus, the above equation can
be rewritten
\begin{equation}
\frac{p}{\rho} + \frac{1}{2}\,v^2 + \Psi - \frac{\partial\phi}{\partial t} = C,\label{e3.36}
\end{equation}
where $C$ is constant in both space and time. Expression~(\ref{e3.36}) is a generalization of Bernoulli's theorem (see Section~\ref{sqbern})
that takes non-steady flow into account. However, this generalization is only valid for irrotational flow. For the special
case of steady flow, we get
\begin{equation}\label{e3.37}
\frac{p}{\rho} + \frac{1}{2}\,v^2 + \Psi = C,
\end{equation}
which demonstrates that for steady irrotational flow the constant in Bernoulli's theorem is the same on all streamlines (see Section~\ref{sqbern}). 

\section{Two-Dimensional Flow}\label{s2d}
Fluid motion is said to be {\em two-dimensional}\/ when the velocity at every point is parallel to
a fixed plane, and is the same everywhere on a given normal to that plane. Thus, in Cartesian coordinates, if the fixed plane is the
$x$-$y$ plane then we can express a general two-dimensional flow patten in the form
\begin{equation}
{\bf v} = v_x(x,y,t)\,{\bf e}_x + v_y(x,y,t)\,{\bf e}_y.
\end{equation}

\begin{figure}
\epsfysize=2.5in
\centerline{\epsffile{Chapter05/stream.eps}}
\caption{\em Two-dimensional flow.}\label{fstream}
\end{figure}

Let $A$ be a fixed point in the $x$-$y$ plane, and let $ABP$ and $ACP$ be two curves, also in the $x$-$y$  plane, 
that join $A$ to an arbitrary point $P$. See Figure~\ref{fstream}. Suppose that fluid is neither created nor destroyed in the region, $R$ (say),  bounded by
these curves. Since the fluid is incompressible, which essentially means that its density is uniform and constant, 
fluid continuity requires that the rate at which the fluid flows into the region $R$, from right to left across the
curve $ABP$, is equal to the rate at which it flows out the of the region, from right to left across the
curve $ACP$. Now, the rate of fluid flow across a surface is generally termed the {\em flux}. Thus, the flux (per unit length parallel to the $z$-axis) from right to left across $ABP$
is equal to the flux from right to left across $ACP$. Since $ACP$ is arbitrary, it follows that the flux from right to left across any curve
joining points $A$ and $P$ is equal to the flux from right to left across $ABP$. In fact, once the base point $A$
has been chosen, this flux only depends on the position of point $P$, and the time $t$. In other words, if we denote the flux by $\psi$ then
it is solely a function of the location of $P$ and the time. Thus, if point $A$ lies at the origin, and point $P$ has Cartesian
coordinates $x$, $y$, then we can write
\begin{equation}
\psi= \psi(x,y,t).
\end{equation}
The function $\psi$ is known as the {\em stream function}. Moreover, the existence of a stream function is a direct consequence of the assumed incompressible nature of the flow. 

Consider two points, $P_1$ and $P_2$, in addition to the fixed point $A$.  See Figure~\ref{fstream1}. Let $\psi_1$ and $\psi_2$ be the fluxes
from right to left across curves $AP_1$ and $AP_2$. Now, using similar arguments to those employed above, the
flux across $AP_2$ is equal to the flux across $AP_1$ plus the flux across $P_1 P_2$. Thus, the
flux across $P_1 P_2$, from right to left, is $\psi_2-\psi_1$. Now, if $P_1$ and $P_2$ both lie on the same streamline
then the flux across $P_1 P_2$ is zero, since the local fluid velocity is directed everywhere parallel to $P_1P_2$. It follows that $\psi_1=\psi_2$. Hence, we conclude that the
stream function is {\em constant}\/ along a streamline. The equation of a streamline is thus $\psi=c$, where $c$
is an arbitrary constant. 

\begin{figure}
\epsfysize=2.5in
\centerline{\epsffile{Chapter05/stream1.eps}}
\caption{\em Two-dimensional flow.}\label{fstream1}
\end{figure}

Let $P_1 P_2=\delta s$ be an infinitesimal arc of a curve that is sufficiently short that it can be regarded as a straight-line.
The fluid velocity in the vicinity of this arc can be resolved into components parallel and perpendicular to the arc. 
The component parallel to $\delta s$ contributes nothing to the flux across the arc from right to left. The component perpendicular to
$\delta s$ contributes $v_\perp\,\delta s$ to the flux. However, the flux is equal to $\psi_2-\psi_1$. Hence,
\begin{equation}
v_\perp = \frac{\psi_2-\psi_1}{\delta s}.
\end{equation}
In the limit $\delta s\rightarrow 0$, the perpendicular velocity from right to left across $ds$ becomes
\begin{equation}
v_\perp = \frac{d\psi}{ds}.
\end{equation}
Thus, in Cartesian coordinates, by considering infinitesimal arcs parallel to the $x$- and $y$-axes, we deduce that
\begin{eqnarray}\label{e3.42}
v_x &=& -\frac{\partial\psi}{\partial y},\\[0.5ex]
v_y&=&\frac{\partial\psi}{\partial x}.\label{e3.43}
\end{eqnarray}
These expressions can be combined to give
\begin{equation}\label{e3.44}
{\bf v} = {\bf e}_z\times \nabla\psi\equiv \nabla z\times \nabla \psi.
\end{equation}
Note that when the fluid velocity is written in this form then it immediately becomes clear
that the incompressibility constraint $\nabla\cdot{\bf v}=0$ is automatically
 satisfied [since $\nabla\cdot(\nabla A\times \nabla B)\equiv 0$---see Equations~(\ref{divvp}) and (\ref{curlgrad})]. 
It is also clear that the stream function is undefined to an arbitrary additive constant. 

The vorticity in two-dimensional flow takes the form
\begin{equation}
\bomega = \omega_z\,{\bf e}_z,
\end{equation}
where
\begin{equation}
\omega_z =  \frac{\partial v_y}{\partial x}-\frac{\partial v_x}{\partial y}.
\end{equation}
Thus, it follows from Equations~(\ref{e3.42}) and (\ref{e3.43}) that
\begin{equation}
\omega_z=\frac{\partial^2\psi}{\partial x^2}+\frac{\partial^2\psi}{\partial y^2}=\nabla^2\psi.  \label{e3.46}                                                                                                                                                                                                                                                                                                                                                                                                                                                                                                                                                                                                                                                                                       
\end{equation}
Moreover, {\em irrotational}\/ two-dimensional  flow is characterized by
\begin{equation}\label{e3.47}
\nabla^2\psi = 0.
\end{equation}

Now, when expressed in terms of cylindrical  coordinates (see Section~\ref{scyl}), Equation~(\ref{e3.44}) yields
\begin{equation}
{\bf v} = v_r(r,\theta,t)\,{\bf e}_r + v_\theta(r,\theta,t)\,{\bf e}_\theta,
\end{equation}
where
\begin{eqnarray}\label{e3.50}
v_r &=& - \frac{1}{r}\,\frac{\partial \psi}{\partial\theta},\\[0.5ex]
v_\theta &=& \frac{\partial\psi}{\partial r}.\label{e3.51}
\end{eqnarray}
Moreover, the vorticity is $\bomega= \omega_z\,{\bf e}_z$, where
\begin{equation}\label{e3.51x}
\omega_z =\frac{1}{r}\,\frac{\partial}{\partial r}\!\left(r\,\frac{\partial\psi}{\partial r}\right) + \frac{1}{r^{\,2}}\,\frac{\partial^2\psi}{\partial \theta^2}.
\end{equation}

\section{Two-Dimensional Uniform Flow}\label{suniform}
Consider a steady two-dimensional flow pattern that is {\em uniform}: {\em i.e.}, such that the fluid velocity
is the same everywhere in the $x$-$y$ plane. For instance, suppose that the common fluid velocity is
\begin{equation}
{\bf v} = V_0\,\cos\theta_0\,{\bf e}_x + V_0\,\sin\theta_0\,{\bf e}_y,
\end{equation}
which corresponds to flow at the uniform speed $V_0$ in a fixed direction that subtends a (counter-clockwise) angle
$\theta_0$ with the $x$-axis.  It follows, from Equations~(\ref{e3.42}) and (\ref{e3.43}), that the
stream function for steady uniform flow takes the form
\begin{equation}\label{e3.53x}
\psi(x,y) = V_0\left(\sin\theta_0\,x-\cos\theta_0\,y\right).
\end{equation}
When written in terms of cylindrical coordinates, this becomes
\begin{equation}
\psi(r,\theta)=-V_0\,r\,\sin(\theta-\theta_0).
\end{equation}

Note, from (\ref{e3.53x}), that $\partial^2\psi/\partial x^2=\partial^2\psi/\partial y^2=0$. Thus, it
follows from Equation~(\ref{e3.46}) that uniform flow is {\em irrotational}. Hence, according to
Section~\ref{qirr}, such flow can also be derived from a velocity potential. In fact, it is easily
demonstrated that
\begin{equation}
\phi(r,\theta)  = -V_0\,r\,\cos(\theta-\theta_0).
\end{equation}

\section{Two-Dimensional Sources and Sinks}\label{ssource}
Consider a uniform {\em line source}, coincident with the $z$-axis,
that emits fluid isotropically at the steady rate of $Q$ unit volumes per unit length per unit time. 
By symmetry, we expect the associated steady flow pattern to be isotropic, and everywhere directed radially away from the source.
See Figure~\ref{fsource}. In other words, we expect
\begin{equation}
{\bf v} = v_r(r)\,{\bf e}_r.
\end{equation}
Consider a cylindrical surface $S$ of unit height and radius $r$ that is co-axial with the source. In a steady state,
the rate at which fluid crosses this surface must be equal to the rate at which the
section of the source enclosed by the surface emits fluid. Hence,
\begin{equation}
\int_S {\bf v}\cdot d{\bf S} = 2\pi\,r\,v_r(r)= Q,
\end{equation}
which implies that
\begin{equation}
v_r(r) = \frac{Q}{2\pi\,r}.
\end{equation}

\begin{figure}
\epsfysize=3.25in
\centerline{\epsffile{Chapter05/source.eps}}
\caption{\em Streamlines of the flow generated by a line source coincident with the $z$-axis.}\label{fsource}
\end{figure}

According to Equations~(\ref{e3.50}) and (\ref{e3.51}), the stream function associated with a line source
of strength $Q$ that is coincident with the $z$-axis is
\begin{equation}\label{e3.58x}
\psi(r,\theta) = -\frac{Q}{2\pi}\,\theta.
\end{equation}
Note that the streamlines, $\psi=c$, are directed radially away from the $z$-axis, as illustrated in Figure~\ref{fsource}. 
Note, also, that the stream function associated with a line source is multivalued. However, this does not cause any
particular difficulty,
since the stream function is continuous, and its gradient single-valued. 

Note, from Equation~(\ref{e3.58x}), that $\partial\psi/\partial r=\partial^2\psi/\partial\theta^2=0$. Hence,
according to (\ref{e3.51x}), $\omega_z=-\nabla^2\psi=0$. In other words, the steady flow pattern
associated with a uniform line source is {\em irrotational}, and can, thus, be derived from a velocity
potential. In fact, it is easily demonstrated that this potential takes the form
\begin{equation}
\phi(r,\theta) = -\frac{Q}{2\pi}\,\ln r.
\end{equation}

A uniform {\em line sink}, coincident with the $z$-axis,
which absorbs  fluid isotropically at the steady rate of $Q$ unit volumes per unit length per unit time
has an associated steady flow pattern
\begin{equation}
{\bf v} = - \frac{Q}{2\pi\,r}\,{\bf e}_r,
\end{equation}
whose stream function is
\begin{equation}\label{e3.58}
\psi(r,\theta) = \frac{Q}{2\pi}\,\theta.
\end{equation}
This flow pattern is also irrotational, and can be derived from the velocity potential
\begin{equation}
\phi(r,\theta) = \frac{Q}{2\pi}\,\ln r.
\end{equation}

Consider a line source and a line sink of equal strength, which  both run parallel to the
$z$-axis, and are located a small distance apart in the $x$-$y$ plane. Such an arrangement
is known as a  {\em dipole}\/ or  {\em doublet}\/ line source. Suppose that the  line source, which is of strength $Q$, is located
at ${\bf r}={\bf d}/2$ (where ${\bf r}$ is a position vector in the $x$-$y$ plane), and that the  line sink, which is also of strength $Q$, is located at ${\bf r}=-{\bf d}/2$. 
Let the function 
\begin{equation}
\psi_Q({\bf r})= -\frac{Q}{2\pi}\,\theta =- \frac{Q}{2\pi}\,\tan^{-1}(y/x)
\end{equation}
 be the stream function associated with a  line source of strength $Q$ located at ${\bf r}={\bf 0}$. 
 Thus, $\psi_Q({\bf r}-{\bf r}_0)$ is the  stream function associated with a  line source of strength $Q$ located at ${\bf r}={\bf r}_0$. 
Furthermore, the stream function associated with a  line sink of strength $Q$ located at ${\bf r}={\bf r}_0$
is $-\psi_Q({\bf r}-{\bf r}_0)$. Now, we expect the flow pattern associated with the combination of a source and a sink to be the vector
sum of the flow patterns generated by the source and sink taken in isolation. It follows that the overall stream function
is the sum of the stream functions generated by the source and the sink taken in isolation. In other words,
\begin{equation}
\psi({\bf r}) = \psi_Q({\bf r}-{\bf d}/2)-\psi_Q({\bf r}+{\bf d}/2)\simeq - {\bf d}\cdot\nabla\psi_Q({\bf r}),
\end{equation}
to first-order in $d/r$. Hence, if ${\bf d} = d\,(\cos\theta_0\,{\bf e}_x+ \sin\theta_0\,{\bf e}_y)=d\,[\cos(\theta-\theta_0)\,{\bf e}_r
-\sin(\theta-\theta_0)\,{\bf e}_\theta]$,
so that the line joining the sink to the source subtends an angle $\theta_0$ with the $x$-axis, 
 then
\begin{equation}
\psi(r,\theta) = -\frac{D}{2\pi}\,\frac{\sin(\theta-\theta_0)}{r},
\end{equation}
where $D=Q\,d$ is termed the strength of the dipole source. Note that the above stream function is antisymmetric
across the line $\theta=\theta_0$ joining the source to the sink. It follows that the
associated dipole flow pattern,
\begin{eqnarray}
v_r(r,\theta)&=& \frac{D}{2\pi}\,\frac{\cos(\theta-\theta_0)}{r^{\,2}},\\[0.5ex]
v_\theta(r,\theta)&=&\frac{D}{2\pi}\,\frac{\sin(\theta-\theta_0)}{r^{\,2}},
\end{eqnarray}
 is symmetric across this line. Figure~\ref{fdipole} shows the streamlines associated with a dipole flow
 pattern characterized by $D>0$ and $\theta_0=0$. Note that the flow speed in a dipole
 pattern falls off like $1/r^{\,2}$. 

\begin{figure}
\epsfysize=3in
\centerline{\epsffile{Chapter05/dipole.eps}}
\caption{\em Streamlines of the flow generated by a dipole line source coincident with the $z$-axis and
aligned along the $x$-axis. The flow is outward along the positive $x$-axis and inward
along the negative $x$-axis.}\label{fdipole}
\end{figure}

 A dipole flow pattern is necessarily irrotational since it is a linear superposition of two irrotational flow patterns.
 The associated velocity potential is
 \begin{equation}
 \phi(r,\theta) = \frac{D}{2\pi}\,\frac{\cos(\theta-\theta_0)}{r}.
 \end{equation}

\section{Two-Dimensional Vortex Filaments}\label{svortex}
Consider a vortex filament of intensity $\Gamma$ that is coincident with the $z$-axis. By symmetry, we expect the associated flow pattern  to circulate isotropically around the filament.
See Figure~\ref{fvortex1}. In other words, we expect
\begin{equation}
{\bf v} = v_\theta(r)\,{\bf e}_\theta.
\end{equation}
Now, according to Section~\ref{scirc}, 
the circulation, $\oint {\bf v}\cdot d{\bf r}$, around any closed curve in the $x$-$y$ plane  is equal to the net intensity of the vortex filaments that pass through the curve. Consider a circular curve of radius $r$ that is concentric with the origin. It follows that
\begin{equation}
\Gamma_r = \oint {\bf v}\cdot d{\bf r}= 2\pi\,r\,v_\theta(r) = \Gamma,
\end{equation}
or
\begin{equation}\label{e3.69}
v_\theta(r) = \frac{\Gamma}{2\pi\,r}.
\end{equation}

\begin{figure}
\epsfysize=3in
\centerline{\epsffile{Chapter05/vortex1.eps}}
\caption{\em Streamlines of the flow generated by a line vortex coincident with the $z$-axis.}\label{fvortex1}
\end{figure}

According to Equations~(\ref{e3.50}) and (\ref{e3.51}), the stream function associated with a vortex filament
of intensity $\Gamma$ that is coincident with the $z$-axis is
\begin{equation}\label{e3.74}
\psi(r,\theta) = \frac{\Gamma}{2\pi}\,\ln r.
\end{equation}
Note that the streamlines, $\psi=c$, circulate around the $z$-axis, as illustrated in Figure~\ref{fvortex1}. 

It can be seen, from Equation~(\ref{e3.74}), that $(\partial/\partial r)(r\,\partial\psi/\partial r)= \partial\psi/\partial\theta=0$. 
Hence,
it follows from  (\ref{e3.51x}) that $\omega_z=-\nabla^2\psi=0$. In other words, the  flow pattern
associated with a straight vortex filament is {\em irrotational}. This is a somewhat surprising result, since there
is a net circulation of the flow around the filament, and, according to Section~\ref{scirc}, non-zero circulation implies
non-zero vorticity. The paradox can be resolved by supposing that the filament has a small,
but finite, radius. In fact, let the filament have the finite radius $a$, and be such that the vorticity is
uniform inside this radius, and zero outside: {\em i.e.}, 
\begin{eqnarray}
\omega_z = \left\{\begin{array}{ccc}
\Gamma/\pi\,a^2&\mbox{\hspace{1cm}}&r\leq a\\[0.5ex]
0&&r>a
\end{array}\right..
\end{eqnarray}
Note that the intensity of the filament ({\em i.e.}, the product of its vorticity and cross-sectional
area) is still $\Gamma$. According to Equation~(\ref{e3.51x}), and assuming that $\psi=\psi(r)$, 
\begin{eqnarray}
\frac{1}{r}\,\frac{d}{\partial r}\!\left(r\,\frac{d\psi}{d r}\right)= \left\{\begin{array}{ccc}
\Gamma/\pi\,a^2&\mbox{\hspace{1cm}}&r\leq a\\[0.5ex]
0&&r>a
\end{array}\right..
\end{eqnarray}
The solution that is well-behaved at $r=0$, and continuous (up to its first derivative) at $r=a$,  is
\begin{eqnarray}
\psi(r,\theta)= \left\{\begin{array}{ccc}
(\Gamma/4\pi)\,(r^2/a^2-1)&\mbox{\hspace{1cm}}&r\leq a\\[0.5ex]
(\Gamma/2\pi)\,\ln(r/a)&&r>a
\end{array}\right..
\end{eqnarray}
Note that this expression is equivalent to (\ref{e3.74}) (apart from an unimportant additive constant) outside the filament, but
 differs inside. The associated circulation velocity, $v_\theta(r)=\partial\psi/\partial r$, is
\begin{eqnarray}
v_\theta(r)= \left\{\begin{array}{ccc}
(\Gamma/2\pi)\,(r/a^2)&\mbox{\hspace{1cm}}&r\leq a\\[0.5ex]
(\Gamma/2\pi)\,(1/r)&&r>a
\end{array}\right.,
\end{eqnarray}
whereas the circulation, $\Gamma_r(r) = 2\pi\,r\,v_\theta(r)$, is written
\begin{eqnarray}
\Gamma_r(r)= \left\{\begin{array}{ccc}
\Gamma\,(r/a)^2&\mbox{\hspace{1cm}}&r\leq a\\[0.5ex]
\Gamma&&r>a
\end{array}\right..
\end{eqnarray}
Thus, we conclude that the flow pattern associated with a straight vortex filament is irrotational
outside the filament, but has finite vorticity inside the filament. Moreover, the  non-zero internal
vorticity generates a constant net circulation of the flow outside the filament. In the limit in which the
radius of the filament tends to zero, the strength of the vorticity within the filament tends to infinity
(in such a way that the product of the vorticity and the cross-sectional area of the filament remains
constant), and the region of the fluid in which the vorticity is non-zero becomes infinitesimal in extent. 

Now, since the flow pattern outside a straight vortex filament is irrotational, it can be derived from a velocity potential. 
In fact, it is easily demonstrated that the appropriate potential takes the form 
\begin{equation}
\phi(r,\theta)= - \frac{\Gamma}{2\pi}\,\theta.
\end{equation}
Note that the above potential is multivalued. However, this does not cause any
particular difficulty,
since the potential is continuous, and its gradient single-valued. 

\section{Two-Dimensional Irrotational Flow}
As we have seen,  in a two-dimensional flow pattern, we can automatically satisfy the
incompressibility constraint, $\nabla\cdot{\bf v}=0$,  by expressing the pattern in terms of a stream function. Suppose, however, that, in addition
to being incompressible, the flow pattern is also {\em irrotational}. In this case,
Equation~(\ref{e3.46}) yields
\begin{equation}
\nabla^2\psi=0.
\end{equation}
In cylindrical coordinates, since $\psi=\psi(r,\theta,t)$, this expression implies that (see Section~\ref{scyl})
\begin{equation}
\frac{1}{r}\,\frac{\partial}{\partial r}\!\left(r\,\frac{\partial\psi}{\partial r}\right) + \frac{1}{r^{\,2}}\,\frac{\partial^2\psi}{\partial\theta^2}
=0.\label{e3.53}
\end{equation}

Let us search for a separable steady-state solution of Equation~(\ref{e3.53}) of the form
\begin{equation}
\psi(r,\theta) = R(r)\,\Theta(\theta).
\end{equation}
It is easily seen that
\begin{equation}
\frac{r}{R}\,\frac{d}{dr}\!\left(r\,\frac{dR}{dr}\right)= - \frac{1}{\Theta}\,\frac{d^2\Theta}{d\theta^2},
\end{equation}
which can only be satisfied if
\begin{eqnarray}
r\,\frac{d}{dr}\!\left(r\,\frac{dR}{dr}\right)&=&m^2\,R,\label{e3.56}\\[0.5ex]
\frac{d^2\Theta}{d\theta^2} &=& -m^2\,\Theta,\label{e3.57}
\end{eqnarray}
where $m^2$ is an arbitrary (positive) constant. The general solution of Equation~(\ref{e3.57})
is a linear combination of $\exp(\,{\rm i}\,m\,\theta)$ and $\exp(-{\rm i}\,m\,\theta)$ factors.
However, assuming that the flow extends over all $\theta$ values, the function $\Theta(\theta)$ must be  {\em single-valued}\/ in $\theta$, otherwise $\nabla\psi$---and,
hence, ${\bf v}$---would not be be single-valued (which is unphysical).  It follows that $m$ can only take {\em integer}\/ values (and that $m^2$
must be a positive, rather than a negative, constant). 
Now, the general solution of Equation~(\ref{e3.56}) is a linear combination of $r^m$ and
$r^{-m}$ factors, except for the special case $m=0$, when it is a linear combination
of $r^0$ and $\ln r$ factors. Thus, the general stream function for steady two-dimensional
irrotational flow (that extends over all values of $\theta$) takes the form
\begin{equation}\label{e3.58y}
\psi(r,\theta) = \alpha_0+ \beta_0\,\ln r + \sum_{m>0} (\alpha_m\,r^m+\beta_m\,r^{-m})\,
\sin[m\,(\theta-\theta_m)],
\end{equation}
where $\alpha_m$, $\beta_m$, and $\theta_m$ are arbitrary constants. 
We can recognize the first few terms on the right-hand side of the above expression. The constant term $\alpha_0$
has zero gradient, and, therefore, does not give rise to any flow. The term $\beta_0\,\ln r$ is the flow pattern generated by 
a vortex filament of intensity $2\pi\,\beta_0$, coincident with the $z$-axis (see Section~\ref{svortex}). The
term $\alpha_1\,r\,\sin(\theta-\theta_1)$  corresponds to uniform flow of speed $\alpha_1$ whose
direction subtends an angle $\theta_1$ with the minus $x$-axis (see Section~\ref{suniform}).
Finally, the term $\beta_1\,\sin(\theta-\theta_1)/r$ corresponds to a dipole flow pattern (see Section~\ref{ssource}).

The velocity potential associated with the irrotational stream function (\ref{e3.58y}) satisfies [see Equations~(\ref{e3.51t}) and
(\ref{e3.44})]
\begin{eqnarray}
\frac{\partial\phi}{\partial r} &=& \frac{1}{r}\,\frac{\partial\psi}{\partial\theta},\\[0.5ex]
\frac{1}{r}\,\frac{\partial\phi}{\partial\theta} &=&-\frac{\partial\psi}{\partial r}.
\end{eqnarray}
It follows that
\begin{equation}
\phi(r,\theta) = \alpha_0-\beta_0\,\theta+\sum_{m>0}(\alpha_m\,r^m-\beta_m\,r^{-m})\,\cos[m\,(\theta-\theta_0)].
\end{equation}

\section{Inviscid Flow Past a Cylindrical Obstacle}\label{scylo}
Consider the steady flow pattern produced when an impenetrable  rigid cylindrical obstacle is placed in a uniformly flowing
incompressible inviscid fluid,  with the cylinder orientated such that its axis is normal to the flow. For instance, suppose that the radius of the cylinder is 
 $a$, and that its axis  corresponds to the line  $x=y=0$. Furthermore, let the unperturbed fluid velocity be of magnitude $V_0$, and 
be directed parallel to the $x$-axis.  Now, we expect the
flow pattern to remain unperturbed very far away from the cylinder. In other words,
we expect  ${\bf v}(r,\theta)\rightarrow V_0\,{\bf e}_x$ as $r/a\rightarrow\infty$, 
which corresponds to a boundary condition on the stream function of the form (see Section~\ref{suniform})
\begin{equation}\label{e3.88}
\psi(r,\theta)\rightarrow -V_0\,r\,\sin\theta\mbox{\hspace{1cm}as $r/a\rightarrow\infty$}.
\end{equation}
Given that the fluid velocity field a large distance downstream of the cylinder is irrotational (since we
have already seen that the flow pattern associated with uniform
flow is irrotational---see Section~\ref{suniform}), it follows from the Kelvin circulation theorem (see Section~\ref{skelvin})
that the velocity field remains irrotational as it is convected past the cylinder. Hence, according to Section~\ref{s2d}, the stream function of the
flow satisfies Laplace's equation:
\begin{equation}\label{e3.89}
\nabla^2\psi = 0.
\end{equation}
The appropriate boundary condition at the surface of the cylinder is simply that the normal fluid
velocity there be zero, since the fluid must stay in contact with the cylinder, but cannot penetrate its
surface. Hence, $v_r(a,\theta) \equiv -(1/a)\,\partial\psi/\partial \theta|_{r=a} = 0$, which implies that
\begin{equation}\label{e3.90}
\psi(a,\theta)= 0,
\end{equation}
since $\psi$ is undetermined to an arbitrary additive constant.  It follows that we are searching for the most
general solution of (\ref{e3.89}) that satisfies the boundary conditions (\ref{e3.88}) and (\ref{e3.90}). 
Comparison with Equation~(\ref{e3.58y}) reveals that this solution takes the form
\begin{equation}
\psi(r,\theta) = V_0\,a\left[-\gamma\,\ln\left(\frac{r}{a}\right)-\left(\frac{r}{a}-\frac{a}{r}\right)\sin\theta\right],
\end{equation}
where
\begin{equation}
\gamma = -\frac{\Gamma}{2\pi\,a\,V_0},
\end{equation}
and $\Gamma$ is the circulation of the flow around the cylinder. 
(Note that the velocity field can be irrotational, but still possess nonzero circulation
around the cylinder, because a loop that encloses the cylinder cannot be spanned
by a surface lying entirely within the fluid. Thus, zero fluid vorticity does not
necessarily imply zero circulation around such a loop from Stokes' theorem.) Let us assume that $\gamma\geq 0$, for the
sake of definiteness. 

\begin{figure}
\epsfysize=3in
\centerline{\epsffile{Chapter05/cylinder.eps}}
\caption{\em Streamlines of the flow generated by a cylindrical obstacle of radius $a$, whose axis runs  along the $z$-axis,
placed in the uniform flow field ${\bf v}= V_0\,{\rm e}_x$. The normalized circulation is $\gamma=0$. }\label{fxcyl}
\end{figure}

Figure~\ref{fxcyl}--\ref{fxcyl2} show streamlines of the flow calculated for various different values of the normalized circulation,
$\gamma$. For $\gamma<2$ there exist  a pair of  points on the surface of the
cylinder at which the flow speed is zero. These are known as {\em stagnation points}, and can be located in Figures~\ref{fxcyl}
and \ref{fxcyl1} as the points at which streamlines intersect the surface of the cylinder at right-angles. Now, the
tangential fluid velocity at the surface of the cylinder is
\begin{equation}
v_t(\theta) \equiv v_\theta(a,\theta) = \left.\frac{\partial\psi}{\partial r}\right|_{r=a} = -V_0\,(\gamma+2\,\sin\theta).
\end{equation}
The stagnation points correspond to the points at which $v_t=0$ (since the normal velocity is automatically 
zero at the surface of the cylinder). Thus, the stagnation points lie at $\theta = \sin^{-1}(-\gamma/2)$. When $\gamma>2$
the stagnation points coalesce and move off the surface of the cylinder, as illustrated in Figure~\ref{fxcyl2} (the stagnation point
corresponds to the point at which two streamlines cross at right-angles). 

\begin{figure}
\epsfysize=3in
\centerline{\epsffile{Chapter05/cylinder1.eps}}
\caption{\em Streamlines of the flow generated by a cylindrical obstacle of radius $a$, whose axis runs  along the $z$-axis,
placed in the uniform flow field ${\bf v}= V_0\,{\rm e}_x$. The normalized circulation is $\gamma=1$. }\label{fxcyl1}
\end{figure}

The irrotational form of Bernoulli's theorem, (\ref{e3.37}), can be combined with the boundary condition $v\rightarrow V_0$
as $r/a\rightarrow \infty$, as well as the fact that $\Psi$ is constant in the present case,  to give
\begin{equation}
p = p_0 + \frac{1}{2}\,\rho\left(V_0^{\,2}-v^2\right),
\end{equation}
where $p_0$ is the constant static fluid pressure a large distance from the cylinder. In particular, the fluid
pressure on the surface of the cylinder is
\begin{equation}\label{e3.95}
P(\theta)\equiv p(a,\theta) = p_0 + \frac{1}{2}\,\rho\left(V_0^{\,2}-v_t^{\,2}\right)= p_1 + \rho\,V_0^{\,2}\left(\cos 2\theta-2\,\gamma\,\sin\theta\right),
\end{equation}
where $p_1=p_0-(1/2)\,\rho\,V_0^{\,2}\,(1+\gamma^2)$. 
The net force per unit length exerted on the cylinder by the fluid has the Cartesian components
\begin{eqnarray}
F_x &=& -\oint P\,\cos\theta\,a\,d\theta,\\[0.5ex]
F_y&=& -\oint P\,\sin\theta\,a\,d\theta.
\end{eqnarray}
Thus, it follows from (\ref{e3.95}) that
\begin{eqnarray}
F_x &=& 0,\\[0.5ex]
F_y&=&  2\pi\,\gamma\,\rho\,V_0^{\,2}\,a= \rho\,V_0\,(-\Gamma).
\end{eqnarray}
Now, the component of the force which a moving fluid exerts on an obstacle, placed in its path, in a direction parallel to that of  the unperturbed  flow is usually called {\em drag}. On the other hand,  the component of the force which
the fluid exerts in a direction perpendicular to that of  the unperturbed flow is usually called {\em lift}. Hence, the above equations imply
that if a cylindrical obstacle is placed in a uniformly flowing inviscid fluid then there is {\em zero drag}. On the other
hand, as long as there is net circulation of the flow around the cylinder, the lift is non-zero. Now, lift is generated because (negative) 
circulation tends to increase the fluid speed directly above, and to decrease it directly below, the cylinder. 
Thus, from Bernoulli's theorem, the fluid pressure is decreased above, and increased below,
the cylinder, giving rise to a net upward force ({\em i.e.}, a force in the $+y$ direction). 

\begin{figure}
\epsfysize=3in
\centerline{\epsffile{Chapter05/cylinder2.eps}}
\caption{\em Streamlines of the flow generated by a cylindrical obstacle of radius $a$, whose axis runs  along the $z$-axis,
placed in the uniform flow field ${\bf v}= V_0\,{\rm e}_x$. The normalized circulation is $\gamma=2.5$. }\label{fxcyl2}
\end{figure}

Suppose that the cylinder  is placed in a fluid which is initially at rest, and that the fluid's uniform flow velocity, $V_0$,
is then very slowly ramped up (in such a manner that no vorticity is induced in the flow at infinity). Since the flow pattern is initially irrotational, and since the
flow pattern well upstream of the cylinder is assumed to remain irrotational, the Kelvin circulation theorem indicates
that the flow pattern around the cylinder also remains irrotational. Consider the time evolution of the circulation, $\Gamma=\oint_C{\bf v}\cdot d{\bf r}$,  around
some {\em fixed}\/  curve $C$ that lies entirely within the fluid, and encloses the cylinder. We have
\begin{equation}
\frac{d\Gamma_C}{dt} = \oint_C\frac{\partial {\bf v}}{\partial t}\cdot d{\bf r} =
\oint_C\left[-\nabla\left(\frac{p}{\rho}+\frac{1}{2}\,v^2\right) + {\bf v}\times \bomega\right]\cdot d{\bf r}= \oint_C {\bf v}\times \bomega\cdot d{\bf r},
\end{equation}
where use has been made of (\ref{eq3.24}) (with $\Psi$ assumed constant). However, $\bomega=\omega_z\,{\bf e}_z$ in two-dimensional flow, and
$d{\bf r}\times {\bf e}_z  =d{\bf S}$, where $d{\bf S}$ is an outward surface element of a unit depth (in the $z$-direction)
surface whose normal lies in the $x$-$y$ plane, and that cuts the $x$-$y$ plane at $C$. In other words,
\begin{equation}
\frac{d\Gamma_C}{dt} = -\oint_S \omega_z\,{\bf v}\cdot d{\bf S}.
\end{equation}
We, thus, conclude that the rate of change of the circulation around $C$ is equal to minus the flux of the vorticity across $S$
[assuming that vorticity is convected by the flow, which follows from (\ref{eq3.25}), the fact that $\bomega=\omega_z\,{\bf e}_z$, and the fact that $\partial/\partial z\equiv 0$ in two-dimensional flow]. 
However, we have already seen that the flow field surrounding the cylinder is {\em irrotational}\/ ({\em i.e.}, such that $\omega_z=0$). It follows that $\Gamma_C$
is  {\em constant}\/ in time. Moreover, since $\Gamma_C=0$ originally, because the fluid surrounding the cylinder was initially
at rest,  we deduce that $\Gamma_C=0$ at all subsequent times. Hence, we conclude that, in an inviscid fluid, 
if the circulation of the flow around the cylinder is initially zero then it remains zero. It follows, from the above
analysis, that, in such a fluid, zero drag force and zero lift force are exerted on the cylinder as a consequence of the
fluid flow.  This result is known as {\em d'Alembert's paradox}, after the French scientist Jean-Baptiste le Rond d'Alembert (1717--1783). D'Alembert's result is  paradoxical because it would seem, at
first sight, to be a reasonable approximation to neglect viscosity alltogether in high Reynolds number flow. However, if
we do this then we end up with the nonsensical  prediction that a high Reynolds number fluid is incapable of
exerting any force on an obstacle placed in its path.

\section{Inviscid Flow Past a Semi-Infinite Wedge}\label{swedge}
Consider the situation, illustrated in Figure~\ref{fwedge}, in which incompressible irrotational flow is incident on a
impenetrable rigid wedge whose apex subtends an angle $\alpha\,\pi$. Let the cross-section of the wedge in the $x$-$y$ plane
be both $z$-independent and symmetric about the $x$-axis. Furthermore, let the apex of
the wedge lie at $x=y=0$. Finally, let the upstream flow a large distance from the
wedge be parallel to the $x$-axis.

\begin{figure}
\epsfysize=3in
\centerline{\epsffile{Chapter05/wedge.eps}}
\caption{\em Inviscid flow past a wedge.}\label{fwedge}
\end{figure}

Since the flow is two-dimensional, incompressible, and irrotational, it can be represented in terms of a stream function that
satisfies Laplace's equation. Moreover, in cylindrical coordinates, this equation takes the form (\ref{e3.53}). The
boundary conditions on the stream function are
\begin{equation}
\psi(r,\alpha\,\pi/2) =\psi(r,-\alpha\,\pi/2)=\psi(r,\pi)=0.
\end{equation}
The first two boundary conditions ensure that the normal velocity at the surface of the wedge is zero. The third boundary condition
follows from the observation that, by symmetry, the streamline that meets the apex of the wedge splits in two, and then flows along its 
top and bottom boundaries, combined with well-known  result that $\psi$ is constant on a streamline.
It is easily demonstrated that
\begin{equation}\label{e3.106}
\psi(r,\theta) = \frac{A}{1+m}\,r^{1+m}\,\sin\left[(1+m)\,(\pi-\theta)\right]
\end{equation}
is a solution of (\ref{e3.53}). Moreover, this solution satisfies the boundary conditions provided  
$(1+m)\,(1-\alpha/2) = 1$, or
\begin{equation}
m=\frac{\alpha}{2-\alpha}.
\end{equation}
Since, as is well-known, the solutions to Laplace's equation (for problems with well-posed boundary conditions) are
{\em unique}, we can be sure that (\ref{e3.106}) is the correct 
solution to the problem under investigation. According to this solution, the tangential velocity on the surface of the
wedge is given by
\begin{equation}
v_t(r) = A\,r^m,
\end{equation}
where $m\geq 0$. 
Note that the tangential velocity is zero at the apex of the wedge. Since the normal velocity is also zero at this point, we
conclude that the apex is a stagnation point of the flow. Figure~\ref{fwedge1} shows the streamlines of the
flow for the case $\alpha=1/2$. 

\begin{figure}
\epsfysize=3in
\centerline{\epsffile{Chapter05/wedge1.eps}}
\caption{\em Streamlines of inviscid incompressible irrotational flow past a $90^\circ$ wedge.}\label{fwedge1}
\end{figure}

\section{Inviscid Flow Over a Semi-Infinite Wedge}\label{swedge1}
Consider the situation illustrated in Figure~\ref{fwedge2} in which an incompressible irrotational fluid flows over an
impenetrable rigid wedge whose apex subtends an angle $\alpha\,\pi$. Let the cross-section of the wedge in the $x$-$y$ plane
be both $z$-independent and symmetric about the $y$-axis. Furthermore, let the apex of
the wedge lie at $x=y=0$. Finally, let the upstream flow a large distance from the
wedge be parallel to the $x$-axis.

\begin{figure}
\epsfysize=3in
\centerline{\epsffile{Chapter05/wedge2.eps}}
\caption{\em Inviscid flow over a wedge.}\label{fwedge2}
\end{figure}

Since the flow is two-dimensional, incompressible, and irrotational, it can be represented in terms of a stream function that
satisfies Laplace's equation. The
boundary conditions on the stream function are
\begin{equation}
\psi\left(r,[3-\alpha]\,\pi/2\right) =\psi\left(r,-[1-\alpha]\,\pi/2\right)=0.
\end{equation}
These boundary conditions ensure that the normal velocity at the surface of the wedge is zero. 
It is easily demonstrated that
\begin{equation}\label{e3.106x}
\psi(r,\theta) = -\frac{A}{1-m}\,r^{1-m}\,\cos\left[(1-m)\,(\theta-\pi/2)\right]
\end{equation}
is a solution of Laplace's equation, (\ref{e3.53}). Moreover, this solution satisfies the boundary conditions provided that 
$(1-m)\,(1-\alpha/2) = 1/2$, or
\begin{equation}
m=\frac{\alpha'}{1+\alpha'},
\end{equation}
where $\alpha'=1-\alpha$. 
Since the solutions to Laplace's equation are
unique, we can again  be sure that (\ref{e3.106x}) is the correct 
solution to the problem under investigation. According to this solution, the tangential velocity on the surface of the
wedge is given by
\begin{equation}
v_t(r) = A\,r^{-m},
\end{equation}
where $m\geq 0$. 
Note that the tangential velocity, and hence the flow speed, is infinite at the apex of the wedge.  However, this
singularity in the flow can be eliminated by slightly rounding the apex.
Figure~\ref{fwedge3} shows the streamlines of the
flow for the case $\alpha=1/2$. 

\begin{figure}
\epsfysize=3in
\centerline{\epsffile{Chapter05/wedge3.eps}}
\caption{\em Streamlines of inviscid incompressible irrotational flow over a $90^\circ$ wedge.}\label{fwedge3}
\end{figure}

\section{Exercises}
{\small 
\renewcommand{\theenumi}{5.\arabic{enumi}}
\begin{enumerate}
\item Liquid is led steadily through a pipeline that passes over a hill of height $h$ into the valley below, the
speed at the crest being $v$. Show that, by properly adjusting the ratio of the cross-sectional areas of the pipe
at the crest and in the valley, the pressure may be equalized at these two places. 

\item For the case of the two-dimensional motion of an incompressible fluid, determine the condition that the velocity components
\begin{eqnarray}
v_x &=& a\,x+b\,y,\nonumber\\[0.5ex]
v_y&=&c\,x+d\,y\nonumber
\end{eqnarray}
satisfy the equation of continuity. Show that the magnitude of the vorticity is $c-b$. 

\item For the case of the two-dimensional motion of an incompressible fluid, show that
\begin{eqnarray}
v_x &=& 2\,c\,x\,y,\nonumber\\[0.5ex]
v_y&=& c\,(a^2+x^2-y^2)\nonumber
\end{eqnarray}
are the velocity components of a possible flow pattern. Determine the stream function and sketch the streamlines. 
Prove that the motion is irrotational, and find the velocity potential. 

\item A cylindrical vortex in an incompressible fluid is co-axial with the $z$-axis, and such that $\omega_z$ takes the constant value
$\omega$ for $r\leq a$, and is zero for $r>a$, where $r$ is a cylindrical coordinate.  Show that
$$
\frac{1}{\rho}\,\frac{dp}{dr} = \frac{\omega^2\,r}{4} = \frac{\kappa^2\,r}{a^4},
$$
where $p(r)$ is the pressure at radius $r$ inside the vortex, and the circulation of the fluid outside the vortex is $2\pi\,\kappa$. 
Deduce that
$$
p(r) = \frac{\kappa^2\,r^2\,\rho}{2\,a^4}+ p_0,
$$
where $p_0$ is the pressure at the center of the vortex. 

\item Consider the cylindrical vortex discussed in Exercise 5.4. If $p(r)$ is the pressure at radius $r$ external to the
vortex, demonstrate that
$$
p(r) = -\frac{\kappa^2\,\rho}{2\,r^2}+ p_\infty,
$$
where $p_\infty$ is the pressure at infinity. 

\item Show that the stream function for the cylindrical vortex discussed in Exercises 5.4 and 5.5 is
$\psi(r)=(1/2)\,\omega\,a^2\,\ln(r/a)$ for $r>a$, and $\psi(r)=(1/4)\,\omega\,(r^2-a^2)$ for $r\leq a$. 

\item Consider a volume $V$ whose boundary is the  surface $S$. Suppose that $V$ contains an
incompressible fluid whose motion is irrotational. Let the velocity potential $\phi$ be constant over
$S$. Prove that $\phi$ has the same constant value throughout $V$. [Hint: Consider the identity
$\nabla\cdot(A\,\nabla A)\equiv \nabla A\cdot \nabla A +A\,\nabla^2 A$.]

\item In Exercise~5.7, suppose that, instead of $\phi$ taking a constant value on the boundary, the
normal velocity is everywhere zero on the boundary. Show that $\phi$ is constant throughout $V$. 

\item Prove that in the two-dimensional motion of a liquid the mean tangential
fluid velocity around any {\em small}\/ circle of radius $r$ is $\omega\,r$, where $2\,\omega$
is the value of
$$
\frac{\partial v_y}{\partial x}- \frac{\partial v_x}{\partial y}
$$
at the center of the circle. Neglect terms of order $r^3$. 

\item Show that the equation of continuity for the two-dimensional motion of an
incompressible fluid can be written
$$
\frac{\partial (r\,v_r)}{\partial r} + \frac{\partial v_\theta}{\partial\theta}=0,
$$
where $r$, $\theta$ are cylindrical coordinates. Demonstrate that this
equation is satisfied when $v_r=a\,k\,r^n\,\exp[-k\,(n+1)\,\theta]$ and
$v_\theta=a\,r^n\,\exp[-k\,(n+1)\,\theta]$. Determine the stream function, and
show that the fluid speed at any point is
$$
(n+1)\,\psi\,\sqrt{1+k^2}/r,
$$
where $\psi$ is the stream function at that point (defined such that $\psi=0$ at $r=0$). 

\item Demonstrate that streamlines cross at right-angles at a stagnation point in two-dimensional incompressible
irrotational flow.

\item Consider two-dimensional incompressible inviscid flow. Demonstrate that the fluid  motion
is governed by the following equations:
\begin{eqnarray}
\frac{\partial\omega}{\partial t} +[\psi,\omega] &=& 0,\nonumber\\[0.5ex]
\nabla^2\psi &=&\omega,\nonumber\\[0.5ex]
\nabla^2\chi &=&\nabla\omega\cdot\nabla\psi + \omega^2,\nonumber
\end{eqnarray}
where ${\bf v} = {\bf e}_z\times \nabla\psi$,  $[A,B] \equiv {\bf e}_z\cdot\nabla A \times\nabla B$, and
$\chi = p/\rho+(1/2)\,v^2+\Psi$. 

\item For irrotational incompressible inviscid motion in two-dimensions show that
$$
\nabla q\cdot\nabla q = q\,\nabla^2 q,
$$
where $q=|{\bf v}|$. 

\end{enumerate}}
