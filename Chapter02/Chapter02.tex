\chapter{Mathematical Models of Fluid Motion}\label{c4}
\section{Introduction}
In this chapter, we  investigate the mathematical models commonly used to describe the equilibrium and
dynamics of fluids. Unless
stated otherwise, all of the analysis is performed using a standard right-handed Cartesian coordinate
system: $x_1$, $x_2$, $x_3$. Moreover, the Einstein summation convention is employed (so repeated roman subscripts are assumed to be summed from 1 to 3---see Appendix~\ref{appb}). 

\section{What is a Fluid?}\label{s4.1}
By definition, a {\em solid}\/ material  is {\em rigid}. Now, although a rigid
material tends to shatter when subjected to very large stresses, it can withstand a moderate
{\em shear stress}\/ ({\em i.e.}, a stress that tends to deform the material by changing its shape, without necessarily changing its volume)
for an indefinite period. To be more exact, when a shear stress is first applied to a rigid material it deforms slightly, but
then springs back to its original shape when the stress is relieved. 

A {\em plastic}\/ material, such as clay,
also possess some degree of rigidity. However, the critical shear stress at which it yields is relatively
small, and once this stress is exceeded the material deforms continuously and
irreversibly, and does not recover its original shape when the stress is relieved. 

By definition, a  {\em fluid}\/ material possesses no rigidity at all.
In other words, a  small fluid element is  unable to withstand any tendency of an applied shear stress to
change its shape.  Incidentally, this  does not
preclude the possibility that such an element may offer resistance
to  shear stress. However, any  resistance must be incapable of preventing the
change in shape from eventually occurring, which  implies that the force of resistance
vanishes with the rate of deformation. An obvious corollary is that the shear stress must
be {\em zero}\/ everywhere inside a fluid that is in mechanical equilibrium. 

Fluids are conventionally classified as either {\em liquids}\/ or {\em gases}. The most important difference
between these two types of fluid lies in their relative  {\em compressibility}: {\em i.e.}, gases can be compressed much
more easily than liquids. Consequently, any motion that involves significant pressure variations is generally accompanied by much larger changes in mass density in the case of a gas than in the
case of a liquid. 

Of course, a macroscopic fluid ultimately consists of a huge number of individual molecules. However, most practical
applications of fluid mechanics are concerned with behavior on  length-scales
that are far larger than the typical intermolecular spacing. Under these circumstances, it
is reasonable to suppose that the bulk properties of  a given fluid are the same as if
it were completely {\em continuous}\/ in structure. A corollary of this assumption is that
when, in the following, we talk about infinitesimal volume elements, we really mean elements which are
sufficiently small that the bulk fluid properties  (such as  mass density, pressure, and velocity) are approximately constant across them, but are still sufficiently large that
they contain a  very great number of molecules (which implies that we can safely neglect any statistical variations
in the bulk properties). The continuum hypothesis also requires infinitesimal volume elements to be much
larger than the molecular mean-free-path.

In addition to the continuum hypothesis, our study of fluid mechanics is premised on three major assumptions:
\begin{enumerate}
\item Fluids are {\em isotropic}\/ media: {\em i.e.}, there is no preferred direction in a fluid.
\item Fluids are {\em Newtonian}: {\em i.e.}, there is a linear relationship between the local shear stress and
the local rate of strain, as first postulated by Newton. It is also assumed that there is a linear relationship
between the local heat flux density and  the local temperature gradient. 
\item Fluids are {\em classical}: {\em i.e.}, the macroscopic motion of ordinary fluids is well-described
by Newtonian dynamics, and both quantum and relativistic effects can be safely ignored.
\end{enumerate}
It should be noted that the above assumptions are not valid for all fluid types ({\em e.g.}, certain
liquid polymers, which are non-isotropic; thixotropic
fluids, such as jelly or paint, which are non-Newtonian; and quantum fluids, such as liquid helium, which
exhibit non-classical effects on macroscopic length-scales). However, most practical
applications of fluid mechanics involve the equilibrium and motion of bodies of water or air, extending over macroscopic length-scales, and situated  relatively close to the Earth's surface. Such bodies are very well-described
as isotropic, Newtonian,  classical fluids.

\section{Volume and Surface Forces}\label{s5.4}
Generally speaking, fluids are acted upon by two distinct types of force. The first type is {\em long-range}\/ in nature---{\em
i.e.}, such that it decreases relatively slowly with increasing distance between interacting elements---and is
capable of completely penetrating into the interior of a fluid.  Gravity
is an obvious example of a long-range force. One consequence of the relatively slow variation of
long-range forces with position  is that they
act equally on all of the fluid contained within a sufficiently small volume element.  In this situation, the net force 
acting on the element becomes directly proportional to its volume. For this reason, long-range forces are often called {\em volume  forces}. In the following, we shall write the total  volume force acting at time $t$ on the fluid contained within a
small volume element of magnitude $d V$, centered on a {\em fixed}\/ point whose position vector is ${\bf r}$,
as
\begin{equation}
{\bf F}({\bf r},t)\,d V.
\end{equation}

The second type of force is {\em short-range}\/ in nature, and
is most conveniently modeled as {\em momentum transport}\/ within the fluid. Such transport is generally due to a combination of
the mutual forces exerted by contiguous molecules, and momentum fluxes caused by  relative molecular motion. 
Suppose that $\bpi_x({\bf r}, t)$ is the net flux density of $x$-directed fluid momentum due to
short-range forces
 at  position ${\bf r}$ and time $t$. In other words, suppose that, at position ${\bf r}$ and 
 time $t$,  as a direct consequence of short-range forces, $x$-momentum is flowing at the rate of
 $|\bpi_x|$ newton-seconds per meter squared per second in the direction of vector $\bpi_x$.  Consider an infinitesimal  plane surface element, $d{\bf S} = {\bf n}\,dS$,  located at point ${\bf r}$. Here, $dS$ is the
 area of the element, and  ${\bf n}$ its unit normal (see Section~\ref{sect22}). The fluid which lies
 on that side of the element toward which ${\bf n}$ points is said to lie on its {\em positive}\/ side, and {\em vice versa}. 
 The net flux of $x$-momentum across the element (in the direction of ${\bf n}$)  is
 $\bpi_x \cdot d{\bf S}$ newtons, which implies (from Newton's second law of motion) that the fluid  on the  positive side of the surface element
 experiences a force in the $x$-direction of $\bpi_x
 \!\cdot\! d{\bf S}$ newtons due to short-range
 interaction  with the fluid on the negative side. 
 According to Newton's third law of motion, the  fluid on the negative side of the surface  experiences a force in the $x$-direction of $-\bpi_x \cdot d{\bf S}$ newtons due to interaction with the  fluid on the positive side. 
 Short-range forces
 are often called {\em surface forces}\/ because they are directly  proportional to the area of the surface element across which
 they act. 
 Let $\bpi_y({\bf r},t)$ and $\bpi_z({\bf r},t)$ be the net flux density of $y$- and $z$-momentum, respectively, at position ${\bf r}$ and
 time $t$. By a straightforward extension of above argument, the net surface force exerted {\em by}\/ the fluid on the positive side of  some
 planar surface element, $d{\bf S}$, {\em on}\/ the fluid
 on its negative side is
 \begin{equation}\label{e4.2}
 {\bf f} = (-\bpi_x\!\cdot\! d{\bf S},\, -\bpi_y\!\cdot\!d{\bf S},\, -\bpi_z\!\cdot\!d{\bf S}).
 \end{equation}
 In tensor notation (see Appendix~\ref{appb}), the above equation can be written
 \begin{equation}\label{e4.3}
 f_i = \sigma_{ij}\,dS_j,
 \end{equation}
 where $\sigma_{11}= -(\bpi_x)_x$, $\sigma_{12} = -(\bpi_x)_y$, $\sigma_{21}=-(\bpi_y)_x$, {\em etc.}
 Here, the $\sigma_{ij}({\bf r}, t)$ are termed the local {\em stresses}\/ in the fluid at position ${\bf r}$ and
 time $t$, and have units of force per unit area. Moreover, the $\sigma_{ij}$ are
 the components of a second-order tensor (see Appendix~\ref{appb}), known as the {\em stress tensor}. This follows because the $f_i$ are the components of a first-order tensor (since all forces are proper vectors), and the $dS_i$ are the components of an {\em arbitrary}\/ first-order tensor [since surface elements are
 also proper vectors---see Section~\ref{sect22}---and (\ref{e4.3}) holds for
  elements whose normals point in {\em any}\/ direction], so application of  the quotient rule (see Section~\ref{strans}) to Equation~(\ref{e4.3}) reveals that the  $\sigma_{ij}$  transform under rotation of the coordinate axes as the components of a second-order tensor. We can interpret $\sigma_{ij}({\bf r},t)$
 as the $i$-component of the force per unit area exerted, at position ${\bf r}$ and time $t$, across a plane surface element normal to the
 $j$-direction. The three diagonal components of $\sigma_{ij}$ are
 termed {\em normal stresses}, since each of them gives the normal component of the  force per unit area acting
 across a plane surface element parallel to one of the Cartesian coordinate planes. The six non-diagonal components
 are termed {\em shear stresses}, since they drive shearing motion in which parallel
 layers of fluid slide relative to one another.
 
\section{General Properties of  Stress Tensor}\label{s4.3}
 The $i$-component of the total force acting on a  fluid element consisting of a fixed volume $V$ enclosed by a surface $S$ is written
 \begin{equation}
 f_i = \int_V F_i\,dV + \oint_S \sigma_{ij}\,dS_j,
 \end{equation}
 where the first term on the right-hand side is the integrated volume force acting throughout $V$, whereas the
 second term is the net surface force acting across $S$. Making use of the tensor divergence theorem (see Section~\ref{stfield}),
 the above expression becomes
 \begin{equation}
 f_i = \int_V F_i\,dV + \int_V \frac{\partial \sigma_{ij}}{\partial x_j}\,dV.
 \end{equation}
In the limit   $V\rightarrow 0$ (note that we are using $V$ to denote both the fluid element and its volume), it is reasonable to suppose that the $F_i$ and $\partial \sigma_{ij}/\partial x_j$ are
 approximately constant across the element. In this situation, both contributions on the right-hand side of the above equation scale as $V$. Now, according to Newtonian dynamics, 
 the $i$-component of the net force acting on the element is equal to the $i$-component of the rate of change of its linear  momentum. However, in the limit
  $V\rightarrow 0$,  the  linear acceleration and mass density of the fluid are  both approximately constant across the element. In this case,
  the 
  rate of change of the  element's linear momentum
 also scales as  $V$. In other words,  the net volume force, surface force, and
 rate of change of linear momentum of an infinitesimal fluid element  all scale as the volume of the element, and
 consequently remain  approximately the same order of magnitude as the volume shrinks to zero. We conclude
 that
 the linear equation of motion of an infinitesimal fluid element places no particular restrictions on the stress tensor. 
 
 The $i$-component of the total torque, taken about the origin $O$ of the coordinate system, acting on a fluid element that consists of
 a fixed volume $V$ enclosed by a surface $S$ is written  [see Equations~(\ref{etorque}) and (\ref{e3.6})]
 \begin{equation}
 \tau_i =\int_V \epsilon_{ijk}\,x_j\,F_k\,dV + \oint_S\epsilon_{ijk}\,x_j\,\sigma_{kl}\,dS_l,
 \end{equation}
 where the first and second terms on the right-hand side are due to volume and surface forces, respectively. 
Making use of the tensor divergence theorem (see Section~\ref{stfield}),
 the above expression becomes
 \begin{equation}
 \tau_i = \int_V \epsilon_{ijk}\,x_j\,F_k\,dV + \int_V \epsilon_{ijk}\frac{\partial (x_j\,\sigma_{kl})}{\partial x_l}\,dV,
 \end{equation}
 which reduces to
 \begin{equation}\label{e4.8}
 \tau_i = \int_V \epsilon_{ijk}\,x_j\,F_k\,dV + \int_V \epsilon_{ijk}\,\sigma_{kj}\,dV+ \int_V\epsilon_{ijk}\,x_j\,\frac{\partial \sigma_{kl}}{\partial x_l}\,dV,
 \end{equation}
 since $\partial x_i/\partial x_j = \delta_{ij}$. Assuming that point $O$ lies {\em within}\/ the fluid element, and taking the
 limit $V\rightarrow 0$ in which the $F_i$, $\sigma_{ij}$, and $\partial \sigma_{ij}/\partial x_j$ are all approximately
 constant across the element, we deduce that the first, second, and third terms on the right-hand
 side of the above equation scale as $V^{\,4/3}$, $V$, and $V^{
 \,4/3}$, respectively. Now, according to Newtonian dynamics, the $i$-component
 of the total torque acting on the fluid element is equal to the $i$-component of the
 rate of change of its net angular momentum about $O$. Assuming that the linear acceleration of the fluid 
 is approximately constant across the element, we deduce that the rate of change of its angular momentum scales as $V^{\,4/3}$. 
 Hence, it is clear that 
 the rotational equation of motion of a  fluid element, surrounding a general point $O$, becomes completely dominated
 by the second term on the right-hand side of (\ref{e4.8}) in the limit that the volume of the element approaches zero (since
 this term is a factor $V^{-1/3}$ larger than the other terms). It follows that the second term must be identically zero (otherwise
an   infinitesimal fluid element would acquire  an absurdly large angular velocity).
 This is only possible, for all choices of the position of point $O$, and the shape of the element, if
 \begin{equation}
 \epsilon_{ijk}\,\sigma_{kj} = 0
 \end{equation}
 throughout the fluid. 
 The above relation shows that the stress tensor must be {\em symmetric}: {\em i.e.},
 \begin{equation}
 \sigma_{ji} = \sigma_{ij}.
 \end{equation}
 It immediately follows that the stress tensor only has {\em six}\/ independent components ({\em i.e.}, $\sigma_{11}$, $\sigma_{22}$, $\sigma_{33}$, $\sigma_{12}$, $\sigma_{13}$, and $\sigma_{23}$). 
 
Now, it is always possible to choose the orientation of a set of Cartesian axes in such a manner
 that the non-diagonal components of a given {\em symmetric}\/ second-order tensor field  are all set to zero at a given point in space
 (see Exercise B.6). 
 With reference to such {\em principal axes}, the
 diagonal components of the stress tensor $\sigma_{ij}$  become so-called {\em principal stresses}---$\sigma_{11}'$, $\sigma_{22}'$, $\sigma_{33}'$, say. Of course, in general, the orientation of the principal axes varies with position. 
 The normal stress $\sigma_{11}'$ acting across a surface
 element perpendicular to the first principal axis  corresponds to a tension (or a compression if $\sigma_{11}'$ is negative)
 in the direction of that axis. Likewise, for $\sigma_{22}'$ and $\sigma_{33}'$. Thus, the general state of the
 fluid, at a particular point in space, can be regarded as a superposition of tensions, or compressions, in three orthogonal directions.
 
 The {\em trace}\/ of the stress tensor, $\sigma_{ii}=\sigma_{11}
 +\sigma_{22}+\sigma_{33}$, is a scalar, and, therefore, independent of the orientation of the coordinate axes (see Appendix~\ref{appb}). Thus,
 it follows that, irrespective of the orientation of the principle axes, the trace of the stress tensor at a given point  is always equal to the sum of the principal stresses: {\em i.e.},
 \begin{equation}\label{e4.11}
 \sigma_{ii} = \sigma_{11}'+\sigma_{22}'+\sigma_{33}'.
 \end{equation}
 
\section{Stress Tensor in a Static Fluid}\label{s5.5}
Consider the surface forces exerted on some infinitesimal cubic volume element of a {\em static}\/ fluid. Suppose that the 
components of the stress tensor are approximately constant across the element. Suppose, further, that the sides of
the cube are aligned parallel to the principal axes of the local stress tensor. This tensor, which now has zero
non-diagonal components, can be regarded as the sum of two tensors: {\em i.e.},
\begin{equation}
\left(
\begin{array}{ccc}
\frac{1}{3}\sigma_{ii}& 0 & 0\\[0.5ex]
0&\frac{1}{3}\sigma_{ii} & 0 \\[0.5ex]
0&0&\frac{1}{3}\sigma_{ii}
\end{array}\right),\label{e4.12}
\end{equation}
and
\begin{equation}
\left(
\begin{array}{ccc}
\sigma_{11}'-\frac{1}{3}\sigma_{ii}& 0 & 0\\[0.5ex]
0&\sigma_{22}'-\frac{1}{3}\sigma_{ii} & 0 \\[0.5ex]
0&0&\sigma_{33}'-\frac{1}{3}\sigma_{ii}
\end{array}\right).\label{e4.13}
\end{equation}

The first of the above tensors is {\em isotropic}\/ (see Section~\ref{siso}), and corresponds to the same
normal force  per unit area acting inward  (since the sign of $\frac{1}{3}\sigma_{ii}$ is invariably negative) on each face of
the  volume element. This uniform compression  acts to change the element's volume,
but not its shape, and can  easily be withstood by the fluid within the element.

The second of the above tensors represents the departure of the stress tensor from an isotropic form. The diagonal
components of this tensor have zero sum, in view of (\ref{e4.11}), and thus represent equal and opposite forces per unit area,
acting on   opposing faces of the  volume element, which are such that the forces on at least one pair of opposing faces constitute a tension, and
the forces on at least one pair constitute a compression. Such forces necessarily tend to change the shape of the volume element, either elongating
or compressing it along  one of its symmetry axes. Moreover, this tendency cannot  be offset by any
volume force acting on the element, since such forces become arbitrarily small compared
to surface forces in the limit that the element's volume tends to zero (see Section~\ref{s4.3}).
Now, we have previously defined a fluid as a material
that is incapable of withstanding any tendency of applied forces to change its shape (see Section~\ref{s4.1}). It follows that
if the diagonal components of the tensor (\ref{e4.13}) are non-zero anywhere inside the fluid then it is impossible for the fluid at that point to be at rest. Hence,
we conclude that the principal stresses, $\sigma_{11}'$, $\sigma_{22}'$, and $\sigma_{33}'$, 
must be equal to one another  at all points in a static fluid. This  implies
that the stress tensor  takes the isotropic form (\ref{e4.12}) everywhere in a stationary fluid. Furthermore, this
is true irrespective of the orientation of the coordinate axes, since the components of an isotropic
tensor are rotationally invariant (see Section~\ref{siso}). 

Fluids at rest are generally in a state of compression, so it is convenient to write the stress tensor of a static
fluid in the form
\begin{equation}
\sigma_{ij} = - p\,\delta_{ij},
\end{equation}
where $p=-\frac{1}{3}\sigma_{ii}$ is termed the {\em static fluid pressure}, and is generally a function of ${\bf r}$ and $t$. 
It follows that, in a stationary fluid, the force per unit area exerted across a plane surface element
with unit normal ${\bf n}$ is $-p\,{\bf n}$ [see Equation~(\ref{e4.3})]. Moreover, this normal force  has the {\em same}\/ value for
all possible orientations of ${\bf n}$. This well-known result---namely, 
that the   pressure is the same in all directions at a given point in a static fluid---is known as {\em Pascal's law}, and is a direct consequence of the
fact that a fluid element cannot withstand shear stresses, or, alternatively, any tendency of applied forces
to change its shape. 

\section{Stress Tensor in a Moving Fluid}\label{sstress}
We have seen that  in a static fluid the stress tensor takes the form
\begin{equation}
\sigma_{ij} = -p\,\delta_{ij},
\end{equation}
where $p=-\frac{1}{3}\,\sigma_{ii}$ is the static pressure: {\em i.e.}, minus the normal stress acting in any direction.
Now,  the normal stress at a given point in a moving fluid generally varies with direction: {\em i.e.}, the principal stresses are
not equal to one another. However, we can still define the mean principal stress as $(\sigma_{11}'+\sigma_{22}'+\sigma_{33}')/3= \frac{1}{3}\,\sigma_{ii}$.
Moreover, given that the  principal stresses are actually normal stresses (in a coordinate frame aligned with the principal axes), we can also regard  $\frac{1}{3}\,\sigma_{ii}$
as the mean normal stress. 
It is convenient to define pressure in a moving fluid as minus the {\em mean normal
stress}: {\em i.e.}, 
\begin{equation}
p= -\frac{1}{3}\,\sigma_{ii}.
\end{equation}
Thus, we can write the stress tensor in a moving fluid as the sum of an isotropic part, $-p\,\delta_{ij}$,
which has the same form as the stress tensor in a static fluid, and a remaining non-isotropic
part, $d_{ij}$, which includes any shear stresses, and also has diagonal components whose
sum is zero. In other words,
\begin{equation}
\sigma_{ij} = - p\,\delta_{ij} + d_{ij},
\end{equation}
where
\begin{equation}
d_{ii} = 0.\label{e4.18}
\end{equation}
Moreover, since $\sigma_{ij}$ and $\delta_{ij}$ are both symmetric tensors, it follows that
$d_{ij}$ is also symmetric: {\em i.e.}, 
\begin{equation}
d_{ji} = d_{ij}.\label{e4.19}
\end{equation}

It is clear that the so-called {\em deviatoric stress tensor}, $d_{ij}$, is a consequence of {\em fluid motion}, since
it is zero in a static fluid. Suppose, however, that we were to view a static fluid in both its rest frame and  in a frame of reference moving at
 some {\em constant}\/ velocity relative to the rest frame. Now, we would expect the force distribution within the fluid to
 be the {\em same}\/ in both frames of reference, since the fluid does not accelerate in either. 
 However, in the first frame, the fluid appears stationary and the deviatoric stress tensor is therefore zero, whilst in the second it has a {\em uniform}\/ velocity and the deviatoric stress tensor is also zero (because it is the same as in the rest frame).
 We, thus, conclude that the deviatoric stress tensor is zero   both in a stationary fluid {\em and}\/ in a moving fluid
 possessing  no spatial velocity gradients. This suggests that the deviatoric stress tensor is driven by {\em velocity gradients}\/
 within the fluid. Moreover, the tensor must vanish as these gradients vanish. 
 
Let the $v_i({\bf r}, t)$ be the Cartesian components of the fluid velocity at point ${\bf r}$ and time $t$. The
various velocity gradients within the fluid then take the form $\partial v_i/\partial x_j$. The simplest
possible assumption, which is consistent with the above discussion,  is that the components of the deviatoric stress tensor are {\em linear}\/ functions
of these velocity gradients: {\em i.e.}, 
\begin{equation}\label{e4.20}
d_{ij} = A_{ijkl}\,\frac{\partial v_k}{\partial x_l}.
\end{equation}
Here, $A_{ijkl}$ is a fourth-order tensor (this follows from the quotient rule because $d_{ij}$ and
$\partial v_i/\partial x_j$ are both proper second-order tensors). Any fluid in which the deviatoric
stress tensor takes the above form is termed a {\em Newtonian fluid}, since Newton was the
first to postulate a linear relationship between shear stresses and velocity gradients.  

Now, in an {\em isotropic}\/ fluid---that is, a fluid in which there is no preferred direction---we would
expect the fourth-order tensor $A_{ijkl}$ to be isotropic---that is, to have a form in which all
distinction between different directions is absent. As demonstrated in Section~\ref{siso}, the
most general expression for  an isotropic fourth-order tensor  is
\begin{equation}\label{e4.21}
A_{ijkl}= \alpha\,\delta_{ij}\,\delta_{kl} + \beta\,\delta_{ik}\,\delta_{jl}+\gamma\,\delta_{il}\,\delta_{jk},
\end{equation}
where $\alpha$, $\beta$, and $\gamma$ are arbitrary scalars (which can be functions of position and time). 
Thus, it follows from (\ref{e4.20}) and (\ref{e4.21}) that
\begin{equation}
d_{ij} = \alpha\,\frac{\partial v_k}{\partial x_k}\,\delta_{ij} + \beta\,\frac{\partial v_i}{\partial x_j}
+\gamma\,\frac{\partial v_j}{\partial x_i}.
\end{equation}
However, according to Equation~(\ref{e4.19}), $d_{ij}$ is a {\em symmetric}\/ tensor, which implies that $\beta=\gamma$, 
and
\begin{equation}
d_{ij} = \alpha\,e_{kk}\,\delta_{ij}+2\,\beta\,e_{ij},
\end{equation}
where
\begin{equation}\label{e4.24}
e_{ij} = \frac{1}{2}\left(\frac{\partial v_i}{\partial x_j} + \frac{\partial v_j}{\partial x_i}\right)
\end{equation}
is called the {\em rate of strain tensor}. 
Finally, according to Equation~(\ref{e4.18}), $d_{ij}$ is a {\em traceless}\/ tensor, which yields 
$3\alpha=-2\beta$, and
\begin{equation}\label{e4.25}
d_{ij} = 2\mu\left(e_{ij} - \frac{1}{3}\,e_{kk}\,\delta_{ij}\right),
\end{equation}
where $\mu=\beta$. 
We, thus, conclude that  the most general expression for the stress tensor in an isotropic Newtonian fluid
is
\begin{equation}\label{e4.26}
\sigma_{ij} = - p\,\delta_{ij} +2\mu\left(e_{ij} - \frac{1}{3}\,e_{kk}\,\delta_{ij}\right),
\end{equation}
where $p({\bf r},t)$ and $\mu({\bf r},t)$ are arbitrary scalars. 

\section{Viscosity}
The significance of the parameter $\mu$, appearing in the above expression for the stress tensor, can be seen from the form taken by the relation (\ref{e4.25}) in the
special case of simple shearing motion. With $\partial v_1/\partial x_2$ as the only non-zero
velocity derivative, all of the components of $d_{ij}$ are zero apart from the
shear stresses
\begin{equation}
d_{12} = d_{21} = \mu\,\frac{\partial v_1}{\partial x_2}.
\end{equation}
Thus, $\mu$ is the constant of proportionality between the rate of shear and the tangential
force per unit area when parallel plane layers of fluid slide over one another. This constant of proportionality is generally
referred to as {\em viscosity}. It is a matter of experience that the force between layers
of fluid undergoing relative sliding motion  always  tends to oppose
the  motion, which implies that  $\mu>0$. 

The viscosities of dry air and pure water at $20^\circ\,{\rm C}$  and atmospheric pressure are about $1.8 \times 10^{-5}\, {\rm kg/(m\,s)}$
and $1.0\times 10^{-3}\, {\rm kg/(m\,s)}$, respectively. In neither case does the viscosity exhibit
much variation with pressure. However, the viscosity of air {\em increases}\/ by about $0.3$ percent, and
that of water {\em decreases}\/ by about 3 percent, per degree Centigrade 
rise in temperature. 

\section{Conservation Laws} 
Suppose that $\theta({\bf r}, t)$ is the density of some bulk fluid property ({\em e.g.},  mass, momentum, 
energy) at position ${\bf r}$ and time $t$. In other words, suppose that, at time $t$,  an infinitesimal fluid element of volume $dV$, located
at position ${\bf r}$, contains an amount $\theta({\bf r},t)\,dV$ of the property in question. Note, incidentally,
that $\theta$ can be either a scalar, a component of a vector, or even a component of a tensor.
 The total 
amount of the property contained within some {\em fixed}\/ volume $V$ is
\begin{equation}
\Theta = \int_V \theta\,dV,
\end{equation}
where the integral is taken over all elements of $V$. Let $d{\bf S}$ be an outward directed 
element of the bounding surface of $V$. Suppose that this element is located at point ${\bf r}$. The volume of fluid  that
flows per second across the element, and so out of  $V$,  is ${\bf v}({\bf r}, t)\!\cdot\! d{\bf S}$. Thus, the
amount of the fluid property under consideration which is {\em convected}\/ across the element per second is
$\theta({\bf r},t)\,{\bf v}({\bf r}, t)\cdot d{\bf S}$. It follows that the net amount of the property that 
is convected out of volume $V$ by fluid flow across its bounding surface $S$ is
\begin{equation}\label{eflux}
\Phi_\Theta = \int_S \theta\,{\bf v}\cdot d{\bf S},
\end{equation}
where the integral is taken over all outward directed  elements of $S$. Suppose, finally, that
the property in question   is created within the volume $V$ at the rate $S_\Theta$ per second. 
The conservation equation for the fluid property takes the form
\begin{equation}
\frac{d\Theta}{dt} = S_\Theta-\Phi_\Theta.
\end{equation}
In other words, the rate of increase in the amount of the property contained within $V$ is the
difference between the  creation rate of the property inside $V$, and the rate at
which the property is convected out of $V$ by  fluid flow.
The above conservation law can also be written
\begin{equation}\label{econs}
\frac{d\Theta}{dt} +\Phi_\Theta= S_\Theta.
\end{equation}
Here, $\Phi_\Theta$ is termed the {\em flux}\/ of the property out of $V$, whereas $S_\Theta$ 
is called the net {\em generation rate}\/ of the property within $V$. 

\section{Mass Conservation}\label{scont}
Let $\rho({\bf r},t)$ and ${\bf v}({\bf r},t)$ be the mass density and velocity  of a given fluid at point ${\bf r}$ and time $t$. Consider a
{\em fixed}\/ volume $V$ surrounded by a surface $S$. The net mass contained within $V$ is
\begin{equation}
M = \int_V \rho\,dV,
\end{equation}
where $dV$ is an element of $V$.
Furthermore, the mass flux across $S$, and out of $V$, is [see Equation~(\ref{eflux})]
\begin{equation}
\Phi_M = \int_S \rho\,{\bf v}\cdot d{\bf S},
\end{equation}
where $d{\bf S}$ is an outward directed element of $S$.  Mass conservation requires 
that the rate of increase of the mass contained within $V$,
plus the net mass flux  out of $V$, should equal zero: {\em i.e.},
\begin{equation}
\frac{dM}{dt} +\Phi_M=0
\end{equation}
[{\em cf.}, Equation~(\ref{econs})].
Here, we are assuming that there is no  mass generation (or destruction) within $V$ (since individual molecules
are effectively indestructible). 
It follows that
\begin{equation}
\int_V \frac{\partial \rho}{\partial t}\,dV + \int_S\rho\,{\bf v}\cdot d{\bf S} = 0,
\end{equation}
since $V$ is non-time-varying. Making use of the divergence theorem (see Section~\ref{sdiv}),
the above equation becomes
\begin{equation}
\int_V\left[\frac{\partial \rho}{\partial t} + \nabla\!\cdot\!(\rho\,{\bf v})\right]dV = 0.
\end{equation}
However, this result is true irrespective of the size, shape, or location of  volume $V$, which is only possible if
\begin{equation}\label{e4.33}
\frac{\partial\rho}{\partial t} + \nabla\!\cdot\!(\rho\,{\bf v}) = 0
\end{equation}
throughout the fluid. The above expression is known as the {\em equation of fluid continuity}, and is
a direct consequence of mass conservation. 

\section{Convective Time Derivative}\label{sconv}
The quantity $\partial \rho({\bf r}, t)/\partial t$, appearing in Equation~(\ref{e4.33}),  represents the time derivative of the fluid mass density at the {\em fixed}\/ point ${\bf r}$. Suppose that ${\bf v}({\bf r},t)$ is the instantaneous fluid velocity at the same point. It follows
that the time derivative of the  density, as seen in a frame of reference which is {\em instantaneously co-moving}\/ with the fluid
at point ${\bf r}$, is
\begin{equation}
\lim_{\delta t\rightarrow 0} \frac{\rho({\bf r}+{\bf v}\,\delta t,t+\delta t)-\rho({\bf r}, t)}{\delta t} = \frac{\partial \rho}{\partial t} + {\bf v}\!\cdot\!\nabla \rho \equiv \frac{D \rho}{Dt},
\end{equation}
where we have Taylor expanded $\rho({\bf r}+{\bf v}\,\delta t, t+\delta t)$ up to first-order in $\delta t$, and where
\begin{equation}\label{e4.35}
\frac{D}{Dt}\equiv  \frac{\partial}{\partial t} + {\bf v}\!\cdot\!\nabla = \frac{\partial}{\partial t} + v_i\,\frac{\partial}{\partial x_i}.
\end{equation}
Clearly, the so-called {\em convective time derivative}, $D/Dt$, represents the time derivative seen in the  local rest frame of the fluid.

The continuity equation (\ref{e4.33}) can be rewritten in the form
\begin{equation}\label{e4.36}
\frac{1}{\rho}\,\frac{D\rho}{Dt}\equiv \frac{D\ln\rho}{Dt}=-\nabla\!\cdot\!{\bf v},
\end{equation}
since $\nabla \cdot(\rho\,{\bf v}) = {\bf v}\cdot\nabla \rho + \rho\,\nabla \cdot{\bf v}$ [see (\ref{divprod})]. Consider a
 volume element $V$ that is {\em co-moving}\/ with the fluid. 
 In general, as the element is convected by the fluid its volume changes. In fact, it is easily
 seen that
 \begin{equation}
 \frac{DV}{Dt} = \int_S{\bf v}\cdot d{\bf S}=\int_S v_i\,dS_i = \int_V \frac{\partial v_i}{\partial x_i}\,dV = \int_V \nabla\!\cdot \! {\bf v}\,dV,
 \end{equation}
 where $S$ is the bounding surface of the element, and use has been made of the divergence theorem. In the limit that $V\rightarrow 0$,
 and $\nabla\!\cdot\!{\bf v}$ is approximately constant across the element, we obtain
 \begin{equation}\label{e4.38}
 \frac{1}{V}\,\frac{DV}{Dt} \equiv \frac{D\ln V}{Dt} = \nabla\!\cdot\!{\bf v}.
 \end{equation}
 Hence, we conclude that the divergence of the fluid velocity at a given point in space specifies
 the fractional rate of increase in the volume of an infinitesimal co-moving fluid element at that point. 
 
\section{Momentum Conservation}\label{smom}
Consider a fixed volume $V$ surrounded by a surface $S$. The $i$-component of the total linear momentum contained within $V$
is
\begin{equation}
P_i = \int_V \rho\,v_i\,dV.
\end{equation}
Moreover, the flux of $i$-momentum across $S$, and out of $V$, is [see Equation~(\ref{eflux})]
\begin{equation}
\Phi_i = \int_S \rho\,v_i\,v_j\,dS_j.
\end{equation}
Finally, the  $i$-component of the net  force acting on the fluid within $V$ is
\begin{equation}
f_i = \int_V F_i\,dV + \oint_S \sigma_{ij}\,dS_j,
\end{equation}
where the first and second terms on the right-hand side are the contributions from volume and surface forces, respectively. 

Momentum conservation requires  that the rate of
increase of the net $i$-momentum of the fluid contained within $V$, plus the flux of $i$-momentum out of $V$, is equal to the rate of $i$-momentum generation
within $V$. Of course, from Newton's second law of motion, the latter quantity is equal to the $i$-component
of the net force acting on the fluid contained within $V$. Thus, we obtain [{\em cf.}, Equation~(\ref{econs})] 
\begin{equation}
\frac{dP_i}{dt} + \Phi_i = f_i,
\end{equation}
which can be written
\begin{equation}
\int_V \frac{\partial(\rho\,v_i)}{\partial t}\,dV +\int_S\rho\,v_i\,v_j\,dS_j =  \int_V F_i\,dV + \oint_S \sigma_{ij}\,dS_j,
\end{equation}
since the volume $V$ is non-time-varying. 
Making use of the tensor divergence theorem, this becomes 
\begin{equation}
\int_V\left[\frac{\partial (\rho\,v_i)}{\partial t} + \frac{\partial(\rho\,v_i\,v_j)}{\partial x_j}\right]dV = 
\int_V\left(F_i + \frac{\partial\sigma_{ij}}{\partial x_j}\right)dV.
\end{equation}
However, the above result is valid irrespective of the size, shape, or location of  volume $V$, which is only
possible if
\begin{equation}
\frac{\partial (\rho\,v_i)}{\partial t} + \frac{\partial(\rho\,v_i\,v_j)}{\partial x_j} = F_i + \frac{\partial\sigma_{ij}}{\partial x_j}
\end{equation}
everywhere inside the fluid. Expanding the derivatives, and rearranging, we obtain
\begin{equation}\label{e4.45}
\left(\frac{\partial\rho}{\partial t} + v_j\,\frac{\partial\rho}{\partial x_j} +\rho\, \frac{\partial v_j}{\partial x_j}\right)v_i
+ \rho\left(\frac{\partial v_i}{\partial t} + v_j\,\frac{\partial v_i}{\partial x_j}\right)= F_i + \frac{\partial\sigma_{ij}}{\partial x_j}.
\end{equation}
Now, in tensor notation, the continuity equation (\ref{e4.33})  is written
\begin{equation}\label{e4.46}
\frac{\partial\rho}{\partial t} + v_j\,\frac{\partial\rho}{\partial x_j} + \rho\,\frac{\partial v_j}{\partial x_j}=0.
\end{equation}
So, combining Equations (\ref{e4.45}) and (\ref{e4.46}), we obtain the following  {\em fluid equation of motion}:
\begin{equation}
 \rho\left(\frac{\partial v_i}{\partial t} + v_j\,\frac{\partial v_i}{\partial x_j}\right)= F_i + \frac{\partial\sigma_{ij}}{\partial x_j}.
\end{equation} 
An alternative form of this equation  is
\begin{equation}\label{e4.48}
\frac{D v_i}{Dt} =  \frac{F_i}{\rho} + \frac{1}{\rho}\frac{\partial\sigma_{ij}}{\partial x_j}.
\end{equation}
The above equation describes how the   net volume and surface forces per unit mass acting on  a co-moving fluid element determine its acceleration.

\section{Navier-Stokes Equation}
Equations~(\ref{e4.24}), (\ref{e4.26}), and (\ref{e4.48}) can be combined to give the equation of motion
of an isotropic, Newtonian, classical fluid:
\begin{equation}
\rho \,\frac{D v_i}{Dt}= F_i - \frac{\partial p}{\partial x_i} + \frac{\partial}{\partial x_j}\!\left[\mu\left(\frac{\partial v_i}{\partial x_j}+ \frac{\partial v_j}{\partial x_i}\right)\right] - \frac{\partial}{\partial x_i}\!\left(\frac{2}{3}\,\mu\,\frac{\partial  v_j}{\partial x_j}\right).
\end{equation}
This equation is generally known as the {\em Navier-Stokes equation}.
Now, in situations in which there are no strong  temperature gradients in the fluid it is a good approximation to treat    viscosity  as a spatially uniform quantity, in which case the
Navier-Stokes equation simplifies
somewhat to give
\begin{equation}
\rho \,\frac{D v_i}{Dt}= F_i - \frac{\partial p}{\partial x_i} +\mu\left[\frac{\partial^2 v_i}{\partial x_j\,\partial x_j}+ \frac{1}{3}\,\frac{\partial^2 v_j}{\partial x_i\,\partial x_j}\right].
\end{equation}
When expressed in vector form, the above expression becomes
\begin{equation}\label{e4.52}
\rho\,\frac{D{\bf v}}{Dt}\equiv \rho\!\left[\frac{\partial {\bf v}}{\partial t} + ({\bf v}\!\cdot\!\nabla)\,{\bf v}\right] = {\bf F} - \nabla p + \mu\left[\nabla^2 {\bf v}
+ \frac{1}{3}\,\nabla(\nabla\!\cdot\!{\bf v})\right],
\end{equation}
where use has been made of Equation~(\ref{e4.35}). Here, 
\begin{eqnarray}
[({\bf a}\!\cdot\!\nabla){\bf b}]_i&\equiv& a_j\,\frac{\partial b_i}{\partial x_j},\label{e4.52x}\\[0.5ex]
(\nabla^2{\bf v})_i &\equiv& \nabla^2 v_i.\label{e4.53}
\end{eqnarray}
Note, however, that the above identities are only valid in {\em Cartesian}\/ coordinates (see Appendix~\ref{ccurv}). 

\section{Energy Conservation}
Consider a fixed volume $V$ surrounded by a surface $S$. The total energy content of the fluid contained within $V$ is
\begin{equation}\label{e4.62}
E = \int_V \rho\,{\cal E}\,dV + \int_V\frac{1}{2}\,\rho\,v_i\,v_i\,dV,
\end{equation}
where the first and second terms on the right-hand side are the net internal and kinetic energies, respectively. Here, 
${\cal E}({\bf r},t)$ is the internal ({\em i.e.}, thermal) energy per unit mass of the fluid. The energy flux
across $S$, and out of $V$, is [{\em cf.},  Equation~(\ref{eflux})]
\begin{equation}\label{e4.63}
\Phi_E = \int_S \rho\left({\cal E} + \frac{1}{2}\,v_i\,v_i\right)v_j\,dS_j = \int_V\frac{\partial}{\partial x_j}\!
\left[\rho\left({\cal E} + \frac{1}{2}\,v_i\,v_i\right)v_j\right]dV,
\end{equation}
where use has been made of the tensor divergence theorem. According to the first law of thermodynamics, the rate of increase of
the energy contained within $V$, plus the net energy flux out of $V$, is equal to the net rate of work done on the fluid
within $V$, minus the
net  heat flux out of $V$: {\em i.e.},
\begin{equation}
\frac{dE}{dt} + \Phi_E = \dot{W} -\dot{Q},
\end{equation}
where $\dot{W}$ is the net rate of work, and $\dot{Q}$  the net heat flux. It can be seen that $\dot{W}-\dot{Q}$ is the
effective energy generation rate within $V$ [{\em cf.}, Equation~(\ref{econs})]. 

Now, the net rate at which volume and surface forces do work on the fluid within $V$ is
\begin{equation}\label{e4.65}
\dot{W} = \int_V v_i\,F_i\,dV + \int_S v_i\,\sigma_{ij}\,dS_j = \int_V\left[v_i\,F_i+ \frac{\partial(v_i\,\sigma_{ij})}{\partial x_j}\right]dV,
\end{equation}
where use has been made of the tensor divergence theorem.

Generally speaking, heat flow in fluids is driven by {\em temperature gradients}. Let the $q_i({\bf r},t)$ be the
Cartesian components of the heat flux density at position ${\bf r}$ and time $t$. It follows that the heat  flux
across a surface element $d{\bf S}$, located at point ${\bf r}$, is ${\bf q}\cdot d{\bf S} = q_i\,dS_i$. Let $T({\bf r},t)$ be
the  temperature of the fluid at position ${\bf r}$ and time $t$. Thus, a general temperature gradient takes the
form $\partial T/\partial x_i$.   Let us assume that there is a {\em linear}\/ relationship between the components of the local heat flux
density and the local temperature gradient: {\em i.e.},
\begin{equation}
q_i = A_{ij}\,\frac{\partial T}{\partial x_j},
\end{equation}
where the $A_{ij}$ are the components of a second-rank tensor (which can be functions of position and time). 
Now, in an {\em isotropic}\/ fluid we would expect $A_{ij}$ to be an isotropic tensor (see Section~\ref{siso}). 
However, the most general second-order isotropic tensor is  simply a multiple of $\delta_{ij}$. Hence, we can write
\begin{equation}
A_{ij} = -\kappa\,\delta_{ij},
\end{equation}
where $\kappa({\bf r},t)$  is termed the {\em thermal conductivity}\/ of the fluid. It follows that the most general
expression for the heat flux density in an isotropic  fluid is
\begin{equation}
q_i = -\kappa\,\frac{\partial T}{\partial x_i},
\end{equation}
or, equivalently, 
\begin{equation}\label{e4.68x}
{\bf q} = -\kappa\,\nabla T.
\end{equation}
Moreover, it is a matter of experience that heat flows {\em down}\/ temperature gradients: {\em i.e.}, $\kappa>0$. 
We conclude that the net heat flux out of volume $V$ is
\begin{equation}\label{e4.69}
\dot{Q} = -\int_S\kappa\,\frac{\partial T}{\partial x_i}\,dS_i = - \int_V\frac{\partial}{\partial x_i}\!\left(\kappa\,\frac{\partial T}{\partial x_i}\right) dV,
\end{equation}
where use has been made of the tensor divergence theorem.

Equations~(\ref{e4.62})--(\ref{e4.65}) and (\ref{e4.69}) can be combined to give the following energy conservation equation:
\begin{eqnarray}
\int_V\left\{\frac{\partial}{\partial t}\!\left[\rho\left({\cal E}+ \frac{1}{2}\,v_i\,v_i\right)\right]
+\frac{\partial}{\partial x_j}\!\left[\rho\left({\cal E}+\frac{1}{2}\,v_i\,v_i\right)v_j\right]\right\}dV &&\nonumber\\[0.5ex]
= \int_V\left[v_i\,F_i + \frac{\partial}{\partial x_j}\!\left(v_i\,\sigma_{ij} + \kappa\,\frac{\partial T}{\partial x_j}\right)\right]dV.&&
\end{eqnarray}
However, this result is valid irrespective of the size, shape, or location of  volume $V$, which is only
possible if
\begin{equation}
\frac{\partial}{\partial t}\!\left[\rho\left({\cal E}+ \frac{1}{2}\,v_i\,v_i\right)\right]
+\frac{\partial}{\partial x_j}\!\left[\rho\left({\cal E}+\frac{1}{2}\,v_i\,v_i\right)v_j\right]
= v_i\,F_i + \frac{\partial}{\partial x_j}\!\left(v_i\,\sigma_{ij} + \kappa\,\frac{\partial T}{\partial x_j}\right)
\end{equation}
everywhere inside the fluid. Expanding some of the derivatives, and rearranging, we obtain
\begin{equation}
\rho\,\frac{D}{D t}\!\left({\cal E}+ \frac{1}{2}\,v_i\,v_i\right)
= v_i\,F_i + \frac{\partial}{\partial x_j}\!\left(v_i\,\sigma_{ij} + \kappa\,\frac{\partial T}{\partial x_j}\right),
\end{equation}
where use has been made of the continuity equation (\ref{e4.36}).
Now, the scalar product of ${\bf v}$ with the fluid equation of motion (\ref{e4.48}) yields
\begin{equation}
\rho\,v_i\,\frac{D v_i}{Dt} \equiv \rho\,\frac{D}{Dt}\!\left(\frac{1}{2}\,v_i\,v_i\right) = v_i\,F_i + v_i\,\frac{\partial \sigma_{ij}}{\partial x_j}.
\end{equation}
Combining the previous two equations, we get
\begin{equation}
\rho\,\frac{D{\cal E}}{Dt} = \frac{\partial v_i}{\partial x_j} \,\sigma_{ij}+ \frac{\partial}{\partial x_j}\!\left(\kappa\,\frac{\partial T}{\partial x_j}\right).
\end{equation}
Finally, making use of (\ref{e4.26}), we deduce that the {\em energy conservation equation}\/ for an isotropic Newtonian fluid
takes the general form
\begin{equation}\label{e4.66}
\frac{D{\cal E}}{Dt} = - \frac{p}{\rho}\,\frac{\partial v_i}{\partial x_i} + \frac{1}{\rho}\left[\chi + \frac{\partial}{\partial x_j}\!\left(
\kappa\,\frac{\partial T}{\partial x_j}\right)\right].
\end{equation}
Here,
\begin{equation}\label{e4.67}
\chi = \frac{\partial v_i}{\partial x_j}\,d_{ij} = 2\mu\left(e_{ij}\,e_{ij} - \frac{1}{3}\,e_{ii}\,e_{jj}\right)
=\mu\left(\frac{\partial v_i}{\partial x_j}\,\frac{\partial v_i}{\partial x_j}+\frac{\partial v_i}{\partial x_j}\,\frac{\partial v_j}{\partial x_i} - \frac{2}{3}\,\frac{\partial v_i}{\partial x_i}\,\frac{\partial v_j}{\partial x_j}\right)
\end{equation}
is the rate of heat generation per unit volume due to viscosity.
When written in vector form, Equation~(\ref{e4.66}) becomes
\begin{equation}\label{e4.68}
\frac{D{\cal E}}{Dt} = - \frac{p}{\rho}\,\nabla\!\cdot\!{\bf v} + \frac{\chi}{\rho} +\frac{\nabla\!\cdot(
\kappa\,\nabla T)}{\rho}.
\end{equation}
According to the above equation, the internal energy per unit mass of a co-moving fluid element
evolves in time as a consequence of work done on the element by pressure, as 
its volume changes, viscous heat generation due to flow shear, and heat conduction.

\section{Equations of Incompressible Fluid Flow}\label{siff}
In most situations of general interest, the flow of a conventional liquid, such as water, is {\em incompressible}\/ to
a high degree of accuracy. Now, a fluid is said to be incompressible  when the mass density of a co-moving volume element 
does not change appreciably as the element moves through regions of varying pressure. In other words,  for an incompressible fluid, the rate of change of $\rho$
following the motion is zero: {\em i.e.},
\begin{equation}\label{e4.78}
\frac{D\rho}{Dt} = 0.
\end{equation}
In this case, the continuity equation (\ref{e4.36}) reduces to
\begin{equation}
\nabla\!\cdot\!{\bf v} = 0.
\end{equation}
We conclude that,  as a consequence of mass conservation, an incompressible fluid must have a {\em divergence-free}, or solenoidal,  velocity field. This immediately implies, from Equation~(\ref{e4.38}), that  the volume of a co-moving fluid element is a constant of the motion. In most practical situations, the initial density distribution in an incompressible fluid is {\em uniform}\/ in space. 
Hence, it follows from (\ref{e4.78}) that the density distribution remains uniform in space and constant in time.
In other words, we can generally treat the density, $\rho$, as a  uniform constant  in incompressible fluid flow problems. 

Suppose that the volume force acting on the fluid is {\em conservative}\/ in nature (see Section~\ref{sgrad}): {\em i.e.},
\begin{equation}
{\bf F} = - \rho\,\nabla\Psi,
\end{equation}
where $\Psi({\bf r}, t)$ is the potential energy per unit mass, and $\rho\,\Psi$ the
potential energy per unit volume. Assuming that the fluid viscosity is a spatially uniform quantity, which is
generally the case (unless there are strong temperature variations within the fluid), the Navier-Stokes equation for an incompressible fluid reduces to
\begin{equation}\label{e5.78}
\frac{D{\bf v}}{Dt}=   - \frac{\nabla p}{\rho}-\nabla\Psi + \nu\,\nabla^2 {\bf v},
\end{equation}
where
\begin{equation}
\nu = \frac{\mu}{\rho}
\end{equation}
is termed the {\em kinematic viscosity}, and has units of meters squared per second. 
Roughly speaking, momentum diffuses a distance of order $\sqrt{\kappa\,t}$ meters in $t$ seconds as a consequence of viscosity.
The kinematic viscosity of water at $20^\circ\,{\rm C}$ is about $1.0\times 10^{-6}\,{\rm m^2/s}$. 
It follows that viscous momentum diffusion in water is a relatively slow process. 

The complete set of equations governing incompressible flow is
\begin{eqnarray}
\nabla\!\cdot\!{\bf v} &=& 0,\label{e4.83}\\[0.5ex]
\frac{D{\bf v}}{Dt}&=& - \frac{\nabla p}{\rho} -\nabla\Psi  + \nu\,\nabla^2 {\bf v}.\label{e4.84}
\end{eqnarray}
Here, $\rho$ and $\nu$ are regarded as known constants, and $\Psi({\bf r},t)$  as a known function. Thus, we have
four equations---namely, Equation~(\ref{e4.83}), plus the three components of Equation~(\ref{e4.84})---for
four unknowns---namely, the pressure, $p({\bf r}, t)$, plus the three components of the velocity, ${\bf v}({\bf r}, t)$. 
Note that an energy conservation equation is redundant in the case of incompressible fluid flow.

\section{Equations of Compressible Fluid Flow}
In many situations of general interest, the flow of gases is {\em compressible}: {\em i.e.}, there are significant changes in the
mass density  as the gas flows from place to place. For the case of compressible flow, the continuity equation 
(\ref{e4.36}), and the Navier-Stokes equation (\ref{e4.52}), must be augmented by the energy
conservation equation (\ref{e4.68}), as well as thermodynamic relations that specify the internal
energy per unit mass  and the temperature in terms of the density and pressure.
For an {\em ideal gas}, these relations take the form
\begin{eqnarray}\label{e4.85}
{\cal E} &=& \frac{c_V}{\cal M}\,T,\\[0.5ex]
T&=& \frac{\cal M}{\cal R}\,\frac{p}{\rho},\label{e4.86}
\end{eqnarray}
where $c_V$ is the molar specific heat  {\em at constant volume}, ${\cal R}=8.3145\,{\rm J\,K^{-1}\,mol^{-1}}$ 
the molar ideal gas constant, ${\cal M}$ the molar mass ({\em i.e.}, the mass of 1 mole of gas molecules), and $T$  the temperature in degrees Kelvin. Incidentally, 1 mole corresponds to $6.0221\times 10^{24}$ molecules. Here, we have assumed, for the sake of simplicity, that $c_V$ is a  uniform constant. 
It is also convenient to  assume that the  thermal conductivity, $\kappa$, is a
 uniform constant. Making use of these approximations, Equations~(\ref{e4.36}), (\ref{e4.68}), (\ref{e4.85}), and (\ref{e4.86}) can be
combined to give
\begin{equation}
\frac{1}{\gamma-1}\left(\frac{Dp}{Dt} - \frac{\gamma\,p}{\rho}\,\frac{D\rho}{Dt}\right)
= \chi + \frac{\kappa\,{\cal M}}{\cal R}\,\nabla^2\!\left(\frac{p}{\rho}\right),
\end{equation}
where 
\begin{equation}
\gamma = \frac{c_p}{c_V} = \frac{c_V+{\cal R}}{c_V}
\end{equation}
is the ratio of the molar specific heat at {\em constant pressure}, $c_p$,  to that at constant volume, $c_V$. (Incidentally, the result that $c_p=c_V+{\cal R}$
for an ideal gas is a standard theorem of thermodynamics.) The ratio of specific heats of dry air
at $20^\circ\,{\rm C}$ is $1.40$. 

The complete set of equations governing compressible ideal gas flow are
\begin{eqnarray}
\frac{D\rho}{Dt} &=& -\rho\,\nabla\!\cdot\!{\bf v},\label{e4.89}\\[0.5ex]
\frac{D{\bf v}}{Dt} &=& -\frac{\nabla p}{\rho} - \nabla\Psi+\frac{\mu}{\rho}\left[\nabla^2{\bf v} + \frac{1}{3}\,\nabla(\nabla\!\cdot\!{\bf v})\right],\label{e4.90}\\[0.5ex]
\frac{1}{\gamma-1}\left(\frac{Dp}{Dt} - \frac{\gamma\,p}{\rho}\,\frac{D\rho}{Dt}\right)
&=& \chi + \frac{\kappa\,{\cal M}}{\cal R}\,\nabla^2\!\left(\frac{p}{\rho}\right),\label{e4.91}
\end{eqnarray}
where the dissipation function $\chi$ is specified in terms of $\mu$ and ${\bf v}$ in Equation~(\ref{e4.67}). Here, $\mu$,
$\gamma$, $\kappa$, ${\cal M}$, and ${\cal R}$ are regarded as known constants, and $\Psi({\bf r}, t)$  as a known function. Thus, we have five equations---namely, Equations~(\ref{e4.89}) and (\ref{e4.91}), plus the three components of Equation~(\ref{e4.90})---for five 
unknowns---namely,
the density, $\rho({\bf r},t)$, the pressure, $p({\bf r}, t)$, and the three components of the velocity, ${\bf v}({\bf r}, t)$. 

\section{Dimensionless Numbers in Incompressible Flow}\label{sdim}
It is helpful to normalize the equations of incompressible fluid flow, (\ref{e4.83})--(\ref{e4.84}), in the following
manner: $\bar{\nabla} = L\,\nabla$, $\bar{\bf v} = {\bf v}/V_0$, $\bar{t} = (V_0/L)\,t$, 
$\bar{\Psi} = \Psi/(g\,L)$, and  $\bar{p} = p/(\rho\,V_0^{\,2} +\rho\,g\,L+ \rho\,\nu\,V_0/L)$. Here, $L$ is a typical spatial variation length-scale,  $V_0$ 
a typical fluid velocity, and $g$  a typical gravitational acceleration (assuming that $\Psi$ represents a gravitational
potential energy per unit mass). All hatted quantities are  dimensionless, and are designed to be comparable with unity. 
The normalized equations of incompressible fluid flow take the form
\begin{eqnarray}\label{e4.89x}
\bar{\nabla}\!\cdot\!\bar{\bf v} &=& 0,\\[0.5ex]
\frac{D\bar{\bf v}}{D\bar{t}}&=&   - \left(1+\frac{1}{{\rm Fr}}+\frac{1}{{\rm Re}}\right)\bar{\nabla} \bar{p}-\frac{\bar{\nabla}\bar{\Psi}}{\rm Fr} + \frac{\bar{\nabla}^2 \bar{\bf v}}{\rm Re},\label{e4.90x}
\end{eqnarray}
where $D/D\bar{t}\equiv \partial/\partial \bar{t} + \bar{\bf v}\!\cdot\!\bar{\nabla}$, and
\begin{eqnarray}
{\rm Re} &=& \frac{L\,V_0}{\nu},\\[0.5ex]
{\rm Fr} &=& \frac{V_0^{\,2}}{g\,L}.
\end{eqnarray}
Here, the dimensionless numbers ${\rm Re}$ and ${\rm Fr}$ are known as the {\em Reynolds number}\/ and the
{\em Froude number}, respectively. The Reynolds number is the typical ratio of inertial to viscous forces within the fluid,
whereas the Froude number is the typical ratio of inertial to gravitational forces.  Thus, viscosity is relatively important
compared to inertia when ${\rm Re}\ll 1$, and {\em vice versa}. Likewise, gravity is relatively important compared to
inertia when ${\rm Fr}\ll 1$, and
{\em vice versa}. 
Note that, in principal,  ${\rm Re}$ and ${\rm Fr}$
are the only quantities in Equations~(\ref{e4.89x}) and (\ref{e4.90x}) that can be significantly greater or smaller
than unity.

For the case of water at $20^\circ\,{\rm C}$, located on the surface of the Earth, 
\begin{eqnarray}
{\rm Re} &\simeq& 1.0\times 10^6\,L({\rm m})\,V_0({\rm m\,s^{-1}}),\\[0.5ex]
{\rm Fr} &\simeq& 1.0\times 10^{-1}\,[V_0({\rm m\,s^{-1}})]^{\,2}/L({\rm m}).
\end{eqnarray}
Thus, if $L\sim 1\,{\rm m}$ and $V_0\sim 1\,{\rm m\,s^{-1}}$, as is often the
case for terrestrial water dynamics, the above expressions suggest that
 ${\rm Re}\gg 1$ and ${\rm Fr}\sim{\cal O}(1)$. 
In this situation, the viscous  term on the right-hand side of (\ref{e4.90x}) becomes negligible, 
and the (unnormalized) incompressible fluid flow equations
reduce to the following {\em inviscid, incompressible, fluid flow equations}:
\begin{eqnarray}
\nabla\!\cdot\!{\bf v} &=& 0,\\[0.5ex]
\frac{D{\bf v}}{Dt}&=& - \frac{\nabla p}{\rho} - \nabla\Psi.
\end{eqnarray}

For the case of lubrication oil at $20^\circ\,{\rm C}$, located on the surface of the Earth, $\nu\simeq 1.0\times 10^{-4}\,{\rm m^2\,s^{-1}}$ ({\em i.e.},
oil is about 100 times more viscous than water), and
so
\begin{eqnarray}
{\rm Re} &\simeq& 1.0\times 10^4\,L({\rm m})\,V_0({\rm m\,s^{-1}}),\\[0.5ex]
{\rm Fr} &\simeq& 1.0\times 10^{-1}\,[V_0({\rm m\,s^{-1}})]^{\,2}/L({\rm m}).
\end{eqnarray}
Suppose that oil is slowly flowing down a narrow lubrication channel such that $L\ll 10^{-3}\,{\rm m}$ and $V_0\ll 10^{-1}\,{\rm m\,s^{-1}}$.
It follows, from the above expressions, that ${\rm Re}\ll 1$ and ${\rm Fr}\ll 1$. In this situation, the inertial term on the left-hand
side of (\ref{e4.90x}) becomes negligible, and the (unnormalized)
 incompressible fluid flow equations
reduce to the following {\em inertia-free, incompressible, fluid flow equations}:
\begin{eqnarray}
\nabla\!\cdot\!{\bf v} &=& 0,\\[0.5ex]
0&=& - \frac{\nabla p}{\rho} - \nabla\Psi+\nu\,\nabla^2{\bf v}.
\end{eqnarray}

\section{Dimensionless Numbers in Compressible Flow}\label{scomp}
It is helpful to normalize the equations of compressible ideal gas flow, (\ref{e4.89})---(\ref{e4.91}), in the following
manner: $\bar{\nabla} = L\,\nabla$, $\bar{\bf v} = {\bf v}/V_0$, $\bar{t} = (V_0/L)\,t$, $\bar{\rho}=\rho/\rho_0$, $\bar{\Psi} = \Psi/(g\,L)$, $\bar{\chi} = (L/V_0)^2\,\chi$, and $\bar{p} = (p-p_0)/(\rho_0\,V_0^{\,2} +\rho_0\,g\,L+ \rho_0\,\nu\,V_0/L)$. Here, $L$ is a typical spatial variation length-scale,  $V_0$ 
a typical fluid velocity, $\rho_0$  a typical mass density, and $g$ a typical gravitational acceleration (assuming that $\Psi$ represents a gravitational
potential energy per unit mass). Furthermore, $p_0$ corresponds to atmospheric pressure at ground level, and is a uniform
constant. It follows that $\bar{p}$ represents  deviations from atmospheric pressure.
All hatted quantities are  dimensionless, and are designed to be comparable with unity. 
The normalized equations of compressible ideal gas flow take the form
\begin{eqnarray}\label{e4.97}
\frac{D\bar{\rho}}{Dt}&=& -\bar{\rho}\,\bar{\nabla}\!\cdot\!\bar{\bf v},\\[0.5ex]
\frac{D\bar{\bf v}}{D\bar{t}}&=&   - \left(1+\frac{1}{{\rm Fr}}+\frac{1}{{\rm Re}}\right)\frac{\bar{\nabla}\bar{p}}{\bar{\rho}}-\frac{\bar{\nabla}\bar{\Psi}}{\rm Fr} + \frac{1}{{\rm Re}\,\bar{\rho}}\left[\bar{\nabla}^2 \bar{\bf v}-\frac{1}{3}\,\bar{\nabla}(\bar{\nabla}\!\cdot\!\bar{\bf v})\right],\label{e4.97x}\\[0.5ex]
\frac{1}{\gamma-1}\left[\frac{D\bar{p}}{D\bar{t}}\right.\!\!&-&\!\! \left.\gamma\left(\frac{\bar{p}_0+\bar{p}}{\bar{\rho}}\right)\frac{D\bar{\rho}}{D\bar{t}}\right]=\frac{\bar{\chi}}{1+{\rm Re}\,(1+1/{\rm Fr})}+ \frac{1}{{\rm Re}\,{\rm Pr}}\,\nabla^2\!\left(
\frac{\bar{p}_0 + \bar{p}}{\bar{\rho}}\right),\nonumber\\[0.5ex]&&\label{e4.98}\\[0.5ex]
\bar{p}_0 &=& \frac{1}{\gamma\,{\rm M}^{\,2}\,(1+1/{\rm Fr}+1/{\rm Re})},\label{e4.99}
\end{eqnarray}
where $D/D\bar{t}\equiv \partial/\partial \bar{t} + \bar{\bf v}\!\cdot\!\bar{\nabla}$, 
\begin{eqnarray}
{\rm Re} &=& \frac{L\,V_0}{\nu},\\[0.5ex]
{\rm Fr} &=& \frac{V_0^{\,2}}{g\,L},\\[0.5ex]
{\rm Pr} &=& \frac{\nu}{\kappa_H},\\[0.5ex]
{\rm M} &=& \frac{V_0}{\sqrt{\gamma\,p_0/\rho_0}},
\end{eqnarray}
and
\begin{eqnarray}
\nu &=&\frac{\mu}{\rho_0},\\[0.5ex]
\kappa_H &=& \frac{\kappa\,{\cal M}}{{\cal R}\,\rho_0}.
\end{eqnarray}
Here, the dimensionless numbers ${\rm Re}$, ${\rm Fr}$, ${\rm Pr}$, and ${\rm M}$  are known as the {\em Reynolds number},  
{\em Froude number}, {\em Prandtl number},  and {\em Mach number}, respectively.  The Reynolds number is the typical ratio of inertial to viscous forces within the gas,
the Froude number  the typical ratio of inertial to gravitational forces, the  Prandtl number 
the typical ratio of  momentum   to  thermal diffusion rates, 
and the Mach number the typical ratio of 
gas flow and sound propagation speeds. Thus, thermal diffusion is far faster than momentum diffusion when ${\rm Pr}\ll 1$,
and {\em vice versa}. Moreover, the gas flow is termed {\em subsonic}\/ when ${\rm M}\ll 1$, {\em supersonic}\/ when
${\rm M}\gg 1$, and {\em transonic}\/ when ${\rm M}\sim {\cal O}(1)$. 
Note that $\sqrt{\gamma\,p_0/\rho_0}$ is the speed of sound in the undisturbed gas. 
The quantity $\kappa_H$ is called the {\em thermal diffusivity}\/ of the gas, and has
units of meters squared per second. Thus, heat typically diffuses through the gas a distance $\sqrt{\kappa_H\,t}$ meters in $t$
seconds. 
The thermal diffusivity of dry air at atmospheric
pressure and $20^\circ\,{\rm C}$ is about
$\kappa_H = 2.1\times 10^{-5}\,{\rm m^2\,s^{-1}}$.
It follows that heat diffusion in air is a relatively slow process. 
The kinematic viscosity of dry air at atmospheric pressure and $20^\circ\,{\rm C}$ is
about $\nu= 1.5\times 10^{-5}\,{\rm m^2\,s^{-1}}$.  Hence, momentum diffusion in air
is also a relatively slow process. 

For the case of dry air at atmospheric pressure and $20^\circ\,{\rm C}$, 
\begin{eqnarray}\label{e4.112}
{\rm Re} &\simeq& 6.7\times 10^4\,L({\rm m})\,V_0({\rm m\,s^{-1}}),\\[0.5ex]
{\rm Fr} &\simeq& 1.0\times 10^{-1}\,[V_0({\rm m\,s^{-1}})]^{\,2}/L({\rm m}),\\[0.5ex]
{\rm Pr} &\simeq & 7.2\times 10^{-1},\\[0.5ex]
{\rm M} &\simeq &2.9\times 10^{-3}\,V_0({\rm m\,s^{-1}}).\label{e4.115}
\end{eqnarray}
Thus, if $L\sim 1\,{\rm m}$ and $V_0\sim 1\,{\rm m\,s^{-1}}$, as is often the case for {\em subsonic}\/
air dynamics close to the Earth's surface, the above expressions suggest that ${\rm Re}\gg 1$,  ${\rm M}\ll 1$, and
${\rm Fr}, {\rm Pr} \sim {\cal O}(1)$. It immediately follows from Equation~(\ref{e4.99}) that $\bar{p}_0\gg 1$. 
However, in this situation, Equation~(\ref{e4.98}) is dominated by the second term in square brackets on
its left-hand side. Hence, this equation can only be satisfied if the term in question is small, which implies  that
\begin{equation}
\frac{D\bar{\rho}}{D\bar{t}} \ll 1.
\end{equation}
Equation~(\ref{e4.97}) then gives
\begin{equation}
\bar{\nabla}\!\cdot\!\bar{\bf v} \ll 1.
\end{equation}
Thus, it is evident that subsonic ({\em i.e.}, ${\rm M}\ll 1$) gas flow is essentially {\em incompressible}.
The fact that ${\rm Re}\gg 1$ implies that such flow is also essentially {\em inviscid}. In the
incompressible inviscid limit (in which $\bar{\nabla}\!\cdot\!\bar{\bf v}=0$ and ${\rm Re}\gg 1$),  the
(unnormalized) compressible ideal gas flow equations
reduce to the previously derived  inviscid, incompressible, fluid flow equations: {\em i.e.}, 
\begin{eqnarray}
\nabla\!\cdot\!{\bf v} &=& 0,\\[0.5ex]
\frac{D{\bf v}}{Dt}&=& - \frac{\nabla p}{\rho} - \nabla\Psi.
\end{eqnarray}
It follows that the equations which govern subsonic gas dynamics close to the surface of the Earth are essentially the
same as those which govern the  flow of water.

Suppose that $L\sim 1\,{\rm m}$ and $V_0\sim 300\,{\rm m\,s^{-1}}$, as is typically the case for {\em transonic}\/
air dynamics ({\em e.g.}, air flow over the wing of a fighter jet). In this situation, 
Equations~(\ref{e4.99}) and (\ref{e4.112})--(\ref{e4.115}) yield ${\rm Re}, {\rm Fr} \gg 1$ and ${\rm M}, {\rm Pr}, \bar{p}_0\sim {\cal O}(1)$. It follows that  the final two terms  on the right-hand sides of Equations~(\ref{e4.97x}) and
(\ref{e4.98}) can be neglected. Thus, the (unnormalized) compressible ideal gas flow equations reduce to the
following set of {\em inviscid  adiabatic ideal gas flow equations}:
\begin{eqnarray}
\frac{D\rho}{Dt} &=& -\rho\,\nabla\!\cdot\!{\bf v},\\[0.5ex]
\frac{D{\bf v}}{Dt} &=& -\frac{\nabla p}{\rho},\\[0.5ex]
\frac{D}{Dt}\!\left(\frac{p}{\rho^\gamma}\right)&=&0. \label{e4.122}
\end{eqnarray}
In particular, if the initial distribution of $p/\rho^\gamma$ is  uniform in space, as is often the case, then Equation~(\ref{e4.122}) ensures that the distribution remains uniform as time progresses. In fact, it can be shown that the entropy
per unit mass of an ideal gas is
\begin{equation}
{\cal S} = \frac{c_V}{{\cal M}}\ln\!\left(\frac{p}{\rho^\gamma}\right).
\end{equation}
Hence, the assumption that $p/\rho^\gamma$ is uniform in space is equivalent to the assumption that the
entropy per unit mass of the gas is a spatial constant. A gas in which this is the case is termed {\em homentropic}. 
Equation~(\ref{e4.122}) ensures that the entropy of a co-moving gas element is a constant of the motion in
transonic flow. A gas in which this is the case is termed {\em isentropic}. 
In the homentropic case, the above compressible gas flow equations simplify
somewhat to give
\begin{eqnarray}
\frac{D\rho}{Dt} &=& -\rho\,\nabla\!\cdot\!{\bf v},\\[0.5ex]
\frac{D{\bf v}}{Dt} &=& -\frac{\nabla p}{\rho},\\[0.5ex]
\frac{p}{p_0} &=& \left(\frac{\rho}{\rho_0}\right)^\gamma.\label{e4.125}
\end{eqnarray}
Here, $p_0$ is atmospheric pressure, and $\rho_0$ is the density of air at atmospheric pressure. Equation~(\ref{e4.125})
is known as the {\em adiabatic gas law}, and is a consequence of the fact that transonic gas dynamics takes place
far too quickly for thermal heat conduction (which is a relatively slow process) to have any appreciable effect on the temperature distribution within the gas. Incidentally, a gas in which thermal diffusion is negligible is generally termed {\em adiabatic}.

\section{Fluid Equations in Cartesian Coordinates}\label{sx2.18}
Let us adopt the conventional Cartesian coordinate system, $x$, $y$, $z$. According to Equation~(\ref{e4.26}), the various components
of the stress tensor are
\begin{eqnarray}
\sigma_{xx} &=& -p + 2\mu\,\frac{\partial v_x}{\partial x},\\[0.5ex]
\sigma_{yy} &=& -p + 2\mu\,\frac{\partial v_y}{\partial y},\\[0.5ex]
\sigma_{zz} &=& -p + 2\mu\,\frac{\partial v_z}{\partial z},\\[0.5ex]
\sigma_{xy}=\sigma_{yx} &=& \mu\left(\frac{\partial v_x}{\partial y}+\frac{\partial v_y}{\partial x}\right),\\[0.5ex]
\sigma_{xz}=\sigma_{zx} &=& \mu\left(\frac{\partial v_x}{\partial z}+\frac{\partial v_z}{\partial x}\right),\\[0.5ex]
\sigma_{yz}=\sigma_{zy} &=& \mu\left(\frac{\partial v_y}{\partial z}+\frac{\partial v_z}{\partial y}\right),
\end{eqnarray}
where ${\bf v}$ is the  velocity, $p$  the pressure, and $\mu$ the viscosity. The equations of compressible
fluid flow, (\ref{e4.89})--(\ref{e4.91}) (from which the equations of incompressible fluid flow
can easily be obtained by setting $D\rho/Dt=0$), become
\begin{eqnarray}
\frac{D\rho}{Dt} &=&-\rho\,\Delta,\\[0.5ex]
\frac{Dv_x}{Dt}&=& - \frac{1}{\rho}\,\frac{\partial p}{\partial x} - \frac{\partial\Psi}{\partial x}
+ \frac{\mu}{\rho}\left(\nabla^2 v_x + \frac{1}{3}\,\frac{\partial\Delta}{\partial x}\right),\\[0.5ex]
\frac{Dv_y}{Dt}&=& - \frac{1}{\rho}\,\frac{\partial p}{\partial y} - \frac{\partial\Psi}{\partial y}
+ \frac{\mu}{\rho}\left(\nabla^2 v_y + \frac{1}{3}\,\frac{\partial\Delta}{\partial y}\right),\\[0.5ex]
\frac{Dv_z}{Dt}&=& - \frac{1}{\rho}\,\frac{\partial p}{\partial z} - \frac{\partial\Psi}{\partial z}
+ \frac{\mu}{\rho}\left(\nabla^2 v_z + \frac{1}{3}\,\frac{\partial\Delta}{\partial z}\right),\\[0.5ex]
\frac{1}{\gamma-1}\left(\frac{D\rho}{Dt} - \frac{\gamma\,p}{\rho}\,\frac{D\rho}{Dt}\right)
&=&\chi + \frac{\kappa\,{\cal M}}{{\cal R}}\,\nabla^2\left(\frac{p}{\rho}\right),
\end{eqnarray}
where $\rho$ is the mass density, $\gamma$ the ratio of specific heats, $\kappa$ the heat conductivity, ${\cal M}$
the molar mass, and ${\cal R}$ the molar ideal gas constant. Furthermore,
\begin{eqnarray}
\Delta&=&\frac{\partial v_x}{\partial x} + \frac{\partial v_y}{\partial y} + \frac{\partial v_z}{\partial z},\\[0.5ex]
\frac{D}{Dt} &=& \frac{\partial}{\partial t} + v_x\,\frac{\partial }{\partial x} + v_y\,\frac{\partial}{\partial y} + v_z\,\frac{\partial}{\partial z},\\[0.5ex]
\nabla^2 &=&\frac{\partial^2}{\partial x^2} + \frac{\partial^2}{\partial y^2}+\frac{\partial^2}{\partial z^2},\\[0.5ex]
\chi &=&2\mu\left[\left(\frac{\partial v_x}{\partial x}\right)^2+\left(\frac{\partial v_y}{\partial y}\right)^2+
\left(\frac{\partial v_z}{\partial z}\right)^2+\frac{1}{2}\left(\frac{\partial v_x}{\partial y}+\frac{\partial v_y}{\partial x}\right)^2\right.\nonumber\\[0.5ex]
&&\left.+\frac{1}{2}\left(\frac{\partial v_x}{\partial z}+\frac{\partial v_z}{\partial x}\right)^2+\frac{1}{2}\left(\frac{\partial v_y}{\partial z}+\frac{\partial v_z}{\partial y}\right)^2\right].
\end{eqnarray}
In the above, $\gamma$, $\mu$, $\kappa$, and ${\cal M}$ are treated as uniform constants. 

\section{Fluid Equations in Cylindrical Coordinates}\label{s2.19}
Let us adopt the cylindrical coordinate system, $r$, $\theta$, $z$. Making use of the results quoted
in Section~\ref{scyl}, the components of the stress tensor are
\begin{eqnarray}
\sigma_{rr} &=&-p + 2\mu\,\frac{\partial v_r}{\partial r},\\[0.5ex]
\sigma_{\theta\theta} &=&-p + 2\mu\left(\frac{1}{r}\frac{\partial v_\theta}{\partial \theta}+ \frac{v_r}{r}\right),\\[0.5ex]
\sigma_{zz} &=&-p + 2\mu\,\frac{\partial v_z}{\partial z},\\[0.5ex]
\sigma_{r\theta}=\sigma_{\theta r} &=&\mu\left(\frac{1}{r}\,\frac{\partial v_r}{\partial\theta} + \frac{\partial v_\theta}{\partial r}-\frac{v_\theta}{r}\right),\\[0.5ex]
\sigma_{rz}=\sigma_{zr} &=&\mu\left(\frac{\partial v_r}{\partial z} + \frac{\partial v_z}{\partial r}\right),\\[0.5ex]
\sigma_{\theta z} = \sigma_{z\theta} &=& \mu\left(\frac{1}{r}\,\frac{\partial v_z}{\partial\theta}+\frac{\partial v_\theta}{\partial z}\right),
\end{eqnarray}
whereas the equations of compressible fluid flow become
\begin{eqnarray}
\frac{D\rho}{Dt} &=&-\rho\,\Delta,\\[0.5ex]
\frac{Dv_r}{Dt}-\frac{v_\theta^{\,2}}{r}&=& - \frac{1}{\rho}\,\frac{\partial p}{\partial r} - \frac{\partial\Psi}{\partial r}
\nonumber\\[0.5ex]&&+ \frac{\mu}{\rho}\left(\nabla^2 v_r -\frac{v_r}{r^2}-\frac{2}{r^2}\,\frac{\partial v_\theta}{\partial\theta}+ \frac{1}{3}\,\frac{\partial\Delta}{\partial r}\right),\\[0.5ex]
\frac{Dv_\theta}{Dt}+\frac{v_r\,v_\theta}{r}&=& - \frac{1}{\rho\,r}\,\frac{\partial p}{\partial \theta} - \frac{1}{r}\frac{\partial\Psi}{\partial \theta}\nonumber\\[0.5ex]&&
+\frac{\mu}{\rho}\left(\nabla^2 v_\theta +\frac{2}{r^2}\,\frac{\partial v_r}{\partial\theta} -\frac{v_\theta}{r^2}+ \frac{1}{3r}\,\frac{\partial\Delta}{\partial \theta}\right),\\[0.5ex]
\frac{Dv_z}{Dt}&=& - \frac{1}{\rho}\,\frac{\partial p}{\partial z} - \frac{\partial\Psi}{\partial z}
+ \frac{\mu}{\rho}\left(\nabla^2 v_z + \frac{1}{3}\,\frac{\partial\Delta}{\partial z}\right),\\[0.5ex]
\frac{1}{\gamma-1}\left(\frac{D\rho}{Dt} - \frac{\gamma\,p}{\rho}\,\frac{D\rho}{Dt}\right)
&=&\chi + \frac{\kappa\,{\cal M}}{\cal R}\,\nabla^2\left(\frac{p}{\rho}\right),
\end{eqnarray}
where 
\begin{eqnarray}
\Delta&=&\frac{1}{r}\,\frac{\partial (r\,v_r)}{\partial r} +\frac{1}{r}\, \frac{\partial v_\theta}{\partial \theta} + \frac{\partial v_z}{\partial z},\\[0.5ex]
\frac{D}{Dt} &=& \frac{\partial}{\partial t} + v_r\,\frac{\partial }{\partial r} + \frac{v_\theta}{r}\,\frac{\partial}{\partial \theta} + v_z\,\frac{\partial}{\partial z},\\[0.5ex]
\nabla^2 &=&\frac{1}{r}\frac{\partial}{\partial r}\!\left(r\,\frac{\partial}{\partial r} \right)+\frac{1}{r^2}\, \frac{\partial^2}{\partial \theta^2}+\frac{\partial^2}{\partial z^2},\\[0.5ex]
\chi &=&2\mu\left[\left(\frac{\partial v_r}{\partial r}\right)^2+\left(\frac{1}{r}\,\frac{\partial v_\theta}{\partial \theta}+\frac{v_r}{r}\right)^2+
\left(\frac{\partial v_z}{\partial z}\right)^2+\frac{1}{2}\left(\frac{1}{r}\,\frac{\partial v_r}{\partial \theta}+\frac{\partial v_\theta}{\partial r}-\frac{v_\theta}{r}\right)^2\right.\nonumber\\[0.5ex]
&&\left.+\frac{1}{2}\left(\frac{\partial v_r}{\partial z}+\frac{\partial v_z}{\partial r}\right)^2+\frac{1}{2}\left(\frac{\partial v_\theta}{\partial z}+\frac{1}{r}\,\frac{\partial v_z}{\partial \theta}\right)^2\right].
\end{eqnarray}

\section{Fluid Equations in Spherical Coordinates}
Let us, finally,  adopt the spherical  coordinate system, $r$, $\theta$, $\phi$. Making use of the results quoted
in Section~\ref{ssph}, the components of the stress tensor are
\begin{eqnarray}
\sigma_{rr} &=&-p + 2\mu\,\frac{\partial v_r}{\partial r},\\[0.5ex]
\sigma_{\theta\theta} &=&-p + 2\mu\left(\frac{1}{r}\frac{\partial v_\theta}{\partial \theta}+ \frac{v_r}{r}\right),\\[0.5ex]
\sigma_{\phi\phi} &=&-p + 2\mu\left(\frac{1}{r\,\sin\theta}\,\frac{\partial v_\phi}{\partial \phi}+\frac{v_r}{r}+\frac{\cot\theta\,v_\theta}{r}\right),\\[0.5ex]
\sigma_{r\theta}=\sigma_{\theta r} &=&\mu\left(\frac{1}{r}\,\frac{\partial v_r}{\partial\theta} + \frac{\partial v_\theta}{\partial r}-\frac{v_\theta}{r}\right),\\[0.5ex]
\sigma_{r\phi}=\sigma_{\phi r} &=&\mu\left(\frac{1}{r\,\sin\theta}\,\frac{\partial v_r}{\partial \phi} + \frac{\partial v_\phi}{\partial r}-\frac{v_\phi}{r}\right),\\[0.5ex]
\sigma_{\theta \phi} = \sigma_{\phi\theta} &=& \mu\left(\frac{1}{r\,\sin\theta}\,\frac{\partial v_\theta}{\partial \phi}+\frac{1}{r}\,\frac{\partial v_\phi}{\partial\theta}-\frac{\cot\theta\,v_\phi}{r}\right),
\end{eqnarray}
whereas the equations of compressible fluid flow become
\begin{eqnarray}
\frac{D\rho}{Dt} &=&-\rho\,\Delta,\\[0.5ex]
\frac{Dv_r}{Dt}-\frac{v_\theta^{\,2}+v_\phi^{\,2}}{r}&=& - \frac{1}{\rho}\,\frac{\partial p}{\partial r} - \frac{\partial\Psi}{\partial r}
+ \frac{\mu}{\rho}\left(\nabla^2 v_r -\frac{2 v_r}{r^2}-\frac{2}{r^2}\,\frac{\partial v_\theta}{\partial\theta}\right.\nonumber\\[0.5ex]&&\left.-\frac{2\cot\theta\,v_\theta}{r^2}-\frac{2}{r^2\,\sin\theta}\,\frac{\partial v_\phi}{\partial\phi}+ \frac{1}{3}\,\frac{\partial\Delta}{\partial r}\right),\\[0.5ex]
\frac{Dv_\theta}{Dt}+\frac{v_r\,v_\theta-\cot\theta\,v_\phi^{\,2}}{r}&=& - \frac{1}{\rho\,r}\,\frac{\partial p}{\partial \theta} - \frac{1}{r}\frac{\partial\Psi}{\partial \theta}
+\frac{\mu}{\rho}\left(\nabla^2 v_\theta +\frac{2}{r^2}\,\frac{\partial v_r}{\partial\theta} \right.\\[0.5ex]
&&\left.-\frac{v_\theta}{r^2\,\sin^2\theta}-\frac{2\,\cot\theta}{r^2\,\sin\theta}\,\frac{\partial v_\phi}{\partial\phi}+ \frac{1}{3r}\,\frac{\partial\Delta}{\partial \theta}\right),\\[0.5ex]
\frac{Dv_\phi}{Dt}+\frac{v_r\,v_\phi + \cot\theta\,v_\theta\,v_\phi}{r}&=& - \frac{1}{\rho\,r\,\sin\theta}\,\frac{\partial p}{\partial \phi} - \frac{1}{r\,\sin\theta}\,\frac{\partial\Psi}{\partial \phi}
+ \frac{\mu}{\rho}\left(\nabla^2 v_\phi-\frac{v_\phi}{r^2\,\sin^2\theta} \right.\nonumber\\[0.5ex]
&&\left. + \frac{2}{r^2\,\sin^2\theta}\,\frac{\partial v_r}{\partial \phi}+ \frac{2\,\cot\theta}{r^2\,\sin\theta}\,\frac{\partial v_\theta}{\partial\phi}+ \frac{1}{3r\,\sin\theta}\,\frac{\partial\Delta}{\partial \phi}\right),\\[0.5ex]
\frac{1}{\gamma-1}\left(\frac{D\rho}{Dt} - \frac{\gamma\,p}{\rho}\,\frac{D\rho}{Dt}\right)
&=&\chi + \frac{\kappa\,{\cal M}}{\cal R}\,\nabla^2\left(\frac{p}{\rho}\right),
\end{eqnarray}
where 
\begin{eqnarray}
\Delta&=&\frac{1}{r^2}\,\frac{\partial (r^2\,v_r)}{\partial r} +\frac{1}{r\,\sin\theta}\, \frac{\partial (\sin\theta\,v_\theta)}{\partial \theta} + \frac{1}{r\,\sin\theta}\,\frac{\partial v_\phi}{\partial \phi},\\[0.5ex]
\frac{D}{Dt} &=& \frac{\partial}{\partial t} + v_r\,\frac{\partial }{\partial r} + \frac{v_\theta}{r}\,\frac{\partial}{\partial \theta} + \frac{v_\phi}{r\,\sin\theta}\,\frac{\partial}{\partial \phi},\\[0.5ex]
\nabla^2 &=&\frac{1}{r^2}\,\frac{\partial}{\partial r}\!\left(r^2\,\frac{\partial}{\partial r} \right)+\frac{1}{r^2\,\sin\theta}\, \frac{\partial}{\partial \theta}\!\left(\sin\theta\,\frac{\partial}{\partial\theta}\right)+\frac{1}{r^2\,\sin^2\theta}\,\frac{\partial^2}{\partial \phi^2},\\[0.5ex]
\chi &=&2\mu\left[\left(\frac{\partial v_r}{\partial r}\right)^2+\left(\frac{1}{r}\,\frac{\partial v_\theta}{\partial \theta}+\frac{v_r}{r}\right)^2+
\left(\frac{1}{r\,\sin\theta}\,\frac{\partial v_\phi}{\partial \phi}+\frac{v_r}{r}+\frac{\cot\theta\,v_\theta}{r}\right)^2\right.\nonumber\\[0.5ex]&&+\frac{1}{2}\left(\frac{1}{r}\,\frac{\partial v_r}{\partial \theta}+\frac{\partial v_\theta}{\partial r}-\frac{v_\theta}{r}\right)^2+\frac{1}{2}\left(\frac{1}{r\,\sin\theta}\,\frac{\partial v_r}{\partial \phi}+\frac{\partial v_\phi}{\partial r}-\frac{v_\phi}{r}\right)^2\nonumber\\[0.5ex]
&&\left.+\frac{1}{2}\left(\frac{1}{r\,\sin\theta}\,\frac{\partial v_\theta}{\partial \phi}+\frac{1}{r}\,\frac{\partial v_\phi}{\partial \theta}-\frac{\cot\theta\,v_\phi}{r}\right)^2\right].
\end{eqnarray}

\section{Exercises}
{\small 
\renewcommand{\theenumi}{2.\arabic{enumi}}
\begin{enumerate}
\item Equations (\ref{e4.68x}), (\ref{e4.68}), and (\ref{e4.89}) can be combined to give the following energy conservation equation
for a non-ideal compressible fluid:
$$
\rho\,\frac{D{\cal E}}{Dt} - \frac{p}{\rho}\,\frac{D\rho}{Dt} = \chi - \nabla\cdot{\bf q},
$$
where $\rho$ is the mass density, $p$ the pressure, ${\cal E}$ the internal energy per unit mass, $\chi$ the
viscous energy dissipation rate per unit volume, and ${\bf q}$ the heat flux density. We also have
\begin{eqnarray}
\frac{D\rho}{Dt} &=&-\rho\,\nabla\cdot{\bf v},\nonumber\\[0.5ex]
{\bf q}&=&-\kappa\,\nabla T,\nonumber
\end{eqnarray}
where ${\bf v}$ is the fluid velocity,  $T$  the temperature, and $\kappa$ the thermal conductivity. Now, according to a standard theorem  in thermodynamics,
$$
T\,d{\cal S} = d{\cal E}- \frac{p}{\rho^2}\,d\rho,
$$
where ${\cal S}$ is the entropy per unit mass. Moreover, the entropy flux density at a given point
in the fluid is
$$
{\bf s} = \rho\,{\cal S}\,{\bf v} + \frac{{\bf q}}{T},
$$
where the first term on the right-hand side is due to direct entropy convection by the fluid, and the second is the entropy
flux density associated with heat conduction. 

Derive an entropy conservation equation of the form
$$
\frac{dS}{dt} + \Phi_S = \Theta_S,
$$
where $S$ is the net amount of entropy contained in some fixed volume $V$, $\Phi_S$ the entropy flux out of $V$, and $\Theta_S$ the
net rate of entropy creation within $V$. Give expressions for $S$, $\Phi_S$, and $\Theta_S$. Demonstrate that
 the entropy creation rate per unit volume is
 $$
 \theta = \frac{\chi}{T} + \frac{{\bf q}\cdot{\bf q}}{\kappa\,T^2}.
 $$
 Finally, show that $\theta\geq 0$, in accordance with the second law of thermodynamics. 
 
 \item The Navier-Stokes equation for an incompressible fluid of uniform mass density $\rho$ takes the form
 $$
 \frac{D{\bf v}}{Dt} = -\frac{\nabla p}{\rho} - \nabla\Psi + \nu\,\nabla^2{\bf v},
 $$
 where ${\bf v}$ is the fluid velocity, $p$ the pressure, $\Psi$ the potential energy per unit mass, and $\nu$
 the (uniform) kinematic viscosity. The incompressibility constraint requires that
 $$
 \nabla\cdot{\bf v} = 0.
 $$
 Finally, the quantity
 $$
 \bomega\equiv \nabla\times {\bf v}
 $$
 is generally referred to as the fluid {\em vorticity}. 
 
 Derive the following {\em vorticity evolution equation}\/ from
 the Navier-Stokes equation:
 $$
 \frac{D\bomega}{Dt} = (\bomega\cdot\nabla)\,{\bf v} + \nu\,\nabla^2\bomega.
 $$
 
 \item Consider two-dimensional incompressible fluid flow. Let the velocity field take the form
 $$
 {\bf v} = v_x(x,y,t)\,{\bf e}_x + v_y(x,y,t)\,{\bf e}_y.
 $$
 Demonstrate that the equations of incompressible fluid flow (see Exercise~2.2) can be satisfied by writing
 \begin{eqnarray}
 v_x &=&-\frac{\partial \psi}{\partial y},\nonumber\\[0.5ex]
 v_y&=&\frac{\partial\psi}{\partial x},\nonumber
 \end{eqnarray}
 where
 $$
 \frac{\partial\omega}{\partial t}+\frac{\partial\psi}{\partial x}\,\frac{\partial \omega}{\partial y} -\frac{\partial\omega}{\partial x}\, \frac{\partial\psi}{\partial y} = \nu\,\nabla^2\omega,
 $$
 and
 $$
 \omega = \nabla^2\psi.
 $$
 Here, $\nabla^2\equiv \partial^2/\partial x^2+ \partial^2/\partial y^2$. Furthermore, the quantity $\psi$ is termed a {\em stream-function}, 
 since ${\bf v}\cdot\nabla\psi = 0$: {\em i.e.}, the fluid flow is everywhere parallel to contours of $\psi$. 
 
 \item Consider incompressible {\em irrotational}\/ flow: {\em i.e.}, flow that satisfies
 \begin{eqnarray}
 \frac{D{\bf v}}{Dt} &=& -\frac{\nabla p}{\rho} -\nabla\Psi +\nu\,\nabla^2{\bf v},\nonumber\\[0.5ex]
 \nabla\cdot{\bf v} &=& 0,\nonumber
 \end{eqnarray}
 as well as
 $$
 \nabla\times {\bf v} = {\bf 0}.
 $$
 Here, ${\bf v}$ is the fluid velocity, $\rho$ the uniform mass density, $p$ the pressure, $\Psi$ the potential energy
 per unit mass, and $\nu$ the uniform kinematic viscosity. 
 
 Demonstrate that the above equations can
 be satisfied by writing
 $$
 {\bf v}= \nabla\phi,
 $$
 where
 $$
 \nabla^2\phi = 0,
 $$
 and
 $$
 \frac{\partial\phi}{\partial t} + \frac{1}{2}\,v^2+ \frac{p}{\rho} + \Psi ={\cal C}(t).
 $$
 Here, ${\cal C}(t)$ is a spatial constant. This type of flow is known as {\em potential flow}, since the velocity
 field is derived from a scalar potential. 
 
 \item The equations of  inviscid adiabatic  ideal gas flow are
 \begin{eqnarray}
 \frac{D\rho}{Dt} &=&-\rho\,\nabla\cdot{\bf v},\nonumber\\[0.5ex]
 \frac{D{\bf v}}{Dt} &=& - \frac{\nabla p}{\rho} - \nabla\Psi,\nonumber\\[0.5ex]
 \frac{D}{Dt}\!\left(\frac{p}{\rho^\gamma}\right) &=& 0.\nonumber
 \end{eqnarray}
 Here, $\rho$ is the mass density, ${\bf v}$ the flow velocity, $p$ the pressure, $\Psi$ the
 potential energy per unit mass, and $\gamma$ the (uniform) ratio of specific heats. 
 Suppose that the pressure and potential energy are both time independent: {\em i.e.}, $\partial p/\partial t=
 \partial\Psi/\partial t=0$. 
 
 Demonstrate that
 $$
 H = \frac{1}{2}\,v^2 + \frac{\gamma}{\gamma-1}\,\frac{p}{\rho} + \Psi
 $$
 is a constant of the motion: {\em i.e.}, $DH/Dt = 0$. This result is known as {\em Bernoulli's theorem}. 
 
\item The equations of  inviscid  adiabatic non-ideal gas flow are
 \begin{eqnarray}
 \frac{D\rho}{Dt} &=&-\rho\,\nabla\cdot{\bf v},\nonumber\\[0.5ex]
 \frac{D{\bf v}}{Dt} &=& - \frac{\nabla p}{\rho} - \nabla\Psi,\nonumber\\[0.5ex]
 \frac{D{\cal E}}{Dt} - \frac{p}{\rho^2}\,\frac {D\rho}{Dt} &=&0.\nonumber
 \end{eqnarray}
 Here, $\rho$ is the mass density, ${\bf v}$ the flow velocity, $p$ the pressure, $\Psi$ the
 potential energy per unit mass, and ${\cal E}$ the internal energy per unit mass. 
 Suppose that the pressure and potential energy are both time independent: {\em i.e.}, $\partial p/\partial t=
 \partial\Psi/\partial t=0$. 
 Demonstrate that
 $$
 H = \frac{1}{2}\,v^2 + {\cal E}+ \frac{p}{\rho} + \Psi
 $$
 is a constant of the motion: {\em i.e.}, $DH/Dt = 0$. This result is a more general form of Bernoulli's theorem. 
 
 \item Demonstrate that Bernoulli's theorem for incompressible, inviscid fluid flow takes the form $DH/Dt=0$, where
 $$
 H = \frac{1}{2}\,v^2 + \frac{p}{\rho} + \Psi.
 $$

\end{enumerate}}
