\chapter{Vectors and Vector Fields}\label{vector}
\section{Introduction}
This Appendix outlines those aspects of vector algebra, vector calculus, and
vector field theory that are helpful in the study of fluid dynamics. 

\section{Scalars and Vectors}
Many physical entities ({\em e.g.}, mass, energy) are 
 entirely defined by a numerical magnitude (expressed in appropriate units). Such entities, which have no directional element, are
known as {\em scalars}. Moreover, since scalars can be represented by real numbers 
it follows that they obey the  laws of ordinary algebra. However, there exits a
second class of physical entities ({\em e.g.}, velocity, acceleration, force) that are
only completely defined when both a numerical magnitude {\em and}\/ a direction in space are specified. 
Such entities are known as {\em vectors}. By definition, a vector obeys the same algebra as
a displacement in space, and may thus be represented geometrically by a
straight-line, $\stackrel{\displaystyle \rightarrow}{PQ}$ (say), where the arrow
indicates the direction of the displacement ({\em i.e.}, from point $P$ to point $Q$). See Figure~\ref{fig1}. 
The magnitude of the vector is represented by the length of the  straight-line.

\begin{figure}
\epsfysize=1.25in
\centerline{\epsffile{AppendixA/figA.01.eps}}
\caption{\em A vector.}\label{fig1}
\end{figure}

It is conventional to denote vectors by bold-faced symbols ({\em e.g.}, ${\bf a}$, ${\bf F}$) and
scalars by non-bold-faced symbols ({\em e.g.}, $r$, $S$). The magnitude of
a general vector, ${\bf a}$, is  denoted  $|{\bf a}|$, or just $a$, and is, by definition, always
greater than or equal to zero. It is convenient to define a vector with zero magnitude---this is
denoted ${\bf 0}$, and has no direction. Finally, two vectors, ${\bf a}$ and ${\bf b}$, are said
to be equal when their magnitudes and directions are both identical. 

\begin{figure}
\epsfysize=2.25in
\centerline{\epsffile{AppendixA/figA.02.eps}}
\caption{\em Vector addition.}\label{fig2}
\end{figure}

\section{Vector Algebra}
Suppose that the displacements $\stackrel{\displaystyle \rightarrow}{PQ}$ and  $\stackrel{\displaystyle \rightarrow}{QR}$
represent the vectors ${\bf a}$ and ${\bf b}$, respectively. See Figure~\ref{fig2}. It can be seen that the result
of combining these two displacements is to give the net displacement $\stackrel{\displaystyle \rightarrow}{PR}$. Hence,
if
 $\stackrel{\displaystyle \rightarrow}{PR}$ represents the vector ${\bf c}$ then we can write
 \begin{equation}
 {\bf c} = {\bf a} + {\bf b}.
 \end{equation}
 This defines {\em vector addition}. 
 By completing the parallelogram $PQRS$, we can also see that
 \begin{equation}
 \stackrel{\displaystyle \rightarrow}{PR} \,= \, \stackrel{\displaystyle \rightarrow}{PQ}+\stackrel{\displaystyle \rightarrow}{QR}\,=\, 
  \stackrel{\displaystyle \rightarrow}{PS}+\stackrel{\displaystyle \rightarrow}{SR}.
  \end{equation}
However, $\stackrel{\displaystyle \rightarrow}{PS}$ has the same length and direction as $\stackrel{\displaystyle \rightarrow}{QR}$,
and, thus, represents the {\em same}\/ vector, ${\bf b}$. Likewise,  $\stackrel{\displaystyle \rightarrow}{PQ}$ and $\stackrel{\displaystyle \rightarrow}{SR}$ both represent the vector ${\bf a}$. Thus, the above equation is equivalent to
\begin{equation}
{\bf c} = {\bf a} + {\bf b} = {\bf b} + {\bf a}.
\end{equation}
We conclude that the addition of vectors is  {\em commutative}.  It can also
be shown that the {\em associative}\/ law  holds: {\em i.e.}, 
\begin{equation}
{\bf a} +
 ({\bf b} + {\bf c}) = ({\bf a} + {\bf b}) + {\bf c}.
 \end{equation}

The null vector, ${\bf 0}$, is represented by a displacement of zero length and arbitrary direction. 
Since the result of combining such a displacement with a  finite length displacement is the same
as the latter displacement by itself, it follows that
\begin{equation}
{\bf a} + {\bf 0} = {\bf a},
\end{equation}
where ${\bf a}$ is a general vector. 
The negative of ${\bf a}$ is defined as that vector which has the same magnitude, but  acts in the opposite direction, and is denoted $-{\bf a}$.
The sum of ${\bf a}$ and  $-{\bf a}$ is  thus
the null vector: {\em i.e.},
\begin{equation}
{\bf a} + (-{\bf a}) = {\bf 0}.
\end{equation}
We can also define the difference of two vectors, ${\bf a}$ and ${\bf b}$, as
\begin{equation}
{\bf c} = {\bf a} - {\bf b} = {\bf a} +(-{\bf b}).
\end{equation}
This definition of {\em vector subtraction}\/ is illustrated in Figure~\ref{fig2a}. 

\begin{figure}
\epsfysize=1.75in
\centerline{\epsffile{AppendixA/figA.03.eps}}
\caption{\em Vector subtraction.}\label{fig2a}
\end{figure}

If $n>0$ is a scalar then the expression $n\,{\bf a}$ denotes a vector whose direction is the same
as ${\bf a}$, and whose magnitude
is $n$ times that of ${\bf a}$. (This definition becomes obvious when $n$ is an integer.)
If $n$ is negative then, since $n\,{\bf a} = |n|\,(-{\bf a})$, it follows
that $n\,{\bf a}$ is a vector whose magnitude is $|n|$ times that of ${\bf a}$, and whose
direction is opposite to ${\bf a}$. These definitions imply that if $n$ and $m$ are
two scalars then
\begin{eqnarray}
n\,(m\,{\bf a}) &=& n\,m\,{\bf a} = m\,(n\,{\bf a}),\\[0.5ex]
(n+m)\,{\bf a} &=& n\,{\bf a} + m\,{\bf a},\\[0.5ex]
n\,({\bf a} + {\bf b}) &=& n\,{\bf a} + n\,{\bf b}.
\end{eqnarray}

\section{Cartesian Components of a Vector}
Consider a Cartesian coordinate system $Oxyz$ consisting of
an origin, $O$, and three mutually perpendicular coordinate axes, $Ox$, $Oy$, and
$Oz$. See Figure~\ref{fig2b}. Such a system is said to be {\em right-handed}\/  if, when looking along the $Oz$ direction, a $90^\circ$ clockwise
rotation about $Oz$  is required to take $Ox$ into $Oy$. Otherwise, it is said to be left-handed. It is conventional to always use  a right-handed coordinate system. 

\begin{figure}
\epsfysize=1.75in
\centerline{\epsffile{AppendixA/figA.04.eps}}
\caption{\em A right-handed Cartesian coordinate system.}\label{fig2b}
\end{figure}

It is convenient to define unit vectors, ${\bf e}_x$, ${\bf e}_y$, and ${\bf e}_z$,  parallel to $Ox$, $Oy$, and $Oz$, respectively.
Incidentally, a unit vector is a vector whose magnitude is unity. The position vector, ${\bf r}$,  of some general point $P$ whose Cartesian coordinates
are ($x$, $y$, $z$) is then given by
\begin{equation}
{\bf r} = x\,{\bf e}_z + y\,{\bf e}_y+z\,{\bf e}_z.
\end{equation}
In other words, we can get from $O$ to $P$ by moving a distance $x$ parallel to $Ox$, then a distance
$y$ parallel to $Oy$, and then a distance $z$ parallel to $Oz$. Similarly, if ${\bf a}$ is an arbitrary vector then 
\begin{equation}
{\bf a} = a_x\,{\bf e}_x+ a_y\,{\bf e}_y+a_z\,{\bf e}_z,
\end{equation}
where $a_x$, $a_y$, and $a_z$ are termed the {\em Cartesian components}\/ of ${\bf a}$. It is coventional to write ${\bf a} \equiv (a_x,\,a_y,\,a_z)$. 
It follows that ${\bf e}_x\equiv (1,\,0,\,0)$, ${\bf e}_y\equiv (0,\,1,\,0)$, and ${\bf e}_z\equiv
(0,\,0,\,1)$. Of course, ${\bf 0} \equiv (0,\,0,\,0)$. 

According to the three-dimensional generalization of the Pythagorean theorem, the distance $OP\equiv |{\bf r}|=r$ is
given by
\begin{equation}
r = \sqrt{x^2 + y^2 + z^2}.
\end{equation}
By analogy, the magnitude of a general vector ${\bf a}$ takes the form
\begin{equation}
a = \sqrt{a_x^{\,2} + a_y^{\,2} + a_z^{\,2}}.
\end{equation}

If ${\bf a}\equiv (a_x,\,a_y,\,a_z)$ and ${\bf b}\equiv (b_x,\,b_y,\,b_z)$ then it is
easily demonstrated that
\begin{equation}
{\bf a} + {\bf b} \equiv (a_x+b_x,\,a_y+b_y,\,a_z+b_z).
\end{equation}
Furthermore, if $n$ is a scalar then it is apparent that
\begin{equation}
n\,{\bf a} \equiv (n\,a_x,\,n\,a_y,\,n\,a_z).
\end{equation}

\section{Coordinate Transformations}\label{stransx}
A Cartesian coordinate system allows  position and direction in space to be represented in a very convenient
manner. Unfortunately, such a coordinate system also introduces arbitrary elements into our analysis. After all, two independent observers might well choose Cartesian coordinate systems with different origins, and
different orientations of the coordinate axes. In general, a given vector ${\bf a}$ will have different
sets of components in these two coordinate systems. However, the direction and magnitude of ${\bf a}$ are the
same in both cases. Hence, the two sets of components must be related to one another in a very particular fashion.  
Actually, since vectors are represented by {\em moveable}\/ line elements in space ({\em i.e.},  in Figure~\ref{fig2},  $\stackrel{\displaystyle \rightarrow}{PQ}$ and  $\stackrel{\displaystyle \rightarrow}{SR}$ represent the {\em same}\/ vector), it follows that
the components of a general vector are not affected by a simple shift in the origin of a Cartesian coordinate system. On the other hand, the
components are modified when the coordinate axes are rotated. 

Suppose that we transform to a new coordinate system, $Ox'y'z'$, which has the
same origin as $Oxyz$, and is obtained by 
rotating the coordinate axes of $Oxyz$ through an angle
$\theta$ about  $Oz$. See Figure~\ref{f3}.
Let the coordinates of a general point $P$ be $(x,\,y,\,z)$ in $Oxyz$ and $(x',\, y',\, z')$ in $Ox'y'z'$. According to simple trigonometry, these two sets of coordinates are related to one another via the transformation 
\begin{eqnarray}
x' &=& x\cos\theta + y\sin \theta,\label{t1}\\[0.5ex]
y' &=& -x\sin\theta + y\cos\theta,\label{t2}\\[0.5ex]
z' &=& z.\label{t3}
\end{eqnarray}
Consider the vector displacement ${\bf r} \equiv\, \stackrel{\displaystyle \rightarrow}{OP}$. 
Note that this displacement is represented by the {\em same}\/ symbol, ${\bf r}$, in both
coordinate systems, since the magnitude and
direction of ${\bf r}$ are manifestly {\em independent}\/ of the orientation of the coordinate axes. The
coordinates of ${\bf r}$ {\em do}\/ depend on the orientation of the axes: {\em i.e.}, ${\bf r}\equiv (x,\,y,\,z)$ in $Oxyz$, and
${\bf r}\equiv (x',\,y',\,z')$ in $Ox'y'z'$. 
However, they must depend in a very specific manner [{\em i.e.}, Equations~(\ref{t1})--(\ref{t3})] which
preserves the magnitude and direction of ${\bf r}$.

\begin{figure}
\epsfysize=1.75in
\centerline{\epsffile{AppendixA/figA.05.eps}}
\caption{\em Rotation of the coordinate axes about $Oz$.}\label{f3}
\end{figure}

The 
components of a general vector ${\bf a}$  transform in an analogous
manner to Equations~(\ref{t1})--(\ref{t3}): {\em i.e.}, 
\begin{eqnarray}
a_{x'} &=& a_x\cos\theta + a_y\sin \theta,\label{tt1}\\[0.5ex]
a_{y'} &=& -a_x\sin\theta + a_y\cos\theta,\\[0.5ex]
a_{z'} &=& a_z.\label{tt3}
\end{eqnarray}
Moreover, there are similar transformation rules for rotation about $Ox$ and $Oy$.
Equations~(\ref{tt1})--(\ref{tt3}) effectively constitute the {\em definition}\/ of a vector: {\em i.e.}, 
 the three
quantities ($a_x,\, a_y,\, a_z$) are the components of a vector provided that
they transform under rotation of the coordinate axes about $Oz$ in accordance with Equations~(\ref{tt1})--(\ref{tt3}). (And also transform
correctly under rotation about $Ox$ and $Oy$).
Conversely, ($a_x,\, a_y,\, a_z$) {\em cannot}\/ be the components of a vector if they
do not transform in accordance with Equations~(\ref{tt1})--(\ref{tt3}). Of course, scalar quantities are {\em invariant}\/
 under rotation of the coordinate axes.
Thus, the individual components of a vector ($a_x$, say) are real numbers, but 
they are
{\em not}\/ scalars.
Displacement vectors, and all vectors derived from
displacements ({\em e.g.}, velocity, acceleration), automatically satisfy Equations~(\ref{tt1})--(\ref{tt3}). There are, however, other
physical quantities that  have both magnitude and direction, but which are not
obviously related to displacements. We need to check carefully to see whether these
quantities are really vectors (see Sections~\ref{sect22} and \ref{srotn}). 


\section{Scalar Product}\label{sscalar}
A scalar quantity is invariant under all possible rotational transformations.
The individual components of a vector are not scalars because they change under
transformation. Can we form a scalar out of some combination of the components
of one, or more, vectors? Suppose that we were to define the
``percent''  product,
\begin{equation}
{\bf a}\,\%\,{\bf b} \equiv a_x \,b_z + a_y \,b_x + a_z \,b_y = {\rm scalar~number},
\end{equation}
for general vectors ${\bf a}$ and ${\bf b}$. Is ${\bf a}\,\%\,{\bf b}$ 
invariant under transformation, as must be the case if it is a scalar number?
Let us consider an example. Suppose that ${\bf a} \equiv (0,\,1,\,0)$ and
${\bf b} \equiv (1,\,0,\,0)$. It is easily seen that ${\bf a}\,\%\,{\bf b}= 1$. Let
us now rotate the coordinate axes through $45^\circ$ about  $Oz$. In the new
coordinate system, ${\bf a} \equiv (1/\sqrt{2},\, 1/\sqrt{2},\, 0)$ and ${\bf b} \equiv (1/\sqrt{2},\,-
1/\sqrt{2},\, 0)$, giving ${\bf a}\,\%\,{\bf b} = 1/2$. Clearly, ${\bf a}\,\%\,{\bf b}$
is {\em not} invariant under rotational transformation, so 
the above definition is a bad one.

Consider, now, 
the {\em dot product}\/ or {\em scalar product}:
\begin{equation}
{\bf a} \cdot {\bf b} \equiv a_x \,b_x + a_y \,b_y + a_z \,b_z =  {\rm scalar~number}.
\end{equation}
Let us rotate the coordinate axes though $\theta$ degrees about $Oz$. According to
Equations~(\ref{tt1})--(\ref{tt3}),  ${\bf a} \cdot {\bf b}$ takes the form
\begin{eqnarray}
 {\bf a} \cdot {\bf b} &= &(a_x\, \cos\theta+a_y\,\sin\theta)\,(b_x\,\cos\theta + b_y\,\sin\theta)\nonumber\\[0.5ex]
&&+(-a_x\,\sin\theta + a_y\,\cos\theta)\,(-b_x\,\sin \theta + b_y\,\cos\theta)
+a_z\, b_z\nonumber\\[0.5ex]
&= &a_x\, b_x + a_y\, b_y + a_z\, b_z
\end{eqnarray}
in the new coordinate system.
Thus, 
${\bf a} \cdot {\bf b}$ {\em is}\/ invariant under rotation about $Oz$. It can easily
be shown that it is also invariant under rotation about $Ox$ and $Oy$.
We conclude that ${\bf a} \cdot {\bf b}$ is a true scalar, and that the above definition is
a good one. Incidentally, ${\bf a} \cdot {\bf b}$ is the {\em only}\/
simple  combination of
the components of two vectors which transforms like a scalar. It is readily 
shown that the dot product  is commutative  and distributive: {\em i.e.},
\begin{eqnarray}
{\bf a} \cdot {\bf b} &=& {\bf b} \cdot {\bf a},\nonumber\\[0.5ex] 
{\bf a}\cdot({\bf b}+{\bf c}) &=& {\bf a} \cdot {\bf b} + {\bf a}\cdot {\bf c}.
\end{eqnarray}
The associative property is meaningless for the dot product, because we cannot
have $({\bf a}\cdot{\bf b}) \cdot{\bf c}$, since  ${\bf a}\cdot{\bf b}$ is scalar.

We have shown that the dot product ${\bf a} \cdot {\bf b}$ is coordinate independent.
But what is the geometric significance of this? Well, in the special case
where ${\bf a} = {\bf b}$, we get
\begin{equation}
{\bf a} \cdot {\bf b} = a_x^{~2}+a_y^{~2} + a_z^{~2} = |{\bf a}|^2=a^2.
\end{equation}
So, the invariance of ${\bf a} \cdot {\bf a}$ is equivalent to the invariance 
of the magnitude of  vector ${\bf a}$ under transformation. 

Let us now investigate the general case. The length squared of $AB$ in the
vector triangle shown in Figure~\ref{f5} is
\begin{equation}
({\bf b} - {\bf a} ) \cdot ({\bf b} - {\bf a} ) = |{\bf a}|^2 + |{\bf b}|^2 - 2\,{\bf a} \cdot
{\bf b}.
\end{equation}
However, according to the ``cosine rule'' of trigonometry,
\begin{equation}
(AB)^2 = (OA)^2 + (OB)^2 - 2 \,(OA)\,(OB)\,\cos\theta,
\end{equation}
where $(AB)$ denotes the length of side $AB$. It follows that
\begin{equation}
{\bf a} \cdot {\bf b} = |{\bf a}|\, |{\bf b}|\, \cos\theta.
\end{equation}
In this case, the invariance of ${\bf a} \cdot {\bf b}$ under transformation is equivalent
to the invariance of the angle subtended between the two vectors. Note that
if ${\bf a} \cdot {\bf b} =0$ then either $|{\bf a}|=0$, $|{\bf b}|=0$, or the vectors
$\bf a$ and $\bf b$ are mutually perpendicular. The angle subtended between two vectors
can easily be obtained from the dot product: {\em i.e.}, 
\begin{equation}
\cos\theta = \frac{{\bf a} \cdot {\bf b}}{|{\bf a}| \,|{\bf b}| }.
\end{equation}

\begin{figure}
\epsfysize=1.5in
\centerline{\epsffile{AppendixA/figA.06.eps}}
\caption{\em A vector triangle.}\label{f5}
\end{figure}

The work $W$ performed by a constant force $\bf F$ which moves an object through a displacement $\bf r$
is the product of the magnitude of $\bf F$ times the displacement in the direction
of $\bf F$. If the angle subtended between  $\bf F$ and $\bf r$ is $\theta$ then
\begin{equation}
W = |{\bf F}| \,(|{\bf r}|\,\cos\theta) = {\bf F}\cdot {\bf r}.
\end{equation}

The work $dW$ performed by a non-constant force ${\bf f}$ which moves an
object through an infinitesimal displacement $d{\bf r}$ in a time interval $dt$ is $dW={\bf f}\cdot d{\bf r}$. Thus, the rate at which the force does
work on the object, which is usually referred to as the power, is
$P = dW/dt = {\bf f}\cdot d{\bf r}/dt$, or $P={\bf f}\cdot {\bf v}$,
where ${\bf v} = d{\bf r}/dt$ is the object's instantaneous velocity.

\section{Vector Area}\label{sect22}
Suppose that we have planar surface of scalar area $S$. We can define a vector
area ${\bf S}$ whose magnitude is $S$, and whose direction is perpendicular to
the plane, in the sense determined  by  a right-hand circulation rule (see Section~\ref{svecp}) applied to  the rim, assuming that a direction of circulation around the rim is specified. See Figure~\ref{f4}.
This quantity clearly possesses both magnitude and direction. But is it a true
vector? We know that if the normal to the surface makes an angle $\alpha_x$ with
the $x$-axis then the area seen looking along the $x$-direction is $S\,\cos\alpha_x$.
This is the $x$-component of ${\bf S}$ (since $S_x={\bf e}_x\cdot {\bf S}={\bf e}_x\cdot{\bf n}\,S=\cos\alpha_x\,S$,
where ${\bf n}$ is the unit normal to the surface).
Similarly, if the normal makes an angle $\alpha_y$ with the $y$-axis then the
area seen looking along the $y$-direction is $S\,\cos\alpha_y$. 
This is the $y$-component of ${\bf S}$. If we limit ourselves to a surface whose
normal is perpendicular to the $z$-direction then $\alpha_x = \pi/2-\alpha_y=\alpha$.
It follows that ${\bf S} = S\,(\cos\alpha,\, \sin\alpha,\, 0)$. If we rotate the
basis about the
$z$-axis by $\theta$ degrees, which is equivalent to rotating the normal to
the surface about the $z$-axis by $-\theta$ degrees, so that $\alpha\rightarrow\alpha-\theta$, then
\begin{equation}\label{e29}
S_{x'} = S\,\cos\,(\alpha-\theta) = S\,\cos\alpha\,\cos\theta + S\,\sin\alpha\,\sin\theta
= S_x\,\cos\theta + S_y\,\sin\theta,
\end{equation}
which is the correct transformation rule for the $x$-component of a vector. The
other components transform correctly as well. This  proves both
that a vector area is a
true vector and that the components of a vector area are the projected areas seen looking
down the coordinate axes. 
\begin{figure}
\epsfysize=1.5in
\centerline{\epsffile{AppendixA/figA.07.eps}}
\caption{\em A vector area.}\label{f4}
\end{figure}

According to the vector addition theorem, the projected area of two plane surfaces,
joined together at a line, 
looking along the $x$-direction (say) is the $x$-component of the resultant of the vector areas of the two surfaces.
Likewise, for many joined-up plane areas, the net area seen looking down the $x$-axis,
which is the same as the  area of the outer {\em rim}\/ seen looking down the $x$-axis, is the
$x$-component  of the resultant of all the vector areas: {\em i.e.}, 
\begin{equation}
{\bf S} = \sum_i {\bf S}_i.
\end{equation}
If we approach a limit,
by letting the number of plane facets increase, and their areas reduce, then we
obtain a continuous surface denoted by the resultant vector area
\begin{equation}
 {\bf S} = \sum_i \delta  {\bf S}_i.
\end{equation}
It is
clear that the area of the rim seen looking down the $x$-axis is just $S_x$. Similarly, for the areas
of the rim seen looking down the other coordinate axes. 
Note that it is the rim of the surface that determines the vector area rather than the nature of
the surface spanning the rim. So, two different surfaces sharing the same rim both possess the {\em same}\/
vector area. 

In conclusion, a loop (not all in one plane) has a vector area ${\bf S}$ which
is the resultant of the component vector areas of any surface ending on the loop. The
components  of ${\bf S}$ are the areas of the loop seen looking down the coordinate axes. As a corollary, a closed surface has ${\bf S} = {\bf 0}$,
since it does not possess a rim. 


\section{Vector Product}\label{svecp}
We have discovered how to construct a {\em scalar}\/ from the components of two
general vectors, $\bf a$ and $\bf b$. Can we also construct a {\em vector}\/ which is not
just a linear combination of $\bf a$ and $\bf b$? Consider the following definition:
\begin{equation}
{\bf a} \ast{\bf b} \equiv (a_x \,b_x,\, a_y\, b_y,\, a_z \,b_z).
\end{equation}
Is ${\bf a} \ast {\bf b}$ a proper vector? Suppose  that ${\bf a} =
(0,\,1,\,0)$, ${\bf b} = (1,\,0,\,0)$. In this case,  ${\bf a} \ast {\bf b}= {\bf 0}$.
However, if we rotate the coordinate axes through $45^\circ$ about $Oz$ then
${\bf a} = (1/\sqrt{2},\, 1/\sqrt{2},\, 0)$, ${\bf b} = (1/\sqrt{2},\, -1/\sqrt{2},\,0)$,
and ${\bf a}\ast {\bf b} = (1/2,\, -1/2,\,0)$. Thus, ${\bf a}\ast {\bf b}$ does
not transform like a vector, because its magnitude depends on the choice of axes.
So, above definition is a bad one.

Consider, now, the {\em cross product}\/ or {\em vector product}:
\begin{equation}
{\bf a}\times{\bf b} \equiv (a_y \, b_z-a_z\, b_y,\, a_z\, b_x - a_x\, b_z,\, a_x\, b_y - a_y\, b_x)
={\bf c}.
\end{equation}
Does this rather unlikely combination transform like a vector? Let us try
rotating the coordinate axes through an angle $\theta$  about $Oz$ using Equations~(\ref{tt1})--(\ref{tt3}).
In the new coordinate system,
\begin{eqnarray}
c_{x'} &= &(-a_x\, \sin\theta + a_y\,\cos\theta)\,b_z - a_z\,(-b_x\, \sin\theta + b_y\,\cos\theta)
\nonumber\\[0.5ex]
&=& (a_y\, b_z - a_z\, b_y)\, \cos\theta + (a_z\, b_x-a_x\, b_z)\,\sin\theta\nonumber\\[0.5ex]
& =& c_x\,\cos\theta
+c_y\,\sin\theta.
\end{eqnarray}
Thus, the $x$-component of ${\bf a}\times{\bf b}$ transforms correctly. It can
easily  be shown that the other components transform correctly as well, and that
all components also transform correctly under rotation about $Ox$ and $Oy$. 
Thus, ${\bf a}\times {\bf b}$ is a proper vector. Incidentally, ${\bf a}\times {\bf b}$
is the {\em only}\/ simple combination of the components of two vectors that transforms
like a vector (which is non-coplanar with ${\bf a}$ and ${\bf b}$).
 The cross product is 
{\em anticommutative},
\begin{equation}
{\bf a}\times{\bf b} = - {\bf b} \times{\bf a},
\end{equation}
distributive,
\begin{equation}
{\bf a}\times({\bf b} +{\bf c})=  {\bf a} \times{\bf b}+{\bf a}\times{\bf c},
\end{equation}
but is {\em not}\/ associative,
\begin{equation}
{\bf a}\times({\bf b} \times{\bf c})\neq ({\bf a}\times{\bf b}) \times{\bf c}.
\end{equation}

The cross product  transforms like a vector, which
means that it must have  a well-defined direction and magnitude. We can show
that ${\bf a}\times{\bf b}$ is {\em perpendicular}\/ to both ${\bf a}$ and ${\bf b}$.
Consider ${\bf a}\cdot {\bf a}\times{\bf b}$. If this is zero then the cross product
must be perpendicular to ${\bf a}$. Now,
\begin{eqnarray}
{\bf a}\cdot {\bf a}\times{\bf b} &= &a_x\,(a_y\, b_z-a_z\, b_y) + a_y\, (a_z\, b_x- a_x \,b_z)
+a_z\,(a_x \,b_y - a_y\, b_x)\nonumber\\[0.5ex]
&=& 0.
\end{eqnarray}
Therefore, ${\bf a}\times {\bf b}$ is perpendicular to ${\bf a}$. Likewise, it can
be demonstrated that ${\bf a}\times{\bf b}$ is perpendicular to ${\bf b}$. 
The vectors $\bf a$, $\bf b$, and ${\bf a}\times{\bf b}$ form a {\em right-handed}\/
set, like the unit vectors ${\bf e}_x$, ${\bf e}_y$, and ${\bf e}_z$. In fact,  ${\bf e}_x\times
{\bf e}_y={\bf e}_z$. This defines a unique direction for ${\bf a}\times{\bf b}$, which
is obtained from a right-hand rule. See Figure~\ref{f6}.

\begin{figure}
\epsfysize=1.75in
\centerline{\epsffile{AppendixA/figA.08.eps}}
\caption{\em The right-hand rule for cross products. Here, $\theta$ is less that $180^\circ$.}\label{f6}
\end{figure}

Let us now evaluate the magnitude of ${\bf a}\times {\bf b}$. We have
\begin{eqnarray}
({\bf a}\times{\bf b})^2 &=& (a_y \,b_z-a_z\, b_y)^2 +(a_z \,b_x - a_x\, b_z)^2 +(a_x \,b_y
-a_y \,b_x)^2\nonumber\\[0.5ex]
&=& (a_x^{~2}+a_y^{~2}+a_z^{~2})\,(b_x^{~2}+b_y^{~2}+b_z^{~2}) -
(a_x\, b_x + a_y \,b_y + a_z\, b_z)^2\nonumber\\[0.5ex]
&=& |{\bf a}|^2 \,|{\bf b}|^2 - ({\bf a}\cdot {\bf b})^2\nonumber\\[0.5ex]
&=& |{\bf a}|^2 \,|{\bf b}|^2 - |{\bf a}|^2 \,|{\bf b}|^2 \,\cos^2\theta = |{\bf a}|^2\,|{\bf b}|^2\, \sin^2\theta.
\end{eqnarray}
Thus,
\begin{equation}
|{\bf a}\times{\bf b}| = |{\bf a}|\,|{\bf b}|\,\sin\theta,
\end{equation}
where $\theta$ is the angle subtended between ${\bf a}$ and ${\bf b}$.
Clearly, ${\bf a}\times{\bf a} = {\bf 0}$ for any vector, since $\theta$ is always
zero in this case. Also, if ${\bf a}\times{\bf b} = {\bf 0}$ then either
$|{\bf a}|=0$, $|{\bf b}|=0$, or ${\bf b}$ is parallel (or antiparallel) to ${\bf a}$.

Consider the parallelogram defined by the vectors ${\bf a}$ and ${\bf b}$. See Figure~\ref{f7}.
The {\em scalar area}\/ of the parallelogram is $a\,b \sin\theta$. By convention, the {\em vector area}\/ has the magnitude of the
scalar area, and is normal to the plane of the parallelogram, in the sense obtained from a right-hand circulation rule by rotating ${\bf a}$ on to
${\bf b}$ (through an acute angle): {\em i.e.}, if the fingers of the right-hand circulate in the direction of
rotation then the thumb of the right-hand indicates the direction of the vector area. So, the vector area is coming out of the
page in Figure~\ref{f7}.
It follows that
\begin{equation}
{\bf S} = {\bf a}\times {\bf b},
\end{equation}

\begin{figure}[b]
\epsfysize=1.25in
\centerline{\epsffile{AppendixA/figA.09.eps}}
\caption{\em A vector parallelogram.}\label{f7}
\end{figure}

Suppose that a force ${\bf F}$ is applied at position ${\bf r}$. See Figure~\ref{f8}.
The torque about the origin $O$ is the product of the magnitude of the force and
the length of the lever arm $OQ$. Thus, the magnitude of the torque is
$|{\bf F}|\,|{\bf r}|\,\sin\theta$. The direction of the torque is conventionally defined as the direction of
the axis through $O$ about which the force tries to rotate objects, in the sense
determined by a right-hand circulation rule. Hence, the torque is out of the page in Figure~\ref{f8}.
 It follows that the vector torque is
given by
\begin{equation}\label{etorque}
\btau = {\bf r}\times{\bf F}.
\end{equation}

\begin{figure}
\epsfysize=2.25in
\centerline{\epsffile{AppendixA/figA.10.eps}}
\caption{\em A torque.}\label{f8}
\end{figure}

The angular momentum, ${\bf l}$, of a particle of linear momentum ${\bf p}$ and position vector ${\bf r}$ is simply defined as the moment of its
momentum about the origin: {\em i.e.}, 
\begin{equation}
{\bf l}={\bf r}\times {\bf p}.
\end{equation}

\section{Rotation}\label{srotn}
Let us try to define a rotation vector \mbox{\boldmath$\theta$} whose magnitude
is the angle of the rotation, $\theta$, and whose direction is parallel to the axis of 
rotation, in the sense determined by a right-hand circulation rule. Unfortunately, this is not a good vector. The problem is that the addition of rotations
is not commutative, whereas vector addition is commuative. 
Figure~\ref{f9} shows the effect of applying two successive $90^\circ$ rotations,
one about $Ox$, and the other about the $Oz$, to a standard six-sided die. In the
left-hand case, the $z$-rotation is applied before the $x$-rotation, and {\em vice
versa} in the right-hand case. It can be seen that the die ends up in two completely
different states. In other words, the $z$-rotation plus the
$x$-rotation does not equal
the  $x$-rotation plus the $z$-rotation. This non-commuting algebra cannot be
represented by vectors. So, although rotations have a well-defined magnitude and
direction, they are {\em not}\/ vector quantities. 
\begin{figure}
\epsfysize=3in
\centerline{\epsffile{AppendixA/figA.11.eps}}
\caption{\em Effect of successive rotations about perpendicular axes on a six-sided die.}\label{f9}
\end{figure}

But, this is not quite the end of the story. Suppose that we take a general vector
$\bf a$ and rotate it about $Oz$ by a {\em small}\/ angle $\delta \theta_z$.
This is equivalent to rotating the coordinate axes about the $Oz$ by $-\delta\theta_z$.
According to Equations~(\ref{tt1})--(\ref{tt3}), we have
\begin{equation}
{\bf a}' \simeq {\bf a} +\delta\theta_z \,{\bf e}_z\times {\bf a},
\end{equation}
where use has been made of the small angle approximations $\sin\theta \simeq \theta$
and $\cos\theta\simeq 1$. The above equation can easily be generalized to allow
small rotations about $Ox$ and $Oy$ by $\delta \theta_x$ and $\delta\theta_y$,
respectively. We find that
\begin{equation}\label{e2.41z}
{\bf a}' \simeq {\bf a} + \delta \mbox{\boldmath$\theta$}\times {\bf a},
\end{equation}
where
\begin{equation}
\delta\mbox{\boldmath$\theta$} = \delta\theta_x \,{\bf e}_x + \delta\theta_y \,{\bf e}_y + 
 \delta\theta_z \,{\bf e}_z.
\end{equation}
Clearly, we can define a rotation vector, \mbox{$\delta$\boldmath$\theta$}, but it only
works for {\em small}\/ angle rotations ({\em i.e.}, sufficiently small that the small
angle approximations of sine and cosine are good). According to the above equation,
a small $z$-rotation plus a small $x$-rotation is (approximately) equal to 
the two rotations applied in the opposite order. 
The fact that infinitesimal rotation is a vector implies that angular velocity,
\begin{equation}
\mbox{\boldmath$\omega$} = \lim_{\delta t\rightarrow 0} \frac{\delta 
\mbox{\boldmath$\theta$} }{\delta t},
\end{equation}
must be a vector as well. Also, if ${\bf a}'$ is interpreted as ${\bf a}(t+\delta t)$
in Equation~(\ref{e2.41z}) then it follows that the equation of motion of a vector
which precesses  about the origin with some angular velocity \mbox{\boldmath$\omega$} is
\begin{equation}
\frac{d {\bf a}}{dt} = \mbox{\boldmath$\omega$}\times {\bf a}.\label{e239}
\end{equation}

\section{Scalar Triple Product}\label{sstp}
Consider three vectors ${\bf a}$, ${\bf b}$, and ${\bf c}$. The {\em scalar triple product}\/ is
defined ${\bf a}\cdot {\bf b}\times {\bf c}$. Now, ${\bf b}\times {\bf c}$ is the vector area of
the parallelogram defined by ${\bf b}$ and ${\bf c}$. So, ${\bf a}\cdot {\bf b} \times{\bf c}$
is the scalar area of this parallelogram multiplied by the component of ${\bf a}$ in the direction
of its  normal. It follows that ${\bf a}\cdot {\bf b}\times{\bf c}$ is
the {\em volume}\/ of the parallelepiped defined by vectors ${\bf a}$, ${\bf b}$, and ${\bf c}$. See Figure~\ref{f9a}.
This volume is independent of how the triple product is formed from ${\bf a}$, ${\bf b}$, 
and ${\bf c}$, except that 
\begin{equation}
{\bf a} \cdot {\bf b}\times{\bf c} = - {\bf a} \cdot {\bf c}\times{\bf b}.
\end{equation}
So, the ``volume'' is positive if ${\bf a}$, ${\bf b}$, and ${\bf c}$ form a right-handed set
({\em i.e.}, if ${\bf a}$ lies above the plane of ${\bf b}$ and ${\bf c}$,
in the sense determined from a right-hand circulation rule by rotating
${\bf b}$ onto ${\bf c}$) and negative if they form a left-handed set. 
The triple product is unchanged if the dot and cross product operators are interchanged,
\begin{equation}
{\bf a} \cdot {\bf b}\times{\bf c} = {\bf a} \times{\bf b} \cdot{\bf c}.
\end{equation}
The triple product is also invariant under any cyclic permutation of ${\bf a}$, ${\bf b}$,
and ${\bf c}$,
\begin{equation}
{\bf a} \cdot {\bf b} \times{\bf c} = {\bf b} \cdot {\bf c} \times{\bf a} = 
{\bf c} \cdot {\bf a} \times{\bf b},
\end{equation}
but any anti-cyclic permutation causes it to change sign,
\begin{equation}
{\bf a} \cdot  {\bf b} \times{\bf c} = - {\bf b} \cdot {\bf a} \times{\bf c}.
\end{equation}
The scalar triple product is zero if any
two of ${\bf a}$, ${\bf b}$, and ${\bf c}$ are parallel, or if ${\bf a}$, ${\bf b}$, and ${\bf c}$
are coplanar. 
\begin{figure}
\epsfysize=1.5in
\centerline{\epsffile{AppendixA/figA.12.eps}}
\caption{\em A vector parallelepiped.}\label{f9a}
\end{figure}


If ${\bf a}$, ${\bf b}$, and ${\bf c}$ are non-coplanar then any vector ${\bf r}$ can be
written in terms of them: {\em i.e.}, 
\begin{equation}
{\bf r} = \alpha \,{\bf a} + \beta\,{\bf b} + \gamma\, {\bf c}.
\end{equation}
Forming the dot product of this equation with ${\bf b}\times{\bf c}$, we then obtain
\begin{equation}
{\bf r}  \cdot {\bf b} \times{\bf c} = \alpha\, {\bf a}\cdot{\bf b} \times{\bf c},
\end{equation}
so
\begin{equation}
\alpha = \frac{{\bf r}\cdot{\bf b}\times{\bf c}}{{\bf a}\cdot{\bf b}\times{\bf c}}.
\end{equation}
Analogous expressions can be written for $\beta$ and $\gamma$. The parameters $\alpha$, $\beta$,
and $\gamma$ are uniquely determined provided $ {\bf a}\cdot{\bf b} \times{\bf c} \neq 0$:
{\em i.e.}, provided that the three vectors are non-coplanar. 

\section{Vector Triple Product}\label{svtp}
For three vectors ${\bf a}$, ${\bf b}$, and ${\bf c}$, the {\em vector triple product}\/ is defined
${\bf a}\times({\bf b} \times {\bf c})$. The brackets are important because
${\bf a}\times({\bf b} \times {\bf c}) \neq ({\bf a}\times{\bf b} )\times {\bf c}$. 
In fact, it can be demonstrated that
\begin{equation}
{\bf a}\times({\bf b} \times {\bf c}) \equiv ({\bf a}\cdot {\bf c}) \,{\bf b} - ({\bf a}\cdot
{\bf b})\,{\bf c}\label{e247}
\end{equation}
and
\begin{equation}
({\bf a}\times{\bf b}) \times {\bf c}\equiv ({\bf a}\cdot {\bf c}) \,{\bf b} - ({\bf b}\cdot
{\bf c})\,{\bf a}.
\end{equation}

Let us try to prove the first of the above theorems. The left-hand side and the
right-hand side are both proper vectors, so if we can prove this result in one particular
coordinate system then it must be true in general. Let us take convenient axes such that 
$Ox$ lies along ${\bf b}$, and ${\bf c}$ lies in the $x$-$y$ plane. It follows that
${\bf b} \equiv (b_x,\,0,\,0)$, ${\bf c} \equiv (c_x,\, c_y,\, 0)$, and ${\bf a} \equiv (a_x,\, a_y,\, a_z)$. 
The vector ${\bf b}\times {\bf c}$ is directed along $Oz$: {\em i.e.}, 
${\bf b}\times{\bf c} \equiv (0,\,0,\,b_x\, c_y)$. Hence, ${\bf a}\times({\bf b}\times{\bf c})$
lies in the $x$-$y$ plane:  {\em i.e.}, ${\bf a}\times({\bf b}\times{\bf c}) \equiv (a_y \,b_x\, c_y,\, -a_x\, b_x\, c_y,\, 0)$.
This is the left-hand side of Equation~(\ref{e247}) in our convenient coordinate system. To evaluate the right-hand side,
we need ${\bf a}\cdot {\bf c} = a_x \,c_x + a_y\, c_y$ and ${\bf a}\cdot {\bf b} = a_x \,b_x$.
It follows that the right-hand side is 
\begin{eqnarray}
{\rm RHS} &= &(\,[a_x\, c_x + a_y \,c_y]\, b_x,\, 0,\, 0) - (a_x \,b_x \,c_x,\, a_x\, b_x\, c_y,\, 0)\nonumber\\[0.5ex]
&=& (a_y\, c_y\, b_x,\, -a_x\, b_x\, c_y,\, 0 ) = {\rm LHS},
\end{eqnarray}
which proves the theorem.

\section{Vector Calculus}\label{s28}
Suppose that  vector ${\bf a}$ varies with time, so that ${\bf a} = {\bf a} (t)$. The time
derivative of the vector is defined
\begin{equation}
\frac{d {\bf a}}{dt} = \lim_{\delta t\rightarrow 0} \left[\frac{{\bf a}(t+\delta t) - {\bf a}(t)}
{\delta t}\right].
\end{equation}
When written out in component form this becomes
\begin{equation}
\frac{d {\bf a}}{dt} \equiv \left(\frac{d a_x}{dt}, \frac{d a_y}{d t}, \frac{d a_z}{ d t}\right).
\end{equation} 

Suppose that ${\bf a}$ is, in fact, the product of a scalar $\phi(t)$ and another vector
${\bf b}(t)$. What now is the time derivative of ${\bf a}$? We have
\begin{equation}
\frac{d a_x}{dt} = \frac{d}{dt}\!\left(\phi\, b_x\right) = \frac{d\phi}{dt}\, b_x + \phi \,
\frac{d b_x}{dt},
\end{equation}
which implies that
\begin{equation}
\frac{d {\bf a}}{dt} = \frac{d\phi}{dt}\, {\bf b} + \phi\, \frac{d {\bf b}}{dt}.
\end{equation}
Moreover, it is easily demonstrated that 
\begin{equation}
\frac{d}{dt}\left({\bf a}\cdot{\bf b}\right) = \frac{d{\bf a}}{dt}\cdot {\bf b} +{\bf a}\cdot\frac{d{\bf b}}{dt},
\end{equation}
and
\begin{equation}
\frac{d}{dt}\left({\bf a}\times{\bf b}\right) = \frac{d{\bf a}}{dt}\times{\bf b} + {\bf a}\times
\frac{d{\bf b}}{dt}.
\end{equation}
Hence, it can be seen that the laws of vector differentiation are analogous to those in 
conventional calculus.

\section{Line Integrals}
Consider a two-dimensional function $f(x,y)$ which is defined for all $x$ and $y$. 
What is meant by the integral of $f$ along a given curve joining the points $P$ and $Q$ in the $x$-$y$ plane?
Well, we first draw out $f$ as a function of length $l$ along the path. See Figure~\ref{f10}. The integral is then simply given
by
\begin{equation}
\int_{P}^Q f(x,y)\,dl = {\rm Area~under~the~curve},
\end{equation}
where $dl=\sqrt{dx^2+dy^2}$. 

\begin{figure}
\epsfysize=2.in
\centerline{\epsffile{AppendixA/figA.13.eps}}
\caption{\em A line integral.}\label{f10}
\end{figure}

As an example of this, consider the integral of $f(x,y)= x\,y^2$ between $P$ and $Q$ along the
two routes indicated in Figure~\ref{f11}.
Along route 1 we have $x=y$, so $dl= \sqrt{2}\, dx$. Thus,
\begin{equation}
\int_P^Q x\,y^2\,dl = \int_0^1 x^3\,\sqrt{2} \,dx = \frac{\sqrt{2}}{4}.
\end{equation}
The integration along route 2 gives
\begin{eqnarray}
\int_P^Q x\,y^2\,dl &=& \left.\int_0^1 x\,y^2\,dx\right|_{y=0}+\left.\int_0^1 x\,y^2 \,dy
\right|_{x=1}\nonumber\\[0.5ex]
&=& 0 + \int_0^1 y^2\,dy = \frac{1}{3}.
\end{eqnarray}
Note that the integral depends on the route taken between the initial and final points. 

\begin{figure}
\epsfysize=1.75in
\centerline{\epsffile{AppendixA/figA.14.eps}}
\caption{\em An example line integral.}\label{f11}
\end{figure}

The most common type of line integral is that in which the contributions from $dx$ and $dy$ are evaluated
separately, rather that through the path-length element $dl$: {\em i.e.}, 
\begin{equation}
\int_P^Q \left[ f(x,y)\,dx + g(x,y)\,dy\right].
\end{equation}
As an example of this, consider the integral
\begin{equation}
\int_P^Q \left[ y\,dx + x^3\,dy\right]
\end{equation}
along the two routes indicated in Figure~\ref{f12}.
Along route 1 we have $x=y+1$ and $dx=dy$, so
\begin{equation}
\int_P^Q \left[ y\,dx + x^3\,dy\right]= \int_{0}^1\left[y\,dy + (y+1)^3\,dy\right] = \frac{17}{4}.
\end{equation}
Along route 2,
\begin{equation}
\int_P^Q \left[ y\,dx + x^3\,dy\right]= \left.\int_0^1 x^3\,dy\right|_{x=1} + \left.\int_1^2 y\,dx\right|_{y=1} = \frac{7}{4}.
\end{equation}
Again, the integral depends on the path of integration. 

\begin{figure}
\epsfysize=1.75in
\centerline{\epsffile{AppendixA/figA.15.eps}}
\caption{\em An example line integral.}\label{f12}
\end{figure}

Suppose that we have a line integral which does {\em not}\/ depend on the path of integration. It
follows that
\begin{equation}
\int_P^Q \left(f\,dx + g\,dy\right) = F(Q) - F(P)
\end{equation}
for some function $F$. Given $F(P)$ for one point $P$ in the $x$-$y$ plane, then
\begin{equation}
F(Q) = F(P) + \int_P^Q \left(f\,dx + g\,dy\right)
\end{equation}
defines $F(Q)$ for all other points in the plane. We can then draw a contour map of $F(x,y)$.
The line integral between points $P$ and $Q$ is simply the change in height in the contour
map between these two points:
\begin{equation}
\int_P^Q \left(f\,dx + g\,dy\right) = \int_P^Q dF(x,y) = F(Q) - F(P).
\end{equation}
Thus,
\begin{equation}
dF(x,y) = f(x,y)\,dx + g(x,y)\,dy.
\end{equation}
For instance, if $F=x^3\,y$ then $dF= 3\,x^2\,y\,dx + x^3\,dy$ and 
\begin{equation}
\int_P^Q \left(3\,x^2\,y\,dx + x^3\,dy\right) = \left[x^3\,y\right]_P^Q 
\end{equation}
is independent of the path of integration. 

It is clear that there are two distinct types of line integral---those which depend only on their
endpoints and not on the path of integration, and those which depend  both on their endpoints
and the integration path. Later on, we shall learn how to distinguish between these two types (see Section~\ref{sgrad}).

\section{Vector Line Integrals}\label{sveclinei}
A {\em vector field}\/ is defined as a set of vectors associated with each point in space.
For instance, the velocity ${\bf v}({\bf r})$ in a moving liquid 
({\em e.g.}, a whirlpool) constitutes
a vector field. By analogy, a {\em scalar field}\/ is a set of scalars associated with each
point in space. An example of a scalar field is the temperature distribution $T({\bf r})$ in
a furnace. 

Consider a general vector field ${\bf A}({\bf r})$. Let $d{\bf r} \equiv (dx,\,dy,\,dz)$ be the
vector element of line length. Vector line integrals often arise as 
\begin{equation}
\int_P^Q {\bf A}\cdot d{\bf r} = \int_P^Q (A_x\,dx+A_y\,dy + A_z\,dz).
\end{equation}
For instance, if ${\bf A}$ is a force-field then the line integral is the work done in going from
$P$ to $Q$. 

As an example, consider the work done by a repulsive inverse-square 
central field, ${\bf F} = - {\bf r}/ |r^3|$. The 
element of work done  is $dW={\bf F}\cdot d{\bf r}$. 
Take $P=(\infty, 0, 0)$ and $Q=(a,0,0)$. The first route considered is along the $x$-axis, so 
\begin{equation}
W = \int_{\infty}^a \left(-\frac{1}{x^2}\right)\,dx = \left[\frac{1}{x}\right]_{\infty}^a
=\frac{1}{a}.
\end{equation}
The second route is, firstly, around a large circle ($r=$ constant) to the point
($a$, $\infty$, 0), and then parallel to the $y$-axis. See Figure~\ref{f13a}. In the first  part, no work is 
done,
since ${\bf F}$ is perpendicular to $d{\bf r}$. In the second part,
\begin{equation}
W = \int_{\infty}^0 \frac{-y\,dy}{(a^2 + y^2)^{3/2}} = \left[\frac{1}{(y^2+a^2)^{1/2}}
\right]^0_\infty = \frac{1}{a}.
\end{equation}
In this case, the integral is independent of the path. However, not all vector line integrals
are path independent.

\begin{figure}
\epsfysize=2.25in
\centerline{\epsffile{AppendixA/figA.16.eps}}
\caption{\em An example vector line integral.}\label{f13a}
\end{figure}

\section{Surface Integrals}
Let us take a surface $S$, that is not necessarily co-planar, and divide it up
into (scalar) elements $\delta S_i$. Then
\begin{equation}
\int\!\int_S f(x,y,z)\, dS = \lim_{\delta S_i\rightarrow 0}\sum_i f(x,y,z)\,
\delta S_i
\end{equation}
is a surface integral. For instance, the volume of water in a lake of depth
$D(x,y)$ is
\begin{equation}
V= \int\!\int D(x,y)\,dS.
\end{equation}
To evaluate this integral we must split the calculation into two ordinary integrals.
The volume in the strip shown in Figure~\ref{f13} is
\begin{equation}
\left[\int_{x_1}^{x_2} D(x,y)\,dx\right]dy.
\end{equation}
Note that the limits $x_1$ and $x_2$ depend on $y$. The total volume is the sum
over all strips: {\em i.e.}, 
\begin{equation}
V = \int_{y_1}^{y_2} dy\left[\int_{x_1(y)}^{x_2(y)} D(x,y)\,dx\right]
\equiv \int\!\int_S D(x,y)\,dx\,dy.
\end{equation} 
Of course, the integral can be evaluated by taking the strips the other way around: {\em i.e.}, 
\begin{equation}
V = \int_{x_1}^{x_2} dx \int_{y_1(x)}^{y_2(x)} D(x,y)\,dy.
\end{equation}
Interchanging the order of integration is a very powerful and useful trick. But
great care must be taken when evaluating the limits.
\begin{figure}
\epsfysize=2.in
\centerline{\epsffile{AppendixA/figA.17.eps}}
\caption{\em Decomposition of a surface integral.}\label{f13}
\end{figure}

As an example, consider 
\begin{equation}
\int\!\int_S x \,y^2\,dx\,dy,
\end{equation}
where $S$ is shown in Figure~\ref{f14}.
Suppose that we evaluate the $x$ integral  first:
\begin{equation}
dy\left(\int_0^{1-y} x\, y^2\,dx\right) = y^2\,dy\left[ \frac{x^2}{2}\right]^{1-y}_0
= \frac{y^2}{2}\,(1-y)^2\,dy.
\end{equation}
Let us now evaluate the $y$ integral:
\begin{equation}
\int_0^1 \left(\frac{y^2}{2}-y^3+ \frac{y^4}{2}\right)dy = \frac{1}{60}.
\end{equation}
We can also evaluate the integral by interchanging the order of integration:
\begin{equation}
\int_0^1 x\,dx \int_0^{1-x} y^2\,dy = \int_0^1\frac{x}{3}\, (1-x)^3\,dx
 = \frac{1}{60}.
\end{equation}
\begin{figure}
\epsfysize=2.in
\centerline{\epsffile{AppendixA/figA.18.eps}}
\caption{\em An example surface integral.}\label{f14}
\end{figure}

In some cases, a surface integral is just the product of two separate integrals.
For instance, 
\begin{equation}
\int\int_S x^2 \,y\,dx\,dy
\end{equation}
where $S$ is a unit square. This integral can be written
\begin{equation}
\int_0^1 dx \int_0^1 x^2 \,y\,dy = \left(\int_0^1 x^2\,dx\right)
\left(\int_0^1 y \,dy\right) = \frac{1}{3}\,\frac{1}{2} = \frac{1}{6},
\end{equation}
since the limits are both independent of the other variable. 

\section{Vector Surface Integrals}
Surface integrals often occur during vector analysis. For instance, the rate of
flow of a liquid of velocity ${\bf v}$ through an infinitesimal
 surface of vector area $d{\bf S}$
is ${\bf v} \cdot d{\bf S}$. The net rate of flow through a surface ${\bf S}$ made up
of
lots of infinitesimal surfaces is 
\begin{equation}
\int\!\int_S {\bf v}\cdot d{\bf S} = \lim_{dS\rightarrow 0}\left[ \sum v\,\cos\theta
\,dS\right],
\end{equation}
where $\theta$ is the angle subtended between the normal to the surface and the
flow velocity. 

Analogously to line integrals, most 
surface  integrals depend both on the surface and the rim.
But some (very important) integrals depend only on the rim, and not on the nature
of the surface which spans it. As an example of this, consider incompressible fluid
flow between two surfaces $S_1$ and $S_2$ which end on the same rim. See Figure~\ref{f19}. 
The volume between the surfaces is constant, so what goes in must come out, and
\begin{equation}
\int\!\int_{S_1} {\bf v}\cdot d{\bf S} = \int\int_{S_2} {\bf v}\cdot d{\bf S}.
\end{equation}
It follows that
\begin{equation}
\int\!\int {\bf v}\cdot d{\bf S} 
\end{equation}
depends only on the rim, and not on the form of surfaces $S_1$ and $S_2$. 

\section{Volume Integrals}
A volume integral takes the form
\begin{equation}
\int\!\int\!\int_V f(x,y,z)\,dV,
\end{equation}
where $V$ is some volume, and $dV = dx \,dy \,dz$ is a small volume element. The
volume element is sometimes written $d^3{\bf r}$, or even $d\tau$. 

As an example
of a volume integral, let us evaluate the center of gravity of a solid pyramid. Suppose that
the pyramid has a square base of side $a$, a  height $a$, and is composed of material of uniform density. Let the centroid of the base lie at the origin, and let
the apex lie at $(0,\,0,\,a)$. 
By symmetry, the center of mass lies on the line joining the centroid  to the apex. 
In fact, the height of the center of mass is given by
\begin{equation}
\overline{z} = \left. \int\!\int\!\int z\,dV\right/ \int\!\int\!\int dV.
\end{equation}
The bottom integral is  just the volume of the pyramid, and can be written
\begin{eqnarray}
\int\!\int\!\int dV &=& \int_0^a dz \int_{-(a-z)/2}^{(a-z)/2} dy\int_{-(a-z)/2}^{(a-z)/2} dx = \int_0^a (a-z)^2\,dz=\int_0^a (a^2-2\,a\,z+z^2)\,dz\nonumber\\[0.5ex]
& =& \left[a^2\,z-a\,z^2+z^3/3\right]_0^a= \frac{1}{3}\,a^3.
\end{eqnarray}
Here, we have evaluated the $z$-integral last because the limits of the $x$- and $y$- integrals are $z$-dependent. 
The top integral takes the form
\begin{eqnarray}
\int\!\int\!\int z\,dV &=& \int_0^a z\,dz \int_{-(a-z)/2}^{(a-z)/2} dy\int_{-(a-z)/2}^{(a-z)/2} dx = \int_0^a z\,(a-z)^2\,dz=\int_0^a (z\,a^2-2\,a\,z^2+z^3)\,dz\nonumber\\[0.5ex]
& =& \left[a^2\,z^2/2-2\,a\,z^3/3+z^4/4\right]_0^a= \frac{1}{12}\,a^4.
\end{eqnarray}
Thus,
\begin{equation}
\bar{z} = \left.\frac{1}{12}\,a^4\right/\frac{1}{3}\,a^3 = \frac{1}{4}\,a.
\end{equation}
In other words, the center of mass of a pyramid lies one quarter of the way between the centroid of the base and the apex. 

\section{Gradient}\label{sgrad}
A one-dimensional function $f(x)$ has a  gradient $df/dx$ which is
defined as the slope of the tangent to the curve at $x$. 
We wish to extend this idea to cover scalar fields in two and three dimensions. 

Consider a two-dimensional scalar field $h(x,\,y)$, which is (say) height above sea-level in a hilly region.
Let  $d{\bf r}\equiv(dx,\,dy)$ be an element of horizontal distance. Consider
$dh/dr$, where $dh$ is the change in height after moving an infinitesimal distance
$d{\bf r}$. This quantity is  somewhat like the one-dimensional gradient, except that
$dh$ depends on the {\em direction}\/ of $d{\bf r}$, as well as its magnitude. 
In the immediate vicinity of some point $P$, the slope reduces to an inclined plane. See Figure~\ref{f15}.
The largest value of $dh/dr$ is straight up the slope. It is easily shown that for any other direction
\begin{equation}
\frac{dh}{dr}= \left(\frac{dh}{dr}\right)_{\rm max}\, \cos\theta,
\end{equation}
where $\theta$ is the angle shown in Figure~\ref{f15}. 
Let us define a two-dimensional vector, ${\bf  grad}\,h$, 
called the {\em gradient}\/ of $h$, whose magnitude is
$(dh/dr)_{\rm max}$, and whose direction is the direction of steepest ascent.
The $\cos\theta$ variation exhibited in the above expression ensures that the component of ${\bf grad}\,h$ in any
direction is equal to $dh/dr$ for that direction. 

\begin{figure}
\epsfysize=2.25in
\centerline{\epsffile{AppendixA/figA.19.eps}}
\caption{\em A two-dimensional gradient.}\label{f15}
\end{figure}

The component of $dh/dr$ in the $x$-direction can be  obtained by plotting out the
profile of $h$ at constant $y$, and then finding the slope of the tangent to the
curve at given $x$. This quantity is known as the {\em partial derivative}\/ of
$h$ with respect to $x$ at constant $y$, and is denoted $(\partial h/\partial x)_y$.
Likewise, the gradient of the profile at constant $x$ is written
$(\partial h/\partial y)_x$. Note that the subscripts denoting constant $x$ and
constant $y$ are usually omitted, unless there is any ambiguity. It follows that
in component form
\begin{equation}
{\bf grad}\,h \equiv \left(\frac{\partial h}{\partial x},\, \frac{\partial h}{\partial y}
\right).
\end{equation}

Now, the equation of the tangent plane at $P=(x_0,\, y_0)$ is
\begin{equation}
h_T(x,\,y)= h(x_0,\,y_0) + \alpha\,(x-x_0)+\beta\,(y-y_0).
\end{equation}
This has the same local gradients as $h(x,\,y)$, so
\begin{equation}
\alpha = \frac{\partial h}{\partial x},~~~~~\beta= \frac{\partial h}{\partial y},
\end{equation}
by differentiation of the above.
For small $dx=x-x_0$ and $dy=y-y_0$, the function $h$ is coincident with the tangent
plane, so
\begin{equation}
dh = \frac{\partial h}{\partial x}\, dx +\frac{\partial h}
{\partial y}\, dy.
\end{equation}
But, ${\bf grad}\,h \equiv (\partial h/\partial x, \,\partial h/\partial y)$ and
$d{\bf r} \equiv (dx, \,dy)$, so
\begin{equation}
dh = {\bf grad}\,h \cdot d{\bf r}.
\end{equation}
Incidentally, the above equation demonstrates that ${\bf grad}\,h$ is a proper vector,
since the left-hand side is a scalar, and, according to the properties of the dot
product, the right-hand side is also a scalar provided that $d{\bf r}$ and
${\bf grad}\,h$ are both 
proper vectors ($d{\bf r}$ is an obvious vector, because it is
directly derived from displacements).

Consider, now, a three-dimensional temperature distribution $T(x,\,y,\,z)$ in 
(say) a
reaction vessel.  Let us define
${\bf grad}\,T$, as before, as a vector whose magnitude is $(dT/dr)_{\rm max}$,
and whose direction is the direction of the maximum gradient. 
This vector is written in component form
\begin{equation}
{\bf grad}\,T \equiv \left(\frac{\partial T}{\partial x}, \,\frac{\partial T}{\partial y},\,
\frac{\partial T}{\partial z}\right).
\end{equation}
Here, $\partial T/\partial x\equiv (\partial T/\partial x)_{y, z}$ is the
gradient of the one-dimensional temperature profile at constant $y$ and $z$. 
The change in $T$ in going from point $P$ to a neighbouring point offset by
$d{\bf r}\equiv (dx,\,dy,\,dz)$ is
\begin{equation}
dT = \frac{\partial T}{\partial x}\,dx +\frac{\partial T}{\partial y}\,dy+
 \frac{\partial T}{\partial z}\,dz.
\end{equation}
In vector form, this becomes
\begin{equation}
dT = {\bf grad}\,T \cdot d{\bf r}.
\end{equation}
Suppose that $dT=0$ for some $d{\bf r}$. It follows that
\begin{equation}
dT = {\bf grad}\,T \cdot d{\bf r} = 0.
\end{equation}
So, $d{\bf r}$ is perpendicular to ${\bf grad}\,T$. Since $dT=0$ along so-called
``isotherms'' ({\em i.e.}, contours of the temperature), we conclude that the isotherms
(contours) are everywhere perpendicular to ${\bf grad}\,T$. See Figure~\ref{f16}.

\begin{figure}
\epsfysize=2.in
\centerline{\epsffile{AppendixA/figA.20.eps}}
\caption{\em Isotherms.}\label{f16}
\end{figure}

It is, of course, possible to integrate $dT$. For instance, the line integral of $dT$ between points $P$ and
$Q$ is written
\begin{equation}
\int_P^Q dT = \int_P^Q {\bf grad}\,T\cdot d{\bf r} = T(Q)-T(P).
\end{equation}
This integral is clearly independent of the path taken between $P$ and $Q$, so
$\int_P^Q {\bf grad }\,T\cdot d{\bf r}$ must be path independent. 


Consider a vector field ${\bf A}({\bf r})$. In general, the line integral $\int_P^Q {\bf A}\cdot d{\bf r}$ depends on the path
taken between the end points. However, for some special vector fields the integral is path independent. Such fields
are called {\em conservative}\/ fields. It can be shown that if ${\bf A}$ is a
conservative field then ${\bf A} = {\bf grad}\,V$ for some scalar field $V$.
The proof of this is straightforward. Keeping $P$ fixed, we have
\begin{equation}
\int_P^Q {\bf A}\cdot d{\bf r} = V(Q),
\end{equation}
where $V(Q)$ is a well-defined function, due to the path independent nature of the
line integral. Consider moving the position of the end point by an infinitesimal
amount $dx$ in the $x$-direction. We have
\begin{equation}
V(Q+dx) = V(Q) + \int_Q^{Q+dx} {\bf A}\cdot d{\bf r} = V(Q) + A_x\,dx.
\end{equation}
Hence,
\begin{equation}
\frac{\partial V}{\partial x} = A_x,
\end{equation}
with analogous relations for the other components of ${\bf A}$. It follows that
\begin{equation}
{\bf A} = {\bf grad} \,V.
\end{equation}

The force field due to gravity is a good example of a conservative field.
Now, if ${\bf A}({\bf r})$ is a force-field then  $\int {\bf A}\cdot d{\bf r}$ is the work done
in traversing some path. If ${\bf A}$ is conservative then
\begin{equation}
\oint {\bf A}\cdot d{\bf r} = 0,
\end{equation}
where $\oint$ corresponds to the line integral around a {\em closed}\/ loop. 
The fact that zero net work is done in going around a closed loop is equivalent
to the conservation of energy (which is why conservative fields are called
``conservative''). A good example of a non-conservative field is the force field due
to friction. Clearly, a frictional system loses energy in going around a closed
cycle, so $\oint {\bf A}\cdot d{\bf r} \neq 0$. 

\section{Grad Operator}\label{sgradx}
It is useful to define the vector {\em operator}
\begin{equation}
\nabla \equiv \left( \frac{\partial}{\partial x},\, \frac{\partial}{\partial y},\,
\frac{\partial }{\partial z}\right),
\end{equation}
which is usually called the {\em grad}\/ or  {\em del}\/ operator. 
This operator acts on everything to
its right in a expression,   until the end of the expression
or a closing bracket is reached.
For instance,
\begin{equation}
{\bf grad}\,f  = \nabla f \equiv \left(\frac{\partial f}{\partial x},\,
\frac{\partial f}{\partial y},\,\frac{\partial f}{\partial z}\right).
\end{equation}
For two scalar fields $\phi$ and $\psi$,
\begin{equation}
{\bf grad}\,(\phi \,\psi) = \phi\,\, {\bf grad}\,\psi +\psi\,\, {\bf grad}\,\phi
\end{equation}
can be written more succinctly as
\begin{equation}
\nabla(\phi\, \psi) = \phi \,\nabla\psi + \psi\, \nabla \phi.
\end{equation}

Suppose that we rotate the coordinate axes through an angle $\theta$  about $Oz$. 
By analogy with Equations~(\ref{t1})--(\ref{t3}), the old coordinates ($x$,\, $y$, \,$z$) are related 
to the new ones ($x'$, \,$y'$,\, $z'$) via
\begin{eqnarray}
x &=& x'\, \cos\theta - y'\,\sin\theta,\\[0.5ex]
y &=& x\,'\sin\theta +y'\,\cos\theta,\\[0.5ex]
z&=& z'.
\end{eqnarray}
Now,
\begin{equation}
\frac{\partial}{\partial x'} = \left(\frac{\partial x}{\partial x'} \right)_{y',z'}
\frac{\partial}{\partial x}+\left(\frac{\partial y}{\partial x'} \right)_{y',z'}
\frac{\partial}{\partial y}+\left(\frac{\partial z}{\partial x'} \right)_{y',z'}
\frac{\partial}{\partial z},
\end{equation}
giving
\begin{equation}
\frac{\partial}{\partial x'} = \cos\theta \,\frac{\partial}{\partial x} + 
\sin\theta \,\frac{\partial}{\partial y},
\end{equation}
and 
\begin{equation}
\nabla_{x'} = \cos\theta\, \nabla_x + \sin\theta \,\nabla_y.
\end{equation}
It can be seen, from  Equations~(\ref{tt1})--(\ref{tt3}), that 
the differential operator $\nabla$ transforms in an analogous manner to
a vector.
 This is another proof that $\nabla f$ is a good  vector.


\section{Divergence}\label{sdiv}
Let us start with a vector field ${\bf A}({\bf r})$. Consider $\oint_S {\bf A}\cdot
d{\bf S}$ over some closed surface $S$, where $d{\bf S}$ denotes an {\em outward}
pointing surface element. This surface integral is usually called the
{\em flux} of ${\bf A}$ out of $S$. If ${\bf A}$ represents the velocity of some fluid
then $\oint_S {\bf A}\cdot d{\bf S}$ is the rate of fluid flow out of $S$. 

If ${\bf A}$ is constant in space then it is easily demonstrated that the net
flux out of $S$ is zero,
\begin{equation}
\oint {\bf A}\cdot d{\bf S} = {\bf A}\cdot \oint d{\bf S} = {\bf A} \cdot {\bf S}=0,
\end{equation}
since the vector area ${\bf S}$ of a closed surface is zero.

Suppose, now, that ${\bf A}$ is not uniform in space. Consider a very small 
rectangular volume over which ${\bf A}$ hardly varies. The contribution to
$\oint {\bf A}\cdot d{\bf S}$ from the two faces normal to the $x$-axis is
\begin{equation}
A_x(x+dx) \,dy\,dz - A_x(x)\, dy\,dz = \frac{\partial A_x}{\partial x}\,dx\,dy\,dz
=  \frac{\partial A_x}{\partial x}\,dV,
\end{equation}
where $dV= dx\,dy\,dz$ is the volume element. See Figure~\ref{f17}.
There are analogous contributions
from the sides normal to the $y$- and $z$-axes, so the total of all the contributions
 is
\begin{equation}\label{e2121}
\oint {\bf A}\cdot d{\bf S} = \left(\frac{\partial A_x}{\partial x}+
\frac{\partial A_y}{\partial y}+\frac{\partial A_z}{\partial z}\right)dV.
\end{equation}
The {\em divergence} of a vector field is defined
\begin{equation}
{\mit div}\,{\bf A} = \nabla\cdot {\bf A} = \frac{\partial A_x}{\partial x}
+\frac{\partial A_y}{\partial y}+\frac{\partial A_z}{\partial z}.
\end{equation}
Divergence  is a good scalar ({\em i.e.}, it is coordinate
 independent), 
since it is the dot product of
the vector operator $\nabla$ with ${\bf A}$. The formal definition of
$\nabla\cdot{\bf A}$ is
\begin{equation}
\nabla\cdot{\bf A} = \lim_{dV\rightarrow 0} \frac{\oint {\bf A}\cdot d{\bf S}}
{dV}.
\end{equation} 
This definition is independent of the shape of the  infinitesimal volume
element. 
\begin{figure}
\epsfysize=2in
\centerline{\epsffile{AppendixA/figA.21.eps}}
\caption{\em Flux of a vector field out of a small box.}\label{f17}
\end{figure}

One of the most important results in vector field theory is the so-called
{\em divergence theorem} or {\em Gauss' theorem}. This states that for any volume
$V$ surrounded by a closed surface $S$,
\begin{equation}
\oint_S {\bf A}\cdot d{\bf S} = \int_V \nabla\cdot{\bf A}\,\,dV,
\end{equation}
where $d{\bf S}$ is an outward pointing volume element. 
The proof is very
 straightforward. We divide up the volume into lots of very small cubes, and
sum $\int {\bf A}\cdot d{\bf S}$ over all of the surfaces. The contributions
from the interior surfaces cancel out, leaving just the contribution from the outer
surface. See Figure~\ref{f18}. We can use Equation~(\ref{e2121}) for each cube individually. This tells us that
the summation is equivalent to $\int \nabla\cdot{\bf A}\,\,dV$ over the whole
volume. Thus, the integral of ${\bf A}\cdot d{\bf S}$ over the outer surface is
equal to the integral of $\nabla\cdot{\bf A}$ over the whole volume, which
proves the divergence theorem.
\begin{figure}
\epsfysize=2.5in
\centerline{\epsffile{AppendixA/figA.22.eps}}
\caption{\em The divergence theorem.}\label{f18}
\end{figure}

Now, for a vector field with $\nabla\cdot{\bf A} = 0$,
\begin{equation}
\oint_S {\bf A}\cdot d{\bf S} =0 
\end{equation}
for any closed surface $S$. So, for two surfaces,
$S_1$ and $S_2$, on the same rim,
\begin{equation}
\int_{S_1} {\bf A}\cdot d{\bf S} = \int_{S_2} {\bf A}\cdot d{\bf S}.
\end{equation}
See Figure~\ref{f19}. (Note that the direction of the surface elements on $S_1$ has been reversed relative to those on the closed surface. Hence, the
sign of the associated surface integral is also reversed.)
Thus, if $\nabla\cdot{\bf A}=0$ then the surface integral depends on the rim but
not on the nature of the surface which spans it. 
On the other hand, if $\nabla\cdot{\bf A}\neq 0 $ then the integral
depends on both the rim and the surface. 
\begin{figure}
\epsfysize=2.in
\centerline{\epsffile{AppendixA/figA.23.eps}}
\caption{\em Two surfaces spanning the same rim (right), and the
equivalent closed surface (left).}\label{f19}
\end{figure}

Consider an incompressible fluid whose velocity field is ${\bf v}$. It is clear that
$\oint {\bf v}\cdot d{\bf S} = 0$ for any closed surface, since what flows into the
surface must flow out again. Thus, according to the divergence theorem,
$\int \nabla\cdot{\bf v}\,\,dV = 0$ for any volume. The only way in which this is
possible is if $\nabla\cdot{\bf v}$ is everywhere zero. Thus,  the velocity components
of an incompressible fluid satisfy the following differential relation:
\begin{equation}
\frac{\partial v_x}{\partial x} + \frac{\partial v_y}{\partial y} + 
\frac{\partial v_z}
{\partial z}=0.
\end{equation}

It is sometimes helpful to represent a vector field ${\bf A}$ by {\em lines of force}
or  {\em field-lines}.
The direction of a line of force at any point is the same as the local direction of
${\bf A}$. The density of lines ({\em i.e.}, the number of lines crossing a unit surface
perpendicular to ${\bf A}$) is equal to $|{\bf A}|$. 
 For instance, in Figure~\ref{f20}, $|{\bf A}|$ is larger at point 1 than at point 2. The number of lines
crossing a surface element $d{\bf S}$ is ${\bf A}\cdot d{\bf S}$. So, the
net number of lines leaving a closed surface is
\begin{equation}
\oint_S {\bf A}\cdot d{\bf S} = \int_V \nabla\cdot{\bf A}\,\,dV.
\end{equation}
If $\nabla\cdot{\bf A}=0$ then there is no net flux of lines out of any surface.
 Such a field is
called a {\em solenoidal}\/ vector field. The simplest example of a solenoidal vector
field is one in which the lines of force all form {\em closed loops}.
\begin{figure}
\epsfysize=1.5in
\centerline{\epsffile{AppendixA/figA.24.eps}}
\caption{\em Divergent lines of force.}\label{f20}
\end{figure}

\section{Laplacian Operator}\label{qlap}
So far we have encountered
\begin{equation}
\nabla\phi = \left(\frac{\partial \phi}{\partial x},\, \frac{\partial \phi}
{\partial y},\, \frac{\partial \phi}{\partial z}\right),
\end{equation}
which is a vector field formed from a scalar field, and 
\begin{equation}
\nabla\cdot{\bf A} = \frac{\partial A_x}{\partial x} + 
\frac{\partial A_y}{\partial y}  +\frac{\partial A_z}{\partial z},
\end{equation}
which is a scalar field formed from a vector field. There are two ways in which
we can combine gradient and divergence. We can either form the vector field 
$\nabla(\nabla\cdot{\bf A})$ or  the scalar field $\nabla\cdot(\nabla\phi)$.
The former is not particularly interesting, but the scalar field 
$\nabla\cdot(\nabla\phi)$ turns up in a great many physical problems, and is, 
therefore, worthy of discussion. 

Let us introduce the heat flow vector ${\bf h}$, which is the rate of flow of heat
energy per unit area across a surface perpendicular to the direction of ${\bf h}$.
In many substances, heat flows directly down the temperature gradient, so that we
can write
\begin{equation}\label{e2127}
{\bf h} = - \kappa \,\,\nabla T,
\end{equation}
where $\kappa$ is the {\em thermal conductivity}. The net rate of heat flow 
$\oint_S {\bf h}\cdot d{\bf S}$ out of some closed surface $S$ must be equal
to the rate of decrease of heat energy in the volume $V$ enclosed by $S$.
Thus, we have
\begin{equation}
\oint_S {\bf h}\cdot d{\bf S} = - \frac{\partial}{\partial t}\left(
\int c\, T\,dV\right),
\end{equation}
where $c$ is the {\em specific heat}. It follows from the divergence theorem that
\begin{equation}\label{e2129}
\nabla\cdot{\bf h} = -c\,\frac{\partial T}{\partial t}.
\end{equation}

Taking the divergence of both sides of Equation~(\ref{e2127}), and making use of Equation~(\ref{e2129}),
 we obtain
\begin{equation}
\nabla\cdot\left(\kappa \,\nabla T\right) = c\,\frac{\partial T}{\partial t}.
\end{equation}
If $\kappa$ is constant then the above equation  can be written
\begin{equation}
\nabla\cdot (\nabla T) = \frac{c}{\kappa} \frac{ \partial T}{\partial t}.
\end{equation}
The scalar field $\nabla\cdot (\nabla T)$ takes the form 
\begin{eqnarray}
\nabla\cdot(\nabla T) &=& 
\frac{\partial}{\partial x}\!\left(\frac{\partial T}{\partial x}\right)+
\frac{\partial}{\partial y}\!\left(\frac{\partial T}{\partial y}\right)+
\frac{\partial}{\partial z}\!\left(\frac{\partial T}{\partial z}\right)
\nonumber\\[0.5ex]
&=& \frac{\partial^2 T}{\partial x^2}+\frac{\partial^2 T}{\partial y^2} +
\frac{\partial^2 T}{\partial z^2} \equiv \nabla^2 T.
\end{eqnarray}
Here, the scalar differential operator
\begin{equation}
\nabla^2 \equiv \frac{\partial^2}{\partial x^2}+ 
\frac{\partial^2}{\partial y^2}+ \frac{\partial^2}{\partial z^2}
\end{equation}
is called the {\em Laplacian}. The Laplacian is a good scalar operator
({\em i.e.}, it is coordinate independent) because it is formed from a
combination of divergence
(another good scalar operator) and gradient (a good vector
operator). 

What is the physical significance of the Laplacian? In one dimension,
$\nabla^2 T$ reduces to $\partial^2 T/\partial x^2$.
Now, $\partial^2 T/\partial x^2$ is positive if $T(x)$ is
concave (from above) and negative if it is convex. So, if $T$ is less than the
average of $T$ in its surroundings then $\nabla^2 T$ is positive, and {\em vice
versa}. 

In two dimensions,
\begin{equation}
\nabla^2 T = \frac{\partial^2 T}{\partial x^2}+ \frac{\partial ^2 T}{\partial y^2}.
\end{equation}
Consider a local minimum of the temperature.
At the minimum, the slope of $T$ increases in all directions, so $\nabla^2 T$ is
positive. Likewise, $\nabla^2 T$ is negative at
 a local maximum.
Consider, now, a steep-sided valley in $T$. Suppose that the bottom of the
valley runs parallel to the $x$-axis. 
At the bottom of the
valley  $\partial^2 T/\partial y^2$ is large and positive, whereas
$\partial^2 T/\partial x^2$ is small and may even be negative. Thus, $\nabla^2 T$
is positive, and this is associated with $T$ being less than the average local
value.

Let us now return to the heat conduction problem:
\begin{equation}
\nabla^2 T = \frac{c}{\kappa} \frac{\partial T}{\partial t}.
\end{equation}
It is clear that if $\nabla^2 T$ is positive then $T$ is locally less than the
average value, so $\partial T/\partial t>0$: {\em i.e.}, the region heats up.
Likewise, if $\nabla^2 T$ is negative then $T$ is locally greater than the average
value, and heat flows out of the region: {\em i.e.}, $\partial T/\partial t<0$. Thus,
the above heat conduction equation makes physical sense.

\section{Curl}\label{scurl}
Consider a vector field ${\bf A}({\bf r})$, and a loop  which lies in one plane. 
The integral of ${\bf A}$ around this loop is written
$\oint {\bf A}\cdot d{\bf r}$, where $d{\bf r}$ is a line element of the
loop. If ${\bf A}$ is a conservative field then 
${\bf A}= \nabla \phi$ and $\oint {\bf A}\cdot d{\bf r}=0$
for all loops. In general, for a non-conservative field, $\oint {\bf A}\cdot d{\bf r}\neq 0$.

For a small loop we expect $\oint {\bf A}\cdot d{\bf r}$ to be proportional to
the area of the loop. Moreover, for a fixed-area loop  we expect
$\oint {\bf A}\cdot d{\bf r}$ to depend on the {\em orientation}\/ of the loop.
One particular orientation will give the maximum value: $\oint {\bf A}\cdot d{\bf r}
= I_{\rm max}$. If the loop subtends an angle $\theta$ with this optimum orientation
then we expect $I= I_{\rm max}\cos\theta$. Let us introduce the vector field
${\bf curl}\,{\bf A}$ whose magnitude is
\begin{equation}
|{\bf curl}\,{\bf A}| = \lim_{dS\rightarrow 0}\frac{\oint {\bf A}\cdot
d{\bf r}}{dS}
\end{equation}
for the orientation giving $I_{\rm max}$. Here, $dS$ is the area of the loop.
The direction of ${\bf curl}\,{\bf A}$ is perpendicular to the plane of the loop,
 when it is
 in the 
orientation giving $I_{\rm max}$, with the sense given by a right-hand circulation rule.

Let us now express ${\bf curl}\,{\bf A}$ in terms of the components of ${\bf A}$.
First, we shall
 evaluate $\oint {\bf A}\cdot d{\bf r}$ around a small rectangle in the
$y$-$z$ plane. See Figure~\ref{f21}.
The contribution from sides 1 and 3 is
\begin{equation}
A_z(y+dy)\,dz - A_z(y)\,dz = \frac{\partial A_z}{\partial y} \,dy\,dz.
\end{equation}
The contribution from sides 2 and 4 is
\begin{equation}
-A_y(z+dz)\,dy + A_y(z)\,dy = -\frac{\partial A_y}{\partial y}\,dy\,dz.
\end{equation}
So, the total of all contributions gives
\begin{equation}
\oint {\bf A}\cdot d{\bf r} = \left(\frac{\partial A_z}{\partial y}-
\frac{\partial A_y}{\partial z}\right)\,dS,
\end{equation}
where $dS=dy\,dz$ is the area of the loop. 
\begin{figure}
\epsfysize=2in
\centerline{\epsffile{AppendixA/figA.25.eps}}
\caption{\em A vector line integral around a small rectangular loop in the $y$-$z$ plane.}\label{f21}
\end{figure}

Consider a non-rectangular (but still small) loop in the $y$-$z$ plane.
We can divide it into rectangular
 elements, and form $\oint {\bf A}\cdot d{\bf r}$ over all the resultant 
loops. The interior 
contributions cancel, so we are just left with the contribution from the outer loop.
Also, the area of the outer loop is the sum of all the areas of the inner loops.
We conclude that 
\begin{equation}
\oint {\bf A} \cdot d{\bf r} = \left(\frac{\partial A_z}{\partial y}-
\frac{\partial A_y}{\partial z}\right) dS_x
\end{equation}
is valid for a small loop 
$d{\bf S} = (dS_x,\,0,\,0)$ of any shape in the $y$-$z$ plane. Likewise, we can show that
if the loop is in the $x$-$z$ plane then $d{\bf S} = (0,\,dS_y,\,0)$ and
\begin{equation}
\oint {\bf A} \cdot d{\bf r} = \left(\frac{\partial A_x}{\partial z}-
\frac{\partial A_z}{\partial x}\right) dS_y.
\end{equation}
Finally, if the loop is in the $x$-$y$ plane then $d{\bf S} = (0,\,0,\,dS_z)$ and
\begin{equation}
\oint {\bf A} \cdot d{\bf r} = \left(\frac{\partial A_y}{\partial x}-
\frac{\partial A_x}{\partial y}\right) dS_z.
\end{equation}

Imagine an arbitrary loop of vector area $d{\bf S} = (dS_x, \,dS_y, \,dS_z)$. We
can construct this out of three vector areas, $1$, $2$, and $3$, directed in the $x$-, $y$-, and $z$-directions, respectively, as
indicated in Figure~\ref{f22}.
If we form the line integral around all three loops then the interior contributions
cancel, and we are left with the line integral around the original loop. Thus,
\begin{equation}
\oint {\bf A} \cdot d{\bf r} = \oint {\bf A}\cdot d{\bf r}_1 +
\oint {\bf A}\cdot d{\bf r}_2+\oint {\bf A}\cdot d{\bf r}_3,
\end{equation}
giving
\begin{equation}\label{e2145}
\oint {\bf A}\cdot d{\bf r} = {\bf curl}\,{\bf A} \cdot d{\bf S} = 
|{\bf curl}\,{\bf A}|\,|d {\bf S}|\,\cos\theta,
\end{equation}
where
\begin{equation}
{\bf curl}\,{\bf A} = \left(\frac{\partial A_z}{\partial y}- \frac{\partial A_y}
{\partial z},\, \frac{\partial A_x}{\partial z}- \frac{\partial A_z}
{\partial x},\, \frac{\partial A_y}{\partial x}- \frac{\partial A_x}
{\partial y}\right),
\end{equation}
and $\theta$ is the angle subtended between the directions of ${\bf curl}\,{\bf A}$ and $d{\bf S}$. 
Note that
\begin{equation}
{\bf curl}\,{\bf A} = \nabla\times {\bf A}.
\end{equation}
This demonstrates that $\nabla\times {\bf A}$ is a good vector field, since it is
the cross product of the $\nabla$ operator (a good vector operator) and 
the vector field ${\bf A}$.
\begin{figure}
\epsfysize=2.5in
\centerline{\epsffile{AppendixA/figA.26.eps}}
\caption{\em Decomposition of a vector area into its Cartesian components.}\label{f22}
\end{figure}

Consider a solid body rotating about the $z$-axis. The angular velocity is given
by $\mbox{\boldmath$\omega$} = (0,\,0,\,\omega)$, so  the rotation velocity at
position ${\bf r}$ is
\begin{equation}
{\bf v} = \mbox{\boldmath$\omega$}\times {\bf r}
\end{equation}
[see Equation~(\ref{e239})].
Let us evaluate $\nabla\times{\bf v}$ on the axis of
rotation. The $x$-component is proportional to the
integral $\oint {\bf v}\cdot d{\bf r}$ around a loop in the $y$-$z$ plane. This is
plainly zero. Likewise, the $y$-component is also zero. The $z$-component
is $\oint {\bf v}\cdot d{\bf r}/ dS$ around some loop in the $x$-$y$ plane. 
Consider a circular loop. We have $\oint {\bf v}\cdot d{\bf r} = 2\pi \,r \,\omega \,r$ 
with $dS = \pi \,r^2$. 
Here, $r$ is the perpendicular distance from the rotation axis.
It follows that $(\nabla\times {\bf v})_z = 2\,\omega$, which
is independent of $r$. So, on the axis, $\nabla\times{\bf v} = (0\,,0\,,2\,\omega)$.
Off the axis, at position ${\bf r}_0$, we can write
\begin{equation}
{\bf v} = \mbox{\boldmath$\omega$}\times ({\bf r}-{\bf r}_0) + 
\mbox{\boldmath$\omega$}\times {\bf r}_0.
\end{equation}
The first part has the same curl as the velocity field on the axis, and the
second part has zero curl, since it is constant. Thus, 
$\nabla\times{\bf v} = (0,\,0,\,2\,\omega)$ everywhere in the body. This allows us to
form a physical picture of $\nabla\times{\bf A}$. If we imagine ${\bf A}({\bf r})$ as the
velocity field of some fluid then $\nabla\times{\bf A}$ at any given point is equal
to twice the local angular rotation velocity: 
{\em i.e.},  2\,\mbox{\boldmath$\omega$}. 
Hence, a vector field with $\nabla\times{\bf A} ={\bf  0}$ everywhere is said to
be {\em irrotational}.

Another important  result of vector field theory is the {\em curl theorem}
or {\em Stokes' theorem}:
\begin{equation}
\oint_C {\bf A} \cdot d{\bf r} = \int_S \nabla\times{\bf A}\cdot d{\bf S},
\end{equation}
for some (non-planar) surface $S$ bounded by a rim $C$. This theorem can easily
be proved by splitting the loop up into many small rectangular loops, and forming
the integral around all of the resultant loops. All of the contributions from the
interior loops cancel, leaving just the contribution from the outer rim. 
Making use of Equation~(\ref{e2145}) for each of the small loops, we can see that the contribution
from all of the loops is also equal to the  integral of $\nabla\times{\bf A}
\cdot d{\bf S}$ across the whole surface. This proves the theorem.

One immediate consequence of  Stokes' theorem is that $\nabla\times{\bf A}$
is ``incompressible.''  Consider any two surfaces, $S_1$ and $S_2$, which share the
same rim. See Figure~\ref{f19}. It is clear from Stokes' theorem that 
$\int \nabla\times {\bf A}\cdot d{\bf S}$ is the same for both surfaces. Thus, it
follows that $\oint\nabla\times {\bf A}\cdot d{\bf S} = 0$ for any closed surface.
However, we have from the divergence theorem that
$\oint\nabla\times{\bf A}\cdot d{\bf S} = \int \nabla\cdot(\nabla\times{\bf A}) \,dV
=0$
 for any volume. Hence,
\begin{equation}
\nabla\cdot\,(\nabla\times{\bf A}) \equiv 0.
\end{equation}
So,
$\nabla\times {\bf A}$ is a solenoidal field. 

We have seen that for a conservative field $\oint {\bf A}\cdot d{\bf r} = 0$
for any loop. This  is entirely equivalent to ${\bf A} = \nabla\phi$. 
However, the magnitude of $\nabla\times{\bf A}$ is 
$\lim_{\,dS\rightarrow 0}\oint{\bf A}\cdot d{\bf r} /dS$ for some
particular loop. It is clear then that
$\nabla\times{\bf A}={\bf 0}$ for a conservative field. In other words,
\begin{equation}
\nabla\times(\nabla\phi)\equiv {\bf 0}.
\end{equation}
Thus, a conservative field is also an irrotational one. 

\section{Useful Vector Identities}
Notation: ${\bf a}$, ${\bf b}$, ${\bf c}$, ${\bf d}$ are general vectors; $\phi$, $\psi$
are general scalar fields; ${\bf A}$, ${\bf B}$ are general vector fields; $({\bf A}\cdot\nabla){\bf B}\equiv
({\bf A}\cdot\nabla B_x,\, {\bf A}\cdot\nabla B_y,\,{\bf A}\cdot\nabla B_z)$ and
$\nabla^2{\bf A} = (\nabla^2 A_x,\, \nabla^2 A_y,\,\nabla^2 A_z)$ (but, only  in Cartesian coordinates---see Appendix~\ref{ccurv}).
\begin{eqnarray}
{\bf a}\times ({\bf b}\times {\bf c}) &=& ({\bf a}\cdot{\bf c})\,{\bf b} - ({\bf a}\cdot{\bf b})\,{\bf c},\\[0.5ex]
({\bf a}\times {\bf b})\times {\bf c} &=& ({\bf c}\cdot{\bf a})\,{\bf b}
- ({\bf c}\cdot{\bf b})\,{\bf a},\\[0.5ex]
({\bf a}\times {\bf b})\cdot({\bf c}\times {\bf d})&=&
({\bf a}\cdot{\bf c})\,({\bf b}\cdot{\bf d}) - ({\bf a}\cdot{\bf d})\,({\bf b}\cdot{\bf c}),\\[0.5ex]
({\bf a}\times {\bf b})\times ({\bf c}\times {\bf d})&=& ({\bf a}\times
{\bf b}\cdot{\bf d})\,{\bf c} - ({\bf a}\times {\bf b}\cdot{\bf c})\,{\bf d},\\[0.5ex]
\nabla(\phi\,\psi) &=&\phi\,\nabla\psi+\psi\,\nabla\phi,\\[0.5ex]
\nabla({\bf A}\cdot{\bf B}) &=& {\bf A}\times(\nabla\times{\bf B}) + {\bf B}
\times(\nabla\times{\bf A}) +({\bf A}\cdot \nabla){\bf B} + ({\bf B} \cdot
\nabla){\bf A},\label{divsp}\\[0.5ex]
\nabla\cdot\nabla \phi&=&\nabla^2\phi,\label{laplacian}\\[0.5ex]
\nabla\cdot\nabla \times {\bf A}& =& 0,\label{divcurl}\\[0.5ex]
\nabla\cdot (\phi\,{\bf A}) &=&\phi \,\nabla\cdot {\bf A} + {\bf A}\cdot \nabla\phi,\label{divprod}\\[0.5ex]
\nabla\cdot({\bf A}\times{\bf B}) &=& {\bf B}\cdot \nabla\times{\bf A} - {\bf A}\cdot
\nabla\times{\bf B},\label{divvp}\\[0.5ex]
\nabla\times\nabla\phi &=& 0,\label{curlgrad}\\[0.5ex]
\nabla\times(\nabla\times{\bf A})&=& \nabla\,(\nabla\cdot{\bf A})- \nabla^2{\bf A},\label{curlcurl}\\[0.5ex]
\nabla\times(\phi\,{\bf A}) &=&\phi\, \nabla\times{\bf A} +\nabla\phi\times{\bf A},\label{curlprod}\\[0.5ex]
\nabla\times({\bf A}\times{\bf B})& =& (\nabla\cdot {\bf B})\,{\bf A}- (\nabla
\cdot{\bf A}) \,{\bf B}+({\bf B}\cdot\nabla){\bf A}- ({\bf A}\cdot\nabla){\bf B}.\label{curlvp}
\end{eqnarray}

\section{Exercises}
{\small 
\renewcommand{\theenumi}{A.\arabic{enumi}}
\begin{enumerate}
\item The position vectors of the four  points $A$, $B$, $C$, and $D$ are ${\bf a}$, ${\bf b}$, $3\,{\bf a}+2\,{\bf b}$,
and ${\bf a}-3\,{\bf b}$, respectively. Express $\stackrel{\displaystyle \rightarrow}{AC}$, $\stackrel{\displaystyle \rightarrow}{DB}$,
$\stackrel{\displaystyle \rightarrow}{BC}$, and $\stackrel{\displaystyle \rightarrow}{CD}$ in terms of ${\bf a}$ and ${\bf b}$. 
\item Prove the trigonometric law of sines 
$$
\frac{\sin a}{A} = \frac{\sin b}{B} = \frac{\sin c}{C}
$$
using vector methods. Here, $a$, $b$, and $c$ are  the three
angles of a plane triangle, and $A$, $B$, and $C$ the lengths of the corresponding opposite sides.

\item Demonstrate using vectors that the diagonals of a parallelogram bisect one another. In addition, show  that if the diagonals of a quadrilateral bisect one another then it is a parallelogram.

\item From the inequality
$$
{\bf a}\cdot{\bf b}= |{\bf a}|\,|{\bf b}|\,\cos\theta\leq |{\bf a}|\,|{\bf b}|
$$
deduce the triangle inequality
$$
|{\bf a} + {\bf b}|\leq |{\bf a}|+|{\bf b}|.
$$

\item Find the scalar product ${\bf a}\cdot{\bf b}$ and the vector product ${\bf a}\times {\bf b}$ when
\begin{enumerate}
\item ${\bf a} = {\bf e}_x + 3\,{\bf e}_y-{\bf e}_z$,\mbox{\hspace{0.5cm}}${\bf b} = 3\,{\bf e}_x+ 2\,{\bf e}_y+{\bf e}_z$,
\item ${\bf a} = {\bf e}_x - 2\,{\bf e}_y+{\bf e}_z$,\mbox{\hspace{0.5cm}}${\bf b} = 2\,{\bf e}_x+ {\bf e}_y+{\bf e}_z$.
\end{enumerate}

\item Which of the following statements regarding the three general vectors ${\bf a}$, ${\bf b}$, and ${\bf c}$ are true?
\begin{enumerate}
\item ${\bf c}\cdot ({\bf a}\times {\bf b}) = ({\bf b}\times {\bf a})\cdot{\bf c}$.
\item ${\bf a}\times ({\bf b}\times {\bf c})= ({\bf a}\times {\bf b})\times {\bf c}$.
\item ${\bf a}\times ({\bf b}\times {\bf c}) = ({\bf a}\cdot{\bf c})\,{\bf b} - ({\bf a}\cdot{\bf b})\,{\bf c}$.
\item ${\bf d} = \lambda\,{\bf a} + \mu\,{\bf b}$ implies that $({\bf a}\times {\bf b})\cdot {\bf d} = 0$. 
\item ${\bf a}\times {\bf c} = {\bf b}\times {\bf c}$ implies that ${\bf c}\cdot{\bf a} - {\bf c}\cdot{\bf b} = c\,|{\bf a}-{\bf b}|$.
\item $({\bf a}\times {\bf b})\times ({\bf c}\times {\bf b}) = [{\bf b}\cdot({\bf c}\times {\bf a})]\,{\bf b}$.
\end{enumerate}

\item Prove that the length of the shortest straight-line from point ${\bf a}$ to the
straight-line joining points ${\bf b}$ and ${\bf c}$ is
$$
\frac{|{\bf a}\times {\bf b} + {\bf b}\times {\bf c} + {\bf c}\times {\bf a}|}{|{\bf b}-{\bf c}|}.
$$

\item Identify the following surfaces:
\begin{enumerate}
\item $|{\bf r}| = a$,
\item ${\bf r}\cdot{\bf n} = b$,
\item ${\bf r}\cdot{\bf n} = c\,|{\bf r}|$,
\item $|{\bf r} -({\bf r}\cdot{\bf n})\,{\bf n}| = d$.
\end{enumerate}
Here, ${\bf r}$ is the position vector, $a$, $b$,  $c$, and $d$ are positive
constants, and ${\bf n}$ is a fixed unit vector.

\item Let ${\bf a}$, ${\bf b}$, and ${\bf c}$ be coplanar vectors related via
$$
\alpha\,{\bf a} + \beta\,{\bf b} + \gamma\,{\bf c} = {\bf 0},
$$
where $\alpha$, $\beta$, and $\gamma$ are not all zero. Show that the condition
for the points with position vectors $u\,{\bf a}$, $v\,{\bf b}$,
and $w\,{\bf c}$ to be colinear is
$$
\frac{\alpha}{u} +\frac{\beta}{v} + \frac{\gamma}{w} = 0.
$$

\item If ${\bf p}$, ${\bf q}$, and ${\bf r}$ are any vectors, demonstrate that 
${\bf a}={\bf q} + \lambda\,{\bf r}$, ${\bf b} = {\bf r}+\mu\,{\bf p}$,
and ${\bf c} = {\bf p} + \nu\,{\bf q}$ are coplanar provided that
$\lambda\,\mu\,\nu=-1$, where $\lambda$, $\mu$, and $\nu$ are scalars.
Show that this condition is satisfied when ${\bf a}$ is perpendicular to ${\bf p}$, ${\bf b}$ to ${\bf q}$, and ${\bf c}$ to ${\bf r}$.

\item The vectors ${\bf a}$, ${\bf b}$, and ${\bf c}$ are non-coplanar, and
form a non-orthogonal vector base. The vectors ${\bf A}$, ${\bf B}$,
and ${\bf C}$, defined by
$$
{\bf A} = \frac{{\bf b}\times {\bf c}}{{\bf a}\cdot{\bf b}\times {\bf c}},
$$
plus cyclic permutations, are said to be {\em reciprocal vectors}. Show that
 $$
 {\bf a} = ({\bf B}\times {\bf C})/({\bf A}\cdot{\bf B}\times {\bf C}),
 $$
plus cyclic permutations.

\item In the notation of the previous question, demonstrate that the plane passing
through points ${\bf a}/\alpha$, ${\bf b}/\beta$, and ${\bf c}/\gamma$
is normal to the direction of the vector
$$
{\bf h} = \alpha\,{\bf A} + \beta\,{\bf B} + \gamma\,{\bf C}.
$$
In addition, show that the perpendicular distance of the plane from the
origin is $|{\bf h}|^{-1}$.

\item Evaluate $\oint {\bf A}\cdot d{\bf r}$ for
$$
{\bf A} = \frac{x\,{\bf e}_x + y\,{\bf e}_y}{\sqrt{x^2 + y^2}}
$$
around the square whose sides are $x=0$, $x=a$, $y=0$, $y=a$. 

\item Consider the following vector field:
$$
{\bf A}({\bf r}) = (8\,x^3+3\,x^2\,y^2,\,2\,x^3\,y+6\,y,\,6).
$$
Is this field conservative? Is it solenoidal? Is it irrotational? Justify your answers. Calculate $\oint_C {\bf A}\cdot d{\bf r}$, where the curve
$C$ is a unit circle in the $x$-$y$ plane, centered on the origin, and the
direction of integration is clockwise looking down the $z$-axis.

\item Consider the following vector field:
$$
{\bf A}({\bf r}) = (3\,x\,y^2\,z^2-y^2,\,-y^3\,z^2+x^2\,y,\,3\,x^2-x^2\,z).
$$
Is this field conservative? Is it solenoidal? Is it irrotational? Justify your answers. Calculate the flux of ${\bf A}$ out of a unit sphere centered
on the origin.


\item Find the gradients of the following scalar functions of the position vector ${\bf r}=(x,\,y,\,z)$:
\begin{enumerate}
\item ${\bf k}\cdot{\bf r}$,
\item $|{\bf r}|^n,$
\item $|{\bf r}-{\bf k}|^{-n}$,
\item $\cos({\bf k}\cdot {\bf r}).$
\end{enumerate}
Here, ${\bf k}$ is a fixed vector.

\item Find the divergences and curls of the following vector fields:
\begin{enumerate}
\item ${\bf k}\times {\bf r}$,
\item $|{\bf r}|^n\,{\bf r}$,
\item $|{\bf r} - {\bf k}|^{n}\,({\bf r}-{\bf k})$,
\item ${\bf a}\,\cos({\bf k}\cdot {\bf r})$.
\end{enumerate}
Here, ${\bf k}$ and ${\bf a}$ are fixed vectors.

\item Calculate $\nabla^2\phi$ when  $\phi=f(|{\bf r} |)$. Find $f$
if $\nabla^2\phi=0$. 


\end{enumerate}
\renewcommand{\theenumi}{\arabic{enumi}}}
