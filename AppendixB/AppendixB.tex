\chapter{Cartesian Tensors}\label{appb}
\section{Introduction}
As we saw in Appendix~\ref{vector},  many physical entities can be represented mathematically as 
either {\em scalars}\/ or
{\em vectors}, depending on their transformation properties under rotation of the coordinate axes. 
However, it turns out that scalars and vectors are particular types of a more general class of 
mathematical constructs called {\em tensors}. In fact, a scalar is a tensor of order zero, and a vector is a tensor of
order one. In fluid mechanics, certain important physical entities ({\em i.e.}, stress and rate of strain) are represented mathematically
by  tensors of order greater than one. It is therefore necessary to supplement our investigation of fluid mechanics with a brief discussion of the mathematics
of tensors. For the sake of simplicity, we shall limit this discussion to Cartesian coordinate systems. Tensors expressed
in such  coordinate systems are known as {\em Cartesian tensors}. 

\section{Tensors and Tensor Notation}
Let the Cartesian coordinates $x$, $y$, $z$ be written as the $x_i$, where $i$ runs from 1 to 3. In other words,
$x\equiv x_1$, $y\equiv x_2$, and $z\equiv x_3$. Incidentally, in the following, any lowercase roman subscript ({\em e.g.}, $i$, $j$, $k$) is
assumed to run from 1 to 3. We can also write the Cartesian components of a general {\em vector}\/ ${\bf v}$
as the $v_i$. In other words, $v_x\equiv v_1$, $v_y\equiv v_2$, and $v_z\equiv v_3$. By contrast, a {\em scalar}\/ is represented as
a variable without a subscript: {\em e.g.}, $a$, $\phi$. Thus, a scalar---which is a tensor of order zero---is
represented as a variable with zero subscripts, and a vector---which is a tensor of order one---is
represented as a variable with one subscript. It stands to reason, therefore, that a tensor of order two is
represented as a variable with two subscripts: {\em e.g.}, $a_{ij}$, $\sigma_{ij}$. Moreover, an $n$th-order tensor is represented as a variable with $n$ subscripts: {\em e.g.}, $a_{ijk}$ is a third-order tensor, and $b_{ijkl}$  a fourth-order tensor. 
Note that a general $n$th-order tensor has $3^n$ independent components. 

Now, the
components of a second-order tensor are conveniently visualized as a two-dimensional matrix, just as
the components of a vector are sometimes visualized as a one-dimensional matrix. However, it
is important to recognize that an $n$th-order  tensor is not simply another name for an $n$-dimensional matrix. A matrix is
just an ordered set of numbers. A tensor, on the other hand, is an ordered set of components
that have specific transformation properties under rotation of the coordinate axes. See Section~\ref{strans}.

Consider two vectors ${\bf a}$ and ${\bf b}$ that are represented as $a_i$ and $b_i$, respectively, in
tensor notation. According to Section~\ref{sscalar}, the {\em scalar product}\/ of these two vectors
takes the form
\begin{equation}
{\bf a}\cdot{\bf b} = a_1\,b_1+a_2\,b_2+a_3\,b_3.
\end{equation}
The above expression can be written more compactly as
\begin{equation}\label{e3.2}
{\bf a}\cdot{\bf b} = a_i\,b_i.
\end{equation}
Here, we have made use of the {\em Einstein summation convention}, according to which, in an expression
containing lower case roman subscripts, any subscript that appears twice (and only twice) in any
term of the expression is assumed to be {\em summed}\/ from 1 to 3 (unless stated otherwise).
Thus, $a_i\,b_i\equiv a_1\,b_1+a_2\,b_2+a_3\,b_3$, and $a_{ij}\,b_j\equiv a_{i1}\,b_1+a_{i2}\,b_2+a_{i3}\,b_3$. 
Note that when an index is summed it becomes a dummy index, and can be written as any
(unique) symbol: {\em i.e.}, $a_{ij}\,b_j$ and $a_{ip}\,b_p$ are equivalent. 
Moreover, only non-summed, or {\em free},  indices  count toward the order of a tensor expression. Thus,
$a_{ii}$ is a zeroth-order tensor (because there are no free indices), and $a_{ij}\,b_j$ is a first-order tensor (because there
is only one free index). The process of reducing the order of a tensor expression by summing indices is known
as {\em contraction}. For example, $a_{ii}$ is a zeroth-order contraction of the second-order tensor $a_{ij}$. 
 Incidentally, when two tensors are multiplied together without contraction the
resulting tensor is called an {\em outer product}: {\em e.g.}, the second-order tensor $a_i\,b_j$ is the
outer product of the two first-order tensors $a_i$ and $b_i$. Likewise, when two tensors are multiplied
together in a manner that  involves contraction then the resulting tensor is called an {\em inner product}:
{\em e.g.}, the first-order tensor $a_{ij}\,b_j$ is an inner product of the second-order tensor $a_{ij}$ and
the first-order tensor $b_i$. Note, from Equation~(\ref{e3.2}), that the scalar product of two
vectors is equivalent to the inner product of the corresponding first-order tensors. 

According to Section~\ref{svecp}, the {\em vector product}\/ of two vectors ${\bf a}$ and ${\bf b}$
takes the form
\begin{eqnarray}
({\bf a}\times {\bf b})_1 &=& a_2\,b_3-a_3\,b_2,\\[0.5ex]
({\bf a}\times {\bf b})_2 &=& a_3\,b_1-a_1\,b_3,\\[0.5ex]
({\bf a}\times {\bf b})_3 &=&a_1\,b_2-a_2\,b_1
\end{eqnarray}
in tensor notation. 
The above expression can be written more compactly as
\begin{equation}\label{e3.6}
({\bf a}\times {\bf b})_i = \epsilon_{ijk}\,a_j\,b_k.
\end{equation}
Here,
\begin{equation}
\epsilon_{ijk} = \left\{
\begin{array}{lll}
+1&\mbox{\hspace{1cm}}& \mbox{if $i, j, k$ is an even permutation of $1, 2, 3$}\\[0.5ex]
-1&&\mbox{if $i, j, k$ is an odd permutation of $1, 2, 3$}\\[0.5ex]
0&&\mbox{otherwise}
\end{array}\right.\label{e3.7}
\end{equation}
is known as the third-order {\em permutation tensor}\/ (or, sometimes, the third-order Levi-Civita tensor). Note, in particular, that $\epsilon_{ijk}$ is zero if one of its indices is
repeated: {\em e.g.}, $\epsilon_{113}=\epsilon_{212}=0$. 
Furthermore, it follows from (\ref{e3.7}) that
\begin{equation}\label{e3.8}
\epsilon_{ijk}=\epsilon_{jki}=\epsilon_{kij}=-\epsilon_{kji}=-\epsilon_{jik}=-\epsilon_{ikj}.
\end{equation}

It is helpful to define the second-order {\em identity tensor}\/ (also known as the Kroenecker delta tensor):
 \begin{equation}
\delta_{ij} = \left\{
\begin{array}{lll}
1&\mbox{\hspace{1cm}}& \mbox{if $i=j$}\\[0.5ex]
0&&\mbox{otherwise}
\end{array}\right. .
\end{equation}
It is easily seen that
\begin{eqnarray}
\delta_{ij} &=& \delta_{ji},\label{e3.10}\\[0.5ex]
\delta_{ii} &=&3,\\[0.5ex]
\delta_{ik}\,\delta_{kj} &=& \delta_{ij},\label{e3.11}\\[0.5ex]
\delta_{ij}\,a_j &=& a_i,\label{e3.11a}\\[0.5ex]
\delta_{ij}\,a_i\,b_j &=& a_i\,b_i,\label{e3.11b}\\[0.5ex]
\delta_{ij}\,a_{ki}\,b_j &=& a_{ki}\,b_i,
\end{eqnarray}
{\em etc.} 

The following is a particularly important tensor identity:
\begin{equation}\label{e3.12}
\epsilon_{ijk}\,\epsilon_{ilm} = \delta_{jl}\,\delta_{km}-\delta_{jm}\,\delta_{kl}.
\end{equation}
In order to establish the validity of the above expression, let us consider the various cases that arise. 
As is easily seen, the right-hand side of (\ref{e3.12}) takes the values
\begin{eqnarray}
+1&\mbox{\hspace{1cm}}&\mbox{if $j=l$ and $k=m\neq j$},\label{e3.13}\\[0.5ex]
-1&&\mbox{if $j=m$ and $k=l\neq j$},\label{e3.14}\\[0.5ex]
0&&\mbox{otherwise}.\label{e3.15}
\end{eqnarray}
Moreover, in each product on the left-hand side, $i$ has the same value in both $\epsilon$  factors. Thus,
for a non-zero contribution,
none of $j$, $k$, $l$, and $m$ can have the same value as $i$ (because each $\epsilon$ factor is zero if any of its
indices are repeated). Since a given subscript
can only take one of three values ($1$, $2$, or $3$), the only possibilities that
generate non-zero contributions are $j=l$ and $k=m$, or $j=m$ and $k=l$, excluding $j=k=l=m$ (since 
each $\epsilon$ factor would then have repeated indices, and so be zero). Thus, the left-hand side reproduces (\ref{e3.15}), as well as the conditions on the indices  in
(\ref{e3.13}) and (\ref{e3.14}). The left-hand side also reproduces the values in (\ref{e3.13}) and (\ref{e3.14}) since if $j=l$ and $k=m$ then $\epsilon_{ijk}=\epsilon_{ilm}$
and the product $\epsilon_{ijk}\,\epsilon_{ilm}$ (no summation) is equal to $+1$, whereas
if $j=m$ and $k=l$ then $\epsilon_{ijk}=\epsilon_{iml}=-\epsilon_{ilm}$ and the
 product $\epsilon_{ijk}\,\epsilon_{ilm}$ (no summation) is equal to $-1$. Here, use has been made of Equation~(\ref{e3.8}).
Hence, the validity of the identity (\ref{e3.12}) has been established. 

In order to illustrate the use of (\ref{e3.12}), consider the vector triple product
identity (see Section~\ref{svtp})
\begin{equation}\label{e3.17}
{\bf a}\times ({\bf b}\times {\bf c}) \equiv ({\bf a}\cdot{\bf c})\,{\bf b} - ({\bf a}\cdot{\bf b})\,{\bf c}.
\end{equation}
In tensor notation, the left-hand side of this identity is written
\begin{equation}
[{\bf a}\times ({\bf b}\times {\bf c})]_i = \epsilon_{ijk}\,a_j\,(\epsilon_{klm}\,b_l\,c_m),
\end{equation}
where use has been made of Equation~(\ref{e3.6}).  Employing Equations~(\ref{e3.8}) and (\ref{e3.12}), this
becomes
\begin{equation}
[{\bf a}\times ({\bf b}\times {\bf c})]_i = \epsilon_{kij}\,\epsilon_{klm}\,a_j\,b_l\,c_m = 
\left(\delta_{il}\,\delta_{jm}-\delta_{im}\,\delta_{jl}\right)a_j\,b_l\,c_m,
\end{equation}
which, with the aid of Equations~(\ref{e3.2}) and (\ref{e3.11a}),  reduces to
\begin{equation}
[{\bf a}\times ({\bf b}\times {\bf c})]_i = a_j\,c_j\,b_i - a_j\,b_j\,c_i = \left[({\bf a}\cdot{\bf c})\,{\bf b} - ({\bf a}\cdot{\bf b})\,{\bf c}\right]_i.
\end{equation}
 Thus, we have established the validity of the vector identity (\ref{e3.17}). 
Moreover, our proof is much more rigorous than that given earlier (in Section~\ref{svtp}). 

\section{Tensor Transformation}\label{strans}
As we saw in Appendix~\ref{vector}, scalars and vectors are defined according to their
transformation properties under rotation of the coordinate axes. In fact, a scalar is
{\em invarient}\/ under rotation of the coordinate axes. On the other
hand, according to Equations~(\ref{e2.41z}) and (\ref{e3.6}), the components of a general vector ${\bf a}$ transform under an {\em infinitesimal}\/ 
rotation of the coordinate axes according to
\begin{equation}\label{e3.21}
a_i' = a_i + \epsilon_{ijk} \,\delta\theta_j\,a_k.
\end{equation}
Here, the $a_i$ are the components of the vector in the original coordinate system, the $a_i'$ are the
components in the rotated coordinate system, and the latter system is obtained from the former via
a combination of an infinitesimal rotation through an angle $\delta\theta_1$ about coordinate axis 1, an infinitesimal rotation through an angle $\delta\theta_2$ about axis 2, and an infinitesimal rotation through an angle $\delta\theta_3$ about axis 3. These three  rotations can take place in
{\em any}\/ order. Incidentally, a finite rotation can be built up out of a great many infinitesimal rotations, so if a vector
transforms properly under an infinitesimal rotation of the coordinate axes then it will also transform properly under a finite rotation. 

Equation~(\ref{e3.21}) can also be written
\begin{equation}\label{e3.22}
a_i' = {\cal R}_{ij}\,a_j,
\end{equation}
where
\begin{equation}\label{e3.21x}
{\cal R}_{ij}  = \delta_{ij} - \delta{\theta}_k\,\epsilon_{kij}
\end{equation}
is a {\em rotation matrix}\/ (which is not a tensor, since it is specific to the two coordinate systems it transforms
between).
To first-order in the $\delta\theta_i$,  Equation~(\ref{e3.22}) can be inverted to give
\begin{equation}\label{e3.24}
a_i = {\cal R}_{ji}\,a_j'.
\end{equation}
This follows because, to first-order in the $\delta\theta_i$,
\begin{eqnarray}
{\cal R}_{ik}\,{\cal R}_{jk}&=& (\delta_{ik}-\delta{\theta}_l\,\epsilon_{lik})\,(\delta_{jk}-\delta\theta_m\,\epsilon_{mjk})
= \delta_{ik}\,\delta_{jk}-\delta\theta_l\,\delta_{jk}\,\epsilon_{lik} -\delta\theta_m\,\delta_{ik}\,\epsilon_{mjk}\nonumber\\[0.5ex]
&=& \delta_{ij} -\delta\theta_l\,\epsilon_{lij}-\delta\theta_l\,\epsilon_{lji}=\delta_{ij},\label{e3.25}
\end{eqnarray} 
where the dummy index $m$ has been relabeled $l$, and use has been made of Equations~(\ref{e3.8}), (\ref{e3.10}), and (\ref{e3.11}). Likewise, it is easily demonstrated that
\begin{equation}
{\cal R}_{ki}\,{\cal R}_{kj } = \delta_{ij}.\label{e3.27}
\end{equation}
It can also be shown that, to first-order in the $\delta\theta_i$, 
\begin{equation}
\epsilon_{ijk}\,{\cal R}_{li}\,{\cal R}_{mj}\,{\cal R}_{nk} = \epsilon_{lmn}.
\end{equation}
This follows because
\begin{eqnarray}
\epsilon_{ijk}\,{\cal R}_{li}\,{\cal R}_{mj}\,{\cal R}_{nk} &=& \epsilon_{ijk}\,(\delta_{li}-\delta\theta_a\,\epsilon_{ali})\,(\delta_{mj}-\delta\theta_b\,\epsilon_{bmj})\,(\delta_{nk}-\delta\theta_c\,\epsilon_{cnk})\nonumber\\[0.5ex]
&=&\epsilon_{ijk}\left[\delta_{li}\,\delta_{mj}\,\delta_{nk}-\delta\theta_a\left(\epsilon_{ali}\,\delta_{mj}\,\delta_{nk}
+\epsilon_{amj}\,\delta_{li}\,\delta_{nk} +\epsilon_{ank}\,\delta_{li}\,\delta_{mj}\right)\right]\nonumber\\[0.5ex]
&=& \epsilon_{lmn} -\delta\theta_a\left(\epsilon_{imn}\,\epsilon_{ial}+\epsilon_{inl}\,\epsilon_{iam}
+\epsilon_{ilm}\,\epsilon_{ian}\right)\nonumber\\[0.5ex]
&=& \epsilon_{lmn}-\delta\theta_a\left(\delta_{ma}\,\delta_{nl}-\delta_{ml}\,\delta_{na}+\delta_{na}\,\delta_{lm}-\delta_{nm}\,\delta_{la}+\delta_{la}\,\delta_{mn}-\delta_{ln}\,\delta_{ma}\right)\nonumber\\[0.5ex]
&=&\epsilon_{lmn}.\label{e3.30}
\end{eqnarray}
Here, there has been much relabeling of dummy indices, and use has been made of Equations~(\ref{e3.10}) and (\ref{e3.12}). It can similarly be shown that
\begin{equation}
\epsilon_{ijk}\,{\cal R}_{il}\,{\cal R}_{jm}\,{\cal R}_{kn} = \epsilon_{lmn}.
\end{equation}

As a direct generalization of Equation~(\ref{e3.22}), a second-order tensor transforms under
rotation as
\begin{equation}
a_{ij}' = {\cal R}_{ik}\,{\cal R}_{jl}\,a_{kl},
\end{equation}
whereas a third-order tensor transforms as
\begin{equation}
a_{ijk}' = {\cal R}_{il}\,{\cal R}_{jm}\,{\cal R}_{kn}\,a_{lmn}.
\end{equation}
The generalization to higher-order tensors is straight-forward. For the  case of a scalar, which is a zeroth-order
tensor, the transformation rule is particularly simple: {\em i.e.}, 
\begin{equation}
a' = a.
\end{equation}
By analogy with Equation~(\ref{e3.24}), the inverse transform is
exemplified by 
\begin{equation}
a_{ijk} = {\cal R}_{li}\,{\cal R}_{mj}\,{\cal R}_{nk}\,a_{lmn}'.
\end{equation}
Incidentally, since all tensors of the same order transform in the same manner, it immediately
follows that two tensors of the same order whose components are equal in one particular Cartesian coordinate system
will have their components equal in {\em all}\/ coordinate systems that can be obtained from the original
system via rotation of the coordinate axes. In other words,
if
\begin{equation}
a_{ij} = b_{ij}
\end{equation}
in one particular Cartesian coordinate system then
\begin{equation}
a_{ij}'=b_{ij}'
\end{equation}
in all Cartesian coordinate systems (with the same origin and system of units as the original system).
Conversely, it does not make sense to equate tensors of different order, since such an equation would
only be valid in one particular coordinate system, and so could not have any physical significance (since the
laws of physics are coordinate independent). 

It can easily be shown that the outer product of two tensors transforms as a tensor of the appropriate order.
Thus, if
\begin{equation}
c_{ijk} = a_i\,b_{jk},
\end{equation}
and
\begin{eqnarray}
a_i' &=& {\cal R}_{ij}\,a_j,\\[0.5ex]
b_{ij}'&=&{\cal R}_{ik}\,{\cal R}_{jl}\,b_{kl},
\end{eqnarray}
then
\begin{eqnarray}
c_{ijk}' &=& a_i'\,b_{jk}' = {\cal R}_{il}\,a_l\,{\cal R}_{jm}\,{\cal R}_{kn}\,b_{mn} = {\cal R}_{il}\,{\cal R}_{jm}\,{\cal R}_{kn}\,a_l\,b_{mn}\nonumber\\[0.5ex]
&=&{\cal R}_{il}\,{\cal R}_{jm}\,{\cal R}_{kn}\,c_{lmn},
\end{eqnarray}
which is the correct transformation rule for a third-order tensor.

The tensor transformation rule  can be combined with the identity (\ref{e3.27}) to show that the
scalar product of two vectors transforms as a scalar. Thus,
\begin{equation}
a_i'\,b_i' = {\cal R}_{ij}\,a_j\,{\cal R}_{ik}\,b_k ={\cal R}_{ij}\,{\cal R}_{ik}\,a_j\,b_k = \delta_{jk}\,a_j\,b_k = a_j\,b_j=a_i\,b_i,
\end{equation}
where use has been made of Equation~(\ref{e3.11b}). Again, the above proof is more rigorous than that given
previously (in Section~\ref{sscalar}). The proof also indicates that the inner product of
two tensors transforms as a tensor of the appropriate order. 

The result that both the inner and outer products of two tensors transform as  tensors
of the appropriate order is known as the {\em product rule}.
Closely related to this rule is the so-called {\em quotient rule}, according to which 
if (say)
\begin{equation}
c_{ij} = a_{ik}\,b_{jk},
\end{equation}
where $b_{jk}$ is an {\em arbitrary}\/ tensor, 
and $c_{ij}$  transforms as a tensor under {\em all}\/ rotations
of the coordinate axes, then $a_{ik}$---which can be thought of as the quotient of $c_{ij}$ and $b_{jk}$---also transforms as a tensor.
The proof is as follows:
\begin{eqnarray}
a_{ik}'\,b_{jk}'&=&c_{ij}' = {\cal R}_{il}\,{\cal R}_{jm}\,c_{lm}= {\cal R}_{il}\,{\cal R}_{jm}\,a_{lk}\,b_{mk}={\cal R}_{il}\,{\cal R}_{jm}\,a_{lk}\,{\cal R}_{pm}\,{\cal R}_{qk}\,b_{pq}'
\nonumber\\[0.5ex]
&=& {\cal R}_{il}\,{\cal R}_{qk}\,a_{lk}\,b_{jq} = {\cal R}_{il}\,{\cal R}_{km}\,a_{lm}\,b_{jk}',
\end{eqnarray}
where use has been made of the fact that $c_{ij}$ and  $b_{ij}$ transform as tensors, as well as Equation~(\ref{e3.25}).
Rearranging, we obtain
\begin{equation}
(a_{ik}'- {\cal R}_{il}\,{\cal R}_{km}\,a_{lm})\,b_{jk}' = 0.
\end{equation}
However, the $b_{ij}'$ are arbitrary, so the above equation can only be satisfied, in general, if
\begin{equation}
a_{ik}' = {\cal R}_{il}\,{\cal R}_{km}\,a_{lm},
\end{equation}
which is the correct transformation rule for a tensor.
Incidentally, the quotient rule applies to any type of valid tensor product.

The components of the second-order identity tensor, $\delta_{ij}$, have the special property that they
are invariant under rotation of the coordinate axes. This follows because
\begin{equation}
\delta_{ij}' = {\cal R}_{ik}\,{\cal R}_{jl}\,\delta_{kl}= {\cal R}_{ik}\,{\cal R}_{jk} = \delta_{ij},
\end{equation}
where use has been made of Equation~(\ref{e3.25}). The components of the third-order permutation
tensor, $\epsilon_{ijk}$, also have this special property. This follows because
\begin{equation}
\epsilon_{ijk}' = {\cal R}_{il}\,{\cal R}_{jm}\,{\cal R}_{ln}\,\epsilon_{lmn} = \epsilon_{ijk},
\end{equation}
where use has been made of Equation~(\ref{e3.30}). The fact that $\epsilon_{ijk}$ transforms as a proper
third-order tensor immediately implies, from the product rule,
that the vector product of two vectors transforms as a proper vector: {\em i.e.},
$\epsilon_{ijk}\,a_j\,b_k$ is a first-order tensor provided that $a_i$ and $b_i$ are both first-order tensors.
This proof is much more rigorous that that given earlier (in Section~\ref{svecp}).

\section{Tensor Fields}\label{stfield}
We saw in Appendix~\ref{vector} that  a {\em scalar field}\/ is a set of scalars  associated with every point in space: {\em e.g.},
$\phi({\bf x})$, where ${\bf x}= (x_1,\,x_2,\,x_3)$ is a position vector. We also saw that   a {\em vector field}\/
 is a set of vectors  associated with every point in space: {\em e.g.}, $a_i({\bf x})$. It stands to reason, then, that
a {\em tensor field}\/  is a set of tensors  associated with every point in space: {\em e.g.}, $a_{ij}({\bf x})$. 
It immediately follows that a scalar field is a zeroth-order tensor field, and a vector field is a first-order
tensor field.

Most tensor field encountered in physics are smoothly varying and {\em differentiable}. Consider the first-order tensor field $a_i({\bf x})$. 
The various partial derivatives of the components of this field with respect to the Cartesian coordinates $x_i$ are written
\begin{equation}
\frac{\partial a_i}{\partial x_j}.
\end{equation}
Moreover, this set of derivatives transform as the components of a  second-order tensor. In order to demonstrate this, we need the transformation
rule for the $x_i$, which is the same as that for a first-order tensor: {\em i.e.},
\begin{equation}
x_i' = {\cal R}_{ij}\,x_j.
\end{equation}
Thus,
\begin{equation}
\frac{\partial x_i'}{\partial x_j} = {\cal R}_{ij}.
\end{equation}
It is also easily shown that
\begin{equation}\label{e3.52}
\frac{\partial x_i}{\partial x_j'} = {\cal R}_{ji}.
\end{equation}
Now,
\begin{equation}
\frac{\partial a_i'}{\partial x_j'} = \frac{\partial a_i'}{\partial x_k}\,\frac{\partial x_k}{\partial x_j'}
= \frac{\partial ({\cal R}_{il}\,a_l)}{\partial x_k}\,{\cal R}_{jk}= {\cal R}_{il}\,{\cal R}_{jk}\,\frac{\partial a_l}{\partial x_k},
\end{equation}
which is the correct transformation rule for a second-order tensor. Here, use has been made of the chain rule, as
well as Equation~(\ref{e3.52}). [Note, from Equation~(\ref{e3.21x}), that the ${\cal R}_{ij}$ are not
functions of position.]
It follows, from the above argument, that differentiating a
tensor field increases its order by one: {\em e.g.}, $\partial a_{ij}/\partial x_k$ is a third-order tensor. The only
exception to this rule occurs when  differentiation and contraction are combined. Thus, $\partial a_{ij}/\partial x_j$
is a first-order tensor, since it only contains a single free index.  

The {\em gradient}\/ (see Section~\ref{sgrad}) of a scalar field is an example of a first-order tensor field ({\em i.e.}, a vector field): 
\begin{equation}
(\nabla \phi)_i = \frac{\partial \phi}{\partial x_i}.
\end{equation}
The {\em divergence}\/ (see Section~\ref{sdiv}) of a vector field is a contracted second-order tensor field that transforms as a scalar:
\begin{equation}
\nabla\cdot {\bf a} = \frac{\partial a_i}{\partial x_i}.
\end{equation}
Finally, the {\em curl}\/ (see Section~\ref{scurl}) of a vector field is a contracted fifth-order tensor that transforms as a vector
\begin{equation}
(\nabla\times {\bf a})_i = \epsilon_{ijk}\,\frac{\partial a_k}{\partial x_j}.
\end{equation}


The above definitions can be used to prove a number of useful results. 
For instance,
\begin{equation}
(\nabla\times \nabla\phi)_i=\epsilon_{ijk}\,\frac{\partial }{\partial x_j}\!\left(\frac{\partial \phi}{\partial x_k}\right)= \epsilon_{ijk}\,\frac{\partial^2\phi}{\partial x_j\,\partial x_k}=0,
\end{equation}
which follows from symmetry because $\epsilon_{ikj}=-\epsilon_{ijk}$ whereas $\partial^2\phi/\partial x_k\,\partial x_j
= \partial^2\phi/\partial x_j\,\partial x_k$. Likewise,
\begin{equation}
\nabla\cdot( \nabla\times {\bf a}) = \frac{\partial}{\partial x_i}\!\left(\epsilon_{ijk}\,\frac{\partial a_k}{\partial x_j}\right)= \epsilon_{ijk}\,\frac{\partial a_k}{\partial x_i\,\partial x_j} = 0,
\end{equation}
which again follows from symmetry. As a final example,
\begin{eqnarray}
\nabla\cdot ({\bf a}\times {\bf b}) &=& \frac{\partial}{\partial x_i}\!\left(\epsilon_{ijk}\,a_j\,b_k\right)
= \epsilon_{ijk}\,\frac{\partial a_j}{\partial x_i}\,b_k + \epsilon_{ijk}\,a_j\,\frac{\partial b_k}{\partial x_i}\nonumber\\[0.5ex]
&=&b_i\, \epsilon_{ijk}\,\frac{\partial a_k}{\partial x_j}-a_i\,\epsilon_{ijk} \,\frac{\partial b_k}{\partial x_j}
={\bf b}\cdot(\nabla \times {\bf a}) - {\bf a}\cdot(\nabla\times {\bf b}).
\end{eqnarray}

According to the {\em divergence theorem}\/ (see Section~\ref{sdiv}),
\begin{equation}
\oint_S a_i\,dS_i = \int_V \frac{\partial a_i}{\partial x_i}\,dV,
\end{equation}
where $S$ is a closed surface surrounding the volume $V$. The above theorem is
easily generalized to give, for example, 
\begin{equation}
\oint_S a_{ij}\,dS_i = \int_V \frac{\partial a_{ij}}{\partial x_i}\,dV,
\end{equation}
or
\begin{equation}
\oint_S a_{ij}\,dS_j = \int_V \frac{\partial a_{ij}}{\partial x_j}\,dV,
\end{equation}
or even
\begin{equation}
\oint_S a\,dS_i = \int_V \frac{\partial a}{\partial x_i}\,dV.
\end{equation}

\section{Isotropic Tensors}\label{siso}
A tensor which has the special property that its components take the {\em same}\/ value in all
Cartesian coordinate systems is called an {\em isotropic tensor}. We have already encountered two
such tensors: namely, the second-order identity tensor, $\delta_{ij}$, and the third-order
permutation tensor, $\epsilon_{ijk}$. Of course, all scalars are isotropic. Moreover, as is easily
demonstrated, there are
no isotropic vectors (other than the null vector). 
It turns out that the
most general isotropic Cartesian tensors of second-, third-, and fourth-order are $\lambda\,\delta_{ij}$, $\mu\,\epsilon_{ijk}$,
and $\alpha\,\delta_{ij}\,\delta_{kl} + \beta\,\delta_{ik}\,\delta_{jl} + \gamma\,\delta_{il}\,\delta_{jk}$, respectively,
where $\lambda$, $\mu$, $\alpha$, $\beta$, and $\gamma$ are scalars. Let us prove these important results.\footnote{This proof
is adapted from P.G.~Hodge, Jr., American Mathematical Monthly {\bf 68}, 793 (1961).} 

The most general second-order isotropic tensor, $a_{ij}$,  is such that
\begin{equation}\label{e3.60}
a_{ij}' = {\cal R}_{ip}\,{\cal R}_{jq}\,a_{pq} = a_{ij}
\end{equation}
for arbitrary rotations of the coordinate axes. 
It follows from Equation~(\ref{e3.21}) that, to first-order in the $\delta\theta_i$, 
\begin{equation}
\delta \theta_m \left(\epsilon_{mis}\,a_{sj} + \epsilon_{mjs}\,a_{is}\right) = 0.
\end{equation}
However, the $\delta\theta_i$ are arbitrary, so we can write
\begin{equation}\label{e3.62}
\epsilon_{mis}\,a_{sj} + \epsilon_{mjs}\,a_{is}=0.
\end{equation}
Let us multiply by $\epsilon_{mik}$. With the aid of Equation~(\ref{e3.12}), we obtain
\begin{equation}
(\delta_{ii}\,\delta_{ks}-\delta_{is}\,\delta_{ki})\,a_{sj} + (\delta_{ij}\,\delta_{ks} -\delta_{is}\,\delta_{kj})\,a_{is} = 0,
\end{equation}
which reduces to 
\begin{equation}
2\,a_{ij} + a_{ji} = a_{ss}\,\delta_{ij}.
\end{equation}
Interchanging the labels $i$ and $j$, and then taking the difference between the two equations thus obtained, we deduce that
\begin{equation}
a_{ij} = a_{ji}.
\end{equation}
Hence,
\begin{equation}
a_{ij} = \frac{a_{ss}}{3}\,\delta_{ij},
\end{equation}
which implies that
\begin{equation}
a_{ij } =\lambda\,\delta_{ij}.
\end{equation}

For the case of an isotropic third-order tensor, Equation~(\ref{e3.62}) generalizes to
\begin{equation}\label{e3.68}
\epsilon_{mis}\,a_{sjk} + \epsilon_{mjs}\,a_{isk}+\epsilon_{mks}\,a_{ijs}=0.
\end{equation}
Multiplying by $\epsilon_{mit}$,  $\epsilon_{mjt}$, and $\epsilon_{mkt}$, and then
setting $t=i$, $t=j$, and $t=k$, respectively, we obtain
\begin{eqnarray}
2\,a_{ijk} + a_{jik} + a_{kji} &=&a_{ssk}\,\delta_{ij} + a_{sjs}\,\delta_{ik},\\[0.5ex]
2\,a_{ijk} + a_{jik} + a_{ikj} &=& a_{ssk}\,\delta_{ij} + a_{iss}\,\delta_{jk},\\[0.5ex]
2\,a_{ijk} +a_{kji} +a_{ikj} &=&a_{sjs}\,\delta_{ik} + a_{iss}\,\delta_{jk},
\end{eqnarray}
respectively. However, multiplying the above equations by $\delta_{jk}$, $\delta_{ik}$, and $\delta_{ij}$, and
then setting $i=i$, $j=i$, and $k=i$, respectively, we
obtain
\begin{eqnarray}
2\,a_{iss} + a_{sis} + a_{ssi} &=&a_{ssi}+a_{sis},\\[0.5ex]
2\,a_{sis} + a_{iss} + a_{ssi} &=& a_{ssi}+a_{iss},\\[0.5ex]
2\,a_{ssi} +a_{iss} +a_{sis} &=&a_{sis} + a_{iss},
\end{eqnarray}
respectively,
which implies that
\begin{equation}
a_{iss}=a_{sis}= a_{ssi} = 0.
\end{equation}
Hence, we deduce that
\begin{eqnarray}
2\,a_{ijk} + a_{jik} + a_{kji} &=&0,\\[0.5ex]
2\,a_{ijk} + a_{jik} + a_{ikj} &=& 0,\\[0.5ex]
2\,a_{ijk} +a_{kji} +a_{ikj} &=&0.
\end{eqnarray}
The solution to the above equation must satisfy
\begin{equation}
a_{ikj} = a_{jik}=a_{kji} = -a_{ijk}.
\end{equation}
This implies, from Equation~(\ref{e3.8}), that
\begin{equation}
a_{ijk} = \mu\,\epsilon_{ijk}.
\end{equation}

For the case of an isotropic fourth-order tensor, Equation~(\ref{e3.68}) generalizes to
\begin{equation}
\epsilon_{mis}\,a_{sjkl} + \epsilon_{mjs}\,a_{iskl}+\epsilon_{mks}\,a_{ijsl} +\epsilon_{mls}\,a_{ijks}=0.
\end{equation}
Multiplying the above by $\epsilon_{mit}$,  $\epsilon_{mjt}$, $\epsilon_{mkt}$, $\epsilon_{mlt}$, and then
setting $t=i$, $t=j$, $t=k$, and $t=l$, respectively, we obtain
\begin{eqnarray}
2\,a_{ijkl} + a_{jikl} + a_{kjil}+a_{ljki} &=&a_{sskl}\,\delta_{ij} + a_{sjsl}\,\delta_{ik}+ a_{sjks}\,\delta_{il},\\[0.5ex]
2\,a_{ijkl} + a_{jikl} + a_{ikjl} +a_{iljk}&=& a_{sskl}\,\delta_{ij} +a_{isks}\,\delta_{jl}+ a_{issl}\,\delta_{jk},\\[0.5ex]
2\,a_{ijkl} +a_{kjil} +a_{ikjl}+a_{ijlk} &=&a_{ijss}\,\delta_{kl}+a_{sjsl}\,\delta_{ik} + a_{issl}\,\delta_{jk},\\[0.5ex]
2\,a_{ijkl}+a_{ljki} +a_{ilkj}+a_{ijlk} &=&a_{ijss}\,\delta_{kl}+a_{isks}\,\delta_{jl}+a_{sjks}\,\delta_{il},
\end{eqnarray}
respectively.
Now, if $a_{ijkl}$ is an isotropic fourth-order tensor then $a_{sskl}$ is clearly an isotropic second-order tensor, which means that is a multiple of $\delta_{kl}$. This, and similar arguments, allows us to deduce that
\begin{eqnarray}
a_{sskl} &=&\lambda\,\delta_{kl},\\[0.5ex]
a_{sjsl} &=&\mu\,\delta_{jl},\\[0.5ex]
a_{sjks} &=& \nu\,\delta_{jk}.
\end{eqnarray}
Let us assume, for the moment, that
\begin{eqnarray}
a_{ijss} &=&a_{ssij},\label{e3.91}\\[0.5ex]
a_{isks} &=&a_{sisk},\\[0.5ex]
a_{issl} &=& a_{sils}.\label{e3.91x}
\end{eqnarray}
Thus, we get 
\begin{eqnarray}\label{e3.92a}
2\,a_{ijkl} + a_{jikl} +a_{kjil} +a_{ljki} &=&\lambda\,\delta_{ij}\,\delta_{kl} + \mu\,\delta_{ik}\,\delta_{jl} 
+\nu\,\delta_{il}\,\delta_{jk},\\[0.5ex]
2\,a_{ijkl} + a_{jikl} +a_{ikjl} +a_{ilkj}&=&\lambda\,\delta_{ij}\,\delta_{kl} + \mu\,\delta_{ik}\,\delta_{jl} 
+\nu\,\delta_{il}\,\delta_{jk},\\[0.5ex]
2\,a_{ijkl} +a_{kjil} +a_{ikjl} +a_{ijlk} &=&\lambda\,\delta_{ij}\,\delta_{kl} + \mu\,\delta_{ik}\,\delta_{jl} 
+\nu\,\delta_{il}\,\delta_{jk},\\[0.5ex]
2\,a_{ijkl} +a_{ljki} +a_{ilkj} +a_{ijlk} &=&\lambda\,\delta_{ij}\,\delta_{kl} + \mu\,\delta_{ik}\,\delta_{jl} 
+\nu\,\delta_{il}\,\delta_{jk}.\label{e3.92}
\end{eqnarray}
Relations of the form
\begin{equation}\label{e3.93}
a_{ijkl} = a_{jilk} = a_{klij}=a_{lkji}
\end{equation}
can be obtained by subtracting the sum of one pair of Equations~(\ref{e3.92a})--(\ref{e3.92}) from the sum of the other pair.
These relations justify Equations~(\ref{e3.91})--(\ref{e3.91x}). 
Equations~(\ref{e3.92a}) and (\ref{e3.93}) can be combined to give
\begin{eqnarray}\label{e3.97}
2\,a_{ijkl}+(a_{ijlk}+a_{ikjl}+a_{ilkj})&=&\lambda\,\delta_{ij}\,\delta_{kl} + \mu\,\delta_{ik}\,\delta_{jl}
+\nu\,\delta_{il}\,\delta_{jk},\\[0.5ex]
2\,a_{iklj}+(a_{ikjl}+a_{ilkj}+a_{ijlk})&=&\lambda\,\delta_{ik}\,\delta_{jl} + \mu\,\delta_{il}\,\delta_{jk}
+\nu\,\delta_{ij}\,\delta_{kl},\\[0.5ex]
2\,a_{iljk}+(a_{ilkj}+a_{ijlk}+a_{ikjl})&=&\lambda\,\delta_{il}\,\delta_{jk} + \mu\,\delta_{ij}\,\delta_{kl}
+\nu\,\delta_{ik}\,\delta_{jl}.\label{e3.99}
\end{eqnarray}
The latter two equations are obtained from the first via cyclic permutation of $j$, $k$, and $l$, with $i$ remaining unchanged.
Summing Equations (\ref{e3.97})--(\ref{e3.99}), we get
\begin{equation}
2\,(a_{ijkl}+a_{iklj}+a_{iljk}) + 3\,(a_{ijlk}+a_{ikjl} +a_{ilkj}) =(\lambda+\mu+\nu)\,(\delta_{ij}\,\delta_{kl}
+\delta_{ik}\,\delta_{jl}+\delta_{il}\,\delta_{jk}).
\end{equation}
It follows from symmetry that
\begin{equation}
a_{ijkl}+a_{iklj}+a_{iljk}= a_{ijlk}+a_{ikjl}+a_{ilkj} = \frac{1}{5}\,(\lambda+\mu+\nu)\,(\delta_{ij}\,\delta_{kl}
+\delta_{ik}\,\delta_{jl}+\delta_{il}\,\delta_{jk}).
\end{equation}
This  can be seen by swapping the indices $k$ and $l$ in the above expression.
Finally, substitution into Equation~(\ref{e3.97}) yields
\begin{equation}
a_{ijkl} = \alpha\,\delta_{ij}\,\delta_{kl} + \beta\,\delta_{ik}\,\delta_{jl}
+\gamma\,\delta_{il}\,\delta_{jk},
\end{equation}
where
\begin{eqnarray}
\alpha &=& (4\,\lambda-\mu-\nu)/10,\\[0.5ex]
\beta &=& (4\,\mu-\nu-\lambda)/10,\\[0.5ex]
\gamma &=& (4\,\nu-\lambda-\mu)/10.
\end{eqnarray}

\section{Exercises}
{\small 
\renewcommand{\theenumi}{B.\arabic{enumi}}
\begin{enumerate}
\item Show that a general second-order tensor $a_{ij}$ can be decomposed into three tensors
$$
a_{ij} = u_{ij} + v_{ij} + s_{ij},
$$
where $u_{ij}$ is symmetric ({\em i.e.}, $u_{ji}=u_{ij}$) and traceless ({\em i.e.}, $u_{ii}=0$), $v_{ij}$ is isotropic,
and $s_{ij}$ only has three independent components.

\item Use tensor methods to establish the following vector identities:
\begin{enumerate}
\item ${\bf a}\cdot{\bf b}\times {\bf c}={\bf a}\times {\bf b}\cdot{\bf c}=-{\bf b}\cdot{\bf a}\times{\bf c}$. 
\item $({\bf a}\times {\bf b})\times {\bf c} = ({\bf a}\cdot{\bf c})\,{\bf b} - ({\bf c}\cdot{\bf b})\,{\bf a}$.
\item $({\bf a}\times {\bf b})\cdot({\bf c}\times {\bf d})=
({\bf a}\cdot{\bf c})\,({\bf b}\cdot{\bf d}) - ({\bf a}\cdot{\bf d})\,({\bf b}\cdot{\bf c})$.
\item $({\bf a}\times {\bf b})\times ({\bf c}\times {\bf d})= ({\bf a}\times
{\bf b}\cdot{\bf d})\,{\bf c} - ({\bf a}\times {\bf b}\cdot{\bf c})\,{\bf d}$.
\item $\nabla\cdot(\phi\,{\bf a}) = \phi\,\nabla\cdot {\bf a} + {\bf a}\cdot\nabla \phi$. 
\item $\nabla\times (\phi\,{\bf a})= \phi\,\nabla\times {\bf a} + \nabla\phi\times {\bf a}$.
\item $\nabla\times({\bf a}\times {\bf b}) = ({\bf b}\cdot\nabla)\,{\bf a} - ({\bf a}\cdot\nabla)\,{\bf b} + (\nabla\cdot {\bf b})\,{\bf a}
-(\nabla\cdot{\bf a})\,{\bf b}$. 
\item $\nabla ({\bf a}\cdot{\bf a}) = 2\,{\bf a}\times (\nabla\times{\bf a}) + 2\,({\bf a}\cdot\nabla)\,{\bf a}$.
\item $\nabla\times(\nabla\times {\bf a}) = \nabla(\nabla\cdot{\bf a}) - \nabla^2 {\bf a}$.
\end{enumerate}
Here, $[({\bf b}\cdot{\nabla}){\bf a}]_i\equiv b_j\,\partial a_i/\partial x_j$, and $(\nabla^2 {\bf a})_i \equiv \nabla^2 a_i$. 

\item A {\em quadric}\/ surface has an equation of the form
$$
a\,x_1^{\,2} + b\,x_2^{\,2}+c\,x_3^{\,2} + 2\,f\,x_1\,x_2+2\,g\,x_1\,x_3+2\,h\,x_2\,x_3=1.
$$
Show that the coefficients in the above expression transform under rotation of
the coordinate axes like the components of a symmetric second-order tensor. Hence, demonstrate that the equation for the surface can be written in the form
$$
x_i\,T_{ij}\,x_j = 1,
$$
where the $T_{ij}$ are the components of the aforementioned tensor.

\item The {\em determinant}\/ of a second-order tensor $A_{ij}$ is defined 
$$
{\rm det}(A) = \epsilon_{ijk}\,A_{i1}\,A_{j2}\,A_{k3}.
$$
\begin{enumerate}
\item Show that 
$$
{\rm det}(A) = \epsilon_{ijk}\,A_{1i}\,A_{2j}\,A_{3k}
$$
is an alternative, and entirely equivalent, definition.
\item Demonstrate that ${\rm det}(A)$ is invariant under rotation of the coordinate axes.
\item Suppose that $C_{ij} = A_{ik}\,B_{kj}$. Show that
$$
{\rm det}(C) = {\rm det}(A)\,{\rm det}(B).
$$
\end{enumerate}

\item If
$$
A_{ij}\,x_j = \lambda\,x_i
$$
then $\lambda$ and $x_j$ are said to be  {\em eigenvalues}\/ and {\em eigenvectors}\/ of the second-order  tensor $A_{ij}$, respectively.
The eigenvalues of $A_{ij}$ are calculated by solving the related homogeneous matrix equation
$$
(A_{ij}-\lambda\,\delta_{ij})\,x_j = 0.
$$
Now, it is a standard result in linear algebra that an equation of the  above form  only has a non-trivial
solution when 
$$
{\rm det}(A_{ij}-\lambda\,\delta_{ij}) = 0.
$$
Demonstrate that the eigenvalues of $A_{ij}$ satisfy the cubic polynomial
$$
\lambda^3 - {\rm tr}(A) \,\lambda^2 + \Pi(A)\,\lambda - {\rm det}(A) = 0,
$$
where
${\rm tr}(A) = A_{ii}$ and $\Pi(A)= (A_{ii}\,A_{jj}-A_{ij}\,A_{ji})/2$. 
Hence, deduce that $A_{ij}$ possesses {\em three}\/ eigenvalues---$\lambda_1$, $\lambda_2$, and $\lambda_3$ (say).
Moreover, show that
\begin{eqnarray}
{\rm tr}(A) &=&\lambda_1+\lambda_2+\lambda_3,\nonumber\\[0.5ex]
{\rm det}(A)&=&\lambda_1\,\lambda_2\,\lambda_3.\nonumber
\end{eqnarray}

\item Suppose that $A_{ij}$ is a {\em symmetric}\/ second-order tensor: {\em i.e.}, $A_{ji}=A_{ij}$. 
\begin{enumerate}
\item Demonstrate that the eigenvalues of $A_{ij}$ are all {\em real}, and that the eigenvectors
can be chosen to be real. 
\item Show that eigenvectors of $A_{ij}$ corresponding to different eigenvalues
are {\em orthogonal}\/ to one another. Hence, deduce that the three eigenvectors of $A_{ij}$ are, or can be chosen to be, mutually orthogonal. 
\item Demonstrate that $A_{ij}$ takes the {\em diagonal}\/ form $A_{ij}=\lambda_i\,\delta_{ij}$ (no sum) in a
Cartesian coordinate system in which the coordinate axes are each parallel to one of the eigenvectors. 
\end{enumerate}

\item In an isotropic elastic medium under stress the displacement $u_i$ satisfies
\begin{eqnarray}
\frac{\partial\sigma_{ij}}{\partial x_j} &=&\rho\,\frac{\partial^2 u_i}{\partial t^2},\nonumber\\[0.5ex]
\sigma_{ij} &=&c_{ijkl}\,\frac{1}{2}\!\left(\frac{\partial u_k}{\partial x_l}+ \frac{\partial u_l}{\partial x_k}\right),\nonumber
\end{eqnarray}
where $\sigma_{ij}$ is the {\em stress tensor}\/ (note that $\sigma_{ji}=\sigma_{ij}$), $\rho$  the mass density
(which is a uniform constant), and 
$$
c_{ijkl} =  K\,\delta_{ij}\,\delta_{kl} + \mu\,[\delta_{ik}\,\delta_{jl}+\delta_{il}\,\delta_{jk} - (2/3)\,\delta_{ij}\,\delta_{kl}].
$$
the isotropic {\em stiffness tensor}.
Here, $K$ and $\mu$ are the {\em bulk modulus}\/ and {\em shear modulus}\/ of the medium, respectively.
Show that the divergence and the curl of ${\bf u}$ both satisfy wave equations. Furthermore, demonstrate that
the characteristic wave velocities of the divergence and curl waves are $[(K+4\mu/3)/\rho]^{1/2}$ and
$(\mu/\rho)^{1/2}$, respectively.

\end{enumerate}}
