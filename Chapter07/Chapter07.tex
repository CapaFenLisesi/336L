\chapter{Potential Flow}
\section{Introduction}
This chapter investigates uniform, high Reynolds number,  flow around  a stationary obstacle which is sufficiently streamlined that there
is no appreciable separation of the flow from its back surface. In this case, the flow is essentially irrotational and inviscid
everywhere apart from a thin boundary layer covering the obstacle's surface. It follows that, in the region outside the
boundary layer, we can write
\begin{equation}
{\bf v} = -\nabla\phi,
\end{equation}
which automatically ensures that the flow is irrotational. If the flow is also incompressible, so that $\nabla\cdot{\bf v}=0$,
then the velocity potential, $\phi$, satisfies Laplace's equation: {\em i.e.},
\begin{equation}
\nabla^2\phi = 0.
\end{equation}
This type of flow is known as {\em potential flow}, since it can be derived from a velocity potential. 
The appropriate boundary condition at the surface of the obstacle is that the normal flow velocity be zero. In other
words, ${\bf n}\cdot\nabla\phi=0$, where ${\bf n}$ is a unit  vector normal to the surface. In general, the
tangential velocity at the surface of the obstacle is non-zero. Of course, this is inconsistent with the {\em no slip condition}, which
demands that the tangential velocity be zero at the surface (see Section~\ref{snoslip}). 
However, as described in the previous chapter,  this inconsistency is resolved by the {\em boundary layer}, across which the tangential velocity is
effectively discontinuous, being non-zero on the outer edge of the layer (where it interfaces with the irrotational
flow), and zero on the inner edge (where it interfaces with the obstacle). The discontinuity in the tangential
velocity across the layer gives rise to a {\em friction drag}\/ acting on the obstacle in the direction of the
  external flow. However, the magnitude of this drag scales as ${\rm Re}^{-1/2}$,  where ${\rm Re}$
is the Reynolds number of the  flow. Hence, such drag becomes negligibly small in the high Reynolds number limit.
In the following, we shall assume that any {\em form drag}\/ due to residual separation of the flow at the back of
the obstacle is also negligibly small. Moreover, for the sake of simplicity, we shall mostly  restrict our
discussion to essentially {\em two-dimensional}\/ situations in which a high Reynolds number fluid moves 
transversely around a stationary obstacle of infinite length and uniform cross-section. 

\section{Velocity Potentials and Stream Functions}
Consider a {\em two-dimensional}\/ velocity field in which the flow is everywhere parallel to the $x$-$y$ plane, and there is
no variation along the $z$-direction. It follows that
\begin{equation}
{\bf v} = v_x(x,y,t)\,{\bf e}_x + v_y(x,y,t)\,{\bf e}_y.
\end{equation}
If the flow is {\em irrotational}\/ then $\nabla\times {\bf v}={\bf 0}$ is automatically satisfied by writing ${\bf v}=-\nabla\phi$, where
$\phi(x,y,t)$ is termed the {\em velocity potential} (see Section~\ref{qirr}). Hence, 
\begin{eqnarray}\label{ep7.4}
v_x &=& - \frac{\partial\phi}{\partial x},\\[0.5ex]
v_y&=&-\frac{\partial\phi}{\partial y}.\label{ep7.5}
\end{eqnarray}
On the other hand, if the flow is {\em incompressible}\/ then $\nabla\cdot{\bf v}=0$ is automatically
satisfied by writing ${\bf v} = \nabla z\times \nabla\psi$, where $\psi(x,y,t)$ is termed the
{\em stream function} (see Section~\ref{s2d}). Hence,
\begin{eqnarray}
v_x &=& - \frac{\partial\psi}{\partial y},\label{ep7.6}\\[0.5ex]
v_y&=&\frac{\partial\psi}{\partial x}.\label{ep7.7}
\end{eqnarray}
Finally, if the flow is both irrotational and incompressible then Equations~(\ref{ep7.4})--(\ref{ep7.5}) and
(\ref{ep7.6})--(\ref{ep7.7}) hold simultaneously, and  so
\begin{eqnarray}\label{ep7.8}
\frac{\partial\phi}{\partial x} &=&\frac{\partial\psi}{\partial y},\\[0.5ex]
\frac{\partial\psi}{\partial x}&=&-\frac{\partial\phi}{\partial y}.\label{ep7.9}
\end{eqnarray}
It immediately follows, from the previous two expressions, that
\begin{equation}
\frac{\partial^2\phi}{\partial x^2} = \frac{\partial^2\psi}{\partial x\,\partial y} = \frac{\partial^2\psi}{\partial y\,\partial x} = 
-\frac{\partial^2\phi}{\partial y^2},
\end{equation}
or
\begin{equation}
\frac{\partial^2\phi}{\partial x^2}+\frac{\partial^2\phi}{\partial y^2} = 0.
\end{equation}
Likewise, it can also be shown that
\begin{equation}
\frac{\partial^2\psi}{\partial x^2}+\frac{\partial^2\psi}{\partial y^2} = 0.
\end{equation}
We conclude that, for two-dimensional flow,  both the velocity potential and the stream function
satisfy Laplace's equation when the fluid is irrotational and incompressible. 
Equations~(\ref{ep7.8}) and (\ref{ep7.9}) also imply that
\begin{equation}
\nabla\phi\cdot\nabla\psi = 0:
\end{equation}
{\em i.e.},  the contours of the velocity potential and the stream function cross at right-angles.

\section{Complex Velocity Potential}
The complex variable is conventionally written
\begin{equation}
z = x + {\rm i}\,y,
\end{equation}
where ${\rm i}$ represents the square root of minus one. Here, $x$ and $y$ are both real, and are identified with the corresponding
Cartesian coordinates.
(Incidentally, $z$ should not be confused with a $z$-coordinate: this is a strictly two-dimensional discussion.) 
We can also write
\begin{equation}
z = r\,{\rm e}^{\,{\rm i}\,\theta}
\end{equation}
where $r=\sqrt{x^2+y^2}$ and $\theta=\tan^{-1}(y/x)$ are termed the modulus and
argument of $z$, respectively, but can also be identified with the corresponding cylindrical polar coordinates. 
Finally, De\,Moivre's theorem,
\begin{equation}
{\rm e}^{\,{\rm i}\,\theta}\equiv \cos\theta+{\rm i}\,\sin\theta,
\end{equation}
implies that
\begin{eqnarray}
x &=&r\,\cos\theta,\\[0.5ex]
y&=&r\,\sin\theta.
\end{eqnarray}

We can write functions  of the complex variable, $F(z)$, just like
we would write functions of a real variable. For instance,
\begin{eqnarray}
F(z) &=& z^2,\\[0.5ex]
F(z)&=& \frac{1}{z}.
\end{eqnarray}
For a given function, $F(z)$, we can substitute  $z=x +{\rm i}\,y$ and write
\begin{equation}\label{ek5.137}
F(z) = \phi(x, y) + {\rm i}\,\psi(x, y),
\end{equation}
where $\phi$ and $\psi$ are {\em real}\/ two-dimensional functions. Thus, if
\begin{equation}
F(z) = z^2,
\end{equation}
then
\begin{equation}
F(x + {\rm i}\,y) = (x+{\rm i}\,y)^2 = (x^2-y^2) + 2\,{\rm i}\, x\,y,
\end{equation}
giving
\begin{eqnarray}
\phi(x, y) &=& x^2 - y^2,\\[0.5ex]
\psi(x, y) &=& 2\, x\,y.
\end{eqnarray}

We can define the derivative of a complex function in just the same manner as
we would  define the derivative of a real function: {\em i.e.}, 
\begin{equation}\label{ek5.141}
\frac{dF}{dz} = ~_{\lim |\delta z|\rightarrow\infty}
\frac{F(z+\delta z) - F(z) }{\delta z}.
\end{equation}
However, we now have a slight problem. If $F(z)$ is a ``well-defined''
function ({\em i.e.}, finite, single-valued, and differentiable) then it should not matter from which direction in the complex
plane we approach $z$ when taking the limit in Equation~(\ref{ek5.141}).
 There are, of course, many
different directions we could approach $z$ from, but if we look at a regular complex
function, $F(z) = z^2$  (say), then
\begin{equation}
\frac{dF}{dz} = 2 \,z
\end{equation}
is perfectly well-defined, and is, therefore,  completely independent of the details of
how the limit is taken in Equation~(\ref{ek5.141}).

The fact that Equation~(\ref{ek5.141})
 has to give the same result, no matter from which direction we approach
$z$, means that there are some restrictions on the forms of the functions $\phi$ and $\psi$ in
 Equation~(\ref{ek5.137}).
Suppose that we approach $z$ along the real axis, so that $\delta z = \delta x$.
We obtain
\begin{eqnarray}\label{ep7.28x}
\frac{dF}{dz}& =& ~_{\lim |\delta x|\rightarrow 0}
\frac{\phi(x+\delta x, y) + {\rm i}\, \psi(x+\delta x, y) - \phi(x, y) - {\rm i}\,
\psi(x,y)}{\delta x} 
\nonumber\\[0.5ex]
&=& \frac{\partial \phi}{\partial x} + {\rm i}\, \frac{\partial \psi}
{\partial x}.
\end{eqnarray}
Suppose that we now approach $z$ along the imaginary axis, so that $\delta z
= {\rm i}\,\delta y$. We get
\begin{eqnarray}
\frac{dF}{dz} &=& ~_{\lim |\delta y|\rightarrow 0}
\frac{\phi(x, y+\delta y) + {\rm i}\, \psi(x, y+\delta y) - \phi(x, y) - {\rm i}\,
\psi(x,y)}{{\rm i}\,\delta y}  \nonumber\\[0.5ex]
&=&
-{\rm i}\,\frac{\partial \phi}{\partial y} + \frac{\partial \psi}
{\partial y}.
\end{eqnarray}
But, if $F(z)$ is a well-defined function then its derivative must also be 
well-defined,
which implies that the above two expressions are equivalent. This 
requires that
\begin{eqnarray}\label{ep7.27}
\frac{\partial \phi}{\partial x} &=& \frac{\partial \psi}{\partial y},\\[0.5ex]
\frac{\partial \psi}{\partial x} &=& -\frac{\partial \phi}{\partial y}.\label{ep7.28}
\end{eqnarray}
These expressions are called the {\em Cauchy-Riemann relations}, and are, in fact, sufficient to ensure 
that  all possible ways of taking the limit (\ref{ek5.141})  give the same answer.

Note that Equations~(\ref{ep7.27})--(\ref{ep7.28}) are identical to Equations~(\ref{ep7.8})--(\ref{ep7.9}).
This suggests that the real and imaginary parts of a well-behaved function of the complex variable
can be interpreted as the velocity potential and stream function, respectively, of some two-dimensional,
irrotational, incompressible, flow pattern. For instance,
suppose that
\begin{equation}\label{ep7.32}
F(z)=U\,z,
\end{equation}
where $U$ is real.  It
follows that
\begin{eqnarray}
\phi (r,\theta)&=&U\,\cos\theta,\\[0.5ex]
\psi(r,\theta)&=&U\,\sin\theta.
\end{eqnarray}
It can be seen, by comparison with the analysis in Section~\ref{suniform}, that the complex potential (\ref{ep7.32}) corresponds to {\em uniform flow}\/ of speed $U$ directed along the negative $x$-axis. 
As a simple generalization of this result, the complex potential
\begin{equation}
F(z) = U\,z\,{\rm e}^{\,{\rm i}\,\alpha}
\end{equation}
corresponds to uniform flow of speed $U$ which subtends a (clockwise) angle $\alpha$ with the negative $x$-axis (see Figure~\ref{funi}).

\begin{figure}
\epsfysize=3in
\centerline{\epsffile{Chapter07/uniform.eps}}
\caption{\em Uniform flow.}\label{funi}
\end{figure}

Suppose that
\begin{equation}\label{ep7.37}
F(z)= \frac{Q}{2\pi}\,\ln z,
\end{equation}
where $Q$ is real.
Since $\ln z \equiv \ln r + {\rm i}\,\theta$, it follows that
\begin{eqnarray}
\phi(r,\theta)&=&\frac{Q}{2\pi}\,\ln r,\\[0.5ex]
\psi(r,\theta)&=&\frac{Q}{2\pi}\,\theta.
\end{eqnarray}
Thus, according to the analysis in Section~\ref{ssource}, the complex potential (\ref{ep7.37}) corresponds
to the flow pattern of a {\em line source}\/ of strength $Q$ running along the $z$-axis (see Figure~\ref{fsource}). 

Suppose that 
\begin{equation}\label{ep7.40}
F(z) = {\rm i}\,\frac{\Gamma}{2\pi}\,\ln z,
\end{equation}
where $\Gamma$ is real. It follows that
\begin{eqnarray}
\phi(r,\theta)&=&-\frac{\Gamma}{2\pi}\,\theta,\\[0.5ex]
\psi(r,\theta)&=&\frac{\Gamma}{2\pi}\,\ln r.
\end{eqnarray}
Thus, according to the analysis in Section~\ref{svortex}, the complex potential (\ref{ep7.40}) corresponds
to the flow pattern of a {\em vortex filament}\/ of intensity  $\Gamma$ running along the $z$-axis (see Figure~\ref{fvortex1}). 

Suppose, finally, that
\begin{equation}\label{ep7.43}
F(z)= U\left(z+\frac{a^2}{z}\right) + {\rm i}\,\frac{\Gamma}{2\pi}\,\ln\left(\frac{z}{a}\right),
\end{equation}
where $U$, $a$, and $\Gamma$, are real. It follows that
\begin{eqnarray}
\phi(r,\theta)&=&U\left(r+\frac{a^2}{r}\right)\cos\theta-\frac{\Gamma}{2\pi}\,\theta,\\[0.5ex]
\psi(r,\theta)&=&U\left(r-\frac{a^2}{r}\right)\sin\theta + {\rm i}\,\frac{\Gamma}{2\pi}\,\ln\left(\frac{r}{a}\right).
\end{eqnarray}
Thus, according to the analysis in Section~\ref{scylo}, the complex potential (\ref{ep7.43}) corresponds to uniform
flow of unperturbed speed $U$, running parallel to the minus $x$-axis, around a cylindrical obstacle of radius $a$ whose
axis coincides with the $z$-axis (see Figures~\ref{fxcyl}, \ref{fxcyl1}, and \ref{fxcyl2}). Here, $\Gamma$ is the circulation of the flow about the cylinder. Note that $\psi=0$
on the  surface of the cylinder ($r=a$), which implies that the normal velocity is zero there, as must be the case.

\section{Complex Velocity}
It follows from Equations~(\ref{ep7.4}), (\ref{ep7.7}), and (\ref{ep7.28x}) that
\begin{equation}
\frac{dF}{dz} = \frac{\partial\phi}{\partial x} + {\rm i}\,\frac{\partial\psi}{\partial x} = -v_x+{\rm i}\,v_y.
\end{equation}
Consequently, $dF/dz$ is termed the {\em complex velocity}. Note that
\begin{equation}
\left|\frac{dF}{dz}\right|^2 = v_x^{\,2} + v_y^{\,2} =v^2,
\end{equation}
where $v$ is the flow speed. 

A {\em stagnation point}\/  is defined as a point in a flow pattern where the flow speed, $v$, falls to zero. 
It follows, from the previous equation, that
\begin{equation}
\frac{dF}{dz} = 0
\end{equation}
at a stagnation point. For instance, the stagnation points of uniform flow, with circulation, around a
cylindrical obstacle---whose complex potential is given
by Equation~(\ref{ep7.43})---satisfy the quadratic equation
\begin{equation}
\frac{dF}{dz}= U\left(1-\frac{a^2}{z^2}\right) + {\rm i}\,\frac{\Gamma}{2\pi\,z} = 0.
\end{equation}
The roots of this equation are
\begin{equation}
\frac{z}{a} = -{\rm i}\,\zeta \pm \sqrt{1-\zeta^2},
\end{equation}
where $\zeta=\Gamma/(4\pi\,U\,a)$, 
with the proviso that $|z|/a>1$, since the region $|z|/a<1$ is occupied by the cylinder. Thus, if
$\zeta \leq 1$ then there are two stagnation points on the surface of the cylinder at $x/a=\pm\sqrt{1-\zeta^2}$ and $y/a=-\zeta$. 
On the other hand, if $\zeta>1$ then there is a single stagnation point below the cylinder at $x/a=0$ and $y=-\zeta-\sqrt{\zeta^2-1}$.

Now, according to Section~\ref{qirr}, Bernoulli's theorem in an steady, irrotational, incompressible, fluid takes the form
\begin{equation}
p +\frac{1}{2}\,\rho\,v^2 = p_0,
\end{equation}
where $p_0$ is a uniform constant, and where gravity (or any other body force) has been neglected. Thus, the
pressure distribution in such a fluid can be written
\begin{equation}\label{ep7.51}
p = p_0 -\frac{1}{2}\,\rho\left|\frac{dF}{dz}\right|^{\,2}.
\end{equation}

\section{Theorem of Blasius}\label{sblasius}
Consider a stationary aerofoil of infinite length and uniform cross-section whose axis runs parallel to the $z$-axis. See Figure~\ref{faero} Let the
curve $C$ represent the intersection of the surface of the aerofoil with the $x$-$y$ plane. Suppose that the aerofoil
is placed in a uniformly flowing incompressible fluid, moving perpendicular to the $z$-axis, and that the two-dimensional steady flow pattern set up around the aerofoil in the $x$-$y$ plane is specified by the
complex potential $F(z)$.  Assuming that the aerofoil is sufficiently streamlined that there is no separation of the flow
from its surface, the  pressure distribution over this surface can be determined from Equation~(\ref{ep7.51}), we yields
\begin{equation}\label{ep7.52}
P = p_0  -\frac{1}{2}\,\rho\left|\frac{dF}{dz}\right|^{\,2}_C.
\end{equation}
Let us evaluate the resultant force (per unit length) and the resultant torque (per unit length) acting on the aerofoil
as a consequence of the pressure distribution.

\begin{figure}
\epsfysize=3in
\centerline{\epsffile{Chapter07/aerofoil.eps}}
\caption{\em A two-dimensional aerofoil.}\label{faero}
\end{figure}

Consider a small element of $C$, lying between $x$, $y$ and $x+dx$, $y+dy$, which is sufficiently
short that it can be approximated as a straight line. Let $P$ be the local fluid pressure. As illustrated
in Figure~\ref{fforce}, the pressure force (per unit length) acting on this element has a component $P\,dy$ in the minus
$x$-direction, and a component $P\,dx$ in the plus $y$-direction. Thus, if $X$ and $Y$ are the components of the
resultant force (per unit length) in the $x$- and $y$- directions, respectively, then
\begin{eqnarray}
dX &=&-P\,dy,\\[0.5ex]
dY&=&P\,dx.
\end{eqnarray}
The pressure force (per unit length) acting on the element also contributes to a moment (per unit length), $M$, acting
on the aerofoil about the $z$-axis. In fact,
\begin{equation}
dM = x\,dX-y\,dY = P\,(x\,dx+y\,dy).
\end{equation}
Thus, the $x$- and $y$- components of the resultant force (per unit length) acting on the aerofoil, as
well as the resultant moment (per unit length) about the $z$-axis, are given by
\begin{eqnarray}
X &=&-\oint_C P\,dy,\\[0.5ex]
Y&=&\oint_C P\,dx,\\[0.5ex]
M &=&\oint_C P\,(x\,dx+y\,dy),
\end{eqnarray}
respectively,
where the integrals are taken (counter-clockwise) around the curve $C$. Finally, given that the pressure distribution takes the
form (\ref{ep7.52}), and that a constant pressure obviously yields zero force and zero moment acting on the aerofoil, 
we find that
\begin{eqnarray}
X &=&\frac{1}{2}\,\rho\oint_C \left|\frac{dF}{dz}\right|^{\,2}dy,\\[0.5ex]
Y&=&-\frac{1}{2}\,\rho\oint_C \left|\frac{dF}{dz}\right|^{\,2}dx,\\[0.5ex]
M &=&-\frac{1}{2}\,\rho\oint_C \left|\frac{dF}{dz}\right|^{\,2}(x\,dx+y\,dy).
\end{eqnarray}

\begin{figure}
\epsfysize=3in
\centerline{\epsffile{Chapter07/force.eps}}
\caption{\em Short section of the surface of a two-dimensional aerofoil.}\label{fforce}
\end{figure}

Now, $z=x+{\rm i}\,y$, and $\bar{z}=x-{\rm i}\,y$, where $\bar{~}$ indicates a complex conjugate.
Hence, $d\bar{z} = dx -{\rm i}\,dy$, and ${\rm i}\,d\bar{z} = dy+{\rm i}\,dx$.  It follows that
\begin{equation}
X - {\rm i}\,Y = \frac{1}{2}\,{\rm i}\,\rho\oint_C \left|\frac{dF}{dz}\right|^{\,2}d\bar{z}.
\end{equation}
However,
\begin{equation}
 \left|\frac{dF}{dz}\right|^{\,2}d\bar{z} = \frac{dF}{dz}\,\frac{d\bar{F}}{d\bar{z}}\,d\bar{z} = \frac{dF}{dz}\,d\,\bar{F},
 \end{equation}
 where $dF = d\phi+{\rm i}\,d\psi$ and $d\bar{F}=d\phi-{\rm i}\,d\,\psi$. However, on the curve $C$,
 which corresponds to the surface of the aerofoil, we must have $\psi={\rm constant}$ in order to ensure that
 the normal velocity is zero. Thus, $d\psi=0$ on $C$, and so $d\bar{F}=dF$. 
 Hence,  on $C$, 
\begin{equation}
 \left|\frac{dF}{dz}\right|^{\,2}d\bar{z} =  \frac{dF}{dz}\,dF = \left(\frac{dF}{dz}\right)^{\,2},
 \end{equation}
 giving
 \begin{equation}\label{ep7.65}
X - {\rm i}\,Y = \frac{1}{2}\,{\rm i}\,\rho\oint_C \left(\frac{dF}{dz}\right)^{\,2}dz.
\end{equation}
This result is known as the {\em Blasius theorem}. 

Note that $x\,dx+ y\,dy={\rm Re}(z\,d\bar{z})$. This implies that
\begin{equation}
M = {\rm Re}\left(-\frac{1}{2}\,\rho \oint_C \left|\frac{dF}{dz}\right|^{\,2} z\,d\bar{z}\right),
\end{equation}
or, making use of an analogous argument to that employed above, 
\begin{equation}\label{ep7.67}
M = {\rm Re}\left[-\frac{1}{2}\,\rho \oint_C \left(\frac{dF}{dz}\right)^{\,2} z\,dz\right],
\end{equation}
 
\section{Complex Integrals}\label{scauchy}
Consider the integral of some function $F(z)$ of the complex variable around some closed curve $C$ in the complex
plane:
\begin{equation}
J = \oint_C F(z)\,dz.
\end{equation}
Since $dz=dx+{\rm i}\,dy$, and writing $F=\phi+{\rm i}\,\psi$, where $\phi$ and $\psi$ are real functions, 
we find that $J=J_r+{\rm i}\,J_i$, where
\begin{eqnarray}
J_r&=&\oint_C (\phi\,dx-\psi\,dy),\\[0.5ex]
J_i&=&\oint_C (\psi\,dx+\phi\,dy).
\end{eqnarray}
However, we can also write the above expressions in the two-dimensional vector form
\begin{eqnarray}
J_r &=&\oint_C {\bf A}\cdot d{\bf r},\\[0.5ex]
J_i &=&\oint_C {\bf B}\cdot d{\bf r},
\end{eqnarray}
where $d{\bf r}=(dx,\,dy)$, ${\bf A}=(\phi,\,-\psi))$, and ${\bf B} = (\psi,\,\phi)$. However, according to
the Stokes' theorem (see Section~\ref{scurl}), 
\begin{eqnarray}
\oint_C {\bf A}\cdot d{\bf r}&=&\int_S (\nabla\times {\bf A})_z\,dS,\\[0.5ex]
\oint_C {\bf B}\cdot d{\bf r} &=&\int_S (\nabla\times {\bf B})_z\,dS,
\end{eqnarray}
where $S$ is the region of the $x$-$y$ plane enclosed by $C$. Hence, we obtain
\begin{eqnarray}
J_r &=&-\int_S\left(\frac{\partial\psi}{\partial x} + \frac{\partial\phi}{\partial y}\right)dS,\\[0.5ex]
J_i&=&\int_S\left(\frac{\partial\phi}{\partial x} - \frac{\partial\psi}{\partial y}\right)dS.
\end{eqnarray}

Let 
\begin{eqnarray}
J&=&\oint_{C}F(z)\,dz,\\[0.5ex]
J'&=&\oint_{C'}  F(z)\,dz,
\end{eqnarray}
where $C'$ is a closed curve in the complex plane that surrounds the smaller curve $C$. 
Consider
\begin{equation}
\Delta J = J-J'.
\end{equation}
Writing $\Delta J=\Delta J_r+{\rm i}\,\Delta J_i$, where $\Delta J_r$ and $\Delta J_i$ are real, a direct generalization of the previous analysis
reveals that
\begin{eqnarray}
\Delta J_r &=&-\int_{S}\left(\frac{\partial\psi}{\partial x} + \frac{\partial\phi}{\partial y}\right)dS',\\[0.5ex]
\Delta J_i&=&\int_{S}\left(\frac{\partial\phi}{\partial x} - \frac{\partial\psi}{\partial y}\right)dS',
\end{eqnarray}
where $S'$ is now the region of the $x$-$y$ plane lying between the curves $C$ and $C'$. 
Suppose that $F(z)$ is well-defined throughout $S$ ({\em i.e.}, it is finite, single-valued, and
differentiable). It immediately follows that its real and imaginary components, $\phi$ and $\psi$,
respectively, satisfy the Cauchy-Riemann relations, (\ref{ep7.27})--(\ref{ep7.28}), throughout $S'$. 
However, if this is the case then it is clear from the previous two expressions that $\Delta J_r=\Delta J_i=0$. 
In other words, if $F(z)$ is well-defined throughout $S$ then $J=J'$. 

Consider, now, the complex integral
\begin{equation}
K=\oint_C \left(\frac{dF}{dz}\right)^{\,2}dz,
\end{equation}
where the closed curve $C$ corresponds to the surface of our aerofoil, and $dF/dz$ is the complex velocity
in the region of the complex plane lying outside this curve. Now, in order to be physical,  the velocity field of the fluid
surrounding the aerofoil must be finite, single-valued, and differentiable. It follows that $dF/dz$, and, hence,
$(dF/dz)^{\,2}$, is well-defined in the region external to $C$. 
Consider the second integral
\begin{equation}
K'=\oint_{C'} \left(\frac{dF}{dz}\right)^{\,2}dz,
\end{equation}
where $C'$ is any closed curve that encloses the aerofoil (and, hence, the curve $C$). Since $(dF/dz)^{\,2}$
is well-defined in the region external to $C$ it is also well-defined in the region lying between $C$ and $C'$. 
Hence, according to our previous analysis, $K'=K$. We, thus, conclude that the complex integral appearing in the
Blasius theorem, (\ref{ep7.65}), can, in fact, be taken around any closed curve that encircles the aerofoil
without invalidating the theorem. The same is true for the complex integral in appearing in formula (\ref{ep7.67}).

