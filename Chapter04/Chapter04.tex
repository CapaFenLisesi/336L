\chapter{Surface Tension}\label{c6x}
\section{Introduction}\label{s6x1}
As is well-known, small drops of water in air, and small bubbles of gas in water, tend to adopt  spherical shapes.
This phenomenon, and a host of other natural phenomena, can  only be accounted for on the hypothesis that an interface between two different media is associated with a
particular form of energy whose magnitude is directly proportional to the interface area.
To be more exact, if $S$ is the  area then the contribution of the interface to the Helmholtz free energy
of the system takes the form $\gamma\,S$, where  $\gamma$ only depends  on the temperature and  
chemical composition
of the two media on either side of the interface. It follows, from standard thermodynamics, that $\gamma\,S$ is the work that
must be performed on the system in order to create the
interface via an isothermal and reversible process. However, this work is exactly the same as that which we would calculate
on the assumption that the interface is in a state of uniform constant tension per unit length $\gamma$. Thus, $\gamma$
can be interpreted as both a free energy per unit area of the interface, and a {\em surface tension}. This
 tension is such that a force of magnitude
$\gamma$ per unit length is exerted across any line drawn on the interface,  in a direction normal to the line, and tangential to 
the interface. 

Surface tension originates from intermolecular cohesive forces. The average free energy of a molecule
in a given isotropic medium possessing an interface with a second medium is independent of its position, provided that the molecule does not lie too close to the interface. However, the free energy is modified when the molecule's distance from
the interface becomes less than the range of the cohesive forces (which is typically $10^{-9}\,{\rm m}$). Since this
range is so small, the number of molecules in a macroscopic system whose free energies are affected by the presence of an interface is directly
proportional to the interface area. Hence, the contribution of the interface to the total free energy
of the system is also proportional to the  area. If only one of the two media
in question is a condensed phase then the parameter $\gamma$ is invariably {\em positive}\/ ({\em i.e.}, such that a
reduction in the surface area is energetically favorable). This follows because
the molecules of a liquid or a solid are subject to an {\em attractive}\/ force from neighboring molecules.
However, molecules that are near to an interface with a gas lack neighbors on one side, and
so experience an unbalanced cohesive force directed toward the interior of the liquid/solid. The
existence of this force makes it energetically favorable for  the interface to contract ({\em i.e.}, $\gamma>0$). 
On the other hand, if the interface
separates a liquid and a solid, or a liquid and another liquid, then the sign of $\gamma$ cannot be predicted
by this argument. In fact, it is possible for both signs of $\gamma$ to occur at liquid/solid and liquid/liquid interfaces. 

The surface tension of a water/air interface at $20^\circ\,{\rm C}$ is $\gamma=7.28\times 10^{-2}\,{\rm N\,m^{-1}}$. The
surface tension at  most
oil/air interfaces is much lower---typically, $\gamma\simeq 2\times 10^{-2}\,{\rm N\,m^{-1}}$. On the other hand, 
interfaces between  liquid metals and air generally have  very large surface tensions. For instance, the surface 
tension of a mercury/air interface at $20^\circ\,{\rm C}$
is $4.87\times 10^{-1}\,{\rm N\,m^{-1}}$. 

For some pairs of liquids, such as water and alcohol, an
interface cannot generally be observed because it is in compression ({\em i.e.}, $\gamma<0$). Such an interface tends
to become as large as possible, leading to complete mixing of the two liquids. In other words, liquids for
which $\gamma>0$ are immiscible, whereas those for which $\gamma<0$ are miscible.  

Finally, the 
surface tension at a liquid/gas or a liquid/liquid interface can be 
affected by the presence of  adsorbed impurities at the interface. For instance, the surface tension at a water/air interface
is significantly deceased in the presence of adsorbed soap molecules. Impurities
that tend to reduce surface tension at interfaces are termed {\em surfactants}. 

\section{Young-Laplace Equation}
Consider an interface separating two immiscible fluids that are in equilibrium with one another. Let these two
fluids be denoted 1 and 2. Consider an arbitrary segment $S$ of this interface that is enclosed by some closed curve $C$. 
Let ${\bf t}$ denote a unit tangent to the curve, and let ${\bf n}$ denote a unit normal to the
interface directed from fluid 1 to fluid 2. (Note that $C$ circulates around ${\bf n}$ in a
right-handed manner.) See Figure~\ref{f6x.01}. Suppose that $p_1$ and $p_2$ are the 
pressures of fluids 1 and 2, respectively, on either side of $S$. Finally, let $\gamma$ be the (uniform) 
surface tension at the interface. 
 
\begin{figure}
\epsfysize=1.5in
\centerline{\epsffile{Chapter04/fig6.01.eps}}
\caption{\em Interface between two immiscible fluids.}\label{f6x.01}
\end{figure}

The net force acting on $S$ is
\begin{equation}\label{e6x.1}
{\bf f} =\int_S  (p_1-p_2)\,{\bf n}\,dS + \gamma \oint_C {\bf t}\times {\bf n}\,dr,
\end{equation}
where $d{\bf S} = {\bf n}\,dS$ is an element of $S$, and $d{\bf r}= {\bf t}\,dr$ an element of $C$. Here, the first term on the
right-hand side is the net normal force due to the pressure difference across the interface, whereas the
second term is the net surface tension force. 
Note that body forces play no role in (\ref{e6x.1}), because the interface has zero volume. Furthermore, 
viscous forces can be neglected, since both fluids are static.
Now, in equilibrium,
the net  force acting on $S$ must be zero: {\em i.e.}, 
\begin{equation}\label{e6x.2}
\int_S (p_1-p_2)\,{\bf n}\,dS =- \gamma \oint_C {\bf t}\times {\bf n}\,dr.
\end{equation}
(In fact, the net force would be zero even in the absence of equilibrium, because the interface has zero mass.)

Applying Stokes' theorem (see Section~\ref{scurl}) to the curve $C$, we find that
\begin{equation}
\oint_C {\bf F}\cdot d{\bf r} = \int_S \nabla\times {\bf F}\cdot d{\bf S},
\end{equation}
where ${\bf F}$ is a general vector field. This theorem can also be written
\begin{equation}
\oint_C {\bf F}\cdot {\bf t}\,dr= \int_S \nabla\times {\bf F}\cdot {\bf n}\,dS.
\end{equation}
Suppose that ${\bf F} = {\bf g}\times {\bf b}$, where ${\bf b}$ is an arbitrary constant vector. We obtain
\begin{equation}
\oint_C ({\bf g}\times {\bf b})\cdot {\bf t}\,dr= \int_S \nabla\times({\bf g}\times{\bf b})\cdot {\bf n}\,dS.
\end{equation}
However, the vector identity (\ref{curlvp}) yields
\begin{equation}
\nabla\times ({\bf g}\times {\bf b}) = - (\nabla\cdot{\bf g})\,{\bf b} + ({\bf b}\cdot\nabla)\,{\bf g},
\end{equation}
since ${\bf b}$ is a {\em constant}\/ vector.
Hence, we get
\begin{equation}
{\bf b}\cdot\oint_C {\bf t}\times {\bf g}\,dr = {\bf b}\cdot\!\int_S[(\nabla {\bf g})\cdot{\bf n} - (\nabla\cdot{\bf g})\,{\bf n}]\,dS,
\end{equation}
where ${\bf b} \cdot(\nabla {\bf g})\cdot{\bf n} \equiv b_i\,(\partial g_j/\partial x_i)\,n_j$. 
Now, since ${\bf b}$ is also an {\em arbitrary}\/ vector, the above equation gives
\begin{equation}
\oint_C {\bf t}\times {\bf g}\,dr = \int_S\left[(\nabla {\bf g})\cdot{\bf n} - (\nabla\cdot{\bf g})\,{\bf n}\right]dS.
\end{equation}
Taking  ${\bf g} = \gamma\,{\bf n}$, we find that
\begin{equation}
\gamma\oint_C {\bf t}\times {\bf n}\,dr= \gamma\int_S \left[(\nabla {\bf n})\cdot {\bf n} - (\nabla\cdot {\bf n})\,{\bf n}\right]dS.
\end{equation}
But, $(\nabla{\bf n})\cdot {\bf n}\equiv (1/2)\,\nabla(n^2)=0$, because ${\bf n}$ is a {\em unit}\/ vector. Thus, we obtain
\begin{equation}\label{e6x.10}
\gamma\,\oint_C {\bf t}\times{\bf n}\,dr = - \gamma\int_S (\nabla \cdot{\bf n})\,{\bf n}\,dS,
\end{equation}
which can be combined with (\ref{e6x.2}) to give
\begin{equation}
\int_S\left[(p_1-p_2)-\gamma\,(\nabla\cdot{\bf n})\right]{\bf n}\,dS = 0.
\end{equation}
Finally, given that $S$ is {\em arbitrary}, the above expression reduces to the pressure balance constraint
\begin{equation}\label{e6x.11}
\Delta p = \gamma\,\nabla\cdot{\bf n},
\end{equation}
where $\Delta p = p_1-p_2$.
The above relation is generally known as the {\em Young-Laplace equation}, and
can also be derived by minimizing the free energy of the interface (see Section~\ref{sbubble}). Note that $\Delta p$ is
the jump in pressure seen when crossing the interface in the {\em opposite}\/ direction to ${\bf n}$.
Of course, a plane interface is characterized by $\nabla\cdot{\bf n}=0$. On the other hand, a curved
interface generally has $\nabla\cdot{\bf n}\neq 0$. In fact, $\nabla\cdot{\bf n}$ measures the {\em local
 mean curvature}\/ of the interface. Thus, according to the Young-Laplace equation, there is a pressure jump across  
a curved interface between two immiscible fluids, the
magnitude of the jump being proportional to the surface tension. 

\section{Spherical Interfaces}\label{s6x3}
Generally speaking, the equilibrium shape of an interface between two immiscible fluids is determined by solving
the force balance equation (\ref{e6.1}) in each fluid, and then applying the Young-Laplace equation to the
interface. However, in situations in which a mass of one fluid is completely
immersed in a second fluid---{\em e.g.}, a mist droplet in air, or a gas bubble in water---the shape of the interface is fairly obvious.  Provided that either the
size of the droplet or bubble, or the difference in densities on the two sides of the interface, is sufficiently small,
we can safely ignore the effect of gravity. This implies that the pressure is {\em uniform}\/ in each fluid, and consequently that the pressure jump
$\Delta p$ is {\em constant}\/ over the interface. Hence, from (\ref{e6x.11}), the mean curvature $\nabla\cdot{\bf n}$
of the interface is also {\em constant}.  Since a sphere is the only closed surface which possesses  a constant mean curvature, we conclude that
 the interface is {\em spherical}. This result also follows from the argument that a stable equilibrium state
is one which {\em minimizes}\/ the free energy of the interface, subject to the constraint that the enclosed volume be 
{\em constant}. In other words, the equilibrium shape of the interface is that which
has the least surface area for a given volume: {\em i.e.}, a sphere.

Suppose that the interface corresponds to the spherical surface $r=R$, where $r$ is a spherical coordinate (see Section~\ref{ssph}).
It follows that ${\bf n}= \left.{\bf e}_r\right|_{r=R}$. (Note, for future reference, that ${\bf n}$ points away from the center of
curvature of the interface.)
 Hence, from (\ref{ec65}), 
\begin{equation}\label{e6x12}
\nabla\cdot{\bf n} = \left.\frac{1}{r^2}\,\frac{\partial r^2}{\partial r}\right|_{r=R} = \frac{2}{R}.
\end{equation}
The Young-Laplace equation, (\ref{e6x.11}), then gives
\begin{equation}
\Delta p = \frac{2\,\gamma}{R}.
\end{equation}
Thus, given that $\Delta p$ is the pressure jump seen crossing the interface in the opposite direction to ${\bf n}$, 
we conclude that the
pressure {\em inside}\/ a droplet or bubble {\em exceeds}\/ that outside by an amount proportional to the surface tension, and
inversely proportional to the droplet or bubble radius. This explains why small bubbles are louder that large
ones when they burst at a free surface: {\em e.g.}, champagne fizzes louder than beer. Note that soap bubbles in air
have {\em two}\/ interfaces  defining  the inner and outer extents of the soap film. Consequently, the net pressure difference is
{\em twice}\/ that across a single interface.  

\section{Capillary Length}\label{sclen}
Consider an interface separating the atmosphere from a liquid of uniform density $\rho$ that is at rest on the surface of the
Earth. Neglecting the density of air compared to that of the liquid, the pressure in the atmosphere can
be regarded as constant. On the other hand, the pressure in the liquid varies as $p=p_0-\rho\,g\,z$ (see Chapter~\ref{c6}), where
$p_0$ is the pressure of the atmosphere, $g$  the acceleration due to gravity, and $z$ measures vertical height (relative
to the equilibrium height of the interface in the absence of surface tension). Note that $z$
increases upward. In this situation, the Young-Laplace equation (\ref{e6x.11})
yields
\begin{equation}\label{e6x.14}
\rho\,g\,z = -\gamma\,\nabla\cdot{\bf n},
\end{equation}
where ${\bf n}$ is the normal to the interface directed from liquid to air.  Now, if $l$ represents the typical radius of curvature of the
interface then the left-hand side of the above equation dominates the right-hand side whenever $l\gg \lambda$,
and {\em vice versa}. Here, 
\begin{equation}\label{e6x16}
\lambda = \left(\frac{\gamma}{\rho\,g}\right)^{1/2}
\end{equation}
is known as the {\em capillary length}, and takes the value $2.7\times 10^{-3}\,{\rm m}$
for pure water at $20^\circ\,{\rm C}$. We conclude that the effect of surface tension on the shape of an liquid/air interface
is likely to dominate the effect of gravity when the  interface's radius of curvature is much less than the capillary length, and {\em vice versa}. 

\begin{figure}
\epsfysize=2in
\centerline{\epsffile{Chapter04/fig6.02.eps}}
\caption{\em Interface between two fluids and a solid.}\label{f6x.02}
\end{figure}

\section{Angle of Contact}
Suppose that a liquid/air interface is in contact with a solid, as would be  the case for water in a glass tube, or a drop
of mercury resting on a table. Figure~\ref{f6x.02} shows a section perpendicular to the edge at which the
liquid, $1$, the air, $2$, and the solid, $3$, meet. Suppose that the free energies per unit
area at the liquid/air, liquid/solid, and air/solid interfaces are $\gamma_{12}$, $\gamma_{13}$, and $\gamma_{23}$,
respectively. If the boundary between the three media is slightly modified in the neighborhood of the edge, as indicated
by the dotted line in the figure, then the area of contact of the air with the solid is increased by a small amount $\delta r$ per unit
breadth (perpendicular to the figure), whereas that of the liquid with the solid is decreased by
$\delta r$ per unit breadth, and that of the liquid with the air is decreased by $\delta r\,\cos\theta$ per unit
breadth. Thus, the net change in free energy per unit breadth is
\begin{equation}
\gamma_{23}\,\delta r - \gamma_{13}\,\delta r - \gamma_{12}\,\delta r\,\cos\theta.
\end{equation}
However, an equilibrium state is one which {\em minimizes}\/ the free energy, implying that the above expression is {\em zero}\/ for arbitrary
(small) $\delta r$: {\em i.e.}, 
\begin{equation}\label{e6x.16}
\cos\theta = \frac{\gamma_{23}-\gamma_{13}}{\gamma_{12}}.
\end{equation}
We conclude that, in equilibrium, the {\em angle of contact}, $\theta$, between the liquid and the solid takes a {\em fixed}\/ value that depends on the
free energies per unit area at the liquid/air, liquid/solid, and  air/solid interfaces. Note that the above formula
could also be obtained from the requirement that the various surface tension forces acting at the edge  balance one another, assuming that
it is really appropriate to interpret $\gamma_{13}$ and $\gamma_{23}$ as surface tensions when one of
the media making up the interface is a solid. 

As explained in Section~\ref{s6x1}, we would generally expect $\gamma_{12}$ and $\gamma_{23}$ to be positive. 
On the other hand, $\gamma_{13}$ could be either positive or negative. Now, since $|\cos\theta|\leq 1$, Equation~(\ref{e6x.16})
can only be solved when $\gamma_{13}$ lies in the range
\begin{equation}
\gamma_{23}+\gamma_{12}> \gamma_{13} > \gamma_{23}-\gamma_{12}.
\end{equation}
If $\gamma_{13}> \gamma_{23}+\gamma_{12}$ then the angle of contact is $180^\circ$, which corresponds to the case
 where the
free energy at the liquid/solid interface is so large that the liquid does not wet the solid at all, but instead breaks up into beads
 on its surface. On the other hand, if $\gamma_{13}< \gamma_{23}-\gamma_{12}$ then the
angle of contact is $0^\circ$, which corresponds to the case where the free energy at the liquid/solid
interface is so small that the liquid completely wets the solid, spreading out indefinitely until it either covers the
whole surface, or its thickness reaches molecular dimensions. 

The angle of contact between water and glass typically lies in the range $25^\circ$ to $29^\circ$, whereas
that between mercury and glass is about $127^\circ$. 

\begin{figure}
\epsfysize=3.5in
\centerline{\epsffile{Chapter04/fig6.03.eps}}
\caption{\em Elevation of liquid level in a capillary tube.}\label{f6x.03}
\end{figure}

\section{Jurin's Law}
Consider a situation in which a narrow, cylindrical,  glass tube of radius $a$ is
dipped vertically  into a liquid of density $\rho$, and the liquid level within the tube rises a height $h$ above the free surface
 as a consequence of surface tension. See Figure~\ref{f6x.03}. Suppose that the radius of the
 tube  is much less than the capillary length. A tube for which this is the case is generally
known as a {\em capillary tube}. 
 According to the discussion in  Section~\ref{sclen}, the shape of the internal liquid/air interface within a capillary tube
  is not significantly affected by gravity.  Thus,  from Section~\ref{s6x3}, the interface is
 a segment of a sphere of radius $R$ (say). If $\theta$ is the angle of contact of interface with the glass then  simple geometry (see Figure~\ref{f6x.03}) reveals that 
 \begin{equation}
 R = \frac{a}{\cos\theta}.
 \end{equation}
 Hence, from Equation~(\ref{e6x12}), the mean curvature of the interface is given by
 \begin{equation}
 \nabla\cdot{\bf n} = -\frac{2}{R} =- \frac{2\,\cos\theta}{a},
 \end{equation}
 where $\gamma$ is the associated surface tension. [The minus sign in the above expression arises from the
 fact that ${\bf n}$ points towards the center of curvature of the interface, whereas the opposite is true for Equation~(\ref{e6x12}).] Finally, from
 (\ref{e6x.14}), application of the Young-Laplace equation to the interface yields
 \begin{equation}
 \rho\,g\,h = \frac{2\,\gamma\,\cos\theta}{a},
 \end{equation}
 which  can be rearranged to give
 \begin{equation}\label{e6x.22}
 h \simeq  \frac{2\,\gamma\,\cos\theta}{\rho\,g\,a}.
 \end{equation}
 This result, which relates the height, $h$, to which a liquid rises in a capillary tube of radius $a$ to the
liquid's  surface tension, $\gamma$, is known as {\em Jurin's law}. 
Note that the assumption that the radius of the tube is much less than the capillary length
is equivalent to the assumption that the height  of the interface above the free surface of the
liquid is much greater than the radius of the tube.
This follows, from (\ref{e6x16}) and (\ref{e6x.22}), because
\begin{equation}
\frac{h}{a} = 2\,\cos\theta\,\frac{\lambda^2}{a^2}.
\end{equation}
Thus, the ordering $a\ll \lambda$  implies that $h\gg a$. 

For the case of water at $20^\circ$, assuming a contact angle of $25^\circ$, Jurin's law yields
$h({\rm mm}) = 13.5/a({\rm mm})$. Thus, water rises a height $13.5\,{\rm mm}$ in a capillary tube of
radius $1\,{\rm mm}$, but rises $13.5\,{\rm cm}$ in a capillary tube of radius $0.1\,{\rm mm}$. 
Note that in the case of a liquid, such a mercury, that has an {\em oblique}\/
angle of contact with glass, so that $\cos\theta<0$, the  liquid level in a capillary tube is {\em depressed}\/
below that of the free surface ({\em i.e.}, $h<0$). 
 
\section{Capillary Curves}
Let adopt Cartesian coordinates on the Earth's surface such that $z$ increases vertically upward. Suppose
that the interface of a liquid of density $\rho$ and surface tension $\gamma$ with the atmosphere corresponds to the
surface $z=f(x)$, where the liquid occupies the region $z<f(z)$. Note that the shape of the interface is $y$-independent. The unit normal to the interface (directed from liquid
to air) is thus 
\begin{equation}
{\bf n} = \frac{\nabla(z-f)}{|\nabla(z-f)|}=\frac{{\bf e}_z - f_x\,{\bf e}_x}{(1+f_x^{\,2})^{1/2}},
\end{equation}
where $f_x\equiv df/dx$. Hence, the mean curvature of the interface is
\begin{equation}
\nabla\cdot {\bf n} = - \frac{f_{xx}}{(1+f_x^{\,2})^{3/2}},
\end{equation}
where $f_{xx}\equiv d^2f/dx^2$. 
According to  (\ref{e6x.14}), the shape of the interface is governed by the nonlinear differential
equation
\begin{equation}
f = \frac{\lambda^2\,f_{xx}}{(1+f_x^{\,2})^{3/2}}.
\end{equation}
where  the vertical height, $f$, of the interface
is measured relative to its equilibrium height in the absence of surface tension. 
Multiplying the above equation by $f_x/\lambda^2$, and integrating with respect to $x$, we obtain
\begin{equation}\label{e6x.27}
 \frac{1}{(1+f_x^{\,2})^{1/2}} = C-\frac{f^{\,2}}{2\,\lambda^{\,2}},
\end{equation}
where $C$ is a constant. It follows that
\begin{equation}\label{e6x.28}
C - \frac{f^{\,2}}{2\,\lambda^{\,2}}\geq 1,
\end{equation}
and
\begin{equation}\label{e6x.29}
\frac{1}{f_x} = \mp \frac{C-f^{\,2}/2\,\lambda^{\,2}}{[1-(C-f^{\,2}/2\,\lambda^{\,2})^2]^{1/2}}.
\end{equation}

\begin{figure}
\epsfysize=3.5in
\centerline{\epsffile{Chapter04/fig6.04.eps}}
\caption{\em Capillary curves for $\pi/4\leq \phi\leq 3\pi/4$ and  (in order from the top to the bottom) $k=0.6$, $0.7$, $0.8$, $0.9$, and $0.99$.}\label{f6x.04}
\end{figure}

 Let
\begin{equation}\label{e6x.30}
C = \frac{2}{k^2}-1,
\end{equation}
where $0< k< 1$, and
\begin{equation}\label{e6x.31}
f = \pm\frac{2\,\lambda}{k}\,(1-k^2\,\sin^2\phi)^{1/2}.
\end{equation}
Thus, from (\ref{e6x.30}) and (\ref{e6x.31}), 
\begin{equation}\label{e6x.32}
C - \frac{f^{\,2}}{2\,\lambda^{\,2}} =-\cos(2\,\phi),
\end{equation}
and so the constraint (\ref{e6x.28}) implies that $\pi/4\leq \phi\leq 3\pi/4$. Moreover, Equations~(\ref{e6x.29})
and (\ref{e6x.32})
reduce to
\begin{equation}\label{e6x.33}
\frac{1}{f_x}\equiv \frac{dx}{df} = \pm \frac{1}{\tan(2\,\phi)}.
\end{equation}
It follows from (\ref{e6x.31}) and (\ref{e6x.33}) that
\begin{equation}
\frac{dx}{d\phi} = \frac{dx}{df}\,\frac{df}{d\phi} = -\frac{\lambda\,k\,\cos(2\,\phi)}{(1-k^2\,\sin^2\phi)^{1/2}},
\end{equation}
which can be integrated to give
\begin{equation}
\frac{x}{\lambda} = \int_\phi^{\pi/2} \frac{k\,\cos(2\,\phi)}{(1-k^2\,\sin^2\phi)^{1/2}}\,d\phi,
\end{equation}
assuming that $x=0$ when $\phi=\pi/2$. 
Thus, we get
\begin{equation}
\frac{x}{\lambda} = \left(k-\frac{2}{k}\right)\tilde{F}(\phi,k) + \frac{2}{k}\,\tilde{E}(\phi,k),
\end{equation}
where
\begin{eqnarray}
\tilde{E}(\phi,k) &=& E(\pi/2,k)- E(\phi,k),\\[0.5ex]
\tilde{F}(\phi,k) &=&  F(\pi/2,k)- F(\phi,k),
\end{eqnarray}
and
\begin{eqnarray}
E(\phi,k) &=& \int_0^\phi (1-k^2\,\sin^2\phi)^{1/2},\label{e8x56}\\[0.5ex]
F(\phi,k) &=& \int_0^\phi (1-k^2\,\sin^2\phi)^{-1/2},\label{e8x57}
\end{eqnarray}
are types of {\em incomplete elliptic integral}.\footnote{See {\em Handbook of Mathematical Functions}, M.~Abramowitz, and
I.A.~Stegun (Dover, New York NY, 1965).}  In conclusion, the  interface shape is  determined parametrically by 
\begin{eqnarray}\label{e6x.38}
\frac{x}{\lambda}& =& \left(k-\frac{2}{k}\right)\tilde{F}(\phi,k) + \frac{2}{k}\,\tilde{E}(\phi,k),\\[0.5ex]
\frac{z}{\lambda} &=& \pm\frac{2}{k}\,(1-k^2\,\sin^2\phi)^{1/2},\label{e6x.39}
\end{eqnarray}
where  $\pi/4\leq \phi\leq 3\pi/4$. Here, the parameter $k$ is restricted to lie in the range $0< k< 1$. 

\begin{figure}
\epsfysize=3.5in
\centerline{\epsffile{Chapter04/fig6.05.eps}}
\caption{\em Liquid/air interface for a liquid trapped between two vertical parallel plates located at $x=\pm\lambda$.
The contact angle of the interface with the plates is $\theta=30^\circ$. }\label{f6x.05}
\end{figure}

Figure~\ref{f6x.04} shows the capillary curves predicted by (\ref{e6x.38}) and (\ref{e6x.39}) for various different values of $k$. Here, we have chosen the plus sign in (\ref{e6x.39}). 
However, if the minus sign is chosen then
the curves  are simply inverted: {\em i.e.}, $x\rightarrow x$ and $z\rightarrow -z$. In can be seen that
all of the  curves shown in the figure are {\em symmetric}\/ about $x=0$: {\em i.e.}, $z\rightarrow z$ as $x\rightarrow -x$. 
Consequently, we can use these curves to determine the shape of the liquid/air interface which arises when a liquid is trapped between
two  flat vertical  plates (made of the same material) that are parallel to one another. Suppose that the plates in question lie at $x=\pm d$. Furthermore, let the angle of contact of the interface 
with the plates be $\theta$, where $\theta <\pi/2$.  Since the angle of contact is {\em acute}, we expect the liquid to
be drawn {\em upward}\/  between the plates, and the interface to be {\em concave}\/ (from above). This corresponds to the
positive sign in (\ref{e6x.39}). In order for the interface to meet the plates at the correct angle, we require
$f_x=-1/\tan\theta$ at $x=-d$ and $f_x=1/\tan\theta$ at $x=+d$. However, if one of these boundary conditions is
satisfied then, by symmetry, the other is automatically satisfied.
From Equation~(\ref{e6x.33}) (choosing the positive sign), the latter boundary condition yields $\tan(2\,\phi)=1/\tan\theta$ at $x=+d$,
which is equivalent to $x=+d$ when $\phi=3\pi/4-\theta/2$. Substituting this value of $\phi$ into Equation~(\ref{e6x.38}), we can numerically determine the
value of $k$ for which $x=d$. The interface shape is then given by Equations~(\ref{e6x.38}) and (\ref{e6x.39}), using
the aforementioned value of $k$, and $\phi$ in the range $\pi/4+\theta/2$ to $3\pi/4-\theta/2$. 
For instance, if $d=\lambda$ and $\theta=30^\circ$ then $k=0.9406$, and the associated interface  is shown
in Figure~\ref{f6x.05}. Furthermore, if we invert this interface ({\em i.e.}, $x\rightarrow x$ and $z\rightarrow -z$) then
we obtain the interface which corresponds to the same plate spacing, but an obtuse contact angle of $\theta=180^\circ-30^\circ=150^\circ$. 

Consider the limit $k\ll 1$, which is such that the distance between the two plates is much less than the capillary
length. It is easily demonstrated that, at small $k$,\,\footnote{See {\em Handbook of Mathematical Functions}, M.~Abramowitz, and
I.A.~Stegun (Dover, New York NY, 1965).} 
\begin{eqnarray}
\tilde{E}(\phi,k)&\simeq & \tilde{\phi} - \frac{k^2}{4}\,(\tilde{\phi}+\sin\tilde{\phi}\,\cos\tilde{\phi}),\\[0.5ex]
\tilde{F}(\phi,k) &\simeq & \tilde{\phi} + \frac{k^2}{4}\,(\tilde{\phi}+\sin\tilde{\phi}\,\cos\tilde{\phi}),
\end{eqnarray}
where $\tilde{\phi}=\pi/2-\phi$. Thus, Equations~(\ref{e6x.38}) and (\ref{e6x.39}) reduce to
\begin{eqnarray}\label{e6x.42}
\frac{x}{\lambda}&\simeq & -\frac{k}{2}\,\sin(2\,\tilde{\phi}),\\[0.5ex]
\frac{z}{\lambda}&\simeq &\frac{2}{k}-\frac{k}{2}\,[1-\cos(2\,\tilde{\phi})].
\end{eqnarray}
It follows that the interface is a segment of the curved surface of a {\em cylinder}\/ whose axis runs parallel to the
$y$-axis. If the distance between the
plates is $2\,d$, and the contact angle is $\theta$, then we require $x=d$ when $\phi=3\pi/4-\theta/2$ (which
corresponds to $\tilde{\phi}=-\pi/4+\theta/2$). 
From Equation~(\ref{e6x.42}), this constraint yields
\begin{equation}
\frac{d}{\lambda}\simeq \frac{k}{2}\,\cos\theta.
\end{equation}
Thus, the height that the liquid rises between the two plates---{\em i.e.}, $h\equiv z(x=0)\equiv z(\tilde{\phi}=\pi/2)\simeq 2\,\lambda/k$---is given by
\begin{equation}
h \simeq \frac{\gamma\,\cos\theta}{\rho\,g\,d}.
\end{equation}
This result is the form taken by Jurin's law, (\ref{e6x.22}), for a liquid drawn up between two parallel plates
of spacing $2\,d$. 

\begin{figure}
\epsfysize=3.5in
\centerline{\epsffile{Chapter04/fig6.06.eps}}
\caption{\em Liquid/air interface for a liquid in contact with a vertical plate located at $x=0$.
The contact angle of the interface with the plate is $\theta=25^\circ$. }\label{f6x.06}
\end{figure}

Consider the case $k=C =1$, which is such that the distance between the two plates is infinite. Let the leftmost plate
lie at $x=0$, and let us completely neglect the rightmost plate, since it lies at infinity. Suppose that $h\equiv z(x=0)$ is the
height of the interface  above the free surface  of the liquid at the point where the interface meets the leftmost plate. 
If $\theta$ is the angle of contact of the interface with the plate then we require $f_x=-1/\tan\theta$ at $x=0$. 
Since $C=1$, it follows from (\ref{e6x.27}) that
\begin{equation}
\frac{h^2}{2\,\lambda^2} = 1-\sin\theta,
\end{equation}
or
\begin{equation}\label{e6x.47}
h=2\,\lambda\,\sin(\pi/4-\theta/2).
\end{equation}
Furthermore, again recalling that $C=1$, Equation~(\ref{e6x.29}) can be integrated to
give
\begin{equation}
x = \int_z^h \frac{df}{f_x}= \lambda \int_{z/2\lambda}^{h/2\lambda}\frac{1-2\,y^2}{y\,(1-y^2)^{1/2}}\,dy,
\end{equation}
where we have chosen the minus sign, and $y=f/2\lambda$. Making the substitution $y=\sin u$, this becomes
\begin{equation}
\frac{x}{\lambda} = \int_{\sin^{-1}(z/2\lambda)}^{\sin^{-1}(h/2\lambda)}\left(\frac{1}{\sin u}- 2\,\sin u\right)du=
\left[-\ln\left(\frac{1+\cos u}{\sin u}\right)+2\,\cos u\right]_{\sin^{-1}(z/2\lambda)}^{\sin^{-1}(h/2\lambda)},
\end{equation}
which reduces to
\begin{equation}\label{e6x.50}
\frac{x}{\lambda} = \cosh^{-1}\left(\frac{2\,\lambda}{z}\right)-\cosh^{-1}\left(\frac{2\,\lambda}{h}\right)
+ \left(4-\frac{h^{\,2}}{\lambda^{\,2}}\right)^{1/2}- \left(4-\frac{z^{\,2}}{\lambda^{\,2}}\right)^{1/2},
\end{equation}
since $\cosh^{-1}(z)\equiv \ln[z+(z^2-1)^{1/2}]$. Thus, Equations~(\ref{e6x.47}) and (\ref{e6x.50})
specify the shape of a liquid/air interface that meets an isolated vertical plate at $x=0$. In particular,
(\ref{e6x.47}) gives the height that the interface climbs up the plate (relative to the free surface) due to the action of
surface tension. Note that this height is restricted to lie in the range $-2\,\lambda\leq h\leq 2\,\lambda$, irrespective
of the angle of contact.
Figure~\ref{f6x.06} shows an example interface calculated for $\theta=25^\circ$. 

\section{Axisymmetric Soap-Bubbles}\label{sbubble}
Consider an axisymmetric soap-bubble whose surface takes the form $r=f(z)$ in cylindrical coordinates. See Section~\ref{scyl}. The
unit normal to the surface is
\begin{equation}
{\bf n} \equiv \frac{\nabla(r-f)}{|\nabla(r-f)|} = \frac{{\bf e}_r- f_z\,{\bf e}_z}{(1+f_z^{\,2})^{1/2}},
\end{equation}
where $f_z\equiv df/dz$. Hence, from (\ref{ec39}), the mean curvature of the surface is given by
\begin{equation}
\nabla\cdot {\bf n} = \frac{1}{f\,f_z}\,\frac{d}{dz}\!\left[\frac{f}{(1+f_z^{\,2})^{1/2}}\right].
\end{equation}
The Young-Laplace equation, (\ref{e6x.11}), then yields
\begin{equation}\label{e6x.54}
\frac{f\,f_z}{a} = \frac{d}{dz}\!\left[\frac{f}{(1+f_z^{\,2})^{1/2}}\right],
\end{equation}
where
\begin{equation}
a = \frac{\gamma}{p_0}.
\end{equation}
Here, $\gamma$ is the net surface tension, including the contributions from the internal and external soap/air interfaces. Moreover,
$p_0=\Delta p$ is the pressure difference between the interior and the exterior of the bubble. Equation~(\ref{e6x.54})
can be integrated to give
\begin{equation}\label{e6x.59}
\frac{f}{(1+f_z^{\,2})^{1/2}}  = \frac{f^{\,2}}{2\,a} + C,
\end{equation}
where $C$ is a constant. 

Suppose that the bubble occupies the region $z_1\leq z\leq z_2$, where $z_1<z_2$, and has a fixed
radius at its two end-points, $z=z_1$ and $z=z_2$. 
This could most easily be achieved by supporting the
bubble on two rigid parallel co-axial rings located at $z=z_1$ and $z=z_2$. 
The net free energy required to create the bubble  can be written
\begin{equation}
{\cal E} = \gamma\,S - p_0\,V,
\end{equation}
where $S$ is area of the bubble surface, and $V$ the enclosed volume. The first term on the right-hand side of
the above expression represents the work needed to overcome surface tension, whilst the second term
represents the work required to overcome the  pressure difference, $-p_0$, between the exterior and
the interior of the bubble. Now, from the general principles of statics, we  expect a stable equilibrium state
of a mechanical system to be such as to {\em minimize}\/ the net free energy, subject to any dynamical
constraints.  It follows that the equilibrium shape of the bubble 
is such as to minimize
\begin{equation}
{\cal E} = \gamma\int_{z_1}^{z_2} 2\pi\,f\,(1+f_z^{\,2})^{1/2}\,dz-p_0\int_{z_1}^{z_2}\pi\,f^{\,2}\,dz,
\end{equation}
subject to the constraint that the bubble radius, $f$, be fixed at $z=z_1, z_2$.
Hence, we need to find the function $f(z)$ that minimizes the integral
\begin{equation}
\int_{z_1}^{z_2} {\cal L}(f,f_z)\,dz,
\end{equation}
where
\begin{equation}
{\cal L}(f,f_z) = 2\pi\,\gamma\,f\,(1+f_z^{\,2})^{1/2} - \pi\,p_0\,f^{\,2},
\end{equation}
subject to the constraint that $f$ is fixed at the limits. This is a standard problem in the {\em calculus of
variations}. See Appendix~\ref{cvar}. In fact, since the functional ${\cal L}(f,f_z)$ does not depend explicitly
on $z$, the minimizing function is the solution of [see Equation~(\ref{e11.14})]
\begin{equation}
{\cal L}-f_z\,\frac{\partial {\cal L}}{\partial f_z}  = C',
\end{equation}
where $C'$ is an arbitrary constant. 
Thus, we obtain 
\begin{equation}
2\pi\,\gamma\left[\frac{f}{(1+f_z^{\,2})^{1/2}}-\frac{f^{\,2}}{2\,a}\right] = C',
\end{equation}
which can be rearranged to give Equation~(\ref{e6x.59}). Hence, we conclude that application of the Young-Laplace
equation does indeed lead to a bubble shape that minimizes the net free energy of the soap/air interfaces.

Consider the case $p_0=0$, in which there is no pressure difference across the surface of the bubble. 
In this situation, writing
$C=b>0$, Equation~(\ref{e6x.59}) reduces to
\begin{equation}\label{e6x.57}
f = b\,(1+f_z^{\,2})^{1/2}.
\end{equation}
Moreover,  according to the previous discussion, the bubble shape specified by (\ref{e6x.57}) is such as to  minimize  the {\em surface area}\/ of the bubble (since the only contribution to the free energy of the soap/air interfaces is directly proportional
to the bubble area). 
The above equation can be rearranged to give
\begin{equation}
f_z=\pm \left(\frac{f^{\,2}}{b^{\,2}}-1\right)^{1/2},
\end{equation}
which leads to
\begin{equation}
z-z_0 = \int_b^r\,\frac{df}{f_z} = \pm\int_b^r\frac{df}{(f^{\,2}/b^{\,2}-1)^{1/2}}=\pm b\,\cosh^{-1}(r/b),
\end{equation}
or
\begin{equation}\label{e6x.67}
r = b\,\cosh(|z-z_0|/b),
\end{equation}
where $z_0$ is a constant. 
This expression describes an axisymmetric surface known as a {\em catenoid}. 

\begin{figure}
\epsfysize=3.5in
\centerline{\epsffile{Chapter04/fig6.07.eps}}
\caption{\em Radius  versus axial distance for a catenoid soap bubble supported by two parallel co-axial rings
of radius $c$ located at $z=\pm 0.65\,c$.}\label{f6x.07}
\end{figure}

Suppose, for instance, that the soap bubble is
supported by identical rings of radius $c$ that are located a perpendicular
distance $2\,d$ apart. Without loss of generality, we can specify that the rings lie at $z=\pm d$. It thus follows,
from (\ref{e6x.67}), that $z_0=0$, and
\begin{equation}\label{e6x.61}
r = b\,\cosh(z/b).
\end{equation}
Here, the parameter $b$ must be chosen so as to satisfy
\begin{equation}
c = b\,\cosh(d/b).
\end{equation}
For example, if $d=0.65\,c$ then $b=0.6416\,c$, and the resulting  bubble shape is illustrated in Figure~\ref{f6x.07}. 

Let $d/c=\zeta$ and $d/b=u$, in which case the above equation becomes
\begin{equation}\label{e6x.70}
G(u) \equiv u - \zeta\,\cosh u = 0.
\end{equation}
Now, the function $G(u)$ attains a maximum value
\begin{equation}
G(u_0) = u_0-\frac{1}{\tanh u_0},
\end{equation}
when $ u_0= \sinh^{-1}(1/\zeta)$. Moreover, if $G(u_0)>0$ then Equation~(\ref{e6x.70}) possesses {\em two}\/ roots. It turns
out that the root associated with the smaller value of $u$ {\em minimizes}\/ the interface system energy, whereas
the other root {\em maximizes}\/ the free energy. Hence, the former root corresponds to a {\em stable}\/
equilibrium state, whereas the latter  corresponds to an {\em unstable}\/
equilibrium state. On the other hand, if $G(u_0)<0$ then Equation~(\ref{e6x.70}) 
possesses {\em no}\/ roots, implying the absence of any equilibrium state. The critical case $G(u_0)=0$ 
corresponds to $u=u_c$ and $\zeta=\zeta_c$, where $u_c\,\tanh u_c=1$ and $\zeta_c=1/\sinh u_c$. 
It is easily demonstrated that $u_c=1.1997$ and $\zeta_c=0.6627$. 
We conclude that  a stable equilibrium state of a catenoid bubble only exists when $\zeta\leq \zeta_c$, which corresponds to
$d\leq 0.6627\,c$. If the relative ring spacing $d$ exceeds the critical value $0.6627\,c$ then the bubble presumably bursts. 

\begin{figure}
\epsfysize=3.5in
\centerline{\epsffile{Chapter04/fig6.08.eps}}
\caption{\em Radius  versus axial distance for an unduloid soap bubble  calculated with $k=0.95$.}\label{f6x.08}
\end{figure}

Consider the case $p_0\neq 0$, in which there is a pressure difference across the surface of the bubble. 
In this situation, writing
\begin{eqnarray}
2\,a &=&\alpha+\beta,\\[0.5ex]
2a\,C &=&\alpha\,\beta,
\end{eqnarray}
Equation~(\ref{e6x.59}) becomes
\begin{equation}\label{e6x.74}
\frac{(\alpha+\beta)\,f}{(1+f_z^{\,2})^{1/2}}= f^{\,2} + \alpha\,\beta,
\end{equation}
which can be rearranged to give
\begin{equation}
f_z = \mp \frac{(\alpha^2-f^{\,2})^{1/2}\,(f^{\,2}-\beta^2)^{1/2}}{f^{\,2}+\alpha\,\beta}.
\end{equation}
We can assume, without loss of generality, that $|\alpha|>|\beta|$. 
It follows, from the above expression,
that $|\alpha|\leq f\leq |\beta|$. Hence, we can write
\begin{eqnarray}
f^{\,2} &=& \alpha^2\,\cos^2\phi + \beta^{\,2}\,\sin^2\phi,\\[0.5ex]
k^2 &=& \frac{\alpha^2-\beta^2}{\alpha^2},
\end{eqnarray}
where $0\leq\phi\leq \pi/2$ and $0< k\leq 1$. 
It follows that
\begin{eqnarray}
f&=& |\alpha|\,(1-k^2\,\sin^2\phi)^{1/2},\\[0.5ex]
\beta &=&{\rm sgn}(\beta)\,|\alpha|\,(1-k^2)^{1/2},
\end{eqnarray}
and
\begin{equation}
\frac{dz}{d\phi} = \frac{1}{f_z}\,\frac{df}{d\phi} = \pm \left(f + \frac{\alpha\,\beta}{f}\right),
\end{equation}
which can be integrated to give
\begin{equation}
|z|= |\alpha|\left[E(\phi,k) + {\rm sgn}(\alpha\,\beta)\,(1-k^2)^{1/2}\,F(\phi,k)\right],
\end{equation}
where $E(\phi,k)$ and $F(\phi,k)$ are {\em incomplete elliptic integrals}\/ [see Equations~(\ref{e8x56})
and (\ref{e8x57})]. Here, we have assumed that $\phi=0$ when $z=0$. 
There are three cases of interest. 

\begin{figure}
\epsfysize=3.5in
\centerline{\epsffile{Chapter04/fig6.09.eps}}
\caption{\em Radius  versus axial distance for a positive  pressure nodoid soap bubble calculated with $k=0.95$.}\label{f6x.09}
\end{figure}

In the first case, $\alpha>0$ and $\beta>0$. It follows that $(1-k^2)^{1/2}\leq r/\alpha\leq 1$ for
$\pi/2\geq\phi\geq 0$, and $0.5\leq \gamma/(p_0\,\alpha) < 1$ for $1\geq k> 0$, where
\begin{eqnarray}
r &=&\alpha\,(1-k^2\,\sin^2\phi),\\[0.5ex]
|z|&=& \alpha\left[E(\phi,k) +(1-k^2)^{1/2}\,F(\phi,k)\right].
\end{eqnarray}
The axisymmetric curve parameterized by the above pair of equations is known as an {\em unduloid}. Note that an
unduloid bubble always has {\em positive}\/ internal pressure (relative to the external pressure): {\em i.e.}, $p_0>0$. 
An example unduloid soap bubble is illustrated in Figure~\ref{f6x.08}

\begin{figure}
\epsfysize=3.5in
\centerline{\epsffile{Chapter04/fig6.10.eps}}
\caption{\em Radius  versus axial distance for a negative  pressure nodoid soap bubble calculated with $k=0.95$.}\label{f6x.10}
\end{figure}

In the second case, $\alpha>0$ and $\beta<0$. It follows that $(1-k^2)^{1/4}\leq r/\alpha \leq 1$
for $\phi_0\geq \phi\geq 0$, and $0<\gamma/(p_0\,\alpha)\leq 0.5$ for $0< k \leq 1$, where
$\phi_0= \sin^{-1}([1-(1-k^2)^{1/2}]^{1/2}/k)$, and 
\begin{eqnarray}
r &=&\alpha\,(1-k^2\,\sin^2\phi),\\[0.5ex]
|z|&=& \alpha\left[E(\phi,k) -(1-k^2)^{1/2}\,F(\phi,k)\right].
\end{eqnarray}
The axisymmetric curve parameterized by the above pair of equations is known as an {\em nodoid}. This
particular type of nodoid bubble has  {\em positive}\/ internal  pressure: {\em i.e.}, $p_0>0$. 
An example positive pressure nodoid soap bubble is illustrated in Figure~\ref{f6x.09}.

In the third case, $\alpha<0$ and $\beta>0$. It follows that $(1-k^2)^{1/2}|\leq r/|\alpha|\leq \alpha|\,(1-k^2)^{1/4}$
for $\pi/2\geq \phi\geq \phi_0$ (or $\pi/2\leq \phi\leq \pi-\phi_0$), and $0>\gamma/(p_0\,|\alpha|)\geq -0.5$ for $0< k \leq1$, where
\begin{eqnarray}
r &=&|\alpha|\,(1-k^2\,\sin^2\phi),\\[0.5ex]
|z|&=& |\alpha|\left[E(\phi,k) -(1-k^2)^{1/2}\,F(\phi,k)\right].
\end{eqnarray}
The axisymmetric curve parameterized by the above pair of equations is again a nodoid. However, this
particular type of nodoid bubble has  {\em negative}\/ internal pressure: {\em i.e.}, $p_0<0$.
An example negative pressure nodoid soap bubble is illustrated in Figure~\ref{f6x.10}. 

\section{Exercises}
{\small 
\renewcommand{\theenumi}{4.\arabic{enumi}}
\begin{enumerate}
\item Show that if $N$ equal spheres of water coalesce so as to form a single spherical drop then the
surface energy is decreased by a factor $1/N^{2/3}$. 
\item A circular cylinder of radius $a$, height $h$, and specific gravity $s$ floats upright in water. Show
that the depth of the base below the general level of the water surface is
$$
s\,h + \frac{2\,\gamma}{a}\,\cos\theta,
$$
where $\gamma$ is the surface tension at the air/water interface, and $\theta$ the contact angle of the
interface with the cylinder. 

\item A film of water is held between two parallel plates of glass a small distance $d$ apart. Prove
that the apparent attraction between the plates is
$$
\frac{2\,A\,\gamma\,\cos\theta}{d} + L\,\gamma\,\sin\theta,
$$
where $\gamma$ is the surface tension at the air/water interface, and $\theta$ the angle of contact of the
interface with glass.

\item Show that if the surface of a sheet of water is slightly corrugated then the surface energy is
increased by
$$
\frac{\gamma}{2}\int \left(\frac{\partial \zeta}{\partial x}\right)^2\,dx
$$
per unit breadth of the corrugations. Here, $x$ is measured horizontally, perpendicular to the corrugations. 
Moreover, $\zeta$ denotes the elevation of the surface above the mean level. Finally, $\gamma$ is the
surface tension at an air/water interface.  If the corrugations are sinusoidal, such that
$$
\zeta = c\,\sin(k\,x),
$$
show that the average increment of the surface energy per unit area is $(1/4)\,\gamma\,k^2\,c^2$.

\item A mass of liquid, which is held together by surface tension alone, revolves about a
fixed axis at a small angular velocity $\omega$, so as to assume a slightly spheroidal shape of
mean radius $a$. Prove that the ellipticity of the spheroid is 
$$
\frac{\rho\,\omega^2\,a^3}{8\,\gamma},
$$
where $\rho$ is the uniform mass density, and $\gamma$ the surface tension.

\item A liquid mass rotates, in the form of a circular ring of radius $a$ and small cross-section, with a constant angular
velocity $\omega$, about an axis normal to the plane of the ring, and passing through its center. The mass is
held together by surface tension alone. Show that the section of the ring must be approximately circular. 
Demonstrate that
$$
\omega = \left(\frac{2\,\gamma}{\rho\,a\,c^2}\right)^{1/2},
$$
where $\rho$ is the density, $\gamma$ the surface tension, and $c$ the radius of the cross-section.

\item Two spherical soap bubbles of radii $a_1$ and $a_2$ are made to coalesce. Show that when the
temperature of the gas in the resulting bubble has returned to its initial value the radius $a$ of
the bubble satisfies
$$
p_0\,a^3+ 4\,\gamma\,a^2= p_0\,(a_1^{\,3}+a_2^{\,3})+4\,\gamma\,(a_1^{\,2}+a_2^{\,2}),
$$
where $p_0$ is the ambient pressure, and $\gamma$ the surface tension of the soap/air interfaces. 

\item A rigid sphere of radius $a$ rests on a flat rigid surface, and a small amount of liquid surrounds
the contact point, making a concave-planar lens whose diameter is small compared to $a$. The
angle of contact of the liquid/air interface with each of the solid surfaces is zero, and the surface tension of the interface is $\gamma$. Show that there is an adhesive force of magnitude $4\pi\,a\,\gamma$ acting
on the sphere. (It is interesting to note that the force is independent of the volume of  liquid.)

\item Two small solid bodies are floating on the surface of a liquid. Show that the effect of
surface tension is to make the objects approach one another if the liquid/air interface has either an acute or an obtuse angle
of contact with both bodies, and to make them move away from one another if the interface has an acute
angle of contact with one body, and an obtuse angle of contact with the other.
\end{enumerate}}
