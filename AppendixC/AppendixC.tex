\chapter{Non-Cartesian Coordinates}\label{ccurv}
\section{Introduction}
In fluid mechanics {\em non-Cartesian coordinates}\/ are often used to exploit the symmetry of particular 
fluid systems. For example, it is convenient  to employ cylindrical coordinates to describe systems possessing
axial symmetry. In this Appendix, we  investigate  a particularly useful class of
non-Cartesian coordinates known as {\em orthogonal curvilinear coordinates}. The two most commonly occurring  examples of this class  in fluid mechanics are {\em cylindrical}\/ 
and {\em spherical}\/ coordinates. (Note, incidentally, that the Einstein summation convention is {\em not}\/ used in this Appendix.)

\section{Orthogonal Curvilinear Coordinates}
Let $x_1$, $x_2$, $x_3$ be a set of standard right-handed Cartesian coordinates. Furthermore, let $u_1(x_1, x_2, x_3)$, 
$u_2(x_1, x_2, x_3)$, $u_3(x_1, x_2, x_3)$ be three independent functions of these coordinates which are
such that each unique triplet of $x_1$, $x_2$, $x_3$ values is associated with a unique triplet of
$u_1$, $u_2$, $u_3$ values. It follows that $u_1$, $u_2$, $u_3$ can be used as an alternative set of coordinates  to
distinguish different points in space. Since the surfaces of constant $u_1$, $u_2$, and $u_3$ are not
generally parallel planes, but rather curved surfaces, this type of coordinate system is termed {\em curvilinear}. 

Let $h_1=|\nabla u_1|^{-1}$, $h_2=|\nabla u_2|^{-1}$, and $h_3=|\nabla u_3|^{-1}$. It follows that
${\bf e}_1=h_1\,\nabla u_1$, ${\bf e}_2=h_2\,\nabla u_2$, and ${\bf e}_3=h_3\,\nabla u_3$ are
a set of unit basis vectors which are  normal to surfaces of constant $u_1$, $u_2$, and $u_3$, respectively, at all points
in space. Note, however,  that the direction of these basis vectors is generally a function of position. Suppose that
the ${\bf e}_i$, where $i$ runs from 1 to 3, are {\em mutually orthogonal}\/ at all points in space: {\em i.e.},
\begin{equation}\label{c4.2}
{\bf e}_i\cdot {\bf e}_j = \delta_{ij}.
\end{equation}
In this case, $u_1$, $u_2$,  $u_3$ are said to constitute an  {\em orthogonal}\/ coordinate system.
Suppose, further, that
\begin{equation}\label{c4.3}
{\bf e}_1\cdot {\bf e}_2\times {\bf e}_3 = 1
\end{equation}
at all points in space, so that $u_1$, $u_2$,  $u_3$   also constitute a {\em right-handed}\/
 coordinate system. It follows that
\begin{equation}\label{c4.4}
{\bf e}_i\cdot{\bf e}_j\times {\bf e}_k = \epsilon_{ijk}.
\end{equation}
Finally, a general vector ${\bf A}$, associated with  a particular point in space, can be written
\begin{equation}
{\bf A} = \sum_i A_i\,{\bf e}_i,
\end{equation}
where the ${\bf e}_i$ are the local basis vectors of the $u_1$, $u_2$, $u_3$ system, and $A_i={\bf e}_i\cdot {\bf A}$ is termed the $i$th component of ${\bf A}$ in this system. 

Consider two neighboring points in space whose coordinates in the $u_1$, $u_2$, $u_3$ system  are $u_1$, $u_2$, $u_3$ and $u_1+d u_1$, $u_2+d u_2$, $u_3+d u_3$.
It is easily shown that the vector directed from the first to the second of these points takes the form
\begin{equation}
d {\bf x} = \frac{du_1}{|\nabla u_1|}\,{\bf e}_1 + \frac{du_2}{|\nabla u_2|}\,{\bf e}_2+\frac{du_3}{|\nabla u_3|}\,{\bf e}_3=\sum_i h_i\,du_i\,{\bf e}_i.
\end{equation}
Hence, from (\ref{c4.2}), an element of length (squared) in the $u_1$, $u_2$, $u_3$ coordinate system is written
\begin{equation}\label{c4.7}
d{\bf x}\cdot d{\bf x}= \sum_i h_i^{\,2}\,du_i^{\,2}.
\end{equation}
Here, the $h_i$, which are generally functions of position, are known as the {\em scale factors}\/ of the  system. 
Elements of area that are normal to ${\bf e}_1$, ${\bf e}_2$, and ${\bf e}_3$, at a given point in space,  take the form $dS_1 = h_2\,h_3\,du_2\,du_3$,
$dS_2 = h_1\,h_3\,du_1\,du_3$,  and $dS_3 = h_1\,h_2\,du_1\,du_2$, respectively. Finally, an element of
volume, at a given point in space, is written $dV = h\,du_1\,du_2\,du_3$, where
\begin{equation}
h = h_1\,h_2\,h_3.
\end{equation}

Note that [see Equation~(\ref{curlgrad})]
\begin{equation}\label{c4.8}
\nabla\times \nabla u_i = 0,
\end{equation}
and 
\begin{equation}\label{c4.9}
\nabla\cdot \left(\frac{h_i^{\,2}}{h}\,\nabla u_i\right) = 0.
\end{equation}
The latter result follows from Equations~(\ref{divvp}) and (\ref{curlgrad}) because $(h_1^{\,2}/h)\,\nabla u_1= \nabla u_2\times \nabla u_3$, 
{\em etc.} Finally, it is easily demonstrated from (\ref{c4.2}) and (\ref{c4.4}) that
\begin{eqnarray}
\nabla u_i\cdot \nabla u_j &=&h_i^{\,-2}\,\delta_{ij},\label{c4.10}\\[0.5ex]
\nabla u_i\cdot\nabla u_j\times \nabla u_k&=& h^{-1}\,\epsilon_{ijk}.\label{c4.11}
\end{eqnarray}

Consider a scalar field $\phi(u_1,u_2,u_3)$. It follows from the chain rule, and the relation ${\bf e}_i = h_i\,\nabla u_i$,
that 
\begin{equation}\label{c4.12}
\nabla \phi = \sum_i \frac{\partial \phi}{\partial u_i}\,\nabla u_i=\sum_i\frac{1}{h_i}\,\frac{\partial \phi}{\partial u_i}\,{\bf e}_i.
\end{equation}
Hence, the components of $\nabla\phi$  in the $u_1$, $u_2$, $u_3$ coordinate system are 
\begin{equation}\label{c4.13}
(\nabla\phi)_i = \frac{1}{h_i}\,\frac{\partial \phi}{\partial u_i}.
\end{equation}

Consider a vector field ${\bf A}(u_1,u_2,u_3)$. We can write
\begin{eqnarray}
\nabla\cdot{\bf A} &=&\sum_i\nabla\cdot(A_i\,{\bf e}_i) =\sum_i\nabla\cdot (h_i\,A_i\,\nabla u_i)=
\sum_i\nabla\!\cdot\!\left(\frac{h}{h_i}\,A_i\,\frac{h_i^{\,2}}{h}\,\nabla u_i\right)\nonumber\\[0.5ex]
&=&\sum_i\frac{h_i^{\,2}}{h}\nabla u_i\cdot\nabla\left(\frac{h}{h_i}\,A_i\right)=\sum_i
\frac{1}{h}\,\frac{\partial}{\partial u_i}\!\left(\frac{h}{h_i}\,A_i\right),
\end{eqnarray}
where use has been made of Equations~(\ref{divprod}), (\ref{c4.9}), and (\ref{c4.10}). Thus, the
divergence of ${\bf A}$ in the $u_1$, $u_2$, $u_3$ coordinate system takes the form
\begin{equation}\label{c4.14}
\nabla\cdot{\bf A} =\sum_i
\frac{1}{h}\,\frac{\partial}{\partial u_i}\!\left(\frac{h}{h_i}\,A_i\right).
\end{equation}

We can write
\begin{eqnarray}
\nabla\times {\bf A} &=& \sum_k\nabla\times (A_k\,{\bf e}_k)=\sum_k \nabla\times (h_k\,A_k\,\nabla u_k)
=\sum_k\nabla (h_k\,u_k)\,\times \nabla u_k\nonumber\\[0.5ex]
&=&\sum_{j,k}\frac{\partial (h_k\,A_k)}{\partial u_j}\,\nabla u_j\times\nabla u_k,
\end{eqnarray}
where use has been made of Equations~(\ref{curlprod}), (\ref{c4.8}), and (\ref{c4.12}).
It follows from (\ref{c4.11}) that
\begin{equation}
(\nabla\times {\bf A})_i = {\bf e}_i\cdot\nabla\times {\bf A} = \sum_{j,k}h_i\,\frac{\partial (h_k\,A_k)}{\partial u_j}\,\nabla u_i\cdot\nabla u_j\times \nabla u_k = \sum_{j,k}\epsilon_{ijk}\,\frac{h_i}{h}\,\frac{\partial (h_k\,A_k)}{\partial u_j}.
\end{equation}
 Hence, the components of $\nabla\times {\bf A}$ in the
$u_1$, $u_2$, $u_3$ coordinate system are
\begin{equation}\label{c4.17}
(\nabla\times{\bf A})_i =  \sum_{j,k}\epsilon_{ijk}\,\frac{h_i}{h}\,\frac{\partial (h_k\,A_k)}{\partial u_j}.
\end{equation}

Now, $\nabla^2\phi = \nabla\cdot\nabla\phi$ [see (\ref{laplacian})], so Equations~(\ref{c4.12}) and (\ref{c4.14})
yield the following expression for $\nabla^2\phi$ in the $u_1$, $u_2$, $u_3$ coordinate system:
\begin{equation}\label{c4.19}
\nabla^2\phi =\sum_i\frac{1}{h}\,\frac{\partial}{\partial u_i}\!\left(\frac{h}{h_i^{\,2}}\,\frac{\partial\phi}{\partial u_i}\right).
\end{equation}

The vector identities (\ref{divsp}) and (\ref{curlvp}) can be combined to give the
following expression for $({\bf A}\cdot\nabla){\bf B}$ that is valid in a general coordinate system:
\begin{eqnarray}\label{c4.20}
({\bf A}\cdot\nabla) {\bf B} &=& \frac{1}{2}\left[\nabla({\bf A}\cdot{\bf B}) - \nabla\times ({\bf A}\times {\bf B})
- (\nabla\cdot{\bf A})\,{\bf B} + (\nabla\cdot{\bf B})\,{\bf A} \right.\nonumber\\[0.5ex]
&&\left.
- {\bf A}\times(\nabla\times {\bf B})-{\bf B}\times (\nabla\times {\bf A})\right].
\end{eqnarray}
Making use of Equations~(\ref{c4.13}), (\ref{c4.14}), and (\ref{c4.17}), as well
as the easily demonstrated results
\begin{eqnarray}\label{c4.21}
{\bf A}\cdot{\bf B} &=&\sum_i A_i\,B_i,\\[0.5ex]
{\bf A}\times {\bf B} &=&\sum_{j,k} \epsilon_{ijk}\,A_j\,B_k,\label{c4.22}
\end{eqnarray}
and the tensor identity (\ref{e3.12}),  Equation~(\ref{c4.20}) reduces (after a great deal of tedious algebra) to the
following expression for the components of $({\bf A}\cdot\nabla){\bf B}$ in  the $u_1$, $u_2$, $u_3$
coordinate system:
\begin{equation}\label{c4.23}
[({\bf A}\cdot\nabla)\,{\bf B}]_i= \sum_j\left(\frac{A_j}{h_j}\,\frac{\partial B_i}{\partial u_j}- \frac{A_j\,B_j}{h_i\,h_j}\,\frac{\partial h_j}{\partial u_i} + \frac{A_i\,B_j}{h_i\,h_j}\,\frac{\partial h_i}{\partial u_j}\right).
\end{equation}
Note, incidentally, that the commonly quoted result $[({\bf A}\cdot\nabla){\bf B}]_i={\bf A}\cdot\nabla B_i$ is only valid in  Cartesian coordinate systems (for which $h_1=h_2=h_3=1$). 

Let us define the gradient $\nabla{\bf A}$ of a vector field ${\bf A}$ as the tensor whose components in a Cartesian coordinate
system take the form
\begin{equation}
(\nabla {\bf A})_{ij} = \frac{\partial A_i}{\partial x_j}.
\end{equation}
In an orthogonal curvilinear coordinate system, the above
expression generalizes to 
\begin{equation}
(\nabla{\bf A})_{ij} = [({\bf e}_j\cdot\nabla)\,{\bf A}]_i.
\end{equation}
It thus follows from (\ref{c4.23}), and the relation $({\bf e}_i)_j = {\bf e}_i\cdot {\bf e}_j=\delta_{ij}$, that
\begin{equation}\label{c4.26}
(\nabla{\bf A})_{ij} = \frac{1}{h_j}\,\frac{\partial A_i}{\partial u_j} - \frac{A_j}{h_i\,h_j}\,\frac{\partial h_j}{\partial u_i}
+\delta_{ij}\sum_k\frac{A_k}{h_i\,h_k}\,\frac{\partial h_i}{\partial u_k}.
\end{equation}

The vector identity (\ref{curlcurl}) yields the
following expression for $\nabla^2{\bf A}$ that is valid in a general coordinate system:
\begin{equation}
\nabla^2{\bf A} = \nabla(\nabla\cdot{\bf A}) - \nabla\times(\nabla\times {\bf A}).
\end{equation}
Making use of Equations~(\ref{c4.14}), (\ref{c4.17}), and (\ref{c4.19}), as well
as (\ref{c4.21}) and (\ref{c4.22}), and the tensor identity (\ref{e3.12}),  the above equation reduces (after a great deal of
tedious algebra) to the following expression for the components of $\nabla^2{\bf A}$
in the $u_1$, $u_2$, $u_3$ coordinate system:
\begin{eqnarray}\label{c4.28}
(\nabla^2{\bf A})_i&=& \nabla^2 A_i +\sum_j\left\{\frac{2}{h_i\,h_j}\left(\frac{1}{h_i}\,\frac{\partial h_i}{\partial u_j}\,\frac{\partial}{\partial u_i}-
\frac{1}{h_j}\,\frac{\partial h_j}{\partial u_i}\,\frac{\partial}{\partial u_j}\right) A_j\right.\nonumber\\[0.5ex]
&&+\frac{h}{h_i\,h_j^{\,2}}\left[\frac{A_j}{h_i^{\,2}}\,\frac{\partial h_j}{\partial u_i}\,\frac{\partial}{\partial u_j}\!\left(
\frac{h_i^{\,2}}{h}\right)- \frac{A_i}{h_j^{\,2}}\,\frac{\partial h_i}{\partial u_j}\,\frac{\partial}{\partial u_j}\!\left(
\frac{h_j^{\,2}}{h}\right) \right]\nonumber\\[0.5ex]
&&+\frac{A_j}{h_i}\,\frac{h}{h_j^{\,3}}\left[\frac{1}{h_j}\,\frac{\partial h_j}{\partial u_i}\,\frac{\partial}{\partial u_j}\!\left(\frac{h_j^{\,2}}{h}\right)+\frac{h}{h_j^{\,2}}\,\frac{\partial}{\partial u_i}\!\left(\frac{h_j^{\,2}}{h}\right)\frac{\partial}{\partial u_j}\!\left(\frac{h_j^{\,2}}{h}\right)-\frac{\partial^2}{\partial u_i\,\partial u_j}\!\left(\frac{h_j^{\,2}}{h}\right)\right]\nonumber\\[0.5ex]
&&\left.-\frac{A_i}{h_i\,h_j^{\,2}}\left[\frac{2}{h_i}\left(\frac{\partial h_i}{\partial u_j}\right)^2-\frac{\partial^2 h_i}{\partial
u_j^{\,2}}\right]\right\}.
\end{eqnarray}
Note, again,  that the commonly quoted result $(\nabla^2 {\bf A})_i=\nabla^2 A_i$ is only valid in Cartesian coordinate systems (for which $h_1=h_2=h_3=h=1$). 

\section{Cylindrical Coordinates}\label{scyl}
In the {\em cylindrical}\/ coordinate system, $u_1=r$, $u_2=\theta$, and $u_3=z$, 
 where $r=\sqrt{x^2+y^2}$,  $\theta=\tan^{-1}(y/x)$, and $x$, $y$, $z$ are standard Cartesian coordinates. 
Thus, $r$ is the perpendicular distance from the $z$-axis, and $\theta$
the angle subtended between the projection of the radius vector ({\em i.e.}, the vector connecting the origin to
a general point in space) onto the $x$-$y$ plane and the $x$-axis. See
Figure~\ref{fcyl}.

\begin{figure}
\epsfysize=2.75in
\centerline{\epsffile{AppendixC/figC.01.eps}}
\caption{\em Cylindrical coordinates.}\label{fcyl}
\end{figure}

 A general  vector ${\bf A}$ is  written
\begin{equation}
{\bf A} = A_r\,{\bf e}_r+ A_\theta\,{\bf e}_\theta + A_z\,{\bf e}_z,
\end{equation}
where ${\bf e}_r=\nabla r/|\nabla r|$, ${\bf e}_\theta = \nabla\theta/|\nabla\theta|$, and ${\bf e}_z=\nabla z/|\nabla z|$. See Figure~\ref{fcyl}. Of course, the unit basis vectors
${\bf e}_r$, ${\bf e}_\theta$, and ${\bf e}_z$ are mutually orthogonal, so
$A_r = {\bf A}\cdot {\bf e}_r$, {\em etc.} 

As is easily demonstrated, an element of length (squared) in the cylindrical coordinate system takes the form
\begin{equation}
d{\bf x}\cdot d{\bf x} = dr^{\,2} + r^2\,d\theta^{\,2} + dz^2.
\end{equation}
Hence,  comparison with Equation~(\ref{c4.7}) reveals that the scale factors for this system are
\begin{eqnarray}
h_r &=& 1,\\[0.5ex]
h_\theta &=& r,\\[0.5ex]
h_z &=& 1.
\end{eqnarray}
Thus,  surface elements normal to ${\bf e}_r$, ${\bf e}_\theta$, and ${\bf e}_z$ are
written
\begin{eqnarray}
dS_r& =& r\,d\theta\,dz,\\[0.5ex]
dS_\theta &=& dr\,dz,\\[0.5ex]
dS_z &=& r\,dr\,d\theta,
\end{eqnarray}
respectively, whereas  
a
volume element takes the form
\begin{equation}
dV = r\,dr\,d\theta\,dz.
\end{equation}
 
 According to Equations~(\ref{c4.13}), (\ref{c4.14}), and (\ref{c4.17}), gradient, divergence, and curl in the cylindrical
 coordinate system are written
\begin{eqnarray}
\nabla \psi &=& \frac{\partial \psi}{\partial r}\,{\bf e}_r
+ \frac{1}{r}\frac{\partial \psi}{\partial\theta}\,{\bf e}_\theta
+ \frac{\partial \psi}{\partial z}\,{\bf e}_z,\\[0.5ex]
\nabla\cdot {\bf A} &=&\frac{1}{r}\,\frac{\partial}{\partial r}\,(r\,A_r) + \frac{1}{r}\,\frac{\partial A_\theta}{\partial\theta} + \frac{\partial A_z}{\partial z},\label{ec39}\\[0.5ex]
\nabla\times{\bf A} &=& \left(\frac{1}{r}\,\frac{\partial A_z}{\partial \theta}-\frac{\partial A_\theta}{\partial z}\right){\bf e}_r
+\left(\frac{\partial A_r}{\partial z}-\frac{\partial A_z}{\partial r}\right){\bf e}_\theta
+ \left(\frac{1}{r}\,\frac{\partial}{\partial r}\,(r\,A_\theta) - \frac{1}{r}\,\frac{\partial A_r}{\partial\theta}\right){\bf e}_z,\nonumber\\[0.5ex]&&\label{spcurl}
\end{eqnarray}
respectively. Here, $\psi({\bf r})$ is a general scalar field, and ${\bf A}({\bf r})$ a general vector field. 

According to Equation~(\ref{c4.19}), when expressed in cylindrical coordinates, the Laplacian of a scalar field becomes
\begin{equation}
\nabla^2 \psi = \frac{1}{r}\,\frac{\partial}{\partial r}\left(r\,\frac{\partial \psi}{\partial r}\right) + \frac{1}{r^2}\,\frac{\partial^2 \psi}{\partial\theta^2} + \frac{\partial^2 \psi}{\partial z^2}.
\end{equation} 

Moreover, from Equation~(\ref{c4.23}), the components of $({\bf A}\cdot\nabla){\bf A}$ in the cylindrical coordinate system are
\begin{eqnarray}
[({\bf A}\cdot\nabla){\bf A}]_r&=& {\bf A}\cdot\nabla A_r - \frac{A_\theta^{\,2}}{r},\\[0.5ex]
[({\bf A}\cdot\nabla){\bf A}]_\theta&=& {\bf A}\cdot\nabla A_\theta + \frac{A_r\,A_\theta}{r},\\[0.5ex]
[({\bf A}\cdot\nabla){\bf A}]_z&=& {\bf A}\cdot\nabla A_z.
\end{eqnarray}

Let us define the symmetric gradient  tensor
\begin{equation}
\widetilde{\nabla {\bf A}} = \frac{1}{2}\left[\nabla {\bf A} + (\nabla{\bf A})^T\right].
\end{equation}
Here, the superscript $T$ denotes a {\em transpose}. Thus, if the $ij$ element of some second-order tensor ${\bf S}$ is $S_{ij}$ then the
corresponding element of ${\bf S}^T$ is $S_{ji}$. 
According to Equation~(\ref{c4.26}), the components of $\widetilde{\nabla {\bf A}}$ in the cylindrical
coordinate system are 
\begin{eqnarray}
(\widetilde{\nabla {\bf A}})_{rr} &=& \frac{\partial A_r}{\partial r},\\[0.5ex]
(\widetilde{\nabla {\bf A}})_{\theta\theta} &=&\frac{1}{r} \frac{\partial A_\theta}{\partial \theta}+ \frac{A_r}{r},\\[0.5ex]
(\widetilde{\nabla {\bf A}})_{zz} &=& \frac{\partial A_z}{\partial z},\\[0.5ex]
(\widetilde{\nabla {\bf A}})_{r\theta}=(\widetilde{\nabla {\bf A}})_{\theta r} &=& \frac{1}{2}\left(\frac{1}{r}\,\frac{\partial A_r}{\partial\theta} + \frac{\partial A_\theta}{\partial r} - \frac{A_\theta}{r}\right),\\[0.5ex]
(\widetilde{\nabla {\bf A}})_{rz}=(\widetilde{\nabla {\bf A}})_{zr} &=&\frac{1}{2}\left(\frac{\partial A_r}{\partial z} + \frac{\partial A_z}{\partial r}\right),\\[0.5ex]
(\widetilde{\nabla {\bf A}})_{\theta z}=(\widetilde{\nabla {\bf A}})_{z\theta} &=&\frac{1}{2}\left(\frac{\partial A_\theta}{\partial z} + \frac{1}{r}\frac{\partial A_z}{\partial \theta}\right).
\end{eqnarray}

Finally, from Equation~(\ref{c4.28}), the components of $\nabla^2{\bf A}$ in the
cylindrical coordinate system are
\begin{eqnarray}
(\nabla^2{\bf A})_r &=&\nabla^2 A_r - \frac{A_r}{r^2} - \frac{2}{r^2}\,\frac{\partial A_\theta}{\partial\theta},\\[0.5ex]
(\nabla^2{\bf A})_\theta &=& \nabla^2 A_\theta + \frac{2}{r^2}\,\frac{\partial A_r}{\partial \theta}-\frac{A_\theta}{r^2},\\[0.5ex]
(\nabla^2{\bf A})_z &=&\nabla^2 A_z.
\end{eqnarray}

\section{Spherical Coordinates}\label{ssph}
In the {\em spherical}\/ coordinate system, $u_1=r$, $u_2=\theta$, and $u_3=\phi$, 
 where $r=\sqrt{x^2+y^2+z^2}$,  $\theta=\cos^{-1}(z/r)$,  $\phi=\tan^{-1}(y/x)$, and $x$, $y$, $z$ are standard Cartesian coordinates. 
Thus, $r$ is the length of the radius vector,  $\theta$
the angle subtended between the radius vector and the $z$-axis,  and $\phi$ the angle subtended between the projection of the radius vector
onto the $x$-$y$ plane and the $x$-axis. See Figure~\ref{fsphx}.

\begin{figure}
\epsfysize=2.75in
\centerline{\epsffile{AppendixC/figC.02.eps}}
\caption{\em Spherical coordinates.}\label{fsphx}
\end{figure}

A general  vector ${\bf A}$ is written
\begin{equation}
{\bf A} = A_r\,{\bf e}_r+ A_\theta\,{\bf e}_\theta + A_\phi\,{\bf e}_\phi,
\end{equation}
where ${\bf e}_r=\nabla r/|\nabla r|$, ${\bf e}_\theta = \nabla\theta/|\nabla\theta|$, and ${\bf e}_\phi=\nabla \phi/|\nabla \phi|$. See Figure~\ref{fsphx}. Of course, the unit vectors
${\bf e}_r$, ${\bf e}_\theta$, and ${\bf e}_\phi$ are mutually orthogonal, so
$A_r = {\bf A}\cdot {\bf e}_r$, {\em etc.} 

 As is easily demonstrated,  an element of length (squared) in the spherical coordinate system takes the form
\begin{equation}
d {\bf x}\cdot d{\bf x} = dr^{\,2} + r^2\,d\theta^{\,2} + r^2\,\sin^2\theta\,d\phi^2.
\end{equation}
Hence,  comparison with Equation~(\ref{c4.7}) reveals that the scale factors for this system are
\begin{eqnarray}
h_r &=& 1,\\[0.5ex]
h_\theta &=& r,\\[0.5ex]
h_\phi &=& r\,\sin\theta.
\end{eqnarray}
Thus,  surface elements normal to ${\bf e}_r$, ${\bf e}_\theta$, and ${\bf e}_\phi$ are
written
\begin{eqnarray}
dS_r& =& r^2\,\sin\theta\,d\theta\,d\phi,\\[0.5ex]
dS_\theta &=& r\,\sin\theta\,dr\,d\phi,\\[0.5ex]
dS_\phi &=& r\,dr\,d\theta,
\end{eqnarray}
respectively, whereas  
a
volume element takes the form
\begin{equation}
dV = r^2\,\sin\theta\,dr\,d\theta\,d\phi.
\end{equation}
 
 According to Equations~(\ref{c4.13}), (\ref{c4.14}), and (\ref{c4.17}), gradient, divergence, and curl in the spherical
 coordinate system  are written
 \begin{eqnarray}
\nabla \psi &=& \frac{\partial \psi}{\partial r}\,{\bf e}_r
+ \frac{1}{r}\frac{\partial \psi}{\partial\theta}\,\,{\bf e}_\theta
+ \frac{1}{r\,\sin\theta}\,\frac{\partial \psi}{\partial \phi}\,{\bf e}_\phi,\\[0.5ex]
\nabla\cdot {\bf A} &=&\frac{1}{r^2}\,\frac{\partial}{\partial r}\,(r^2\,A_r) + \frac{1}{r\,\sin\theta}\,\frac{\partial }{\partial\theta} \,(\sin\theta\,A_\theta)+ \frac{1}{r\,\sin\theta}\,\frac{\partial A_\phi}{\partial \phi},\label{ec65}\\[0.5ex]
\nabla\times{\bf A} &=& \left[\frac{1}{r\,\sin\theta}\,\frac{\partial}{\partial \theta}\,(\sin\theta \,A_\phi)-\frac{1}{r\,\sin\theta}\,\frac{\partial A_\theta}{\partial \phi}\right]{\bf e}_r\nonumber\\[0.5ex]
&&+\left[\frac{1}{r\,\sin\theta}\,\frac{\partial A_r}{\partial \phi}-\frac{1}{r}\frac{\partial}{\partial r}\,(r\,A_\phi)\right]{\bf e}_\theta\nonumber\\[0.5ex]
&&+ \left[\frac{1}{r}\,\frac{\partial}{\partial r}\,(r\,A_\theta) - \frac{1}{r}\,\frac{\partial A_r}{\partial\theta}\right]{\bf e}_\phi,\
\end{eqnarray}
respectively. Here, $\psi({\bf r})$ is a general scalar field, and ${\bf A}({\bf r})$ a general vector field. 

According to Equation~(\ref{c4.19}), when expressed in spherical coordinates, the Laplacian of a scalar field becomes
\begin{equation}
\nabla^2 \psi= \frac{1}{r^2}\,\frac{\partial}{\partial r}\left(r^2\,\frac{\partial \psi}{\partial r}\right) + \frac{1}{r^2\,\sin\theta}\,\frac{\partial }{\partial\theta}\left(\sin\theta\,\frac{\partial \psi}{\partial \theta}\right) + \frac{1}{r^2\,\sin^2\theta}\,\frac{\partial^2 \psi}{\partial \phi^2}.
\end{equation}

Moreover, from  Equation~(\ref{c4.23}), the components of $({\bf A}\cdot\nabla){\bf A}$ in the spherical coordinate system are
\begin{eqnarray}
[({\bf A}\cdot\nabla){\bf A}]_r&=& {\bf A}\cdot\nabla A_r - \frac{A_\theta^{\,2}+A_\phi^{\,2}}{r},\\[0.5ex]
[({\bf A}\cdot\nabla){\bf A}]_\theta&=& {\bf A}\cdot\nabla A_\theta + \frac{A_r\,A_\theta-\cot\theta\,A_\phi^{\,2}}{r},\\[0.5ex]
[({\bf A}\cdot\nabla){\bf A}]_\phi&=& {\bf A}\cdot\nabla A_\phi+ \frac{A_r\,A_\phi+\cot\theta\,A_\theta\,A_\phi}{r}.
\end{eqnarray}

Now, according to Equation~(\ref{c4.26}), the components of $\widetilde{\nabla {\bf A}}$ in the spherical
coordinate system are 
\begin{eqnarray}
(\widetilde{\nabla {\bf A}})_{rr} &=& \frac{\partial A_r}{\partial r},\\[0.5ex]
(\widetilde{\nabla {\bf A}})_{\theta\theta} &=&\frac{1}{r} \frac{\partial A_\theta}{\partial \theta}+ \frac{A_r}{r},\\[0.5ex]
(\widetilde{\nabla {\bf A}})_{\phi\phi} &=& \frac{1}{r\,\sin\theta}\,\frac{\partial A_\phi}{\partial \phi}+ \frac{A_r}{r} + \frac{\cot\theta\,A_\theta}{r},\\[0.5ex]
(\widetilde{\nabla {\bf A}})_{r\theta}=(\widetilde{\nabla {\bf A}})_{\theta r} &=& \frac{1}{2}\left(\frac{1}{r}\,\frac{\partial A_r}{\partial\theta} + \frac{\partial A_\theta}{\partial r} - \frac{A_\theta}{r}\right),\\[0.5ex]
(\widetilde{\nabla {\bf A}})_{r\phi}=(\widetilde{\nabla {\bf A}})_{\phi r} &=&\frac{1}{2}\left(\frac{1}{r\,\sin\theta}\,\frac{\partial A_r}{\partial \phi} + \frac{\partial A_\phi}{\partial r}-\frac{A_\phi}{r}\right),\\[0.5ex]
(\widetilde{\nabla {\bf A}})_{\theta \phi}=(\widetilde{\nabla {\bf A}})_{\phi\theta} &=&\frac{1}{2}\left(\frac{1}{r\,\sin\theta}\,\frac{\partial A_\theta}{\partial \phi} + \frac{1}{r}\frac{\partial A_\phi}{\partial \theta}-\frac{\cot\theta\,A_\phi}{r}\right).
\end{eqnarray}

Finally, from Equation~(\ref{c4.28}), the components of $\nabla^2{\bf A}$ in the
spherical coordinate system are
\begin{eqnarray}
(\nabla^2{\bf A})_r &=&\nabla^2 A_r  -\frac{2 A_r}{r^2}-\frac{2}{r^2}\,\frac{\partial A_\theta}{\partial\theta}-\frac{2\cot\theta\, A_\theta}{r^2} -\frac{2}{r^2\,\sin\theta}\,\frac{\partial A_\phi}{\partial\phi},\\[0.5ex]
(\nabla^2{\bf A})_\theta &=& \nabla^2 A_\theta +\frac{2}{r^2}\,\frac{\partial A_r}{\partial\theta}-\frac{A_\theta}{r^2\,\sin^2\theta}
-\frac{2}{r^2\,\sin\theta}\,\frac{\partial A_\phi}{\partial\phi},\\[0.5ex]
(\nabla^2{\bf A})_\phi &=&\nabla^2 A_\phi-\frac{A_\phi}{r^2\,\sin^2\theta} + \frac{2}{r^2\,\sin^2\theta}\,\frac{\partial A_r}{\partial\phi}+\frac{2 \cot\theta}{r^2\,\sin\theta}\,\frac{\partial A_\theta}{\partial\phi}.
\end{eqnarray}

\section{Exercises}
{\small 
\renewcommand{\theenumi}{C.\arabic{enumi}}
\begin{enumerate}
\item Find the Cartesian components of the basis vectors ${\bf e}_r$,
${\bf e}_\theta$, and ${\bf e}_z$ of the cylindrical coordinate
system. Verify that the vectors are mutually orthogonal. Do the
same for the basis vectors ${\bf e}_r$, ${\bf e}_\theta$, and ${\bf e}_\phi$ of the spherical  coordinate system.

\item Use cylindrical coordinates to prove that the volume of a right cylinder of radius $a$ and
length $l$ is $\pi\,a^2\,l$. Demonstrate that the moment of inertia of a uniform cylinder of mass $M$ and radius $a$ about
its symmetry axis is $(1/2)\,M\,a^2$.

\item Use spherical coordinates to prove that the volume of a sphere of radius $a$ is $(4/3)\,\pi\,a^3$. Demonstrate that the moment of inertia of a uniform sphere of mass $M$ and radius $a$ about
an axis passing through its center  is $(2/5)\,M\,a^2$.

\item For what value(s) of $n$ is $\nabla\cdot(r^n\,{\bf e}_r)=0$, where $r$ is a
spherical coordinate?

\item For what value(s) of $n$ is $\nabla\times(r^n\,{\bf e}_r)={\bf 0}$, where $r$ is a
spherical coordinate?

\item \begin{enumerate}
\item Find a vector field ${\bf F} = F_r(r)\,{\bf e}_r$ satisfying $\nabla\cdot {\bf F} = r^m$ for $m\geq 0$. Here,
$r$ is a spherical coordinate.
\item Use the divergence theorem to show that
$$
\int_V r^m\,dV = \frac{1}{m+3}\int_S r^{m+1}\,{\bf e}_r\cdot d{\bf S},
$$
where $V$ is a volume enclosed by a surface $S$.
\item Use the above result (for $m=0$) to demonstrate that the volume of a right cone is
one third the volume of the right cylinder having the same base and height. 
\end{enumerate}

\item The electric field generated by a $z$-directed electric dipole of moment $p$, located
at the origin, is
$$
{\bf E}({\bf r})= \frac{1}{4\pi\epsilon_0} \left[\frac{3\,({\bf e}_r\cdot {\bf p})\,{\bf e}_r-{\bf p}}{r^3}\right],
$$
where ${\bf p} = p\,{\bf e}_z$, and $r$ is a spherical coordinate. Find the components of ${\bf E}({\bf r})$ in
the spherical coordinate system. Calculate $\nabla\cdot{\bf E}$ and $\nabla\times {\bf E}$. 

\item Show that the {\em parabolic cylindrical coordinates}\/ $u$, $v$, $z$, defined
by the equations $x=(u^2-v^2)/2$, $y=u\,v$, $z=z$, where $x$, $y$, $z$ are Cartesian
coordinates, are orthogonal. Find the scale factors $h_u$, $h_v$, $h_z$.
What shapes are the $u={\rm constant}$ and $v={\rm constant}$ surfaces?
Write an expression for $\nabla^2 f$ in parabolic cylindrical coordinates.

\item Show that the {\em elliptic cylindrical coordinates}\/ $\xi$, $\eta$, $z$, defined
by the equations $x=\cosh\xi\,\cos\eta$, $y=\sinh\xi\,\sin\eta$, $z=z$, where $x$, $y$, $z$ are Cartesian
coordinates, and $0\leq \xi\leq \infty$, $-\pi<\eta \leq \pi$, are orthogonal. Find the scale factors $h_\xi$, $h_\eta$, $h_z$.
What  shapes are the $\xi={\rm constant}$ and $\eta={\rm constant}$ surfaces?
Write an expression for $\nabla f$ in elliptical cylindrical coordinates.

\end{enumerate}}