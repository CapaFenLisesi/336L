\chapter{Hydrostatics}\label{c6}
\section{Introduction}
Part~II of this book is devoted to the study of {\em incompressible}\/ fluids---in practice, such fluids are  generally  either liquids, or gases whose
motion is subsonic (see Section~\ref{scomp}). 
This particular chapter discusses {\em hydrostatics}, which is the study of the {\em mechanical equilibrium}\/ of incompressible fluids. 

\section{Hydrostatic Pressure}\label{s6.2}
Consider a    body of water that is stationary   in a reference frame that is fixed with respect to the Earth's surface. In this chapter, such a  frame is
treated as approximately inertial.  Let $z$ measure vertical height, and suppose that the region $z\leq 0$ is occupied
by water, and the region $z>0$  by air. 
According to Equation~(\ref{e5.78}), the air/water system remains in mechanical equilibrium  ({\em i.e.}, ${\bf v} = D{\bf v}/Dt= {\bf 0}$) provided
\begin{equation}\label{e6.1}
{\bf 0} = \nabla p + \rho\,\nabla\Psi,
\end{equation}
where $p$ is the static fluid pressure, $\rho$ the mass density,  $\Psi=g\,z$ the gravitational potential energy per unit mass,  and $g$
the (approximately uniform) acceleration due to gravity.
Now,
\begin{equation}\label{e6.2}
\rho(z) = \left\{ \begin{array}{ccc}
0&\mbox{\hspace{1cm}}&z>0\\[0.5ex]
\rho_0&&z\leq 0
\end{array}\right.,
\end{equation}
where $\rho_0$ is the (approximately uniform) mass density of water. Here, the comparatively small mass density of air has been neglected. Since $\rho=\rho(z)$ and $\Psi=\Psi(z)$, it immediately follows, from (\ref{e6.1}), that $p=p(z)$,
where
\begin{equation}\label{e6.3}
\frac{dp}{dz} = -\rho\,g.
\end{equation}
 We conclude that constant pressure
surfaces in a stationary body of water take the form of {\em horizontal planes}. 
Making use of (\ref{e6.2}), the above equation can be integrated to give
\begin{equation}\label{e6.4}
p(z)=\left\{\begin{array}{llc}
p_0&\mbox{\hspace{1cm}}&z>0\\[0.5ex]
p_0-\rho_0\,g\,z&&z\leq 0
\end{array}\right.,
\end{equation}
where $p_0\simeq 10^5\,{\rm N\,m}^{-2}$ is  atmospheric pressure at ground level. According to this expression,  pressure in stationary water increases {\em linearly}\/ with increasing depth  ({\em i.e.}, with decreasing $z$, for $z<0$). 
In fact, given that $g\simeq 9.8\,{\rm m\,s}^{-2}$ and $\rho_0\simeq 10^3\,{\rm kg\,m^{-3}}$, we deduce that
hydrostatic  pressure rises
at the rate of 1 atmosphere ({\em i.e.}, $10^5\,{\rm N\,m}^{-2}$) every $10.2\,{\rm m}$ increase in depth below the
surface.

\section{Buoyancy}\label{sbouy}
Consider the air/water system described in the previous section.
Let $V$ be some volume, bounded by a closed surface $S$, that straddles the plane $z=0$, and is thus partially occupied by water, and partially
by air. The $i$-component of the net force acting on the fluid ({\em i.e.}, either water or air) contained within $V$ is written (see Section~\ref{s5.4})
\begin{equation}\label{e6.5}
f_i = \int_S \sigma_{ij}\,dS_j + \int_V F_i\,dV,
\end{equation}
where
\begin{equation}
\sigma_{ij} = - p\,\delta_{ij}
\end{equation}
is the stress tensor for a static fluid (see Section~\ref{s5.5}), and
\begin{equation}\label{e6.7}
{\bf F} =- \rho\,g\,{\bf e}_z
\end{equation}
the gravitational force density. (Recall that the indices $1$, $2$, and $3$ refer to the $x$-, $y$-, and $z$-axes, respectively. Thus,
$f_3\equiv f_z$, {\em etc.})
The first term on the right-hand side of (\ref{e6.5}) represents the net {\em surface force}\/ acting across
 $S$, whereas the second term represents the net {\em volume force}\/ distributed throughout $V$. 
 Making use of the tensor divergence theorem (see Section~\ref{stfield}), Equations~(\ref{e6.5})--(\ref{e6.7})
yield the following expression for  the net force:
\begin{equation}
{\bf f}= {\bf B} + {\bf W},
\end{equation}
where 
\begin{equation}\label{e6.9}
B_i =- \int_V\frac{\partial p}{\partial x_i}\,dV,
\end{equation}
and
\begin{eqnarray}\label{e6.10}
W_x&=& W_y = 0,\\[0.5ex]
W_z &=&-\int_V \rho\,g\,dV.\label{e6.10b}
\end{eqnarray}
Here, ${\bf B}$ is the  net surface force, and ${\bf W}$   the net volume force. 

It follows from Equations~(\ref{e6.4}) and (\ref{e6.9}) that 
\begin{equation}\label{e6.16}
{\bf B}  = M_0\,g\,{\bf e}_z,
\end{equation}
where 
$M_0 = \rho_0\,V_0$.
Here, $V_0$ is volume of that part of $V$ which lies below the waterline, and $M_0$  the
total mass of water contained within $V$. 
 Moreover, from Equations~(\ref{e6.2}), (\ref{e6.10}), and (\ref{e6.10b}),
\begin{equation}
{\bf W} = -M_0\,g\,{\bf e}_z.
\end{equation}
It can be seen that the net surface force, ${\bf B}$, is directed vertically upward, and exactly balances the
net volume force, ${\bf W}$, which is directed vertically downward. Of course, ${\bf W}$ is the {\em weight}\/ of the water contained
within $V$. On the other hand,  ${\bf B}$,  which is generally known as the {\em buoyancy}\/ force,
is the resultant pressure of the water immediately surrounding $V$.  We conclude that, in  equilibrium,   the net buoyancy force acting across $S$ exactly balances the weight of the water inside $V$, so that the total force acting on the contents of $V$ is zero, as
must be the case for a system in mechanical equilibrium. We can  also deduce that 
the line of action of ${\bf B}$ (which is vertical) passes through the  center of gravity of the water inside $V$.  Otherwise, a  net torque would act on the contents of $V$, which would contradict our  assumption that the system is in 
mechanical equilibrium.

\section{Equilibrium of Floating Bodies}\label{s6.4}
Consider the situation described in the previous section.
Suppose that the fluid contained within  $V$ is replaced by a partially submerged {\em solid body}\/ whose outer surface
corresponds to $S$. Furthermore, suppose that this body is in mechanical equilibrium with the surrounding fluid ({\em i.e.}, it is stationary, and floating on the surface of the water). 
It follows that the pressure distribution in the surrounding fluid is unchanged [since the force balance criterion (\ref{e6.3}) can be integrated to give the pressure distribution (\ref{e6.4}) at all contiguous points in the fluid, provided that the fluid remains 
in mechanical equilibrium]. 
We conclude that the net surface force acting across  $S$ is also unchanged (since this is directly related to the
pressure distribution in the fluid immediately surrounding $V$), which implies that the 
buoyancy force acting on the
floating body is the {\em same}\/ as that acting on the displaced water: {\em i.e.}, the water that previously
occupied $V$. 
In other words, from (\ref{e6.16}), 
\begin{equation}
{\bf B} = W_0\,{\bf e}_z,
\end{equation}
where $W_0=M_0\,g$ and $M_0$ are the weight and mass of the displaced water, respectively.  The fact that the
buoyancy force is unchanged also implies that the vertical line of action of ${\bf B}$ passes through the center of gravity, $H$ (say), of the displaced water. Incidentally, $H$ is generally
known as the {\em center of buoyancy}. 

A floating body of weight $W$ is acted upon by two forces: namely, its own weight, 
\begin{equation}
{\bf W} =- W\,{\bf e}_z,
\end{equation}
and the buoyancy force, ${\bf B}=W_0\,{\bf e}_z$,  due to the pressure of the surrounding water.
Of course, the line of action of ${\bf W}$ passes through the body's center of gravity, $G$ (say).
Now,   to remain in equilibrium, the body must be subject to zero net force and zero net torque. 
The  requirement of zero net force yields  $W_0=W$. In other words, {\em in equilibrium,  the weight of the water displaced by a floating body 
 is equal to the weight of the body}, or, alternatively, {\em in equilibrium, the magnitude of the buoyancy force acting on a floating body  is equal to the weight of the
displaced water}. This famous result is known as {\em Archimedes' principle}. The requirement of
zero net torque implies that, {\em in equilibrium, the center of gravity, $G$, and center of buoyancy, $H$, of
a floating body lie on the same vertical straight-line}.

Consider a floating body of mass $M$ and volume $V$. 
 Let $\rho = M/V$ be the body's
mean mass density. Archimedes' principle implies that, in equilibrium, 
\begin{equation}\label{e6.20}
\frac{V_0}{V} = s,
\end{equation}
where
\begin{equation}
s = \frac{\rho}{\rho_0}
\end{equation}
is termed the body's {\em specific gravity}. (Recall, that $V_0$ is the submerged volume, and $\rho_0$ the
mass density of water.)
We conclude, from (\ref{e6.20}), that the  volume fraction of a floating body  that is submerged  is equal to the body's 
specific gravity. Obviously, the specific gravity must be {\em less}\/ than unity, since the
submerged volume fraction cannot exceed unity.
In fact, if the specific gravity exceeds unity then it is impossible for the buoyancy force to balance the body's weight, and the body
consequently sinks. 

Consider a body of volume $V$ and specific gravity $s$ that floats in equilibrium. It follows, from Equation~(\ref{e6.20}),
that the submerged volume is $V_0=s\,V$. Hence, the volume above the waterline is $V_1= V-V_0 = (1-s)\,V$.
Suppose that the body is inverted such that its previously submerged part is raised above the waterline, and {\em vice versa}: {\em i.e.},
$V_0\leftrightarrow V_1$. According to (\ref{e6.20}),  the body can only remain in equilibrium in this configuration if its specific gravity changes to
\begin{equation}
s' = \frac{V_1}{V}=\frac{V-V_0}{V} =1-s.
\end{equation}
We conclude that for every equilibrium configuration of a floating body of specific gravity $s$ there exists
an inverted equilibrium configuration for a body of the same shape having the complementary
specific gravity $1-s$. 

\section{Vertical Stability of  Floating Bodies}
Consider a floating body of weight $W$ which, in equilibrium, has  a submerged volume $V_0$. Thus, the body's downward weight is balanced by the upward buoyancy force, $B = \rho_0\,V_0\,g$: {\em i.e.},  $\rho_0\, V_0\,g=W$. Let $A_0$ be the cross-sectional area of the body at the waterline ({\em i.e.}, in the
plane $z=0$). It is convenient to define the body's {\em mean draft}\/ (or mean submerged depth) as $\delta_0= V_0/A_0$. Suppose that the
body is displaced slightly downward, without rotation, such that its mean draft becomes $\delta_0+\delta_1$, where
$|\delta_1|\ll \delta_0$. 
Assuming that the
cross-sectional area in the vicinity of the waterline  is constant, the new submerged volume is  $V_0'=A_0\,(\delta_0+\delta_1)= V_0 + A_0\,\delta_1$, and the
new buoyancy force becomes $B'=\rho_0\,V_0'\,g=W+ \rho_0\,A_0\,g\,\delta_1=(1+\delta_1/\delta_0)\,W$. However, the weight of the body  is unchanged. Thus, 
the body's perturbed vertical equation of motion  is written
\begin{equation}
\frac{W}{g}\,\frac{d^2 \delta_1}{dt^2} = W- B' = -\frac{W}{\delta_0}\,\delta_1,
\end{equation}
which reduces to the simple harmonic equation
\begin{equation}
\frac{d^2 \delta_1}{dt^2} = - \frac{g}{\delta_0}\,\delta_1.
\end{equation}
We conclude that when a floating body of mean draft $\delta_0$ is subject to a small vertical displacement it {\em oscillates}\/  about its equilibrium position at the
characteristic frequency
\begin{equation}
\omega = \sqrt{\frac{g}{\delta_0}}.
\end{equation}
It follows that such a body is unconditionally {\em stable}\/ to small vertical displacements. Of course, the above analysis presupposes
that the oscillations take place sufficiently slowly that the water immediately surrounding the  body always remains
in approximate hydrostatic equilibrium. 

\section{Angular Stability of Floating Bodies}\label{s3x6}
Let us now investigate the  stability of  floating bodies to {\em angular}\/ displacements.  For the sake of simplicity, we shall
only consider bodies that have two mutually perpendicular 
planes of symmetry. Suppose that when such a  body is in an equilibrium state the 
two symmetry planes are   {\em vertical},  and correspond to the $x=0$ and $y=0$ planes.
As before, the $z=0$ plane coincides with the surface of the water. It follows, from symmetry,  that when the body is
in an equilibrium
state  its center of gravity, $G$, and center of buoyancy, $H$, both lie on the $z$-axis. 

\begin{figure}
\epsfysize=3in
\centerline{\epsffile{Chapter03/fig3.01.eps}}
\caption{\em Stable and unstable configurations for a floating body.}\label{f6.01}
\end{figure}

Suppose that the body turns through a small angle $\theta$ about some horizontal axis, lying in the plane $z=0$, that goes through the origin. 
Let $GH$ be that, originally vertical, straight-line that passes through the body's center of gravity, $G$, and
 original center of buoyancy, $H$. Owing to the altered shape of the volume of displaced water, the
center of buoyancy is shifted to some new position $H'$. Let the vertical straight-line
going through $H'$ meet $GH$ at $M$. See Figure~\ref{f6.01}. Point $M$ (in the limit $\theta\rightarrow 0$) is called the {\em metacenter}.
In the disturbed state, the body's weight $W$ acts downward through $G$, and
the buoyancy $\rho_0\,V_0\,g$ acts upward through $M$. Let us assume that the submerged volume, $V_0$,
is {\em unchanged}\/ from the equilibrium state (which excludes vertical oscillations from consideration). It follows that the weight and
the buoyancy force are equal and opposite, so
that there is no net force on the body. However, as can be seen from Figure~\ref{f6.01}, the weight and the buoyancy force generate a net {\em torque}\/ of magnitude $\tau=W\,\lambda\,\sin\theta$. Here, $\lambda$ is the length $MG$: {\em i.e.}, the distance  between the metacenter and the
center of gravity. This distance is generally known as the {\em metacentric height}, and is defined such that it is
 positive when $M$ lies above $G$, and {\em vice versa}. 
Moreover, as is also clear from Figure~\ref{f6.01}, when $M$ lies above $G$ the torque acts to reduce $\theta$, and {\em vice versa}.  In the former case, the torque is known as
a {\em righting torque}. We conclude that a floating body is stable to small angular displacements about some horizontal axis lying
in the plane $z=0$
provided that its metacentric height is {\em positive}: {\em i.e.}, provided that the metacenter lies {\em above}\/ the
center of gravity. Since we have already demonstrated that a floating body is unconditionally stable to
small vertical displacements (and since it is also  fairly obvious that such a body is 
neutrally stable to both  horizontal displacements and angular displacements about a vertical axis passing through its center of gravity), it follows that a necessary and sufficient condition for the stability
of a floating body to a general small perturbation (made up of arbitrary linear and angular components)
is that its metacentric height be positive for angular displacements about any horizontal axis. 

\section{Determination of Metacentric Height}
Suppose that the floating body considered in the previous section is in an equilibrium  state.
Let $A_0$ be the  cross-sectional area  at the waterline: {\em i.e.}, in the plane $z=0$. Since  the
body is assumed to be symmetric with respect to
the $x=0$ and $y=0$ planes, we have
\begin{equation}\label{e6.26}
\int_{A_0}x\,dx\,dy = \int_{A_0}y\,dx\,dy=\int_{A_0}x\,y\,dx\,dy=0,
\end{equation}
where the integrals are taken over the whole cross-section at $z=0$. 
Let $\delta(x,y)$ be the body's draft: {\em i.e.}, the vertical distance between the surface of the water and the body's lower boundary. It follows, from symmetry, that $\delta(-x,y)=\delta(x,y)$ and $\delta(x,-y)=\delta(x,y)$. Moreover, the submerged volume is
\begin{equation}\label{e6.27}
V_0 =\int_{A_0}\int_0^\delta dx\,dy\,dz= \int_{A_0}\,\delta (x,y)\,dx\,dy.
\end{equation}
It also follows from symmetry that
\begin{equation}\label{e6.29}
\int_{A_0} x\,
\delta(x,y)\,dx\,dy= \int_{A_0}y\,\delta(x,y)\,dx\,dy = 0.
\end{equation}
Now, the depth of the unperturbed center of buoyancy below the surface of the water is
\begin{equation}\label{e6.30}
 h = \frac{\int_{A_0}\int_0^\delta z\,dx\,dy\,dz}{V_0}=\frac{1}{2V_0}\,\int_{A_0} \delta^2(x,y)\,dx\,dy=\frac{\delta_0^{\,2}\,A_0}{2V_0},
\end{equation}
where
\begin{equation}
\delta_0 = \left(\frac{\int_{A_0}\delta^2(x,y)\,dx\,dy}{A_0}\right)^{1/2}.
\end{equation}
Finally, from symmetry, the unperturbed center of buoyancy lies at $x=y=0$. 

Suppose that the body now turns  through a {\em small}\/ angle $\theta$ about the $x$-axis. 
As is easily demonstrated, the body's new draft becomes
$\delta'(x,y)\simeq \delta(x,y) + \theta\,y$. Hence, the new submerged volume is
\begin{equation}
V_0'=\int_{A_0} [\delta(x,y) + \theta\,y]\,dx\,dy = V_0 + \theta\int_{A_0} y\,dx\,dy = V_0,
\end{equation}
where use has been made of Equations~(\ref{e6.26}) and (\ref{e6.27}). Thus, the submerged volume is
unchanged, as should be the case for a purely angular displacement. The new depth of the center of buoyancy is
\begin{equation}
h' =  \frac{\int_{A_0}\int_0^{\delta'} z\,dx\,dy\,dz}{V_0}=\frac{1}{2V_0}\int_{A_0} [\delta^2(x,y) + 2\theta\,y\,\delta(x,y) + {\cal O}(\theta^2)]\,dx\,dy = h,
\end{equation}
where use has been made of (\ref{e6.29}) and (\ref{e6.30}). Thus, the depth of the center of buoyancy is also unchanged. Moreover, from symmetry, it
is clear that  the center of buoyancy still lies at $x=0$. Finally, the new $y$-coordinate of
the center of buoyancy is
\begin{equation}
 \frac{\int_{A_0}\int_0^{\delta'}y\,dx\,dy\,dz}{V_0}=\frac{\int_{A_0} y\,[\delta(x,y)+\theta\,y]\,dx\,dy}{V_0}=\theta\,\frac{\kappa_x^{\,2}\,A_0}{V_0},
\end{equation}
where use has been made of (\ref{e6.29}). Here,
\begin{equation}
\kappa_x = \left(\frac{\int_{A_0}y^2\,dx\,dy}{A_0}\right)^{1/2},
\end{equation}
is the {\em radius of gyration}\/ of area $A_0$ about the $x$-axis. 

It follows, from the above analysis, that if the floating body under consideration turns  through a
small angle $\theta$ about the $x$-axis then its center of buoyancy shifts {\em horizontally}\/  a distance $\theta\,\kappa_x^{\,2}\,A_0/V_0$ in the plane perpendicular to the axis of rotation.
In other words, the distance $HH'$ in Figure~\ref{f6.01} is $\theta\,\kappa_x^{\,2}\,A_0/V_0$. Simple trigonometry
reveals that $\theta \simeq HH'/MH'$ (assuming that $\theta$ is small). Hence, $MH' = HH'/\theta=\kappa_x^{\,2}\,A_0/V_0$. Now, $MH'$ is the height of the metacenter
relative to the center of buoyancy. However, the center of buoyancy lies a depth $h$ below the surface of the water (which corresponds to the plane $z=0$).
Hence, the $z$-coordinate of the metacenter  is $z_M=\kappa_x^{\,2}\,A_0/V_0- h$. Finally, if $z_G$ and $z_H=-h$ are the $z$-coordinates of the unperturbed
centers of gravity and  buoyancy, respectively,  then 
\begin{equation}
z_M = \frac{\kappa_x^{\,2}\,A_0}{V_0}+ z_H, 
\end{equation}
and
the metacentric height, $\lambda\equiv z_M-z_G$, becomes
\begin{equation}\label{e6.33}
\lambda = \frac{\kappa_x^{\,2}\,A_0}{V_0}-z_{GH},
\end{equation}
where $z_{GH} \equiv z_G-z_H$. Note that, since $\kappa_x^{\,2}\,A_0/V_0>0$, the metacenter always lies {\em above}\/
the center of buoyancy.

A simple extension of the above argument reveals that if  the body turns through
a small angle $\theta$ about the $y$-axis then the metacentric height is
\begin{equation}
\lambda = \frac{\kappa_y^{\,2}\,A_0}{V_0}-z_{GH},
\end{equation}
where 
\begin{equation}
\kappa_y = \left(\frac{\int_{A_0}x^2\,dx\,dy}{A_0}\right)^{1/2},
\end{equation}
 is  radius of gyration of area $A_0$ about the $y$-axis. Finally, as is easily demonstrated, if the body rotates about a
horizontal axis which subtends an angle $\alpha$ with the $x$-axis then
\begin{equation}
\lambda = \frac{\kappa^{\,2}\,A_0}{V_0} -z_{GH},
\end{equation}
where 
\begin{equation}
\kappa^{\,2} = \kappa_x^{\,2}\,\cos^2\alpha+ \kappa_y^{\,2}\,\sin^2\alpha.
\end{equation}
Thus, the minimum value of $\kappa^2$ is the lesser of $\kappa_x^{\,2}$ and $\kappa_y^{\,2}$. It follows that
the equilibrium state in question is unconditionally stable provided it is stable to small
amplitude angular displacements about horizontal axes normal to its two vertical symmetry planes ({\em i.e.}, the $x=0$ and
$y=0$ planes). 

As an example, consider a  uniform rectangular block of specific gravity $s$ floating such that its sides
of length $a$, $b$, and $c$ are parallel to the $x$-, $y$-, and $z$-axes, respectively. Such a block can be
thought of as a very crude model of a ship. 
 The volume of the
block is $V=a\,b\,c$. Hence, the submerged volume is $V_0=s\,V = s\,a\,b\,c$. The cross-sectional
area of the block at the waterline ($z=0$) is $A_0=a\,b$. 
It is easily demonstrated that $\delta(x,y)=\delta_0=V_0/A_0=s\,c$.  Thus, the center of buoyancy lies a depth $h=\delta_0^{\,2}\,A_0/2V_0=s\,c/2$ below
the surface of the water [see Equation~(\ref{e6.30})]. Moreover, by symmetry, the center of gravity is a height $c/2$ above the bottom surface of the block, which is located 
a depth $s\,c$ below the surface of the water. Hence, $z_H = -h=-s\,c/2$, $z_G= c/2-s\,c$, and $z_{GH} = c\,(1-s)/2$. Consider the
stability of the block to small amplitude angular displacements about the $x$-axis. We have
\begin{equation}
\kappa_x^{\,2} = \frac{\int_{-a/2}^{a/2}\int_{-b/2}^{b/2} y^2\,dx\,dy}{a\,b} = \frac{b^2}{12}.
\end{equation}
Hence, from (\ref{e6.33}), the metacentric height is
\begin{equation}
\lambda = \frac{b^2}{12s\,c} - \frac{c}{2}\,(1-s).
\end{equation}
The stability criterion $\lambda>0$ yields
\begin{equation}
\frac{b^2}{6 c^2} - s\,(1-s)> 0.
\end{equation}
Since the maximum value that $s\,(1-s)$ can take is $1/4$, it follows that the block is stable for all
specific gravities when
\begin{equation}
c < c_0 = \sqrt{\frac{2}{3}}\,b.
\end{equation}
On the other hand, if $c>c_0$ then the block is unstable for intermediate specific gravities such that $s_-< s< s_+$, where
\begin{equation}
s_\pm = \frac{1\pm \sqrt{1-c^2/c_0^{\,2}}}{2},
\end{equation}
 and is stable otherwise. 
Assuming that the block is stable, its angular equation of motion is written
\begin{equation}
I\,\frac{d^2\theta}{dt^2} = - W\,\lambda\,\sin\theta\simeq - W\,\lambda\,\theta,
\end{equation}
where 
\begin{equation}
I = \frac{W}{g\,V}\int_{-a/2}^{a/2}\int_{-b/2}^{b/2}\int_{-s\,c}^{c-s\,c}\,(y^2+z^2)\,dx\,dy\,dz= \frac{W}{12g}\left(b^2+ 4\,[(1-s)^3+s^3]\,c^2\right)
\end{equation}
is the moment of inertia of the block about the $x$-axis. Thus, we obtain the the simple harmonic equation 
\begin{equation}
\frac{d^2\theta}{dt^2} = -\omega^2\,\theta,
\end{equation}
where
\begin{equation}\label{e6.46}
\omega^2 = \frac{W\,\lambda}{I}= \frac{g}{s\,c}\,\frac{c_0^{\,2}-4s\,(1-s)\,c^{2}}{c_0^{\,2}+(8/3)\,[(1-s)^3+s^3]\,c^{2}}.
\end{equation} 
We conclude that the block executes small amplitude angular oscillations about the $x$-axis at the angular frequency $\omega$. However, this result is only
accurate in the limit in which the oscillations are sufficiently slow that the water surrounding the block always remains
in approximate hydrostatic equilibrium. For the case of rotation about the $y$-axis, the above analysis
is unchanged except that $a\leftrightarrow b$. 

The metacentric height  of a conventional ship whose length  greatly exceeds its width is typically
much less for rolling ({\em i.e.}, rotation about a horizontal axis
running along the ship's length)  than  for pitching  ({\em i.e.}, rotation about a horizontal axis
perpendicular to the ship's length), since the radius of gyration for pitching greatly exceeds that for rolling.
As is clear
from Equation~(\ref{e6.46}),  a ship with a relatively small  metacentric height (for rolling) has a relatively long roll period, and {\em vice versa}. Now,  an excessively low metacentric height increases the chances of a ship capsizing   if the weather is rough, or if its cargo/ballast shifts, or if it is
damaged and partially flooded.  For this reason, maritime regulatory agencies, such as the International Maritime Organization, specify minimum metacentric heights for various different types of sea-going vessel. A relatively large metacentric height, on the other hand, 
generally renders a ship uncomfortable for passengers and crew, because the ship executes short period rolls, resulting in large g-forces. Such forces also increase the risk that cargo may break loose or shift.  

We saw earlier, in Section~\ref{s6.4}, that if a body of  specific gravity $s$ floats in vertical equilibrium in a certain position
then a body of the same shape, but of specific gravity $1-s$, can float in vertical equilibrium in  the inverted position. We can now
demonstrate that these positions are either both stable, or both unstable, provided the body is
of uniform density. Let $V_1$ and $V_2$ be
the volumes that are above and below the waterline, respectively, in the first position. Let $H_1$ and
$H_2$ be the mean centers of these two volumes, and $H$ that of the whole volume. It follows that
$H_2$ is the center of buoyancy in the first position, $H_1$ the center of buoyancy in the second (inverted) position, and
$H$ the center of gravity in both positions. Moreover,
\begin{equation}
V_1\,\,H_1G = V_2\,\,H_2G = \left(\frac{V_1\,V_2}{V_1+V_2}\right)H_1H_2,
\end{equation}
where $H_1G$ is the distance between points $H_1$ and $G$, {\em etc.}
The metacentric heights in the first and second positions are
\begin{eqnarray}
\lambda_1 &=& \frac{\kappa^{\,2}\,A}{V_1} - H_1G =\frac{1}{V_1}\left[A\,\kappa^{\,2}- \left(\frac{V_1\,V_2}{V_1+V_2}\right)  H_1H_2\right],\\[0.5ex]
\lambda_2 &=& \frac{\kappa^{\,2}\,A}{V_2} - H_2G =\frac{1}{V_2}\left[A\,\kappa^{\,2}- \left(\frac{V_1\,V_2}{V_1+V_2}\right)H_1H_2\right],
\end{eqnarray}
respectively,
where $A$ and $\kappa$ are the area and radius of gyration of the common waterline section, respectively. 
Thus,
\begin{equation}
\lambda_1\,\lambda_2 = \frac{1}{V_1\,V_2}\left[A\,\kappa^{\,2}- \left(\frac{V_1\,V_2}{V_1+V_2}\right)  H_1H_2\right]^2\geq 0,
\end{equation}
which implies that  $\lambda_1\gtreqqless 0$ as $\lambda_2\gtreqqless 0$, and {\em vice versa}. It follows that the first
and second positions are either both stable,  both marginally stable, or both unstable.

\section{Energy of a Floating Body}
The conditions governing the equilibrium and stability of a floating body can also be deduced from
the principle of energy. 

For the sake of simplicity, let us suppose that the water  surface area  is infinite, so that the immersion of the body does not
generate any change in  the water level. The potential energy of the body itself is $W\,z_G$, where 
$W$ is the body's weight, and $z_G$ the height of its center of gravity, $G$,  relative to the surface of the water. If the
body displaces a volume $V_0$ of water then this  effectively means that a weight $\rho_0\,V_0$ of water, whose
center of gravity is located at the center of buoyancy, $H$, is removed, and then spread as an infinitely thin film
over the surface of the water. This involves a gain of potential energy of $-\rho_0\,V_0\,z_H$, where
$z_H$ is the height of $H$ relative to the surface of the water. Now, vertical force balance requires that
$W=\rho_0\,V_0$. Thus, the potential energy of the system is  $W\,z_{GH}$ (modulo an arbitrary additive constant), where 
$z_{GH} \equiv z_G-z_H$ is the height of the center of gravity relative to the center of buoyancy. 

According to the  principles of statics, an equilibrium state corresponds to either  a {\em minimum}\/
or a {\em maximum}\/ 
of the potential energy. However, such an equilibrium is only stable when the potential energy is  {\em minimized}. 
Thus, it follows that a stable equilibrium configuration of a floating body is such as to minimize 
the height of the body's center of gravity relative to its center of buoyancy.

\section{Curve of Buoyancy}
\begin{figure}
\epsfysize=3.5in
\centerline{\epsffile{Chapter03/fig3.02.eps}}
\caption{\em Curve of buoyancy for a floating body.}\label{f6.02}
\end{figure}

Consider a floating body in vertical force balance that is slowly rotated about a horizontal axis normal to one of  its vertical symmetry planes.
Let us take the center of gravity, $G$, which necessarily lies in this plane, as the origin of a coordinate system that is {\em fixed}\/ with respect to the body.  
As illustrated in Figure~\ref{f6.02}, as the body rotates, the locus of its center of buoyancy, $H$, as seen in the fixed reference frame, appears to traces out a curve, $AB$,
in the plane of symmetry.  
This curve is known as the {\em curve of buoyancy}. Let $r$ represent the radial distance from the origin, $G$,
to some point, $H$, on the curve of buoyancy. Note that the tangent to the curve of buoyancy is always orientated
horizontally. This follows because, as was shown in the previous section, small rotations of a floating body in vertical force balance
cause its center of buoyancy to shift horizontally, rather than vertically, in the plane perpendicular to the axis of rotation.
Thus, the  difference in vertical height, $z_{GH}$, between the center of
gravity and the center of buoyancy is equal to the perpendicular distance, $p$, between $G$ and the  tangent
to the curve of buoyancy at $H$. An  equilibrium configuration therefore corresponds to a maximum or a minimum of $p$ as
point $H$ moves along the curve of buoyancy. However, the equilibrium is only stable if $p$ is minimized.  Now, if $R$ is the radius of curvature of the
curve of buoyancy then, according to a standard result in differential calculus, 
\begin{equation}
R = r\,\frac{dr}{dp}.
\end{equation}
Writing this result in the form
\begin{equation}
\delta r = \frac{R}{r}\,\delta p,
\end{equation}
it can be seen that maxima and minima of $\delta p$, which are the points on the
curve of buoyancy where  $\delta p =0$, correspond to the points where  $\delta r= 0$, and are,
thus,  coincident with maxima and minima of $r$. In other words, an equilibrium configuration corresponds to
a point of maximum or minimum $r$ on the curve of buoyancy: {\em i.e.}, a point at which $GH$ meets the
curve at right-angles. At such a point, $r=p$, and the potential energy consequently takes the value $W\,r$. 

Let $H_0$ be a point on the curve of buoyancy, 
and let $r_0$, $p_0$, and $R_0$ be the corresponding values of $r$, $p$, and $R$. For neighboring points on the curve, we can write
\begin{equation}
r-r_0=\left. \frac{dr}{dp}\right|_{H_0}\,(p-p_0),
\end{equation}
or
\begin{equation}
r -r_0 = \frac{R_0}{r_0}\,(p-p_0).
\end{equation}
It follows that $p-p_0$ has the same sign as $r-r_0$ (since $R_0$ and $r_0$ are both positive). [The fact that $R_0$ is positive
({\em i.e.}, $dr/dp>0$)
follows from the previously established result that the metacenter, which is the center of curvature of the curve of buoyancy,   always lies above the center of buoyancy,  implying 
 that the curve of
buoyancy is necessarily  concave upwards.]
Hence,
the minima and maxima of $r$ occur simultaneously with those of $p$. Consequently, a stable equilibrium
configuration corresponds to a point of minimum  $r$ on the curve of buoyancy:
{\em i.e.}, a {\em minimum}\/ in the distance $GH$ between the center of gravity and the center of buoyancy.

We can use the above result to determine the stable equilibrium configurations for a beam of
square cross-section, and uniform specific gravity $s$, that  floats with its length horizontal. 
In order to achieve this goal, we must calculate the distance $GH$ for all possible configurations
of the beam that are in vertical force balance. However, we need only consider cases where $s<1/2$, since, according to
the analysis of Section~\ref{s3x6}, for every stable equilibrium configuration with  $s=s_0<1/2$
there is a corresponding stable inverted configuration with $s = 1-s_0>1/2$, and {\em vice versa}. 

\begin{figure}
\epsfysize=3in
\centerline{\epsffile{Chapter03/fig3.03.eps}}
\caption{\em Beam of square cross-section floating with two corners immersed.}\label{f6.03}
\end{figure}

Let us define fixed rectangular axes, $x$ and $y$,  passing through the center of the middle section of the beam, and running parallel
to its sides. Let us start with the case where the waterline $PQ$ is parallel to a side. See Figure~\ref{f6.03}.
If the length of a side is $2a$ then (\ref{e6.20})
yields
\begin{equation}
AP=BQ=2a\,s.
\end{equation}

Suppose that the beam is  turned through an angle $\theta>0$ such that the waterline assumes the position $P'Q'$, in Figure~\ref{f6.03},
but still intersects two opposite sides. The lengths $AP'$ and $BQ'$ satisfy
\begin{equation}
BQ' - AP' = 2a\,\tan\theta.
\end{equation}
Moreover,  the area of the trapezium $P'ABQ'$ must match that of the  rectangle $PABQ$
in order to ensure that the submerged volume remains invariant (otherwise, the beam would not remain in vertical force balance): {\em i.e.}, 
\begin{equation}
(AP' + BQ')\,a = 4a^2\,s.
\end{equation}
It follows that
\begin{eqnarray}
AP' &=& a\,(2s-\tan\theta),\\[0.5ex]
BQ' &=& a\,(2s+\tan\theta).
\end{eqnarray} 
The constraint that the waterline intersect two opposite sides of the beam implies that $AP'>0$, and, hence, that
\begin{equation}\label{e6.57}
\tan\theta < 2s.
\end{equation}
The coordinates of the center of buoyancy, $H$, which is the mean center of the trapezium $P'ABQ'$, are
\begin{eqnarray}
\bar{x} &=& 
\frac{\int_{-a}^a\int_{a-h(x)}^a x\,dx\,dy}{\int_{-a}^a\int_{a-h(x)}^a dx\,dy} = \frac{(2/3)\,a^3\,\tan\theta}{4a^2\,s}=\frac{a}{6s}\,\tan\theta,\\[0.5ex]
\bar{y}&=&\frac{\int_{-a}^a\int_{a-h(x)}^a y\,dx\,dy}{\int_{-a}^a\int_{a-h(x)}^a dx\,dy} \nonumber\\[0.5ex]
&=& \frac{4a^3\,s\,(1-s)-(1/3)\,a^3\,\tan^2\theta}{4a^2\,s}=(1-s)\,a - \frac{a}{12s}\,\tan^2\theta,
\end{eqnarray} 
where
\begin{equation}
h(x) = 2a\,s+x\,\tan\theta.
\end{equation}
Thus, if $u=r^2/a^2=(\bar{x}^2+ \bar{y}^2)/a^2$ then
\begin{equation}
u = \frac{t^2}{36s^2} + \left[(1-s)-\frac{t^2}{12s}\right]^2,
\end{equation}
where $t=\tan\theta$. Now, a stable equilibrium state corresponds to a minimum of $r$ with respect to $\theta$, and, hence, of $u$  with respect to $t$. 
However,
\begin{eqnarray}
\frac{du}{dt} &=& \frac{t}{36s^2}\left[t^2-12s\,(1-s)+2\right],\\[0.5ex]
\frac{d^2u}{dt^2} &=& \frac{1}{36s^2}\left[3t^2-12s\,(1-s)+2\right].
\end{eqnarray}
The minima and maxima of $u$ occur when $du/dt=0$, $d^2u/dt^2>0$ and $du/dt=0$, $d^2u/dt^2<0$, respectively.
It follows that the symmetrical position, $t=0$, in which the sides of the beam are either parallel or perpendicular to the
waterline, is always an equilibrium,  but is only stable when
\begin{equation}
s^2-s+\frac{1}{6}>0:
\end{equation}
{\em i.e.}, when $s<1/2-1/\sqrt{12}= 0.2113$. 
It is also possible to obtain equilibria in asymmetric positions such that $t$ is the root of
\begin{equation}
t^2= 12s\,(1-s)-2.
\end{equation}
Such equilibria only exist for $s>0.2113$, and are stable. Finally, in order to satisfy the  constraint (\ref{e6.57}), we
must have $t<2s$, which, in combination with the above equation, implies that
\begin{equation}
8s^2-6s +1>0,
\end{equation}
or $s<0.25$. 

\begin{figure}
\epsfysize=3in
\centerline{\epsffile{Chapter03/fig3.04.eps}}
\caption{\em Beam of square cross-section floating with one corner immersed.}\label{f6.04}
\end{figure}

Suppose that the constraint (\ref{e6.57}) is not satisfied, so that the immersed portion of the beam's cross-section
is triangular. See Figure~\ref{f6.04}. It is clear that
\begin{equation}
\frac{BQ'}{BP'} = \tan\theta.
\end{equation}
Moreover, the area of the triangle $P'BQ'$, in Figure~\ref{f6.04}, must match that of the rectangle $PABQ$, in
Figure~\ref{f6.03}, in order to ensure that the submerged volume remain invariant: {\em i.e.},
\begin{equation}
\frac{1}{2}\,BP'\,\,BQ'= 4a^2\,s.
\end{equation}
It follows that
\begin{eqnarray}
BP' &=& (8s/\tan\theta)^{1/2}\,a,\\[0.5ex]
BQ' &=& (8s\,\tan\theta)^{1/2}\,a,
\end{eqnarray}
or, writing $z^2=\tan\theta$ and $\omega^2 = (8/9)\,s$, 
\begin{eqnarray}
BP' &=& 3\omega\,z^{-1}\,a,\\[0.5ex]
BQ'&=& 3\omega\,z\,a.
\end{eqnarray}
The coordinates of the center of buoyancy, $H$, which is the mean center of triangle $P'BQ'$, are
\begin{eqnarray}
\bar{x} &=& a- BP'/3= a\,(1-\omega\,z^{-1}),\\[0.5ex]
\bar{y} &=& a-BQ'/3=a\,(1-\omega\,z),
\end{eqnarray}
since the perpendicular distance of the mean center of a triangle from one of its sides is one third of the perpendicular distance
from the side to the opposite vertex.
Thus, if $u=r^2/a^2=(\bar{x}^2+ \bar{y}^2)/a^2$ then
\begin{eqnarray}
u &=&(1-\omega\,z^{-1})^2 + (1-\omega\,z)^2,\\[0.5ex]
\frac{du}{dz} &=& \frac{2\omega^2\,(z^2-1)\,(z^2-\omega^{-1}\,z+1)}{z^3},\\[0.5ex]
\frac{d^2u}{dz^2} &=& \frac{2\omega^2\,(z^4-2\omega^{-1}\,z+3)}{z^4}.
\end{eqnarray}
Moreover, the constraint (\ref{e6.57}) yields
\begin{equation}
z > \frac{3}{2}\,\omega.
\end{equation}
The stable and unstable equilibria correspond to $du/dz=0$, $d^2u/dz^2>0$ and $du/dz$, $d^2u/dz^2<0$, respectively. 
It follows that the symmetrical position, $z=1$, in which the diagonals of the beam are either parallel or perpendicular to the
waterline is an equilibrium provided $\omega<2/3$, or $s<1/2$,  but is only stable when
$\omega>1/2$, or $s>9/32=0.28125$. 
It is also possible to obtain equilibria in asymmetric positions such that $z$ is the root of
\begin{equation}
z^2-\omega^{-1}\,z+1=0.
\end{equation}
Such equilibria only exist for $\sqrt{2}/3<\omega<1/2$, or $1/4<s<9/32$, and are stable. 

In summary, the stable equilibrium configurations of a beam of square cross-section, floating with its length horizontal, are such that
the sides are either parallel or perpendicular to the waterline for $s<0.2113$, such
that two corners are immersed but the  sides and diagonals are neither parallel nor perpendicular to the waterline for $0.2113<s<0.25$, such that only one corner is immersed but the
sides and diagonals are neither parallel nor perpendicular to the waterline for $0.25<s<0.28125$, and such that the diagonals are either parallel or perpendicular to waterline for $0.28125<s<0.5$. For $s>0.5$, the stable configurations are the same as those for a
beam with the complimentary specific gravity $1-s$. 

\section{Rotational Hydrostatics}
Consider, finally, the equilibrium of an  incompressible fluid that is {\em uniformly}\/ rotating at a
{\em fixed}\/ angular velocity $\bomega$ in some inertial  frame of reference. Of course,  such a fluid appears {\em stationary}\/ in a non-inertial {\em co-rotating}\/ reference frame. Moreover, according to standard Newtonian dynamics, the force balance equation for the fluid in the
co-rotating frame takes the form ({\em cf.}, Section~\ref{s6.2})
\begin{equation}\label{e8.1}
{\bf 0} = \nabla p + \rho\,\nabla\Psi + \rho\,\bomega\times (\bomega\times {\bf r}),
\end{equation}
where $p$ is the static fluid pressure, $\rho$ the mass density, $\Psi$ the gravitational potential energy
per unit mass, and ${\bf r}$ a position vector (measured with respect to an origin that lies on the
axis of rotation). The final term on the right-hand side of the above equation represents the fictitious {\em centrifugal force density}. 
Without loss of generality, we can assume that $\bomega = \omega\,{\bf e}_z$. It follows
that
\begin{equation}\label{e8.2}
{\bf 0} = \nabla p + \rho\,\nabla(\Psi+\Psi') ,
\end{equation}
where 
\begin{equation}\label{e8.3}
\Psi' = -\frac{1}{2}\,\omega^2\,(x^2+y^2)
\end{equation}
is the so-called {\em centrifugal potential}. 
Recall, incidentally, that $\rho$ is a uniform constant in an incompressible fluid.

As an example, consider the equilibrium of a body of water, located on the Earth's surface, that is uniformly rotating about a vertical
axis at the fixed angular velocity $\omega$. It is convenient to adopt cylindrical
coordinates (see Section~\ref{scyl}), $r$, $\theta$, $z$, whose symmetry axis coincides with the axis of rotation. Let $z$ increase upward.  It follows
that $\Psi=-g\,z$ and $\Psi' =-(1/2)\,\omega^2\,r^2$. Assuming that the pressure distribution is {\em axisymmetric},
so that $p=p(r,z)$, the force balance equation, (\ref{e8.2}), reduces to
\begin{eqnarray}
\frac{\partial p}{\partial r} + \rho\,\frac{\partial (\Psi+\Psi')}{\partial r} &=&0,\\[0.5ex]
\frac{\partial p}{\partial z} + \rho\,\frac{\partial (\Psi+\Psi')}{\partial z} &=&0,
\end{eqnarray}
or
\begin{eqnarray}
\frac{\partial p}{\partial r} - \rho\,\omega^2\,r&=&0,\\[0.5ex]
\frac{\partial p}{\partial z} +\rho\,g &=&0.
\end{eqnarray}
The previous two equations can be integrated to give
\begin{equation}
p(r,z) = p_0 + \rho\left(\frac{1}{2}\,\omega^2\,r^2- g\,z\right),
\end{equation}
where $p_0$ is a constant. Thus, constant pressure surfaces in a uniformly rotating
body of water take the form of {\em paraboloids of revolution}\/ about the rotation axis. Suppose that $p_0$ represents
atmospheric pressure. In this case, the surface of the water is the locus of $p(r,z)=p_0$: {\em i.e.},
it is the constant pressure surface whose pressure matches that of the atmosphere. It follows that the
surface of the water is the paraboloid of revolution 
\begin{equation}
z = \frac{\omega^{\,2}}{2g}\,r^2,
\end{equation}
where  $r$ is the perpendicular distance from the axis
of rotation, and $z=0$  the on-axis height of the surface.

Now, from Section~\ref{sbouy}, it is plain that the  buoyancy force acting on any co-rotating solid body which is wholly or
partially immersed in the water is the same as that which would maintain the
mass of water displaced by the body in relative equilibrium. In the case of a floating body, this mass is 
limited by the continuation of the water's curved surface through the body. Let points $G$ and
$H$ represent the centers of gravity and  buoyancy, respectively, of the body. Of course, the latter point is
simply the center of gravity of the displaced water. Suppose that $G$ and $H$ are located perpendicular
distances $r_G$ and $r_H$ from the axis of rotation, respectively. Finally, let $M$ be the mass of the body, and
$M_0$ the mass of the displaced water. It follows that the buoyancy force has an upward vertical
component $M_0\,g$, and an outward horizontal component $-M_0\,\omega^2\,r_H$. Thus, according to
standard Newtonian dynamics, the
equation of horizontal motion of a general co-rotating body is
\begin{equation}
M\,(\ddot{r} - \omega^2\,r_G) = -M_0\,\omega^2\,r_H,
\end{equation}
where $\dot{~}\equiv d/dt$. 
Now,  from Archimedes' principle,  $M_0=M$ for the case of a floating body that is less dense than water. 
However, if the body is of {\em uniform}\/ density then $r_H>r_G$, as a consequence of the curvature of the water's surface. Hence,
we obtain
\begin{equation}
\ddot{r} = -\omega^2\,(r_H-r_G) <0.
\end{equation}
In other words,  a  floating body drifts radially {\em inward}\/ towards the rotation axis. On the other hand, $M_0<M$ for a fully submerged
body that is more dense than water. However, if the body is of {\em uniform}\/ density then its centers of
gravity and buoyancy coincide with one another, so that $r_H=r_G$. Hence, we obtain
\begin{equation}
\ddot{r} = (M-M_0)\,\omega^2\,r_G>0.
\end{equation}
In other words, a fully submerged body drifts radially {\em outward}\/ from the rotation axis. The above analysis accounts for the
common observation that objects heavier than water, such as grains of sand, tend to
collect on the outer side of a bend in a fast flowing river, whilst floating objects, such as sticks,
tend to collect on the inner side. 


\section{Exercises}
{\small 
\renewcommand{\theenumi}{3.\arabic{enumi}}
\begin{enumerate}
\item A hollow vessel floats in a basin. If, as a consequence of a leak, water flows slowly into the vessel, how
will the level of the water in the basin be affected?
\item A hollow spherical shell made up of material of specific gravity $s>1$ has external and internal
radii $a$ and $b$, respectively. Demonstrate that the sphere will only float in water if
$$
\frac{b}{a}> \left(1-\frac{1}{s}\right)^{1/3}.
$$
\item Show that the equilibrium of a solid of uniform density floating with an edge or corner just emerging
from the water is unstable.
\item Prove that if a solid of uniform density floats with a flat face just above the waterline then the
equilibrium is stable.
\item Demonstrate that a uniform solid cylinder floating with its axis horizontal is in a stable equilibrium
provided that its length exceeds the breadth of the waterline section. [Hint: The cylinder is obviously neutrally
stable to rotations about its axis, which means that the corresponding metacentric height is zero.]
\item Show that a uniform solid cylinder of radius $a$ and height $h$ can float in stable equilibrium, with its
axis vertical, if $h/a<\sqrt{2}$. If the ratio $h/a$ exceeds this value, prove that the equilibrium is
 only stable when  the specific gravity of the cylinder  lies outside the range
$$
\frac{1}{2}\left(1\pm \sqrt{1-2\,\frac{a^2}{h^2}}\,\right).
$$
\item A uniform, thin, hollow cylinder of radius $a$ and height $h$ is open at both ends. Assuming that $h>2a$, prove that the
cylinder  cannot
float upright if its specific gravity lies in the range
$$
\frac{1}{2}\pm \sqrt{\frac{1}{4}-\frac{a^2}{h^2}}.
$$
\item Show that the cylinder of the preceding example can float with its axis horizontal provided
$$
\frac{h}{2a}> \sqrt{3}\,\sin(s\,\pi),
$$
where $s$ is the specific gravity of the cylinder.
\item Prove that any segment of a uniform sphere, made up of a substance lighter than water, can float in
stable equilibrium with its plane surface horizontal and immersed.
\item A vessel carries a tank of oil, of specific gravity $s$, amidships. Show that the effect of the fluidity of the
oil on the rolling of the vessel is equivalent to a reduction in the metacentric height by $A\,\kappa^2\,s/V$,
where $V$ is the displacement of the ship, $A$ the surface-area of the tank, and $\kappa$ the radius
of gyration of this area. In what ratio is the effect diminished when a longitudinal partition bisects the
tank?
\item Find the stable equilibrium configurations of a cylinder of elliptic cross-section, with major and minor radii $a$ and
$b<a$, respectively, made up of material of specific gravity $s$, which floats with its axis horizontal.
\item A cylindrical tank has a circular cross-section of radius $a$. 
Let the center of gravity of the tank be located a
distance $c$ above its base. Suppose that the tank is pivoted about a horizontal axis passing through its
center of gravity, and is then filled with fluid up to a depth $h$ above its base. Demonstrate that the position in which the
tank's axis is upright is unstable for all filling depths provided 
$$
c^2< \frac{1}{2}\,a^2.
$$
Show that if $c^2>(1/2)\,a^2$ then the upright position is stable when $h$ lies in the range
$$
 c\pm \sqrt{c^2-a^2/2}.
$$
\item A thin cylindrical vessel of cross-sectional area $A$ floats upright, being immersed to a depth $h$, and
contains water to a depth $k$. Show that the work required to pump out the water is $\rho_0\,A\,k\,(h-k)$. 
\item A sphere of radius $a$ is just immersed in water that is contained in a cylindrical vessel of radius $R$ whose
axis is vertical. Prove that if the sphere is raised just clear of the water then the water's loss of potential energy is 
$$
W\,a\left(1-\frac{2}{3}\,\frac{a^2}{R^2}\right),
$$
where $W$ is the weight of the water originally displaced by the sphere.
\item A sphere of radius $a$, weight $W$, and specific gravity $s>1$, rests on the bottom of
a cylindrical vessel of radius $R$ whose axis is vertical, and which contains
water to a depth $h>2a$. Show that the work required to lift the sphere out of the 
vessel is less than if the water had been absent by an amount
$$
\left(h-a-\frac{2}{3}\,\frac{a^3}{R^2}\right)\frac{W}{s}.
$$
\item A lead weight is immersed in water that is steadily rotating at an angular velocity $\omega$ about a vertical axis, the
weight being suspended from a fixed point on this axis by a string of length $l$. Prove that the
position in which the weight hangs vertically downward is stable or unstable depending on whether $l< g/\omega^2$ or $l>g/\omega^2$,
respectively. Also, show that if the vertical position is unstable then there exists a stable inclined position in which the
string is normal to the surface of equal pressure passing though the weight.
\item A thin cylindrical vessel of radius $a$ and height $H$ is orientated such that its axis is vertical. Suppose that the
vessel is filled with liquid of density $\rho$ to 
some height $h<H$ above the base, spun about its axis at a steady angular velocity $\omega$, and the liquid
allowed to attain a steady state. Demonstrate that, provided $\omega^2\,a^2/g < 4\,h$ and $\omega^2\,a^2/g< 4\,(H-h)$,
the net radial thrust on the vertical walls of the vessel is
$$
\pi\,a\,h^2\,\rho\,g\left(1+ \frac{\omega^2\,a^2}{4\,g\,h}\right)^2.
$$
\item A thin cylindrical vessel of radius $a$ with a plane horizontal lid is just filled with liquid of density $\rho$, and
the whole rotated about a vertical axis at a fixed angular velocity $\omega$. Prove that the net upward thrust of the
fluid on the lid is
$$
\frac{1}{4}\,\pi\,a^4\,\rho\,\omega^2.
$$
\item A liquid-filled thin spherical vessel of radius $a$ spins about a vertical diameter at the fixed angular velocity $\omega$.
Assuming that the liquid co-rotates with the vessel, and that $\omega^2>g/a$, show that the pressure on the wall of the
vessel is greatest  a depth $g/\omega^2$ below the center. Also prove that the net normal thrusts on the lower
and upper hemispheres are
$$
\frac{5}{4}\,M\,g+ \frac{3}{16}\,M\,\omega^2\,a,
$$
and
$$
\frac{1}{4}\,M\,g - \frac{3}{16}\,M\,\omega^2\,a,
$$
respectively,
where $M$ is the mass of the liquid.
\item A closed cubic vessel filled with water is rotating about a vertical axis passing through the centers of two opposite sides.
Demonstrate that, as a consequence of the rotation, the net thrust on a side is increased by
$$
\frac{1}{6}\,a^4\,\rho\,\omega^2,
$$
where $a$ is the length of an edge of the cube, and $\omega$  the angular velocity of rotation.
\item A closed vessel filled with water is rotating at constant angular velocity $\omega$ about a horizontal axis. Show that,
in the state of relative equilibrium, the constant pressure surfaces in the water are circular cylinders whose common axis
is a height $g/\omega^2$ above the axis of rotation.

\end{enumerate}}